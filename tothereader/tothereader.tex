\documentclass[main.tex]{subfiles}



\setlength{\columnsep}{30pt}

\begin{document}
\pagenumbering{Roman}
\pagestyle{style1}
\newgeometry{left=0.8in,right=2.8in, top=3.5cm,bottom=2cm}

\begin{fullwidth}
\begin{multicols*}{2}
First and foremost, I genuinely care about the progress of each and every one of my students and I want to see you all succeed. This is why I decided to write this manuscript. This set of lecture notes was designed with a focus on the student--with a focus on you. It introduces the basic concepts of college chemistry in a way that a student of any level can hopefully understand. These notes start with the fundamentals and--at this point--end with solids and liquids. Some of the chapters included in this guide can be challenging. Success is not an accident. Only with hard work, patience and perseverance you will be able to achieve what you want. I hope to encourage you not only to successfully pass this class. More importantly, I hope to inspire you to see that you can do this. 

College chemistry is not an easy subject. You may experience frustration due to the terminology or the math content. This guide is developed in chapters and sections in order to break down the very basics of the chemistry concepts. One of my mail goals is to help you solve chemistry problems. Solving problems--not only chemistry problems but problems of any kind--is an extremely useful skill in life. Chemists approach the solving of problems in a very specific way. They use critical thinking and previous knowledge in order to find the solution based on the information presented. As you study this set of lecture, I encourage you to read the different section of a chapter, highlight the main ideas and find key words that represent new concepts. Numerous examples are presented along the chapters with the full solution. A lot of examples are also presented without the worked solution, just including the answer. Plenty of end of the chapter problems are further included. After you read the content of a chapter I highly encourage you to work on the end of the chapter problems. As with any skill, practice makes perfect.

I used numerous tools along this guide to help you focus on the most relevant content. For example, \emph{yellow notes} are used to indicate important formulas or tables. Also, when the numerical problems get to complex, an \emph{analyze the problem box} is included to help you identify what is given and what is asked in the problem.

This set of lectures resonates with the open textbook movement that is taking over CUNY as well as SUNY. Education is expensive and you as student often rely on textbooks to learn. These valuable educational resources are often used for a very limited period of time and tossed or returned when a class has finished. The open textbook movement aims to alleviate the cost of education by relying on resources that are free for both the students and for the educators. Still, these sources are imperfect and not as curated as textbooks, and this is the price to pay. I warn you this set of lecture is indeed imperfect, and hence its title. Yet, it is the result of many hours of work--indeed months of work. Still, it contains typos and often times incorrect answers. Your role is key. I encourage you first to be understanding and patient, and then to contribute to the development of this guide. With your input we can make this guide a better educational resource. Mind that this guide was written by an educator and as such it sometimes use terms and a way of thinking that correspond to the educators\textquotesingle point of view.

This set of lecture does not intent to replace any textbook. Indeed, there are many high-quality textbooks in the literature that I recommend:

\begin{small}\begin{itemize}[label=\resizebox{!}{.7em}{\rotatebox[origin=c]{-90}{\includegraphics{./tothereader/lis}}}]
\setlength\itemsep{0.5em}
\item Chemical Principles: The Quest for Insight by Peter Atkins et al.
\item Chemistry: The Central Science by Theodore E. Brown et al.
\item Chemistry by Steven S. Zumdahl et al.
\item Chemistry: The Molecular Nature of Matter and Change by Martin Silberberg et al.
\item Chemistry by Raymond Chang et al.
\item Chemistry: Atoms First by OpenStax
\end{itemize}\end{small}

With the help of the textbooks above you can certainly expand and complement the information presented in this guide.

This guide was fully coded in \LaTeX from the cover or the periodic table to the molecular orbital diagrams or the solid representations. Chemistry is a microscopic science not accessible to the naked eye. Visuals play a very important role in chemistry education. Visuals--in the form of images or diagrams--helps makes chemistry more apparent to the viewer. One of the weak points of many open education chemistry guides are the visuals. They tend to be simplistic with low quality. This guide extensively relies on images and diagrams and uses Tikz and other open-source tools to generate diagrams. All other images used here are open-source images.

The work of chemists is certainly challenging, but also exciting and rewarding. Chemists produce everything from plastics and paints to pharmaceuticals, foods, flavors, fragrances, detergents, and cosmetics. Chemistry students are well-prepared for medical, veterinarian, dentistry, optometry or pharmacy school. I hope you enjoy this guide and more importantly I wish you success in your career.  
\par \medskip
\includegraphics[height=4.5\baselineskip,]{./tothereader/signature} \par
Daniel Torres \par
New York City
\end{multicols*}
\end{fullwidth}
\restoregeometry
\end{document}
