\begin{figure*}[!ht] %%%%%%%% EXTRA PROBLEMS BOX
\begin{summary}[boxed title style={colback=white},colback=white]
{\raggedright\textsc{\textbf{Titrations and buffers}}\par}
\begin{multicols}{2}\setstretch{0.3}
\begin{enumerate}[nolistsep]
\item A 5mL sample of an unknown acid is neutralized with 40 mL of a \ce{KOH} 0.5M solution. Calculate the molarity of the unknown acid.
\begin{flushright}\small Ans: $4M$ \end{flushright}
\item A 0.05L sample of an unknown acid is neutralized with 5 mL of a \ce{KOH} 2M solution. Calculate the molarity of the unknown acid.
\begin{flushright}\small Ans: $2M$ \end{flushright}
\item In order to standardize an HCl solution of unknown concentration, you add 25.00 mL of this acid in a  flask, and then add a few drops of an indicator. On the buret you use 0.2 M \ce{NaOH}. Before the titration, the buret reads 1 mL and 31 mL at the end point. Find the molarity of the HCl solution.
\begin{flushright}\small Ans: 0.24M \end{flushright}
\item What volume of 0.5 M \ce{KOH} would neutralize 10 mL of the 3M-\ce{HCl}. 
\begin{flushright}\small Ans: $60mL$ \end{flushright}
\end{enumerate}
\end{multicols}
\end{summary}
\end{figure*} %%%%%%%% EXTRA PROBLEMS BOX