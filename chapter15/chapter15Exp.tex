\documentclass[main.tex]{subfiles}
\begin{document}\newpage
\setdoublesep{0.35700 em}  % 'Bond Spacing'
\setatomsep{1.78500 em}    % 'Fixed Length'
\setbondoffset{0.18265 em} % 'Margin Width'
\newcommand{\bondwidth}{0.06642 em} % 'Line Width'
\setbondstyle{line width = \bondwidth}

\begin{fullwidth}





%%%%%%%%%%%%HEADING
\begin{multicols}{2}
\begin{tcolorbox}[enhanced jigsaw,breakable,size=title,
colback=mybrown!05,colframe=black,fonttitle=\bfseries,
title=STUDENT INFO,pad at break=1mm, break at=15cm/0pt ]
\vspace{0.2cm}
\noindent Name: \rule{5cm}{0.4pt}Date:\rule{1cm}{0.4pt}\\
Pre-lab Done: \tikzcheckmark[scale=2,black]{no mark}\quad
\end{tcolorbox}
\end{multicols}
\hfill
\vspace{0.2cm}
\begin{center}
{\large \bfseries 
Pre-lab Questions 
\par
\Huge
Organic Molecules
\\[5pt] \par}
\vspace{0.2cm}
\end{center}
\par
\noindent
\uline{  \hfill \normalsize \hfill       }
%%%%%%%%%%%%HEADING

\begin{enumerate}
% PELAB 1
\item Given the following molecular formula, name the following linear alkanes (hydrocarbons):

\begin{multicols}{2}
 \begin{tabular}{ p{3cm} p{4cm}    }
 \ce{CH4} &\rule{4cm}{0.4pt}        \\
   \ce{C4H10} &\rule{4cm}{0.4pt}        \\
    \ce{C5H12} &\rule{4cm}{0.4pt}        \\
\end{tabular}
 \begin{tabular}{ p{3cm} p{3cm}    }
 \ce{C2H6} &\rule{3.8cm}{0.4pt}        \\
   \ce{C3H8} &\rule{3.8cm}{0.4pt}        \\
    \ce{C9H20} &\rule{3.8cm}{0.4pt}        \\
 \end{tabular}
\end{multicols}

\item Given the following molecular formula, name the following cyclic alkanes (hydrocarbons):
\begin{multicols}{2}
 \begin{tabular}{ p{3cm} p{4cm}    }
 \ce{C3H6} &\rule{4cm}{0.4pt}        \\
   \ce{C4H8} &\rule{4cm}{0.4pt}        \\
    \ce{C5H10} &\rule{4cm}{0.4pt}        \\
\end{tabular}
 \begin{tabular}{ p{3cm} p{3cm}    }
 \ce{C6H12} &\rule{3.8cm}{0.4pt}        \\
   \ce{C7H14} &\rule{3.8cm}{0.4pt}        \\
    \ce{C9H18} &\rule{3.8cm}{0.4pt}        \\
 \end{tabular}
\end{multicols}







\item Indicate the molecular, expanded, condensed and skeletal formula for the following linear alkanes:
\begin{center}\begin{tabular}{ |p{1.3cm}|p{1.3cm}|p{4cm}| m{4cm}| m{4cm}| }
\hline
& Molecular Formula    & Expanded Formula &   Condensed Formula & Skeletal Formula   \\
\hline
\vspace{0cm}Hexane\vspace{0.8cm} &  &     & & \\
\hline
\vspace{0cm}Pentane\vspace{0.8cm} &  &  &    & \\
\hline
\end{tabular}\end{center}



\item Identify the functional groups:
\begin{center}\begin{tabular}{ |p{4cm}|p{4cm}|p{4cm}| m{4cm}| }
\hline
Molecule &  Functional Group   &Molecule  &Functional Group       \\
\hline
\vspace{0cm}\begin{center}\chemfig{*6((-H_3C)---N(-C(=[::+60]O)-[::-60])---)}\end{center}\vspace{0.8cm} &  &\chemfig{[7]H_3C-C(=[6]O)-[1]CH_2-CH_2-[1]CH_3}     &  \\
\hline
\vspace{0cm}\begin{center}  \chemfig{-[::+45]=[::-45]-[::+45]OH} \end{center}\vspace{0.8cm} &  &     \chemfig{[7]H-C(=[6]O)-[1]CH_2-CH_2-[1]CH_3}
&  \\
\hline
\end{tabular}\end{center}

\end{enumerate}


\clearpage\mbox{}\clearpage



%%%%%%%%%%%%HEADING
\begin{multicols}{2}
\begin{tcolorbox}[enhanced jigsaw,breakable,size=title,
colback=mybrown!05,colframe=black,fonttitle=\bfseries,
title=STUDENT INFO,pad at break=1mm, break at=15cm/0pt ]
\vspace{0.2cm}
\noindent Name: \rule{5cm}{0.4pt}Date:\rule{1cm}{0.4pt}\\
Pre-lab Done: \tikzcheckmark[scale=2,black]{no mark}\quad
\end{tcolorbox}
\end{multicols}
\hfill
\vspace{0.2cm}
\begin{center}
{\large \bfseries 
Experiment
\par
\Huge
Organic Molecules
\\[5pt] \par}
\vspace{0.2cm}
\end{center}
\par
\noindent
\uline{  \hfill \normalsize \hfill       }
%%%%%%%%%%%%HEADING

\vspace{0.2cm}{\large \bfseries Linear Alkanes}
Use the molecular models set for this experiment. Each sphere represents an element. Carbon is black, hydrogen white, oxygen red and nitrogen blue. Build up the following molecules and complete the table. Show your professor all molecular models before proceeding to next part.
\begin{center}\begin{tabular}{ |p{1.3cm}|p{5cm}| m{5cm}| m{5cm}| }
\hline
    & Expanded Formula &   Condensed Formula & Skeletal Formula   \\
\hline
\vspace{0cm}Methane\vspace{1.5cm} &       & & \begin{center}\Huge N/A\end{center} \\
\hline
\vspace{0cm}Ethane\vspace{1.5cm} &    &    & \\
\hline
\vspace{0cm}Propane\vspace{1.5cm} &  &      & \\
\hline
\vspace{0cm}Butane\vspace{1.5cm} &  &      & \\
\hline
\end{tabular}\end{center}
\end{fullwidth}

\newpage
\begin{fullwidth}
\vspace{0.2cm}{\large \bfseries Cyclic Alkanes}
Use the molecular models set for this experiment. Each sphere represents an element. Carbon is black, hydrogen white, oxygen red and nitrogen blue. Build up the following molecules and complete the table. Show your professor all molecular models before proceeding to next part.
\begin{center}\begin{tabular}{ |p{2cm}|p{5cm}| m{5cm}| m{5cm}| }
\hline
    & Expanded Formula &   Condensed Formula & Skeletal Formula   \\
\hline
\vspace{0cm}Cyclopropane\vspace{1.5cm} &       & &  \\
\hline
\vspace{0cm}Cyclobutane\vspace{1.5cm} &    &    & \\
\hline
\vspace{0cm}Cyclopentane\vspace{1.5cm} &  &      & \\
\hline
\vspace{0cm}Cyclohexane\vspace{1.5cm} &  &      & \\
\hline
\end{tabular}\end{center}



\vspace{0.2cm}{\large \bfseries Short alkanes with substituents}
Use the molecular models set for this experiment. Each sphere represents an element. Carbon is black, hydrogen white, oxygen red and nitrogen blue. Build up the following molecules and complete the table. Show your professor all molecular models before proceeding to next part.
\begin{center}\begin{tabular}{ |p{3cm}|p{5cm}| m{5cm}| m{5cm}| }
\hline
    & Expanded Formula &    Molecular Formula   \\
\hline
\vspace{0cm}Chloromethane\vspace{1.5cm} &        &  \\
\hline
\vspace{0cm}Dichloromethane\vspace{1.5cm} &       & \\
\hline
\vspace{0cm}
\makecell{BromoChloro\\-Fluoromethane}
\vspace{1.0cm} &        & \\
\hline
\vspace{0cm}Chloroethane\vspace{1.5cm} &        & \\
\hline
\end{tabular}\end{center}

\end{fullwidth}


\newpage
\begin{fullwidth}
\vspace{0.2cm}{\large \bfseries Long alkanes with substituents}
Use the molecular models set for this experiment. Each sphere represents an element. Carbon is black, hydrogen white, oxygen red and nitrogen blue. Build up the following molecules and complete the table. Show your professor all molecular models before proceeding to next part.
\begin{center}\begin{tabular}{ |p{4cm}| m{6cm}| m{5cm}| }
\hline
     Name &   Condensed Formula & Skeletal Formula   \\
\hline
 &  \vspace{1cm} \chemfig{CH_3-CH(-[:90]CH_3)-CH_2-CH_2-CH_3}\vspace{0.5cm}     &   \\

\hline
 &  \vspace{1cm}\chemfig{CH_3-CH(-[:90]CH_3)-CH(-[:90]CH_3)-CH_2-CH_3}\vspace{0.5cm}     &   \\

\hline

\end{tabular}\end{center}



\vspace{0.2cm}{\large \bfseries More alkanes with substituents}
There is no need to use the molecular models at this point. Now, name the following molecules:
\begin{center}\begin{tabular}{ |p{7cm}|p{7cm}|  }
\hline
    Formula & name    \\
\hline
\vspace{0cm}\begin{center}\chemfig{CH_3-CH(-[:90]Br)-CH(-[:90]Cl)-CH_2-CH_2-CH_3}
\end{center} \vspace{1.5cm} &        \\
\hline
\vspace{0cm}\begin{center}\chemfig{CH_3-CH(-[:90]CH_2-CH_3)-CH_2-CH_2-CH_3}\end{center} \vspace{1.5cm} &     \\
\hline
\vspace{0cm}\begin{center}\chemfig{CH_3-C(-[:90]CH_3)(-[:-90]CH_2-CH_3)-CH_2-CH_2-CH_3}\end{center} \vspace{1.5cm} &  \\
\hline
\end{tabular}\end{center}
\end{fullwidth}



\newpage
\begin{fullwidth}

\vspace{0.2cm}{\large \bfseries Functional Groups}
Identity the following functional groups:
\begin{center}\begin{tabular}{ |p{7cm}|p{7cm}|  }
\hline
    Formula & name    \\
\hline
\vspace{0cm}
\begin{center}
\chemfig{*6((-H_3C)---N(-C(=[::+60]O)-[::-60])---)}
\end{center}
 \vspace{1.5cm} &        \\
\hline
\vspace{0cm}
\begin{center}
\chemfig{**6(---(-NH _2)---)}
\end{center}\vspace{1.5cm} &     \\
\hline
\vspace{0cm}
\begin{center}
\chemfig{-[::+45]=[::-45]-[::+45]OH}
\end{center}\vspace{1.5cm} &  \\
\hline
\vspace{0cm}
\begin{center}
\chemfig{[7]H-C(=[6]O)-[1]CH_2-CH_2-[1]CH_3}
\end{center}\vspace{1.5cm} &  \\
\hline
\vspace{0cm}
\begin{center}
\chemfig{*6(-=-(-O-[::-60](=[::-60]O)-[::+60])=(-(=[::+60])-[::-60]OH)-=)}
\end{center}\vspace{1.5cm} &  \\
\hline


\end{tabular}\end{center}
\end{fullwidth}



\end{document}