\documentclass[main.tex]{subfiles}
\begin{document}\newpage
\setdoublesep{0.35700 em}  % 'Bond Spacing'
\setatomsep{1.78500 em}    % 'Fixed Length'
\setbondoffset{0.18265 em} % 'Margin Width'
\newcommand{\bondwidth}{0.06642 em} % 'Line Width'
\setbondstyle{line width = \bondwidth}
\newgeometry{left=0.8in,right=0.8in, top=2.5cm,bottom=2cm}
\fancyhfoffset[E,O]{0pt}
\setlength{\columnsep}{30pt}
\begin{conclusion}
\end{conclusion}
\setstretch{0.3}
\begin{multicols*}{2}






{\raggedright\textsc{\textbf{Alkanes}}\par}
\begin{enumerate}

\item Classify the following chemicals in two different categories. Give a rationale for your classification. For the group with more chemicals, further classify those chemicals in two categories.
 \begin{inparaenum}[(a)]
\item \ce{KCl} % (ionic, inorganic)
\item \ce{C2H2} %(covalent, organic)
\item \ce{C4H10} %(covalent, organic)
\item \ce{FeO_{(s)}} %(ionic, inorganic)
\item \ce{C6H12} %(covalent, organic)
\item \ce{PH3} %(covalent, inorganic)
\item \ce{H2O} %(covalent, inorganic)
\end{inparaenum}
 \begin{flushright}Ans:
 \begin{inparaenum}[(a)]
\item \ce{KCl}   (ionic, inorganic)
\item \ce{C2H2}  (covalent, organic)
\item \ce{C4H10}  (covalent, organic)
\item \ce{FeO_{(s)}}  (ionic, inorganic)
\item \ce{C6H12}  (covalent, organic)
\item \ce{PH3}  (covalent, inorganic)
\item \ce{H2O}  (covalent, inorganic)
\end{inparaenum}
  \end{flushright}

\item Working in groups, select an everyday-life object (e.g  a spoon) and guess whether the materials that made this object are mostly organic or inorganic (e.g mostly inorganic). Without revealing your answer, present the material you selected to another team member and ask him or her to give you his or her point or view regarding whether the materials that made this object are mostly organic or inorganic. 


\item The molecular formula of ethane is
 \vspace{0.3cm}\begin{enumerate}[label=(\alph*)]
  \begin{multicols*}{2}
\item \ce{C2H6} 
\item \ce{CH4} 
\item \ce{C4H10} 
\item \ce{C6H14} 
\item \ce{C10H22} 
\end{multicols*}\begin{flushright}\small Ans: (a) \end{flushright}
\end{enumerate}

\item The molecular formula of butane is
\vspace{0.3cm}\begin{enumerate}[label=(\alph*)]
 \begin{multicols*}{2}
\item \ce{C2H6} 
\item \ce{CH4} 
\item \ce{C4H10} 
\item \ce{C6H14} 
\item \ce{C10H22}
\end{multicols*}\begin{flushright}\small Ans: (c) \end{flushright}
\end{enumerate}
\item The molecular formula of methane is
\vspace{0.3cm}\begin{enumerate}[label=(\alph*)]
 \begin{multicols*}{2}
\item \ce{C2H6} 
\item \ce{CH4} 
\item \ce{C4H10} 
\item \ce{C6H14} 
\item \ce{C10H22}
\end{multicols*}\begin{flushright}\small Ans: (b) \end{flushright}
\end{enumerate}
\item The molecular formula of decane is
\vspace{0.3cm}\begin{enumerate}[label=(\alph*)]
 \begin{multicols*}{2}
\item \ce{C2H6} 
\item \ce{CH4} 
\item \ce{C4H10} 
\item \ce{C6H14} 
\item \ce{C10H22}
\end{multicols*}\begin{flushright}\small Ans: (e) \end{flushright}
\end{enumerate}

\item The name of the following alkane is: \ce{C3H8} 
\vspace{0.3cm}\begin{enumerate}[label=(\alph*)]
 \begin{multicols*}{2}
\item Ethane
\item Octane 
\item Hexane
\item Pentane
\item Propane
\end{multicols*}\begin{flushright}\small Ans: (e) \end{flushright}
\end{enumerate}

\item The name of the following alkane is: \ce{C8H18} 
\vspace{0.3cm}\begin{enumerate}[label=(\alph*)]
 \begin{multicols*}{2}
\item Ethane
\item Octane 
\item Hexane
\item Pentane
\item Propane
\end{multicols*}\begin{flushright}\small Ans: (b) \end{flushright}
\end{enumerate}

\item The name of the following alkane is: \ce{C5H12} 
\vspace{0.3cm}\begin{enumerate}[label=(\alph*)]
 \begin{multicols*}{2}
\item Ethane
\item Octane 
\item Hexane
\item Pentane
\item Propane
\end{multicols*}\begin{flushright}\small Ans: (d) \end{flushright}
\end{enumerate}


\item Write down the condensed formula for: Hexane.
\begin{flushright}\small Ans: \small\chemfig{CH_3-CH_2-CH_2-CH_2-CH_2-CH_3} \end{flushright}

\item Write down the condensed formula for: Propane.
\begin{flushright}\small Ans: \small\chemfig{CH_3-CH_2-CH_3} \end{flushright}

\item Write down the expanded formula for: Pentane.
\begin{flushright}\small Ans: \small \setpolymerdelim()
\chemfig{H-C(-[2]H)(-[6]H) -[@{A,0.5}:0,2]C(-[2]H)(-[6]H) -[@{B,0.5}:0,2]C(-[2]H)(-[6]H)-H }\makebraces(15pt,15pt){$\scriptstyle\!\!3$}{A}{B}
 \end{flushright}

\item Write down the expanded formula for: Decane.
\begin{flushright}\small Ans: \small \setpolymerdelim()
\chemfig{H-C(-[2]H)(-[6]H) -[@{A,0.5}:0,2]C(-[2]H)(-[6]H) -[@{B,0.5}:0,2]C(-[2]H)(-[6]H)-H }\makebraces(15pt,15pt){$\scriptstyle\!\!8$}{A}{B}
 \end{flushright}

\item Write down the molecular formula for:
\begin{center}
\setpolymerdelim()
\chemfig{H-C(-[2]H)(-[6]H) -[@{A,0.5}:0,2]C(-[2]H)(-[6]H) -[@{B,0.5}:0,2]C(-[2]H)(-[6]H)-H }\makebraces(15pt,15pt){$\scriptstyle\!\!4$}{A}{B}
\end{center}
\begin{flushright}\small Ans: \ce{C6H14},hexane \end{flushright}


\item Write down the molecular formula for:
\begin{center}
\setpolymerdelim()
\chemfig{-[:45]-[@{A,0.5}:-45,1]-[:45]-[@{B,0.5}:-45,1] -[:45]}\makebraces(12pt,12pt){$\scriptstyle\!\!2$}{A}{B}
\end{center}
\begin{flushright}\small Ans: \ce{C8H18}, octane \end{flushright}

\item Write down the molecular formula for:
\begin{center}
\setpolymerdelim()
\chemfig{-[:45]-[@{A,0.5}:-45,1]-[:45]-[@{B,0.5}:-45,1] -[:45]}\makebraces(12pt,12pt){$\scriptstyle\!\!3$}{A}{B}
\end{center}
\begin{flushright}\small Ans: \ce{C10H22}, decane \end{flushright}


\item Write down the expanded formula for: hexane.
\begin{flushright}\small Ans: \small \setpolymerdelim()
\chemfig{-[:45]-[@{A,0.5}:-45,1]-[:45]-[@{B,0.5}:-45,1] -[:45]}\makebraces(9pt,9pt){$\scriptstyle\!\!1$}{A}{B}
 \end{flushright}



{\raggedright\textsc{\textbf{Cycloalkanes}}\par}

\item The molecular formula of cyclopropane is
\vspace{0.3cm}\begin{enumerate}[label=(\alph*)]
 \begin{multicols*}{2}
\item \ce{C2H4} 
\item \ce{C3H6} 
\item \ce{C4H8} 
\item \ce{C6H12} 
\item \ce{C10H20} 
\end{multicols*}\begin{flushright}\small Ans: (c) \end{flushright}
\end{enumerate}

\item The molecular formula of cyclohexane is
\vspace{0.3cm}\begin{enumerate}[label=(\alph*)]
 \begin{multicols*}{2}
\item \ce{C2H4} 
\item \ce{C3H6} 
\item \ce{C4H8} 
\item \ce{C6H12} 
\item \ce{C10H20} 
\end{multicols*}\begin{flushright}\small Ans: (d) \end{flushright}
\end{enumerate}




\item Name the following cycloalkane:
\begin{center}\chemfig{*9(---------)}\end{center}
\begin{flushright}\small Ans: Cyclononane\end{flushright}

\item Name the following cycloalkane:
\begin{center}\chemfig{*7(---------)}\end{center}
\begin{flushright}\small Ans: Cycloheptane\end{flushright}

\item Name the following cycloalkane:
\begin{center}\chemfig{H_2C*6(-CH_2-CH_2-CH_2-CH_2-H_2C-[,,2])}
\end{center}
\begin{flushright}\small Ans: Cyclohexane\end{flushright}

\item Name the following cycloalkane:
\begin{center}\chemfig{H_2C*4(-CH_2-CH_2-H_2C-[,,2])}
\end{center}
\begin{flushright}\small Ans: Cyclopropane\end{flushright}

{\raggedright\textsc{\textbf{Alkanes with substituents}}\par}

\item Give the name for the following compound:
\begin{center}
\chemfig{CH_3-CH(-[:90]CH_3)-CH_2-CH_2-CH_3}
\end{center}
\begin{flushright}\small Ans: 2-methylpentane\end{flushright}

\item Give the name for the following compound:
\begin{center}
\chemfig{CH_3-C(-[:90]CH_3)(-[:-90]CH_3)-CH_2-CH_2-CH_2-CH_3}
\end{center}
\begin{flushright}\small Ans: 2,2-dimethylhexane\end{flushright}

\item Give the name for the following compound:
\begin{center}
\chemfig{CH_3-CH_2-C(-[:90]CH_3)(-[:-90]CH_3)-CH_2-CH_2-CH_3}
\end{center}
\begin{flushright}\small Ans: 3,3-dimethylhexane\end{flushright}

\item Give the name for the following compound:
\begin{center}
\chemfig{CH_3-CH(-[:90]CH_3)-CH(-[:90]CH_3)-CH_2-CH_2-CH_3}
\end{center}
\begin{flushright}\small Ans: 2,3-dimethylhexane\end{flushright}


\item Give the name for the following compound:
\begin{center}
\chemfig{CH_3-CH(-[:90]C_2H_6)-CH_2-CH_2-CH_3}
\end{center}
\begin{flushright}\small Ans: 2-ethylpentane\end{flushright}

\item Give the name for the following compound:
\begin{center}
\chemfig{CH_3-CH(-[:90]Br)-CH(-[:90]NO_2)-CH_2-CH_2-CH_3}
\end{center}
\begin{flushright}\small Ans: 2-bromo-3-nitrohexane\end{flushright}

\item Give the name for the following compound:
\begin{center}
\chemfig{CH_3-CH_2-CH(-[:90]CH_2-CH_2-CH_3)-CH_2-CH_3}
\end{center}
\begin{flushright}\small Ans: 3-ethylhexane\end{flushright}

\item Give the name for the following compound:
\begin{center}
\chemfig{CH_3-CH(-[:-90]CH_2-CH_3)-CH_2-CH_3}
\end{center}
\begin{flushright}\small Ans: 3-methylpentane\end{flushright}

\item Give the name for the following compound:
\begin{center}
\chemfig{CH_3-C(-[:90]CH_3)(-[:-90]CH_2-CH_3)-CH_3}
\end{center}
\begin{flushright}\small Ans: 2,3-dimethylbutane\end{flushright}







\item Identify the functional groups in the following molecule:
\begin{center}
\chemfig{*6((-H_3C)---N(-C(=[::+60]O)-[::-60])---)}
\end{center}
\begin{flushright}\small Ans: amide.\end{flushright}




\item Identify the functional groups in the following molecule:
\begin{center}
\chemfig{**6(---(-NH _2)---)}
\end{center}
\begin{flushright}\small Ans: amine and aromatic.\end{flushright}

\item Identify the functional groups in the following molecule:
\begin{center}
\chemfig{**6(--(-OH)----)}
\end{center}
\begin{flushright}\small Ans: alcohol and aromatic.\end{flushright}

\item Identify the functional groups in the following molecule:
\begin{center}
\chemfig{*6(--(-~CH)----)}
\end{center}
\begin{flushright}\small Ans: alkyne.\end{flushright}

\item Identify the functional groups in the following molecule:
\begin{center}
\chemfig{*6(=-(-CH_3)----)}
\end{center}
\begin{flushright}\small Ans: alkene.\end{flushright}

\item Identify the functional groups in the following molecule:
\begin{center}
\chemfig{-[::+45]=[::-45]-[::+45]OH}
\end{center}
\begin{flushright}\small Ans: alkene and alcohol.\end{flushright}


\item Identify the functional groups in the following molecule:
\begin{center}
\chemfig{[7]H_3C-C(=[6]O)-[1]CH_2-CH_2-[1]CH_3}
\end{center}
\begin{flushright}\small Ans: ketone.\end{flushright}

\item Identify the functional groups in the following molecule:
\begin{center}
\chemfig{[7]H-C(=[6]O)-[1]CH_2-CH_2-[1]CH_3}
\end{center}
\begin{flushright}\small Ans: aldehyde.\end{flushright}

\item Identify the functional groups in the following molecule:
\begin{center}
\chemfig{--[::-45]O-}
\end{center}
\begin{flushright}\small Ans: ether.\end{flushright}


\item Identify the functional groups in the following molecule:
\begin{center}
\chemfig{*6(-=-(-O-[::-60](=[::-60]O)-[::+60])=(-(=[::+60])-[::-60]OH)-=)}\end{center}
\begin{flushright}\small Ans: alchol, atomatic, alkene, ester.\end{flushright}


\item Identify the functional groups in the following molecule:
\begin{center}\chemfig{
  *6(-(--[::-45]O-)=-
    (-=^[::-60]N-*6(=(-COOH)-=(-SH)-=(-NH_2)-))
  =(-OH)-(-(=[::-60]O)-[::60])=)
}\end{center}
\begin{flushright}\small Ans: (top left) ketone; (bottom) ether; (top left) alcohol; (top right) amino and thiol; (bottom left) acid; there are also two aromatic rings\end{flushright}

\item Identify the functional groups in the following molecule:
\begin{center}
\chemfig{N*6((-H_3C)---N(-C(=[::+60]O)-[::-60]*6(-=-(--[::-60]-[::+60])
=(-*6(--*5(-(--[::-60]-[::+60]CH_3)=-(-)-=)--(=O)-N(-H)-))-=))---)}
\end{center}
\begin{flushright}\small Ans: there are an amine, two amides, an aromatic ring and two alkanes.\end{flushright}





\restoregeometry
\end{enumerate}
\end{multicols*}
\pagecolor{green!10}\afterpage{\nopagecolor}\newpage
\end{document}
