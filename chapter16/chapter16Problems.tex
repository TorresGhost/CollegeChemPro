\documentclass[main.tex]{subfiles}
\begin{document}\newpage
\setdoublesep{0.35700 em}  % 'Bond Spacing'
\setatomsep{1.78500 em}    % 'Fixed Length'
\setbondoffset{0.18265 em} % 'Margin Width'
\newcommand{\bondwidth}{0.06642 em} % 'Line Width'
\setbondstyle{line width = \bondwidth}
\newgeometry{left=0.8in,right=0.8in, top=2.5cm,bottom=2cm}
\fancyhfoffset[E,O]{0pt}
\setlength{\columnsep}{30pt}
\begin{conclusion}
\end{conclusion}
%\setstretch{0.3}
\begin{multicols*}{2}\setcounter{numA}{1}



{\raggedright\textsc{\textbf{Solutions and colloids }}\par}
%%%%%PROBLEM
\begin{question}[ID=\the\value{numA}]\SetQuestionProperties{section-title=\nameref{sec:units}}
For the following colloids, indicate the nature (liquid, solid or gad) of the dispersed and dispersing medium: 
\begin{inparaenum}[(a)]	
\item  soda water % (dispersed: gas ; dispersing: liquid)
\item  cake % (dispersed: gas ; dispersing: solid)
\item  midst % (dispersed: liquid ; dispersing: gas)
\item  smoke % (dispersed: solid ; dispersing: gas)
\item  froth % (dispersed: gas ; dispersing: liquid)
 \end{inparaenum}
\end{question}
\begin{solution}
\begin{inparaenum}[(a)]
\item  soda water   (dispersed: gas ; dispersing: liquid)
\item  cake   (dispersed: gas ; dispersing: solid)
\item  midst   (dispersed: liquid ; dispersing: gas)
\item  smoke   (dispersed: solid ; dispersing: gas)
\item  froth   (dispersed: gas ; dispersing: liquid)
 \end{inparaenum}
\hspace{0.1cm}\end{solution}\stepcounter{numA}%%%%%%%%%%%%





%%%%%PROBLEM
\begin{question}[ID=\the\value{numA}]\SetQuestionProperties{section-title=\nameref{sec:units}}
For the following colloids, indicate the nature (liquid, solid or gad) of the dispersed and dispersing medium: 
\begin{inparaenum}[(a)]	
\item  pumice stone lava   (dispersed: gas ; dispersing: solid)
\item  cheese   (dispersed: liquid ; dispersing: solid)
\item  paint   (dispersed: solid ; dispersing: liquid)
\item  clouds   (dispersed: liquid ; dispersing: gas)
 \end{inparaenum}
\end{question}
\begin{solution}
\begin{inparaenum}[(a)]
\item  pumice stone lava   (dispersed: gas ; dispersing: solid)
\item  cheese   (dispersed: liquid ; dispersing: solid)
\item  paint   (dispersed: solid ; dispersing: liquid)
\item  clouds   (dispersed: liquid ; dispersing: gas)
 \end{inparaenum}
\hspace{0.1cm}\end{solution}\stepcounter{numA}%%%%%%%%%%%%

{\raggedright\textsc{\textbf{Units of concentration }}\par}




%%%%%%PROBLEM
\begin{question}[ID=\the\value{numA}]\SetQuestionProperties{section-title=\nameref{sec:units}}
Calculate: 
\begin{inparaenum}[(a)]
\item  the percent by mass of a solution made of 5g of NaBr in 70g of solution % (7.14\%)
\item  the percent by mass of a solution made of 5g of NaBr with 20g of water%(20\%)
\item  the percent by mass of a solution made of 5g of NaBr with 10mL of water%(33.3\%)
 \end{inparaenum}
\end{question}
\begin{solution}
\begin{inparaenum}[(a)]
\item   7.14\% 
\item   20\% 
\item   33.3\% 
 \end{inparaenum}\hspace{0.1cm}\end{solution}\stepcounter{numA}%%%%%%%%%%%%

%%%%%%PROBLEM
\begin{question}[ID=\the\value{numA}]\SetQuestionProperties{section-title=\nameref{sec:units}}
Calculate the molality for: 
\begin{inparaenum}[(a)]
\item  a solution made of 5moles of NaBr(MW=103g/mol) in 200g of water % (25m)
\item  a solution made of 20g of NaBr(MW=103g/mol) in 200g of water % (0.97m)
\item  a solution made of 10g of NaBr(MW=103g/mol) in 20mL of water % (4.85m)
 \end{inparaenum}
\end{question}
\begin{solution}
\begin{inparaenum}[(a)]
\item    25m 
\item   0.97m 
\item   4.85m 
 \end{inparaenum}\hspace{0.1cm}\end{solution}\stepcounter{numA}%%%%%%%%%%%%



%%%%%%PROBLEM
\begin{question}[ID=\the\value{numA}]\SetQuestionProperties{section-title=\nameref{sec:units}}
Calculate the density of a solution: 
\begin{inparaenum}[(a)]
\item  containing 3g of solute and 100g of water in 101mL % (1.02g/mL)
\item  containing 1g of solute and 100mL of water in 103mL % (0.98g/mL)
 \end{inparaenum}
\end{question}
\begin{solution}
\begin{inparaenum}[(a)]
\item    1.02g/mL 
\item   0.98g/mL 
 \end{inparaenum}\hspace{0.1cm}\end{solution}\stepcounter{numA}%%%%%%%%%%%%


%%%%%%PROBLEM
\begin{question}[ID=\the\value{numA}]\SetQuestionProperties{section-title=\nameref{sec:units}}
Calculate the mole fraction of solute of a solution: 
\begin{inparaenum}[(a)]
\item  containing 3g of solute (MW=16g/mL) and 100g of water % (0.03)
\item  containing 2 moles of solute and 30 moles of solvent % (0.06)
\item  containing 4 moles of solute and 45 moles of solution % (0.09)
 \end{inparaenum}
\end{question}
\begin{solution}
\begin{inparaenum}[(a)]
\item   0.03
\item   0.06
\item   0.09
 \end{inparaenum}\hspace{0.1cm}\end{solution}\stepcounter{numA}%%%%%%%%%%%%

%%%%%%PROBLEM
\begin{question}[ID=\the\value{numA}]\SetQuestionProperties{section-title=\nameref{sec:units}}
Calculate: 
\begin{inparaenum}[(a)]
\item  the molarity of a 0.2m solution of density 1.2g/mL made of a NaCl(MW=58g/mol) % (0.26M)
\item  the molality of a 0.2M solution of density 1.2g/mL made of a NaCl(MW=58g/mol) % (0.27M)
\item  the percent by mass of a 0.1M solution of density 1.2g/mL made of a NaCl(MW=58g/mol) % (0.45/%)
 \end{inparaenum}
\end{question}
\begin{solution}
\begin{inparaenum}[(a)]
\item  0.26M
\item   0.27M
\item   0.45\%
 \end{inparaenum}\hspace{0.1cm}\end{solution}\stepcounter{numA}%%%%%%%%%%%%



%%%%%%PROBLEM
\begin{question}[ID=\the\value{numA}]\SetQuestionProperties{section-title=\nameref{sec:units}}
Calculate: 
\begin{inparaenum}[(a)]
\item  the molarity of a 20\% solution of density 1.1g/mL made of a \ce{Na2SO4}(MW=142g/mol) % (1.55M)
\item  the mole fraction of a 20\% aqueous solution of density 1.1g/mL made of a \ce{Na2SO4}(MW=142g/mol) % (0.03)
\item  the percent by mass of an aqueous solution of density 1.1g/mL and 0.4mole fraction, made of a \ce{Na2SO4}(MW=142g/mol) % (84\%)
\end{inparaenum}
\end{question}
\begin{solution}
\begin{inparaenum}[(a)]
\item   1.55M 
\item   0.03 
\item   84\% 
 \end{inparaenum}\hspace{0.1cm}\end{solution}\stepcounter{numA}%%%%%%%%%%%%




%%%%%%PROBLEM
\begin{question}[ID=\the\value{numA}]\SetQuestionProperties{section-title=\nameref{sec:units}}
We prepare an \ce{C6H12O6} (MW=180g/mL) solution by mixing 5g of solute in 100mL of water giving 102mL of solution. Calculate:
\begin{inparaenum}[(a)]
\item  the molarity  % (0.27M)
\item  the molality  % (0.28m)
\item  the percent by mass  % (4.76\%)
\item  the mole fraction % (0.005)
\end{inparaenum}
\end{question}
\begin{solution}
\begin{inparaenum}[(a)]
\item   0.27M 
\item   0.28m 
\item   4.76\% 
\item   0.005  
 \end{inparaenum}\hspace{0.1cm}\end{solution}\stepcounter{numA}%%%%%%%%%%%%


%%%%%%PROBLEM
\begin{question}[ID=\the\value{numA}]\SetQuestionProperties{section-title=\nameref{sec:units}}
Calculate the molar weight of the solute of a solution with 0.9g/mL density, 2M and 1.9m.\end{question}
\begin{solution}
155g/mol
\hspace{0.1cm}\end{solution}\stepcounter{numA}%%%%%%%%%%%%

%%%%%%PROBLEM
\begin{question}[ID=\the\value{numA}]\SetQuestionProperties{section-title=\nameref{sec:units}}
Calculate the molar weight of the solute of a solution with 0.9g/mL density, 0.5 mole fraction and 90\% of solute by mass.\end{question}
\begin{solution}
162g/mol
\hspace{0.1cm}\end{solution}\stepcounter{numA}%%%%%%%%%%%%


{\raggedright\textsc{\textbf{Solutions of electrolytes and effective solute particles }}\par}


%%%%%%PROBLEM
\begin{question}[ID=\the\value{numA}]\SetQuestionProperties{section-title=\nameref{sec:units}}
Break down the following electrolytes in ions, if possible:
\begin{inparaenum}[(a)]
\item  \ce{CaI2}  % ( \ce{Ca^{2+} +2\ce{I^{-} })
\item   \ce{KNO3}  % ( \ce{K^{+} + \ce{NO3^{-} })
\item   \ce{CaSO4}  % ( \ce{Ca^{2+} + \ce{SO4^{2-} })
\item   \ce{FeSO4}  % ( \ce{Fe^{2+} + \ce{SO4^{2-} })
\end{inparaenum}
\end{question}
\begin{solution}
\begin{inparaenum}[(a)]
\item  \ce{CaI2}    ( \ce{Ca^{2+}} +2\ce{I^{-} })
\item   \ce{KNO3}    ( \ce{K^{+}} + \ce{NO3^{-} })
\item   \ce{CaSO4}    ( \ce{Ca^{2+}} + \ce{SO4^{2-} })
\item   \ce{FeSO4}    ( \ce{Fe^{2+}} + \ce{SO4^{2-} })
 \end{inparaenum}\hspace{0.1cm}\end{solution}\stepcounter{numA}%%%%%%%%%%%%


%%%%%%PROBLEM
\begin{question}[ID=\the\value{numA}]\SetQuestionProperties{section-title=\nameref{sec:units}}
Break down the following electrolytes in ions, if possible:
\begin{inparaenum}[(a)]
\item   \ce{SnCl4}  % ( \ce{Sn^{4+} + 4\ce{Cl^{-} })
\item   \ce{CuCl2}  % ( \ce{Cu^{2+} + 2\ce{Cl^{-} })
\item   \ce{Ba(OH)2}  % ( \ce{Ba^{2+} + 2\ce{OH^{-} })
\item   \ce{CuSO3}  % ( \ce{Cu^{2+} + \ce{SO3^{2-} })
\item   \ce{MgSO4}  % ( \ce{Mg^{2+} + \ce{SO4^{2-} })
\end{inparaenum}
\end{question}
\begin{solution}
\begin{inparaenum}[(a)]
\item   \ce{SnCl4}    ( \ce{Sn^{4+} }+ 4\ce{Cl^{-} })
\item   \ce{CuCl2}    ( \ce{Cu^{2+}} + 2\ce{Cl^{-} })
\item   \ce{Ba(OH)2}    ( \ce{Ba^{2+}} + 2\ce{OH^{-} })
\item   \ce{CuSO3}    ( \ce{Cu^{2+}} + \ce{SO3^{2-} })
\item   \ce{MgSO4}    ( \ce{Mg^{2+}} + \ce{SO4^{2-} })
 \end{inparaenum}\hspace{0.1cm}\end{solution}\stepcounter{numA}%%%%%%%%%%%%



%%%%%%PROBLEM
\begin{question}[ID=\the\value{numA}]\SetQuestionProperties{section-title=\nameref{sec:units}}
Calculate the $i$ factor for the following chemicals:
\begin{inparaenum}[(a)]
\item   \ce{NaNO3}  % ($i$=2)
\item   \ce{NaCl}  % ($i$=2)
\item   \ce{CaI2}  % ($i$=3)
\item   \ce{MgCl2}  % ($i$=3)
\end{inparaenum}
\end{question}
\begin{solution}
\begin{inparaenum}[(a)]
\item   \ce{NaNO3}  % ($i$=2)
\item   \ce{NaCl}  % ($i$=2)
\item   \ce{CaI2}  % ($i$=3)
\item   \ce{MgCl2}  % ($i$=3)
 \end{inparaenum}\hspace{0.1cm}\end{solution}\stepcounter{numA}%%%%%%%%%%%%




%%%%%%PROBLEM
\begin{question}[ID=\the\value{numA}]\SetQuestionProperties{section-title=\nameref{sec:units}}
Calculate the $i$ factor for the following chemicals:
\begin{inparaenum}[(a)]
\item   \ce{Mg(NO3)2}  % ($i$=3)
\item   \ce{CuSO4}  % ($i$=2)
\item   \ce{FeCl3}  % ($i$=4)
\end{inparaenum}
\end{question}
\begin{solution}
\begin{inparaenum}[(a)]
\item   \ce{Mg(NO3)2}  % ($i$=3)
\item   \ce{CuSO4}  % ($i$=2)
\item   \ce{FeCl3}  % ($i$=4)
 \end{inparaenum}\hspace{0.1cm}\end{solution}\stepcounter{numA}%%%%%%%%%%%%



%%%%%%PROBLEM
\begin{question}[ID=\the\value{numA}]\SetQuestionProperties{section-title=\nameref{sec:units}}
We dissolve 3 moles of solute in 1L of solution. Given that Van't Hoff factor of the solute is 3, calculate the nominal solute concentration and the effective concentration of particles on solution.  
\end{question}
\begin{solution}
$c^{nominal}$=9M; $c^{effective}$=9M
\hspace{0.1cm}\end{solution}\stepcounter{numA}%%%%%%%%%%%%


%%%%%%PROBLEM
\begin{question}[ID=\the\value{numA}]\SetQuestionProperties{section-title=\nameref{sec:units}}
We dissolve 0.5 moles of solute in 1L of solution reaching a effective concentration of solute particles of 0.9M. Calculate the Van't Hoff factor.
\end{question}
\begin{solution}
1.8
\hspace{0.1cm}\end{solution}\stepcounter{numA}%%%%%%%%%%%%


%%%%%%PROBLEM
\begin{question}[ID=\the\value{numA}]\SetQuestionProperties{section-title=\nameref{sec:units}}
Can the percent dissociation of a electrolyte be negative? Elaborate.
\end{question}
\begin{solution}
No as the maximum value should be one.
\hspace{0.1cm}\end{solution}\stepcounter{numA}%%%%%%%%%%%%

{\raggedright\textsc{\textbf{Colligative properties }}\par}

%%%%%%PROBLEM
\begin{question}[ID=\the\value{numA}]\SetQuestionProperties{section-title=\nameref{sec:units}}
Calculate the boiling point of a 3m \ce{C6H12O6} aqueous solution. $T^{solvent}_b$=100$^{\circ}$C and $k_b$=0.512$^{\circ}$C/m.
\end{question}
\begin{solution}
101.5$^{\circ}$C
\hspace{0.1cm}\end{solution}\stepcounter{numA}%%%%%%%%%%%%

%%%%%%PROBLEM
\begin{question}[ID=\the\value{numA}]\SetQuestionProperties{section-title=\nameref{sec:units}}
Calculate the boiling point increase of a 8m \ce{C6H12O6} aqueous solution. $T^{solvent}_b$=100$^{\circ}$C and $k_b$=0.512$^{\circ}$C/m.
\end{question}
\begin{solution}
4.01$^{\circ}$C
\hspace{0.1cm}\end{solution}\stepcounter{numA}%%%%%%%%%%%%

%%%%%%PROBLEM
\begin{question}[ID=\the\value{numA}]\SetQuestionProperties{section-title=\nameref{sec:units}}
Calculate the boiling point of a 3m \ce{CaCl2} aqueous solution. $T^{solvent}_b$=100$^{\circ}$C and $k_b$=0.512$^{\circ}$C/m.
\end{question}
\begin{solution}
104.07$^{\circ}$C
\hspace{0.1cm}\end{solution}\stepcounter{numA}%%%%%%%%%%%%

%%%%%%PROBLEM
\begin{question}[ID=\the\value{numA}]\SetQuestionProperties{section-title=\nameref{sec:units}}
Calculate the boiling point increase of a 8m \ce{KCl} aqueous solution. $T^{solvent}_b$=100$^{\circ}$C and $k_b$=0.512$^{\circ}$C/m.
\end{question}
\begin{solution}
8.19$^{\circ}$C
\hspace{0.1cm}\end{solution}\stepcounter{numA}%%%%%%%%%%%%


%%%%%%PROBLEM
\begin{question}[ID=\the\value{numA}]\SetQuestionProperties{section-title=\nameref{sec:units}}
Calculate the freezing point of a 2m \ce{I2} solution on benzene. $T^{solvent}_b$=5.5$^{\circ}$C and $k_b$=4.9$^{\circ}$C/m.
\end{question}
\begin{solution}
-4.3$^{\circ}$C
\hspace{0.1cm}\end{solution}\stepcounter{numA}%%%%%%%%%%%%


%%%%%%PROBLEM
\begin{question}[ID=\the\value{numA}]\SetQuestionProperties{section-title=\nameref{sec:units}}
Calculate the freezing point depression of a 2m \ce{I2} solution on benzene. $T^{solvent}_f$=5.5$^{\circ}$C and $k_b$=4.9$^{\circ}$C/m.
\end{question}
\begin{solution}
-4.3$^{\circ}$C
\hspace{0.1cm}\end{solution}\stepcounter{numA}%%%%%%%%%%%%


%%%%%%PROBLEM
\begin{question}[ID=\the\value{numA}]\SetQuestionProperties{section-title=\nameref{sec:units}}
The vapor pressure of cyclohexane is 100hPa at 20$^{\circ}$C. Calculate the vapor pressure of the solution resulting of mixing 3moles of cyclohexane and 4 moles of \ce{I2}.
\end{question}
\begin{solution}
42.8hPa
\hspace{0.1cm}\end{solution}\stepcounter{numA}%%%%%%%%%%%%

%%%%%%PROBLEM
\begin{question}[ID=\the\value{numA}]\SetQuestionProperties{section-title=\nameref{sec:units}}
The vapor pressure of cyclohexane is 100hPa at 20$^{\circ}$C. Calculate the vapor pressure lowering of the solution resulting of mixing 3moles of cyclohexane and 4 moles of \ce{I2}.
\end{question}
\begin{solution}
-42.85hPa
\hspace{0.1cm}\end{solution}\stepcounter{numA}%%%%%%%%%%%%



%%%%%%PROBLEM
\begin{question}[ID=\the\value{numA}]\SetQuestionProperties{section-title=\nameref{sec:units}}
The vapor pressure of cyclohexane is 100hPa at 20$^{\circ}$C. Calculate the vapor pressure lowering of the solution resulting of mixing 3moles of cyclohexane and 3 moles of \ce{I2}.
\end{question}
\begin{solution}
-50hPa
\hspace{0.1cm}\end{solution}\stepcounter{numA}%%%%%%%%%%%%


%%%%%%PROBLEM
\begin{question}[ID=\the\value{numA}]\SetQuestionProperties{section-title=\nameref{sec:units}}
The vapor pressure of cyclohexane is 100hPa at 20$^{\circ}$C. Calculate the vapor pressure of the solution resulting of mixing 6moles of cyclohexane and 1 moles of \ce{I2}.
\end{question}
\begin{solution}
85.71hPa
\hspace{0.1cm}\end{solution}\stepcounter{numA}%%%%%%%%%%%%

%%%%%%PROBLEM
\begin{question}[ID=\the\value{numA}]\SetQuestionProperties{section-title=\nameref{sec:units}}
The vapor pressure of cyclohexane is 100hPa at 20$^{\circ}$C. Calculate the vapor pressure lowering of the solution resulting of mixing 6moles of cyclohexane and 1 moles of \ce{I2}.
\end{question}
\begin{solution}
-14.28hPa
\hspace{0.1cm}\end{solution}\stepcounter{numA}%%%%%%%%%%%%


%%%%%%PROBLEM
\begin{question}[ID=\the\value{numA}]\SetQuestionProperties{section-title=\nameref{sec:units}}
Calculate the osmotic pressure of a 3M \ce{NaCl} solution at 298K.
\end{question}
\begin{solution}
146.62atm
\hspace{0.1cm}\end{solution}\stepcounter{numA}%%%%%%%%%%%%


%%%%%%PROBLEM
\begin{question}[ID=\the\value{numA}]\SetQuestionProperties{section-title=\nameref{sec:units}}
A semipermeable membrane separate two \ce{NaCl} solutions with concentration 0.2M (on the left) and 0.1M (on the right). What side of the membrane will receive an osmotic flow of water?
\end{question}
\begin{solution}
The more concentrated (left side).
\hspace{0.1cm}\end{solution}\stepcounter{numA}%%%%%%%%%%%%

%%%%%%PROBLEM
\begin{question}[ID=\the\value{numA}]\SetQuestionProperties{section-title=\nameref{sec:units}}
A semipermeable membrane separate two solutions, 0.2M \ce{NaCl} (on the left) and 0.1M \ce{CaCl2} (on the right). What side of the membrane will receive an osmotic flow of water?
\end{question}
\begin{solution}
none
\hspace{0.1cm}\end{solution}\stepcounter{numA}%%%%%%%%%%%%



{\raggedright\textsc{\textbf{Factors affecting the solubility of solids and gases }}\par}
%%%%%%PROBLEM
\begin{question}[ID=\the\value{numA}]\SetQuestionProperties{section-title=\nameref{sec:units}}
The solubility of Ar in water at 298K when the partial pressure of Ar is 0.9atm is $1.26\times 10^{-3}$M. Calculate:
\begin{inparaenum}[(a)]
\item  Henry's law constant at that temperature  % ($1.4\times 10^{-3}$ M/atm)
\item  The solubility of Ar at a partial pressure of 0.5atm % ($7.00\times 10^{-4}$M)
\end{inparaenum}
\end{question}
\begin{solution}
\begin{inparaenum}[(a)]
\item   $1.4\times 10^{-3}$ M/atm
\item     $7.00\times 10^{-4}$M
\end{inparaenum}\hspace{0.1cm}\end{solution}\stepcounter{numA}%%%%%%%%%%%%






\end{multicols*}

\newpage
\begin{answersenvironment}
\begin{minipage}[c]{1\textwidth}
\begin{localsize}{10}
{\Large \bf Answers}
\SetupExSheets{
  headings = inline-nr , % numbered and inline
  counter-format = qu) , % numbers 1) 2) ... 
}
%\printsolutions 
 %\printsolutions[byID={1,3,5,7,9,11,13,15,17,19,21,23,25,27,29,31,33,35,37}]
\end{localsize}
\end{minipage}\end{answersenvironment}
\end{document}

