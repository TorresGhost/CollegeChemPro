\documentclass[main.tex]{subfiles}
\newcommand\chapterlabel{physicalsolutions}
\setcounter{figurenewcounter}{0}
\newsavebox\unitcellwater
\newsavebox\trianglenew

\begin{document}
\linenumbers
%\setcounter{chapter}{5}
  
\chapter[Physical properties of solutions]{Physical properties of solutions}
%\label{ch:atoms}


      \begin{marginfigure}
      \begin{tikzpicture} \node (a) at (0,0) {\includegraphics[width=4cm]{chapter16/figure1}} node[rotate=90, font=\tiny] at ([yshift=.5cm,xshift=.1cm]a.south east) {\textsuperscript{\textcopyright} PxFuel} ;
\end{tikzpicture}
\end{marginfigure}


\lettrine[lines=4]{\color{black!45}S}{olutions} are homogeneous mixtures of a solvent and one or more solutes. In your everyday life, you will encounter numerous liquid-based solutions, from recovery drinks to fancy champagne or plain milk. However, solutions are not just liquid-based as you can also find solid solutions such as an alloy, and even gas solutions such as the air. This chapter will cover the physical properties of solutions--boiling, freezing, and melting--and in particular, we will gain insight into the impact of the solute on the physical properties of the solvent. The chapter will also address the meaning of different concentration units and the relationship between them, as well as the properties of colloids--a special type of unstable mixture.
 
\begin{marginfigure}%LEARNING GOALS BOX
\begin{mytcbox}{GOALS}
\begin{enumerate}[label=\protect\circled{\color{white}\arabic*}]
\item Compute different concentration units
\item Interconvert concentration units
\item Predict freezing and boiling point of solutions
\item Predict the vapor pressure of solutions
\item Identify colloids
\end{enumerate}
\end{mytcbox}
\vspace{1cm}
\begin{tcolorbox}[enhanced,colback=red!5!white,colframe=black!50!red,boxrule=1pt,
  arc=0pt,outer arc=0pt,drop heavy lifted shadow]
\faGears\ 
\docenvdef{Discussion:} What does it means the term colligative, like in colligative properties of solutions? List four colligative properties of a solution.  \end{tcolorbox}
\end{marginfigure}%LEARNING GOALS BOX



\section{Solutions and colloids}
A solution is a homogeneous mixture of a solvent and one or more solutes. Homogeneous means that a solution consists of only one visible phase (e.g. wine) in contrast to heterogeneous which means that a mixture would be composed of two or more distinct phases (e.g. a chocolate chip cookie). 

\sloppy 
\begin{description}

\item[\docfilehook{Solutions in terms of phase and solubility}{}] 
One can find gas, liquid, or solid solutions depending on the final phase of the resulting solutions. For example, brass is a solid solution of copper and zinc and the air is a gaseous solution of oxygen, nitrogen, and other components. Solutions can be classified in terms of solubility. The solubility of a given solute in a solvent is the maximum amount of solute one can dissolve in a volume of solvent. Solutions can be saturated when they contain the maximum amount of solute one can fit, unsaturated when they contain less than the maximum amount of solute one can fit, or supersaturated when they contain more than a saturated solution. Supersaturated solutions tend to be unstable and the solute tends to eventually precipitate.
 \begin{center}
\refstepcounter{table}  \label{tab:{\chapterlabel}1}
\fontfamily{ppl}\selectfont
\begin{tabular}{llll}
\rowcolor{black!45}
\toprule
\multicolumn{4}{l}{\hypersetup{colorlinks,linkcolor={white}} \cellcolor{black}\color{white}\bfseries\small Table \ref{tab:{\chapterlabel}1} Types of solutions } \\
\midrule
 \rowcolor{gray!10} Solute & Solvent & Solution state & Example\\
\midrule
Gas	& \multicolumn{1}{c}{Gas}&\multicolumn{1}{c}{Gas} &\multicolumn{1}{c}{Air} \\ 
Gas	& \multicolumn{1}{c}{Liquid}&\multicolumn{1}{c}{Liquid} &\multicolumn{1}{c}{Club soda} \\ 
Gas	& \multicolumn{1}{c}{Solid}&\multicolumn{1}{c}{Solid} &\multicolumn{1}{c}{\ce{H2} in Pd} \\ 
Liquid	& \multicolumn{1}{c}{Liquid}&\multicolumn{1}{c}{Liquid} &\multicolumn{1}{c}{Acetone in water} \\ 
Liquid	& \multicolumn{1}{c}{Solid}&\multicolumn{1}{c}{Solid} &\multicolumn{1}{c}{Easy light charcoal  } \\ 
Solid	& \multicolumn{1}{c}{Liquid}&\multicolumn{1}{c}{Liquid} &\multicolumn{1}{c}{Saltwater  } \\ 
Solid	& \multicolumn{1}{c}{Solid}&\multicolumn{1}{c}{Solid} &\multicolumn{1}{c}{Brass  } \\ 
 \bottomrule
\end{tabular}\end{center} 
Table \ref{tab:{\chapterlabel}1} reports different types of solutions with examples.


\item[\docfilehook{Colloids: not a solution}{}] 
We have that solutions are homogeneous mixtures in the form of only one visible phase. If we add an ionic solute to a solvent, the solute particles would break down into ions and these ions would be solvated by the solvent molecules. As such,  the solute in solution exists in a different state than the solid solute. We also have that a heterogeneous mixture would result from mixing sand and water. The particles of sand will suspend on the liquid but eventually, they will deposit on the bottom of the container. Between homogeneous and heterogenous mixtures we have colloids.

Colloids are special types of homogeneous mixtures. Colloids are suspensions of two components that are indeed immiscible in a non-homogeneous way. Think for example of milk, that is a colloid containing small particles of fat and protein suspended on a liquid. The particles of fat are called the dispersed phase and the liquid matrix is called the dispersing medium. The suspended particles of a colloid are larger than the particles of a normal homogeneous solution. Also, and perhaps more importantly, the particles of a colloid can be separated and for example, by adding a few drops of lemon to a glass of milk you will be able to separate both the fat and the liquid. In contract, solutions are made of inseparable components and the only way to separate the solvent and solute in a solution is by boiling it. Colloids are named depending on the nature of the dispersed and dispersant phases: aerosols (e.g. fog), foams (e.g. whipped cream), emulsions (e.g. mayonnaise), sols (e.g. milk of magnesia), and gels (e.g. jelly) are just a few examples of different types of colloids.

How to differentiate a solution and colloid? The Tyndall effect exposes the differences between solutions and colloids. A focused beam of light can easily pass through a solution as the particles of the solute are smaller than the light wavelength. Differently, when a colloid is exposed to the same light it will be scattered by the dispersed phase of the colloid which has a larger size. Therefore, you will be able to see the bean passing through the colloid. An example of the Tyndall effect is the scattering of light from the car headlights on a foggy day.
 \begin{center}
\refstepcounter{table}  \label{tab:{\chapterlabel}2}
\fontfamily{ppl}\selectfont
\begin{tabular}{llll}
\rowcolor{black!45}
\toprule
\multicolumn{4}{l}{\hypersetup{colorlinks,linkcolor={white}} \cellcolor{black}\color{white}\bfseries\small Table \ref{tab:{\chapterlabel}2} Types of colloids } \\
\midrule
 \rowcolor{gray!10} Dispersing medium & Dispersed medium& name & Example\\
\midrule
Gas	& \multicolumn{1}{c}{Liquid}&\multicolumn{1}{c}{Aerosol} &\multicolumn{1}{c}{Fog} \\ 
 Gas	& \multicolumn{1}{c}{Solid}&\multicolumn{1}{c}{Aerosol} &\multicolumn{1}{c}{Smoke} \\ 
Liquid	& \multicolumn{1}{c}{Gas}&\multicolumn{1}{c}{Foam} &\multicolumn{1}{c}{Whipped cream} \\ 
Liquid	& \multicolumn{1}{c}{Liquid}&\multicolumn{1}{c}{Emulsion} &\multicolumn{1}{c}{Mayonnaise} \\ 
Liquid	& \multicolumn{1}{c}{Solid}&\multicolumn{1}{c}{Sol} &\multicolumn{1}{c}{Milk of magnesia} \\ 
Solid	& \multicolumn{1}{c}{Gas}&\multicolumn{1}{c}{Foam} &\multicolumn{1}{c}{Styrofoam} \\ 
Solid	& \multicolumn{1}{c}{Liquid}&\multicolumn{1}{c}{Gel} &\multicolumn{1}{c}{jelly} \\ 
Solid	& \multicolumn{1}{c}{Solid}&\multicolumn{1}{c}{Solid sol} &\multicolumn{1}{c}{alloys} \\ 

 \bottomrule
\end{tabular}\end{center} 
Table \ref{tab:{\chapterlabel}2} reports all different types of colloids one can encounter with examples for each of them.
\end{description}


\section{Units of concentration}
There are numerous units to express concentration. Here we will cover molarity, molality, percent of the mass, and mole fraction. We will also review the concept of density under a new light. More importantly, we will also learn how to interchange different concentration units.

\sloppy 
\begin{description}
\item[\docfilehook{The concept of solution}{}] 
Mind that solution results of adding a solute to a solvent.
\begin{equation}
\text{solution}=\text{solvent} + \text{solute}\label{\chapterlabel:equation1}
\end{equation}
\item[\docfilehook{Molarity, $\text{M}$}{}] 
Molarity, M, is defined as the moles of solute divided by the liters of solution. In this chapter it is critical to specify the nature of the moles and the volume, as one can think of moles of solute or moles of solvent, as well as liters of solution or liters of solvent.
\begin{equation}
\boxed{ M=\frac{\text{moles of solute}}{\text{liters of solution}}}\label{\chapterlabel:equation2}
\end{equation}
For example, if we mix 0.4 moles of NaCl and we fill beaker until we reach a 100mL (0.1L) mark, the molarity of the solution will be 4M.
\item[\docfilehook{Mole fraction, $\chi$}{}] 
The mole fraction of a solute is the ratio of the moles of solute over the moles of solution--that is moles of solute plus the moles of solvent.
\begin{equation}
\boxed{ \chi=\frac{\text{moles of solute}}{\text{moles of solute + moles of solvent}}}\label{\chapterlabel:equation3}
\end{equation}
For example, if we mix 0.4 moles of NaCl and 0.6 moles of water, the mole fraction of solute will be 0.4. We can define a similar mole fraction of solvent and both, the mole fraction of solvent and solute should add up to 1.
\item[\docfilehook{Percent by mass, $\%_m$}{}] 
The percent by mass (or percent by weight) of a solute is the ratio of the grams of solute over the grams of solution--that is grams of solute plus the grams of solvent--multiplied by 100.
\begin{equation}
\boxed{ \%_m=\frac{\text{mass of solute}}{\text{mass of solute + mass of solvent}}\times 100}\label{\chapterlabel:equation4}\end{equation}
There is an equivalent concentration measure to the solute percent by mass but based on volume called the solute percent by volume. $\%_v$ is calculated as the ration of the solute volume and the solution volume in percent form. This percentage is useful when the solute and solvent are both liquids and can be measured in terms of volume.
For example, if we mix 5 grams of NaCl and 100 grams of water, the percent by mass of solute will be 5\%. We can define a similar percent by mass of solvent and both, the percent by mass of solvent and solute should add up to 100.
\item[\docfilehook{Molality, m}{}] 
The molality of a solution is the number of moles of solute per kilogram of solvent.
\begin{equation}
\boxed{ \text{m}=\frac{\text{moles of solute}}{\text{kg of solvent}}} \label{\chapterlabel:equation5}
\end{equation}
For example, if we mix 5 grams of NaCl and 100 grams (0.1 kg) of water, the molality of the solution will be 50m. The properties molarity and molality are different. In particular, molarity (M) depends on temperature. As the volume of a liquid slightly increases with temperature, molarity decreases with temperature. Differently, molality (m) is temperature independent.
\item[\docfilehook{Density of a solution, d}{}] 
Density of a solution--often expressed in g/mL--is the ration of the grams of solution and the mL of solution.
\begin{equation}
\boxed{ \text{d}=\frac{\text{grams of solution}}{\text{mL of solution}} }\label{\chapterlabel:equation6}
\end{equation}
Density is used to convert mass of solution into volume of solution, of the opposite. In the case of pure water, density of water is 1g/mL that is the mass in grams of water equals to its volume in mL. 


\begin{example} %%%%%%%%%%%%%%%%%%%%%%%% EXAMPLE BOX
A solution is prepared by mixing 1g of NaCl (MW=59g/mol) in 100g of water to give a final volume of 120mL. Calculate: 
\begin{inparaenum}[(a)]	
\item The percent by mass of solute
\item	 The mole fraction of solute 
\item  The molarity of the solution
\item  The molality of the solution
\item  The density of the solution
\end{inparaenum}\\
 \textlcsc{ \textcolor{dgreen}{\Large \textbf{Solution}} }\\
In this example it is convenient to recall all the parameters we have in terms of solute, solvent and solution. 
Regarding the solute, we have the mass of solute ($m_{\text{solute}}=1g$), and the moles of solute ($n_{\text{solute}}=1g\times \frac{\text{1 mol of NaCl}}{\text{59g of NaCl}}=0.017 \text{moles}$). Regarding the solvent, we have the mass of solvent ($m_{\text{solvent}}=100$g=0.1kg), the moles of water ($n_{\text{solvent}}=100g\times \frac{\text{1 mol of }\ce{H2O}}{\text{18g of }\ce{H2O}}=5.6 \text{moles}$), and the volume of solvent ($v_{\text{solvent}}=100$mL). Mind that in the case of water, very normally, its mass in grams equals to its volume in mL. In terms of solution, we can compute the mass of solution ($m_{\text{solution}}=1+100=101$g), the moles of solution ($n_{\text{solution}}=0.017+5.6=5.617\text{moles}$), and the volume of solution ($v_{\text{solution}}=120$mL=0.12L). Now, we are ready to compute all concentration units of the solution. The percent by mass of solute is the ration between the mass of solute and the mass of solution times 100:
\[\%_m=\frac{\text{mass of solute}}{\text{mass of solute + mass of solvent}}\times 100
=\frac{1\text{g of solute}}{101\text{g of solution}}\times 100=0.99\%\]
The mole fraction is the ration between the moles of solute and moles of solution:
\[\chi=\frac{\text{moles of solute}}{\text{moles of solute + moles of solvent}}
=\frac{0.017\text{moles of solute}}{5.617\text{moles of solution}}=3.02\times 10^{-3}	\]
The molarity of the solution is the ration between the moles of solute and the liters of solution:
\[\text{M}=\frac{\text{moles of solute}}{\text{liters of solution}}=\frac{5.617\text{moles of solution}}{0.12\text{L of solution}}=46.8\text{M}\]
The molality is the ration between the moles of solute and the kg of solvent:
\[	\text{m}=\frac{\text{moles of solute}}{\text{kg of solvent}} =\frac{0.017\text{moles of solute}}{0.1\text{kg of solvent}}=0.17\text{m}
	\]
Finally, density of the solution is the ration of the grams of solution and the volume of solution:
\[\text{d}=\frac{\text{grams of solution}}{\text{mL of solution}} = \frac{101\text{g of solution}}{120\text{mL of solution}}=0.84\text{g/mL}
	\]

\faDiamond\ \textlcsc{ \textcolor{dgreen}{\Large \textbf{Study Check}} }\\
A solution is prepared by mixing 1g of glucose (MW=180g/mol) in 50g of water to give a final volume of 100mL. Calculate: 
\begin{inparaenum}[(a)]	
\item The percent by mass of solute %1.96\%
\item	 The mole fraction of solute %$2\times 10^{-3}$
\item  The molarity of the solution	%0.05M
\item  The molality of the solution	%0.11m
\item  The density of the solution	%0.51g/mL
\end{inparaenum}\\
\begin{flushright} Answer: \begin{inparaenum}[(a)]	
\item  1.96\%
\item	  $2\times 10^{-3}$
\item   0.05M
\item   0.11m
\item   0.51g/mL
\end{inparaenum}\end{flushright}
\end{example}%%%%%%%%%%%%%%%%%%%%%%%% EXAMPLE BOX


\item[\docfilehook{Why relating units of concentration}{}] 
When you prepare a solution you normally weight a given amount of solute and add some volume of solvent. That will give you a given concentration that you can compute in terms of for example molarity. Often, you encounter a solution already prepared, for example a 2M solution and you need to know a different type of concentration unit, such as its molality. That is why relating concentration units is important. In the next sections we will cover how the different concentration units are related.
\item[\docfilehook{Relating molarity and molality, $\text{M }\longleftrightarrow\text{ m}$}{}] 
We can use the following formula in order to relate Molarity and molality
\begin{equation}
\boxed{ \text{m}=\frac{1000\cdot \text{M}}{1000\cdot d - \text{M}\cdot MW}}
\quad\quad or \quad\quad
\boxed{ \text{M}=\frac{ 1000\cdot m\cdot d   }{1000+MW\cdot m   	}}
\label{\chapterlabel:equation7}
\end{equation}
where:
\begin{where}
 \item M   is the molarity of the solution
  \item m   is the molality of the solution
  \item $d$   is the density of the solution 
  \item $MW$   is the molecular weight of the solute 
\end{where}
Mind that this formula only works for water as solvent and used 1g/mL as the density of water.
\item[\docfilehook{Relating molarity and the percent by mass of solute, $\text{M }\longleftrightarrow\%_{m}$}{}] 
We can use the following formula in order to relate Molarity and percent by mass of solute
\begin{equation}
\boxed{ \text{M}=\frac{\%_m \cdot \text{d}\cdot 10}{MW}}
\quad  \text{or }\quad 
\boxed{ \%_m=\frac{\text{M}\cdot MW }{10\cdot \text{d} }}
\label{\chapterlabel:equation8}
\end{equation}
where:
\begin{where}
 \item M   is the molarity of the solution
  \item $\%_m$   is the percent by mass of solute
  \item $d$   is the density of the solution 
  \item $MW$   is the molecular weight of the solute 
\end{where}

\item[\docfilehook{Relating percent by mass and mole fraction of solute, $\chi \longleftrightarrow\%_{m}$}{}] 
We can use the following formula in order to relate Molarity and percent by mass of solute
\begin{equation}
\boxed{ \chi=\frac{18\cdot  \%_{m}  }{100\cdot MW + (18-MW)\cdot \%_{m} }}
\quad  \text{or }\quad 
\boxed{ \%_{m}=\frac{ 100\cdot MW\cdot \chi  }{ 18+(MW-18)\chi }}
\label{\chapterlabel:equation9}
\end{equation}
where:
\begin{where}
 \item $\chi$   is the solute mole fraction
  \item $\%_m$   is the solute percent by mass  
  \item $MW$   is the molecular weight of the solute 
  \item 18 is the molar weight of water
\end{where}
Mind that this formula only works for water as solvent. If using a different solvent you just need to update the $18$ value and use the molar weight of the new solvent instead.
\begin{example} %%%%%%%%%%%%%%%%%%%%%%%% EXAMPLE BOX
A 2M NaCl (MW=59g/mol) solution has a density of 1.2g/mL. Calculate: 
\begin{inparaenum}[(a)]	
\item The molality of the solution
\item  The mass percent of solute
\item	 The mole fraction of solute 
\end{inparaenum}\\
 \textlcsc{ \textcolor{dgreen}{\Large \textbf{Solution}} }\\
 We will first convert molarity into molality using density and the molar mass of the solute.
 \[\text{m}=\frac{1000\cdot \text{M}}{1000\cdot d - \text{M}\cdot MW}=\frac{1000\cdot 2}{1000\cdot 1.2 - 2\cdot 59}=1.84m\]
 We will then convert molarity into mass percent:
 \[ \%_m=\frac{\text{M}\cdot MW }{10\cdot \text{d} }=\frac{2\cdot 59 }{10\cdot 1.2 }=9.83\%\]
 We will finally convert the mass percent of solute into mole fraction:
 \[\chi=\frac{18\cdot  \%_{m}  }{100\cdot MW + (18-MW)\cdot \%_{m} }=\frac{18\cdot  9.83\%  }{100\cdot 59 + (18-59)\cdot 9.83\%  }=0.03\]
\faDiamond\ \textlcsc{ \textcolor{dgreen}{\Large \textbf{Study Check}} }\\
For a 0.11m glucose (MW=180g/mol) solution with density 0.51g/mL, calculate:
\begin{inparaenum}[(a)]	
\item The percent by mass of solute  
\item	 The mole fraction of solute  
\item  The molarity of the solution	 
\end{inparaenum}\\
\begin{flushright} Answer: \begin{inparaenum}[(a)]	
\item  1.76\%
\item	  $1.78\times 10^{-3}$
\item   0.05M
\end{inparaenum}\end{flushright}
\end{example}%%%%%%%%%%%%%%%%%%%%%%%% EXAMPLE BOX

\item[\docfilehook{Compute molecular masses from molality and molarity}{}] 
In numerous applications one needs to compute the molecular weight of a solute by means of a given molality or molarity. It is useful to remember that the molality of a solution is related to the moles of solute and the kilograms of solvent, in contrast to the molarity of a solution that depends on the litters of solution. When we prepare a solution we normally know the mass of solute used and the mass of the solvent or the volume of solution. We can compute the molar mass of the solute by means of:
\begin{equation}
\boxed{ MW=\frac{ \text{g of solute}   }{\text{m}\cdot \text{kg of solvent}   	}}
\quad\quad\text{or} \quad\quad\
\boxed{ MW=\frac{ \text{g of solute}   }{\text{M}\cdot \text{L of solution}}}
\label{\chapterlabel:equation17}
\end{equation}
where:
\begin{where}
 \item MW   is the molar mass of the solute
  \item $\text{g of solute}$   is the mass of solute
  \item $\text{kg of solvent}$   is the mass of solvent
    \item $\text{L of solution}$   is the volume of solution
  \item $\text{m}$   is the molality of the solution
\end{where}
\begin{example} %%%%%%%%%%%%%%%%%%%%%%%% EXAMPLE BOX
We prepare a solution by weighting 5g of solute and adding 10g of solvent in order to prepare a 0.1m solution. Calculate the molar mass of the solute.
\\
 \textlcsc{ \textcolor{dgreen}{\Large \textbf{Solution}} }\\
In this example, we are given molality (m=0.1), the mass of solute ($m_{\text{solute}}$=5g) and the mass of solvent ($m_{\text{solvent}}$=0.01kg). In order to calculate the molar mass of a solute by means of molality we have that:
 \[MW=\frac{ \text{g of solute}   }{\text{m}\cdot \text{kg of solvent}   	}=\frac{5}{0.1\cdot 0.01}=5000g/mol \]
 \faDiamond\ \textlcsc{ \textcolor{dgreen}{\Large \textbf{Study Check}} }\\
We prepare a solution by weighting 1g of solute and adding liquid until 100mL of solution in order to prepare a 2M solution. Calculate the molar mass of the solute.
\\
\begin{flushright} Answer: 5g/mol\end{flushright}
\end{example}%%%%%%%%%%%%%%%%%%%%%%%% EXAMPLE BOX





\end{description}



\section{Solutions of electrolytes and effective solute particles}
Chemicals can be classified based on their electrolyte character in strong electrolytes and non-electrolytes. Non-electrolytes do not dissociate in solution so that each non-electrolyte molecules becomes a solute particle. Differently, strong electrolytes dissociate in solution so that each strong electrolyte molecule gives more than one solute particles. This section covers the concept of effective solute particle and the idea of Van't Hoff factor $i$ that related the amount of moles of solute dissolved and the moles of solute particles.
\sloppy 
\begin{description}
\item[\docfilehook{Strong and non-electrolytes}{}] 
 Strong electrolytes completely dissociate in water. Hence, in a solution of a strong electrolyte you will only have ions and never molecules. Strong electrolytes are typically ionic compounds such as \ce{MgCl2} or \ce{NaCl} (table salt). We represent the dissociation of a strong electrolyte with a single arrow, meaning that the reaction proceeds to completion and for the example below, in the solution we will only have ions (\ce{Mg^{2+}_{(aq)} + 2Cl^{-}_{(aq)}}) and not molecules (\ce{MgCl2_{(s)}}):
\begin{center}\ce{MgCl2_{(s)}  ->[H2O] Mg^{2+}_{(aq)} + 2Cl^{-}_{(aq)} }.\end{center}
In terms of solute dissolved and solute particles, we have that one more of solute dissolved gives three moles of solute particle:
\begin{center}\ce{1 moles of solute dissolved  ->[H2O] 3 moles of solute particles }.\end{center}

\item[\docfilehook{Nonelectrolytes}{Nonelectrolytes}] Nonelectrolytes do not dissociate in water. Hence a solution of a nonelectrolyte will only contains molecules and not ions. Examples of nonelectrolytes are carbon-based chemicals such as methanol, ethanol, urea or sucrose. The dissociation of urea for example \ce{CH4N2O} proceeds as:
\begin{center}\ce{CH4N2O(s)  ->[H2O] CH4N2O_{(aq)} }\end{center}
In terms of solute dissolved and solute particles, we have that one more of solute dissolved gives one mole of solute particle:
\begin{center}\ce{1 moles of solute dissolved  ->[H2O] 1 moles of solute particles }.\end{center}
\item[\docfilehook{Breaking down electrolytes into ions}{}] Electrolytes--in particular strong electrolytes--dissociate producing ions. This way, a solution of for example \ce{NaCl} does not contain \ce{NaCl} molecules but \ce{Na^+_{(aq)}} cations and \ce{Cl^-_{(aq)}} anions. Hence it is important to  correctly break down electrolytes into ions. In order to do this, you need to revert the combination of ions that produce a given chemical while making sure the charges are balanced. For example, let us beak magnesium chloride \ce{MgCl2_{(aq)}} into ions. This is a strong electrolytes formed by magnesium cations and chloride anions. The valence of magnesium is +II and the valence of chlorine is -I. The \ce{MgCl2} formula also tells us we have one magnesium and two chlorines. The overall process is:
\begin{center}\ce{MgCl2_{(aq)} -> Mg^{2+}_{(aq)} + 2Cl^{-}_{(aq)} }\end{center}
Another example, magnesium nitrate \ce{Mg(NO3)2}. This strong electrolyte--as this is an ionic salt--is made of lithium with valence +I and nitrate with valence -I. The formula indicated we have one \ce{Mg^{2+}_{(aq)}} and two \ce{NO3^{-}_{(aq)}}. Hence:
\begin{center}\ce{Mg(NO3)2_{(aq)} -> Mg^{2+}_{(aq)} + 2NO3^{-}_{(aq)} }\end{center}

\begin{example} %%%%%%%%%%%%%%%%%%%%%%%% EXAMPLE BOX
Break down the following chemicals into ions, if possible:\\
\begin{center}\fontfamily{ppl}\selectfont
\begin{tabular}{ll}
\rowcolor{black!45}
\toprule
Chemical &  Particles in solution  \\
\midrule
\ce{K2CrO4_{(aq)}} &\hspace{3cm}  \\
\ce{Ba(NO3)2_{(aq)}} & \hspace{3cm}   \\
\ce{BaCrO4_{(s)}} &\hspace{1cm}    \\
 \ce{KNO3_{(aq)} } &\hspace{1cm}    \\
\bottomrule
\end{tabular}\end{center}
\textlcsc{ \textcolor{dgreen}{\Large \textbf{Solution}} }\\
We can only break down into ions ionic compounds and oxosalts that are not solid. From the list of chemicals in the example, we will not be able to break down \ce{BaCrO4_{(s)}} into ions as it is a solid. From the other chemicals, \ce{K2CrO4_{(aq)}} is named potassium chromate and contains 2\ce{K^+_{(aq)}} and \ce{CrO4^{2-}_{(aq)}} ions. Barium nitrate--\ce{Ba(NO3)2_{(aq)}}--will produce \ce{Ba^{2+}_{(aq)}} and 2\ce{NO3^{-}_{(aq)}}. Finally, potassium nitrate-- \ce{KNO3_{(aq)} }--will produce \ce{K^{+}_{(aq)}} and \ce{NO3^{-}_{(aq)}}. In the table:
\begin{center}\fontfamily{ppl}\selectfont
\begin{tabular}{ll}
\rowcolor{black!45}
\toprule
Chemical &  Particles in solution  \\
\midrule
\ce{K2CrO4_{(aq)}} & 2\ce{K^+_{(aq)}} + \ce{CrO4^{2-}_{(aq)}}  \\
\ce{Ba(NO3)2_{(aq)}} &\ce{Ba^{2+}_{(aq)}} + 2\ce{NO3^{-}_{(aq)}}\\
\ce{BaCrO4_{(s)}} &\ce{BaCrO4_{(s)}}    \\
 \ce{KNO3_{(aq)} } &\ce{K^{+}_{(aq)}} + \ce{NO3^{-}_{(aq)}}   \\
\bottomrule
\end{tabular}\end{center}
\faDiamond\ \textlcsc{ \textcolor{dgreen}{\Large \textbf{Study Check}} }\\
Break down the following chemicals into ions, if possible: \ce{H2O_{(l)}}, \ce{NH3_{(l)}}, \ce{AgNO3_{(aq)}}.
  \\
\flushright  {\small Answer: \ce{H2O_{(l)}}, \ce{NH3_{(l)}}, \ce{Ag^+_{(aq)}}, \ce{NO3^-_{(aq)}}.}
\end{example}%%%%%%%%%%%%%%%%%%%%%%%% EXAMPLE BOX



\item[\docfilehook{Van't Hoff factor $i$}{ }] This factor for a given electrolyte related the number of dissolved particles and the number of solute particles:
\begin{equation}
\boxed{ i=\frac{\text{moles of solute particles} }{\text{moles of dissolved particles} }}
\label{\chapterlabel:equation15}
\end{equation}
For example, all non-electrolytes produce only a single solute particles, as they do not break down in solution and therefore for all non-electrolytes $i$ is 1. Differently, the $i$ values for a strong electrolyte is depends on the salt stoichiometry. For example, for \ce{NaCl} $i$ is two, as one mole of salt produces two moles of ions, and for \ce{CaCl2} $i$ is 3, as one mole of calcium fluoride produces three moles of ions, overall.
It is important to notice the research has found that the Van't Hoff factor indeed depends on the concentration of the salt and for large concentrations the expected $i$ value not always corresponds to the observed value, due to the formation of ion pairs, pairs of ions that associate on solution reducing the effective ion concentration.
 \begin{center}
\refstepcounter{table}  \label{tab:{\chapterlabel}3}
\fontfamily{ppl}\selectfont
\begin{tabular}{llllllll}
\rowcolor{black!45}
\toprule
\multicolumn{8}{l}{\hypersetup{colorlinks,linkcolor={white}} \cellcolor{black}\color{white}\bfseries\small Table \ref{tab:{\chapterlabel}3} Expected and observed Van't Hoff factors for different concentrations. } \\
\midrule
 \rowcolor{gray!10} Solute &  \multicolumn{3}{c}{$i^{\text{Observed}}$} &  \multicolumn{1}{c}{$i^{\text{Expected}}$}   &\multicolumn{3}{c}{$i^{\text{Observed}}/i^{\text{Expected}}*100$}   \\
 \midrule
   & 0.1m & 0.01m & 0.001m &  \multicolumn{1}{c}{ }&0.1m & 0.01m & 0.001m\\

\midrule
\ce{C6H12O6}	& \multicolumn{1}{c}{1.00}& \multicolumn{1}{c}{1.00} & \multicolumn{1}{c}{1.00}& \multicolumn{1}{c}{1.00} &\multicolumn{1}{c}{100\% }&\multicolumn{1}{c}{ 100\%}&\multicolumn{1}{c}{ 100\%}\\ 
\ce{NaCl}	& \multicolumn{1}{c}{1.87}& \multicolumn{1}{c}{1.94} & \multicolumn{1}{c}{1.97}& \multicolumn{1}{c}{2.00} &\multicolumn{1}{c}{ 93.5\%}&\multicolumn{1}{c}{97.0\% }&\multicolumn{1}{c}{98.5\% }\\ 
\ce{K2SO4}	& \multicolumn{1}{c}{2.32}& \multicolumn{1}{c}{2.70} & \multicolumn{1}{c}{2.84}& \multicolumn{1}{c}{3.00} &\multicolumn{1}{c}{ 77.3\%}&\multicolumn{1}{c}{ 90.0\%}&\multicolumn{1}{c}{ 64.6\%}\\ 
\ce{MgSO4}	& \multicolumn{1}{c}{1.21}& \multicolumn{1}{c}{1.53} & \multicolumn{1}{c}{1.82}& \multicolumn{1}{c}{2.00} &\multicolumn{1}{c}{ 60.5\%}&\multicolumn{1}{c}{76.5\% }&\multicolumn{1}{c}{91.0\% }\\ 
 \bottomrule
\end{tabular}\end{center} 
Table \ref{tab:{\chapterlabel}3} reports observed and expected $i$ values for different salts and different concentrations. At low concentration the effect of ion pairs is less pronounced and the expected value tend to resemble the observed value: $i^{\text{Expected}}\simeq i^{\text{Observed}}$. The Van't Hoff factor is reported for strong electrolytes.
A property called percent dissociation of an electrolyte, $\alpha$, is helpful to describe the dissociation of weak electrolytes. Strong electrolytes dissociate completely and hence the effective concentration of ions is the same as the nominal concentration of solute particles. Weak electrolytes, on the other hand, do not completely dissociate in solution and the effective concentration of solute particles tend to be smaller than the nominal concentration. We have that:
\begin{equation}
\boxed{ \alpha = \frac{c^{\text{effective}}}{c^{\text{Nominal}}} \times 100	}
\label{\chapterlabel:equation16}
\end{equation}
Percent dissociation is zero for nonelectrolytes as no ions are produced and hence the effective ion concentration is null. Finally, percent dissociation for strong electrolytes is 100\% as the nominal and effective ion concentration are the same. Note that the percent dissociation can change with concentration.
\begin{example} %%%%%%%%%%%%%%%%%%%%%%%% EXAMPLE BOX
A solution that is 0.02M in HF has a effective ion concentration of 0.015M. Calculate the percent dissociation of the electrolyte.
\\
 \textlcsc{ \textcolor{dgreen}{\Large \textbf{Solution}} }\\
The percent dissociation, $\alpha$, is the ration between the effective ion concentration (the amount of ions in solution) and the nominal concentration (the amount of solute added to the solution). We have that the nominal concentration is 0.02M and the effective concentration is 0.015M. As usual for weak electrolytes, the effective concentration is lower than the nominal one as some molecules did not dissociate. We can calculate the percent dissociation:
\[\alpha=\frac{c^{\text{effective}}}{c^{\text{Nominal}}} \times 100=\frac{0.015}{0.02}\times 100=75\%\]
The results indicate that only the 75\% of the solute dissociate in solution.\\
 \faDiamond\ \textlcsc{ \textcolor{dgreen}{\Large \textbf{Study Check}} }\\
The percent dissociation of a 0.1M weak electrolyte is 40\%. Calculate the effective ion concentration.
\\
\begin{flushright} Answer: 0.04M\end{flushright}
\end{example}%%%%%%%%%%%%%%%%%%%%%%%% EXAMPLE BOX
\end{description}


\section{Colligative properties}
Colligative properties of solutions are properties that depend on the concentration of solute but not on the nature of this solute. In the following, we will elaborate more on the idea of colligative properties. Indeed, there are four colligative properties of the solutions: the freezing point decrease, the boiling point increase, the osmotic pressure, and the vapor pressure. Mind that these are all properties of solutions and not of pure substances.

\sloppy 
\begin{description}
\item[\docfilehook{Boiling point elevation}{}] 
Solutions are made of a solute dissolved in a solvent. Pure solvent have a specific boiling point. When we made a solution, the solution boils at a different temperature than the pure solvent, and in particular, a solution boils at a higher temperature than the solvent. This effect is called boiling point elevation of a solution in comparison with the pure solvent. The boiling point elevation does not depend on the nature of the solute and only depends on the molality (m) of the solution by means of the following formula:
\begin{equation}
\boxed{ T_b^{\text{solution}}=T_b^{\text{solvent}}+k_b\cdot i\cdot m 	}
\quad  \text{or }\quad 
\boxed{\Delta T_b =k_b\cdot i\cdot m}
\label{\chapterlabel:equation10}
\end{equation}
where:
\begin{where}
 \item $T_b^{\text{solution}}$  is the boiling point of the solution
  \item $T_b^{\text{solvent}}$   is the boiling point of the pure solvent  
  \item $k_b$   is called boiling point elevation constant in units of $^{\circ}C/m$
  \item $m$ is the molality of the solution
    \item $i$ is van't Hoff factor of the solute
  \item $\Delta T_b$ is the boiling point increase, that is $T_b^{\text{solution}}-T_b^{\text{solvent}}$
      \item $i\cdot m$ is the effective concentration of solute particles

\end{where}
Mind that $i\cdot m$ represents the effective solute-particles concentration in the solution. For the case of NaCl $i$ is 2 as every NaCl unit dissociates producing 2 ions or two solute-particles. The value of $k_b$ depends on the solvent and in general boiling point increases tend to be modest. For example, a water-based solution containing NaCl boils at a higher temperature than pure water which boils at 100$^{\circ}C$. A 1m NaCl solution boils at 101.04$^{\circ}C$ that is one degree higher than pure water.
The boiling point elevation formula establishes a linear relationship between the boiling point of a solution and molality in which the slope of the relationship is positive and gives the value of $k_b$, the $x$ variable is $m$ and the $y$ variable is $T_b^{\text{solution}}$. In another words, by plotting $T_b^{\text{solution}}$ vs. $m$ we should obtain a straight line with a slope that equals to $k_b$ and an intercept equals to $T_b^{\text{solvent}}$.
\item[\docfilehook{Freezing point depression}{}] 
A pure solvent freezes at a specific temperature and for example water freezes at 0$^{\circ}$C. Solutions freeze at a lower temperature than the pure solvent. We call this effect the freezing point depression. This decrease on the freezing point depends only on the molality of the solution and not on the solute. The freezing point depression--or decrease--is given by the formula:
\begin{equation}
\boxed{ T_f^{\text{solution}}=T_f^{\text{solvent}}-k_f\cdot i\cdot m 	}
\quad  \text{or }\quad 
\boxed{\Delta T_f =-k_f\cdot i\cdot m}
\label{\chapterlabel:equation11}
\end{equation}
where:
\begin{where}
 \item $T_f^{\text{solution}}$  is the freezing point of the solution
  \item $T_f^{\text{solvent}}$   is the freezing point of the pure solvent  
  \item $k_f$   is called freezing point depression constant in units of $^{\circ}C/m$
  \item $m$ is the molality of the solution
      \item $i$ is van't Hoff factor of the solute
  \item $\Delta T_f$, a negative value, is the freezing point depression, that is $T_f^{\text{solution}}-T_f^{\text{solvent}}$
        \item $i\cdot m$ is the effective concentration of solute particles

\end{where}

Mind that $i\cdot m$ represents the effective solute-particles concentration in the solution. For the case of NaCl $i$ is 2 as every NaCl unit dissociates producing 2 ions, or 2 solute particles.
For example, pure water freezes at 0$^{\circ}$C but a 1m NaCl solution freezes at -3.72$^{\circ}$C, that is almost four degrees lower than pure water.
The freezing point depression formula establishes a linear relationship between the freezing point of a solution and molality in which the slope of the relationship is negative and gives the value of $k_f$, the $x$ variable is $m$ and the $y$ variable is $T_f^{\text{solution}}$. In another words, by plotting $T_f^{\text{solution}}$ vs. $m$ we should obtain a straight line with a slope in absolute value that equals to $k_f$ and an intercept equals to $T_f^{\text{solvent}}$.

 \begin{center}
\refstepcounter{table}  \label{tab:{\chapterlabel}4}
\fontfamily{ppl}\selectfont
\begin{tabular}{lllll}
\rowcolor{black!45}
\toprule
\multicolumn{5}{l}{\hypersetup{colorlinks,linkcolor={white}} \cellcolor{black}\color{white}\bfseries\small Table \ref{tab:{\chapterlabel}4} Boiling-point elevation and freezing-point depression for various solvents } \\
\midrule
 \rowcolor{gray!10} Solvent &\multicolumn{1}{c}{$T^{solvent}_b$ ($^{\circ}$C)}&  \multicolumn{1}{c}{$k_b$ ($^{\circ}$C/m)} &  \multicolumn{1}{c}{$T^{solvent}_f$ ($^{\circ}$C)}&\multicolumn{1}{c}{$k_f$ ($^{\circ}$C/m)}\\
\midrule
Acetic acid	& \multicolumn{1}{c}{117.9 }& \multicolumn{1}{c}{3.07 } & \multicolumn{1}{c}{16.6 }& \multicolumn{1}{c}{3.90 } \\ 
 Benzene	& \multicolumn{1}{c}{80.1 }& \multicolumn{1}{c}{2.53 } & \multicolumn{1}{c}{5.5 }& \multicolumn{1}{c}{4.90 } \\ 
Carbon disulfide	& \multicolumn{1}{c}{46.2 }& \multicolumn{1}{c}{2.34 } & \multicolumn{1}{c}{-111.5 }& \multicolumn{1}{c}{3.83 } \\ 
Carbon tetrachloride	& \multicolumn{1}{c}{76.5 }& \multicolumn{1}{c}{5.03 } & \multicolumn{1}{c}{-23 }& \multicolumn{1}{c}{30 } \\ 
Chloroform	& \multicolumn{1}{c}{61.7 }& \multicolumn{1}{c}{3.63 } & \multicolumn{1}{c}{-63.5 }& \multicolumn{1}{c}{4.70 } \\ 
Diethyl ether	& \multicolumn{1}{c}{34.5 }& \multicolumn{1}{c}{2.02 } & \multicolumn{1}{c}{-116.2 }& \multicolumn{1}{c}{1.79 } \\ 
Ethanol	& \multicolumn{1}{c}{78.5 }& \multicolumn{1}{c}{1.22 } & \multicolumn{1}{c}{-117.3 }& \multicolumn{1}{c}{1.99 } \\ 
Water	& \multicolumn{1}{c}{100 }& \multicolumn{1}{c}{0.512 } & \multicolumn{1}{c}{0 }& \multicolumn{1}{c}{1.86 } \\ 

 \bottomrule
\end{tabular}\end{center} 
Table \ref{tab:{\chapterlabel}4} reports values for the boiling-point elevation and freezing-point depression for various solvents.

\begin{example} %%%%%%%%%%%%%%%%%%%%%%%% EXAMPLE BOX
For a solution of 5 g of \ce{I2} (MW=254g/mol) in 100 g of benzene, \ce{C6H6}:
\begin{inparaenum}[(a)]	
\item Calculate its molality %50m  
\item	 Given that benzene boiling point is 80$^{\circ}$C, and that $k_b$ = 2.53 $^{\circ}$C/m, calculate the boiling point and the boiling point elevation of the solution.  
\item	 Given that benzene freezing point is 5$^{\circ}$C, and that $k_f$ = 5.10 $^{\circ}$C/m, calculate the freezing point and the freezing point depression of the solution.   
\end{inparaenum}
\\
 \textlcsc{ \textcolor{dgreen}{\Large \textbf{Solution}} }\\
In order to calculate the molality of the solution, we need the moles of solute ($5g\times \frac{1mol}{254g}=0.019$moles)  and the kilograms of solvent (100g=0.1kg). We have that molality is
\[m=\frac{\text{moles of solute}}{\text{kg of solvent}}=\frac{0.019}{0.1}=0.19m\]
With the molality, we can calculate the freezing and boiling point of the solution. Mind that the pure solvent freezes at 5$^{\circ}$C and boils at 80$^{\circ}$C. The solution will freeze at a lower temperature and will boil at a higher temperature--that is, it will experience a freezing depression and boiling elevation. We will calculate first the freezing point (and freezing point depression):
\[T_f^{\text{solution}}=T_f^{\text{solvent}}-k_f\cdot i\cdot m =5-5.10\cdot 1\cdot 0.19=4.031 ^{\circ} C\]
In other words the freezing depression will be
\[\Delta T_f =-k_f\cdot i\cdot m=-5.10\cdot 1\cdot 0.19=-0.969^{\circ}C\]
We will now calculate the boiling point elevation:
\[T_b^{\text{solution}}=T_b^{\text{solvent}}+k_b\cdot i\cdot m=80+2.53\cdot 1\cdot 0.19=80.48^{\circ}C\]
 In other words the boiling elevation will be
\[\Delta T_b =k_b\cdot i\cdot m=2.53\cdot 1\cdot 0.19=0.4807^{\circ}C \]
\\
\faDiamond\ \textlcsc{ \textcolor{dgreen}{\Large \textbf{Study Check}} }\\
For a solution of 5 g of \ce{NaCl} (MW=58g/mol) in 100 g of acetic acid, \ce{CH3COOH}:
\begin{inparaenum}[(a)]	
\item Calculate its molality %0.86m  
\item	 Given that benzene boiling point is 118$^{\circ}$C, and that $k_b$ = 3.08 $^{\circ}$C/m, calculate the boiling point and the boiling point elevation of the solution.  
\item	 Given that benzene freezing point is 17$^{\circ}$C, and that $k_f$ = 3.59 $^{\circ}$C/m, calculate the freezing point and the freezing point depression of the solution.   
\end{inparaenum}
\begin{flushright} Answer: 
\begin{inparaenum}[(a)]	
\item  50m  
\item	  123.3$^{\circ}$C ; 5.29$^{\circ}$C
\item	  10.82$^{\circ}$C ; -6.17$^{\circ}$C
\end{inparaenum}
\end{flushright}
\end{example}%%%%%%%%%%%%%%%%%%%%%%%% EXAMPLE BOX





\item[\docfilehook{Vapor-pressure lowering}{}] 
Every liquid exerts a certain vapor pressure that depends on temperature. The molecules of the surface of the liquid are less tied than the molecules of the interior part of the liquid called the bulk. As such, they can scape producing what we call vapor pressure of the liquid. Solutions exert lower vapor pressure than the pure solvent. The vapor-pressure lowering is a colligative property that depends on the solute mole fraction:
\begin{equation}
\boxed{ P_{vap}^{\text{solution}}=(1-  \chi )\cdot P_{vap}^{\text{solvent}} 	}
\quad  \text{or }\quad 
\boxed{\Delta P_{vap} =  -  \chi \cdot P_{vap}^{\text{solvent}}  }
\label{\chapterlabel:equation12}
\end{equation}
where:
\begin{where}
 \item $P_{vap}^{\text{solution}}$  is the vapor pressure of the solution
 \item $P_{vap}^{\text{solvent}}$  is the vapor pressure of the pure solvent
  \item $\chi$   is the solute mole fraction 
  \item $\Delta P_{vap} $, a negative value, is the vapor-pressure lowering, that is, $P_{vap}^{\text{solution}}-P_{vap}^{\text{solvent}}$
\end{where}
For example, the vapor pressure of water at 25$^{\circ}$C  is 0.03 atm. If we make a solution with a 0.5 solute mole fraction by adding table salt to the water, the vapor pressure of this solution would be 0.015 atm. In other words, the vapor pressure is lower than the one from pure water.  Equation \ref{\chapterlabel:equation12} is called Raoult's Law.
Raoult's Law establishes a linear relationship between the vapor-pressure of a solution and the mole fraction in which the slope of the relationship gives the vapor pressure of the pure solvent, the $x$ variable is $1-\chi$ and the $y$ variable is $P_{vap}^{\text{solution}}$. Simply put, by plotting $P_{vap}^{\text{solution}}$ vs. $1-\chi$ we obtain a straight line with a slope equals to $P_{vap}^{\text{solution}}$.


\begin{example} %%%%%%%%%%%%%%%%%%%%%%%% EXAMPLE BOX
Given that the vapor pressure of water at 25$^{\circ}$C  is 0.03 atm, calculate the vapor pressure and the vapor pressure lowering of a 20\% of mass NaCl (MW=59g/mol) solution at that temperature.
\\
 \textlcsc{ \textcolor{dgreen}{\Large \textbf{Solution}} }\\
In order to calculate the vapor pressure of a solution or the vapor-pressure lowering of a solution we need to calculate the mole fraction of the solute. We can calculate $\chi$ by means of the solution of the solute mass percent:
\[\chi=\frac{18\cdot  \%_{m}  }{100\cdot MW + (18-MW)\cdot \%_{m} }=\frac{18\cdot  20  }{100\cdot 59 + (18-59)\cdot 20 }=0.07\]
Now, with the mole fraction of the solute and given that the vapor pressure of the solvent is 0.03atm, we can compute first the vapor pressure of the solution
\[ P_{vap}^{\text{solution}}=(1-  \chi ) \cdot P_{vap}^{\text{solvent}}=(1-0.07)\cdot 0.03=0.028atm\]
and then the vapor-pressure lowering
\[\Delta P_{vap}=-2.1\times 10^{-3} atm\]
\faDiamond\ \textlcsc{ \textcolor{dgreen}{\Large \textbf{Study Check}} }\\
Calculate the vapor-pressure lowering of a 3m \ce{I2} (MW=254g/mol) solution in cyclohexane at 279K given  that the vapor pressure of cyclohexane at that temperature is 5.164kPa and the solution density is 1.3g/mL.\\
\begin{flushright} Answer: $\chi=0.05$; $\Delta P_{vap}=-0.26kPa$\end{flushright}
\end{example}%%%%%%%%%%%%%%%%%%%%%%%% EXAMPLE BOX



\item[\docfilehook{Osmotic pressure of a solution}{}] 
The pressure of a gas result from the movement of the gas particles as they hit the walls of their container. The higher the hitting frequency the higher pressure. More specifically, the pressure of a gas depends on the force exerted by the gas molecules per unit of container area. Solutions can also exhibit pressure as solute molecules also hit the walls of their container. This pressure is called osmotic pressure. The higher molarity, that is the number of moles per liter in the solution, the higher the osmotic pressure of a solution:
\begin{equation}
\boxed{ \pi=i\cdot M\cdot RT 	}
\label{\chapterlabel:equation13}
\end{equation}
where:
\begin{where}
 \item $\pi$  is the osmotic pressure of the solution in atm
 \item $M$  is the molarity of the solution
  \item $R$   is the constant of the gases: 0.082 atmL/Kmol
  \item $T $ is the temperature of the solution in K
        \item $i$ is van't Hoff factor of the solute
      \item $i\cdot M$ is the effective concentration of solute particles
\end{where}
The osmotic pressure of solutions is responsible for an effect called osmosis. First, let us talk about what is a semipermeable membrane. These are a type of membranes that allow the pass of solvent molecules without allowing the pass of solute molecules. 
We could set up and experiment in which we separate two solutions of different concentrations by a semipermeable membrane and we will certainly see that the level of liquid on the most concentrated solution will rise, whereas the level of liquid in the most dilute will reduce. 
This effect results from the travel of water molecules from the less to the more concentrated solution equilibrating the osmotic pressure at the semipermeable membrane. 
Solute diffusion normally occurs against the concentration gradient. Solute diffuses from more to the less concentrated solution. The more concentrated solution is referred to as the hypertonic solution, whereas the less concentrated solution is referred to as the hypotonic solution.
Differently, solvent diffusion occurs following the concentration gradient (Low M $\rightarrow$ High M). Osmosis is responsible for many every-day life phenomena such as salting as a method to preserve food: by adding salt one can kill bacteria and preserve fish by extracting the water from their inside.  



\item[\docfilehook{Colligative properties review}{}] 
As you can see from the equations previously presented here, the vapor-pressure lowering, the freezing-point depression, the boiling-point elevation, and the osmotic pressure are all controlled by the concentration of the solute particles, in terms of molality, molarity, or mole fraction and they are unaffected by the nature of the solute. A such, a 1m \ce{NaCl} solutions will experience the same boiling-point increase than a 1m \ce{KCl} solution, even if the solute is different. Colligative properties are associated with the nature of the solvent and the concentration of solute but not the nature of the solute. Furthermore, these formulas work well for very diluted solutions. We call these solutions, in general, ideal solutions.



\item[\docfilehook{Graphical method to calculate colligative constants}{}] 
The formulas for the different colligative properties represent linear relationships. Therefore by means of plotting, we can calculate some of the colligative constants. The next example explains how to obtain colligative constants by means of a graphical method. In order to obtain any colligative constants, the plot involved in the calculation should represent a good quality linear trend and the statistic tool used to assess the goodness of a linear plot is called linear correlation or linear regression. Linear regression uses a linear correlation coefficient, $r^2$, to assess the quality of the fit.
Good linear plots, in general, are characterized by a values of linear correlation coefficient between 0.99 and 1. Differently, $r^2$ values lower than 0.99 in general do not result from a truly linear relationship. No accurate colligative constants shoould be calculated from a set of data characterized by poor linear correlation coefficient. Is important to keep in mind:
\begin{equation}\begin{split}
r^2\leq0.99 \quad \quad (\text{Good linear regression})\label{\chapterlabel:equation16}\\
r^2<0.99 \quad \quad (\text{Poor linear regression}) 
\end{split}\end{equation}
You can fit a linear regression to a set of experimental data either by means of a graphic calculator or using specialized \href{http://www.alcula.com/calculators/statistics/linear-regression/#gsc.tab=0}{internet websites}.
\begin{example} %%%%%%%%%%%%%%%%%%%%%%%% EXAMPLE BOX
The following two sets of data report the change in boiling point of a solution. Assess the date to calculate the boiling elevation constant, and if possible, calculate colligative constant and the boiling point of the pure solvent. 
\begin{center}\begin{tabular}[t]{  c c c c   }
\toprule
\multicolumn{2}{c}{Set A}&\multicolumn{2}{c}{Set B}\\
\toprule
  T($^{\circ}$C)	&m&T($^{\circ}$C)	&m\\
\midrule
 82.81	&1&	82.89	&1\\
 85.02	&2&	84.4	&2\\
 88.03	&3&	88.5	&3\\
 90.99	&4&	89.81	&4\\
 93.25	&5&	94.5	&5\\
\bottomrule
\end{tabular}\end{center}
 \textlcsc{ \textcolor{dgreen}{\Large \textbf{Solution}} }\\
 In order to ensure that the data is good enough to calculate the boiling elevation constant we need to assess the quality of the linear regression. By means of two regression analysis displayed below, we have that for Data set A, $r^2$ is larger than 0.99 and hence this data set fits well a linear regression. On the other hand, the Data set B is not good enough in order to calculate the boiling point elevation constant as  $r^2$ is lower than  0.99. In another words, the Data set B does not represent a real linear relationship between boiling temperature and molality. Therefore, we will use Data set A in order to calculate the boiling elevation constant. From the slope of the line we will calculate $k_b$=2.69$^{\circ}$C/m and from the intersect we will calculate the boiling point of the pure solvent $T^{solvent}_b=79.96^{\circ}$C.
 \pgfplotstableread{    
 y1 			x1	 y2 		x2
 82.81		1 	82.89	 1 
  85.02		2 	84.4	 	2
  88.03		3 	88.5	 	3 
  90.99		4 	89.81	 4
  93.25		5 	94.5	 	5 	 
}\tableLabel
\MakeRegression{x1}{y1}{reg1}{\SlA}{\IntA}{\RsqA}
\MakeRegression{x2}{y2}{reg2}{\SlB}{\IntB}{\RsqB}
\begin{tikzpicture}
\begin{groupplot}[height=5cm,width=6.4cm,
  group style={
    group size=2 by 1,
  }
]
\nextgroupplot[title=SetA, legend style={at={(0.9,1.4)} }]
\addplot +[mark=none] table [x=x1, y=reg1] {\tableLabel};
\addplot +[only marks] table [x=x1, y=y1]  {\tableLabel};   
\addlegendentry{\LegendEntry{\SlA}{\IntA}{\RsqA}}
%\addlegendentry{Set A}
\nextgroupplot[title=SetB, legend style={at={(0.99,1.4)} }]
\addplot +[mark=none] table [x=x2, y=reg2] {\tableLabel};
\addplot +[only marks] table [x=x2, y=y2]  {\tableLabel};   
\addlegendentry{\LegendEntry{\SlB}{\IntB}{\RsqB}}
%\addlegendentry{Set B}
\end{groupplot}
\node[anchor=north]  at (5,0) {molality, m};
\node[anchor=north, rotate=90]  at (-1,2) {T, $^{\circ}$C};

\end{tikzpicture}
\faDiamond\ \textlcsc{ \textcolor{dgreen}{\Large \textbf{Study Check}} }\\
The following two sets of data report the change in boiling point of a solution. Assess the date to calculate the boiling elevation constant, and if possible, calculate colligative constant and the boiling point of the pure solvent. 
\begin{center}\begin{tabular}[t]{  c c c c   }
\toprule
\multicolumn{2}{c}{Set A}&\multicolumn{2}{c}{Set B}\\
\toprule
  T($^{\circ}$C)	&m&T($^{\circ}$C)	&m\\
\midrule
 17.02	&1&	817.21	&1\\
 15.21	&2&	14.42	&2\\
  11.10	&3&	11.63	&3\\
 7.65	&4&	8.84	&4\\
 6.05	&5&	6.05		&5\\
\bottomrule
\end{tabular}\end{center}
%\pgfplotstableread{    
% y1 			x1	 y2 		x2
% 17.02		1 	17.21		 1 
%  15.21		2 	14.42	 	2
% 11.10		3 	11.63	 	3 
% 7.65			4 	8.84	 		4
%  6.05		5 	6.05	 		5 	 
%}\tableLabel
%\MakeRegression{x1}{y1}{reg1}{\SlA}{\IntA}{\RsqA}
%\MakeRegression{x2}{y2}{reg2}{\SlB}{\IntB}{\RsqB}
%\begin{tikzpicture}
%\begin{groupplot}[height=5cm,width=6.4cm,
%  group style={
%    group size=2 by 1,
%  }
%]
%\nextgroupplot[title=SetA, legend style={at={(0.9,1.4)} }]
%\addplot +[mark=none] table [x=x1, y=reg1] {\tableLabel};
%\addplot +[only marks] table [x=x1, y=y1]  {\tableLabel};   
%\addlegendentry{\LegendEntry{\SlA}{\IntA}{\RsqA}}
%%\addlegendentry{Set A}
%\nextgroupplot[title=SetB, legend style={at={(0.99,1.4)} }]
%\addplot +[mark=none] table [x=x2, y=reg2] {\tableLabel};
%\addplot +[only marks] table [x=x2, y=y2]  {\tableLabel};   
%\addlegendentry{\LegendEntry{\SlB}{\IntB}{\RsqB}}
%%\addlegendentry{Set B}
%\end{groupplot}
%\node[anchor=north]  at (5,0) {molality, m};
%\node[anchor=north, rotate=90]  at (-1,2) {T, $^{\circ}$C};
%
%\end{tikzpicture}
\begin{flushright} Answer: Data set B, $k_f=2.79^{\circ}$C, $T^{solvent}_f=20^{\circ}$C\end{flushright}
\end{example}%%%%%%%%%%%%%%%%%%%%%%%% EXAMPLE BOX


\item[\docfilehook{Use of colligative properties to calculate $\alpha$}{}] 
Colligative measurements can be used to calculate the percent dissociation ($\alpha$) of an electrolyte. The idea is to compare the nominal concentration based on the preparation of the electrolyte solution and the effective concentration of ions and molecules in the solution obtained using the colligative measurements. Remember these measurements directly depend on the total number of particles in the solution: ions and molecules.  However, in a weak electrolyte, the nominal concentration used to prepare the solution is not the same as the effective concentration of ions and molecules. This is because the concentration of ions and molecules in solution as weak electrolytes break down based on the degree of dissociation and, at the same time, the number of ions produced depends on the Van't Hoff factor $i$. The following equation gives the effective concentration of solute particles in the solution for a weak electrolyte in terms of the Van't Hoff factor and the nominal concentration:
\[ c^{\text{effective}}=c^{\text{Nominal}}\cdot (1+ (i-1)\frac{\alpha}{100}) \]
For example, imagine we prepare a 0.1M ($c^{\text{Nominal}}$) solution of a weak electrolyte with a 98\% degree of dissociation ($\alpha$) and the electrolyte dissociates into two different ions ($i$=2). The number of solute particles in the solution, that is the effective concentration of particles, will be 0.198M. As you can see this number is larger than 0.1M as it accounts not only for molecules (0.002M) but also for ions (0.196M).
The following example demonstrates how to compute electrolyte's degree of dissociation using osmotic pressure measurements.
  \begin{example} %%%%%%%%%%%%%%%%%%%%%%%% EXAMPLE BOX
A solution of 0.07M HF--a weak electrolyte--has a osmotic pressure of 2atm at 298K. Calculate:
\begin{inparaenum}[(a)]	
\item The nominal solute concentration.
\item The effective ion concentration.
\item  The percent dissociation of the acid.
\end{inparaenum}\\
 \textlcsc{ \textcolor{dgreen}{\Large \textbf{Solution}} }\\
(a) The nominal solute concentration is 0.07M.
(b) The osmotic pressure depends on molarity and temperature. We know the temperature, R is 0.082atmL/Kmol, and the vale of the osmotic pressure. With this information we can calculate the effective concentration of particles in solution:
\[i\cdot M=\frac{\pi}{RT}=\frac{2}{0.082\cdot 298}=0.08M\]
This value accounts for the overall concentration of solute particles in the form of molecules and ions.
(c)  The osmotic pressure is related to the effective molarity and given that we know the nominal solute concentration, we can calculate the degree of dissociation of the acid:
\[c^{\text{effective}}=c^{\text{Nominal}}\cdot (1+ (i-1)\frac{\alpha}{100})\]
Plugging the given values we have:
\[0.08=0.07\cdot (1+ (2-1)\frac{\alpha}{100}) \]
We can solve for $\alpha$:
\[ \alpha=100\cdot (0.08/0.07-1)=14.28\% \]
This results indicated that 14\% of the electrolyte dissociates  forming \ce{H^+ + F-}, whereas 86\% stays in the form of \ce{HF}.\\
\faDiamond\ \textlcsc{ \textcolor{dgreen}{\Large \textbf{Study Check}} }\\
We prepare a 0.1M solution of a weak electrolyte with $i$=3. Given that the degree of dissociation of the electrolyte is 95\%, calculate the osmotic pressure of the solution at 298K.
\\
\begin{flushright} Answer: 7.23atm\end{flushright}
\end{example}%%%%%%%%%%%%%%%%%%%%%%%% EXAMPLE BOX
  


\item[\docfilehook{Use of colligative properties to calculate $MW$}{}] 
Colligative properties measurements are also useful in order to calculate molar masses of solutes as most of these properties are related to concentration and concentration is related to the molar mass of the solute. The method is based on carrying any colligative measurement, for example, the boiling point elevation, the freezing point depression, or the osmotic pressure, and with these measurements computing the corresponding concentration, molarity of molarity. Once we have the concentration we can use Equation \ref{\chapterlabel:equation17} in order to compute the molar mass of the solute. We will work on an example:
\begin{example} %%%%%%%%%%%%%%%%%%%%%%%% EXAMPLE BOX
We prepare a solution by adding 1g of solute--a non-electrolyte--into 100mL of water. The solution experience a freezing point depression of 2.1$^{\circ}$C. Given than the freezing depression constant of water is 0.512$^{\circ}$C/m, calculate the molar mass of the solute.
\\
 \textlcsc{ \textcolor{dgreen}{\Large \textbf{Solution}} }\\
We will first calculate the molality of the solution by means of the freezing point depression. 
\[\Delta T_f =-i\cdot k_f\cdot m\]
We will use the freezing point depression ($\Delta T_f $=-2.1$^{\circ}$C), Van't Hoff's factor ($i$=1) and the freezing point depression constant ($k_f$=0.512$^{\circ}$C/m):
\[-2.1 =-1\cdot 0.512\cdot m\]
we have that the molality of the solution is $m$=4.10m. Now that we have the molality we will calculate the molar mass of the solute using the information provided regarding the solution preparation: mass of solute ($m_{solute}$=1g) and volume of water ($m_{solvent}$=0.1kg), taking into account that the density of water is 1g/mL. We have that the molar mass of the solute is:
\[MW=\frac{ \text{g of solute}   }{\text{m}\cdot \text{kg of solvent}   	}=\frac{1}{4.10\cdot 0.1}=5.12g/mol\]
\\
\faDiamond\ \textlcsc{ \textcolor{dgreen}{\Large \textbf{Study Check}} }\\
We prepare a solution by adding 5g of of solute--a non-electrolyte--until filling 50mL of solution. The solution experience a boiling point elevation of 5.3$^{\circ}$C. Given the boiling elevation constant of water, 1.86$^{\circ}$C/m, calculate the molar mass of the solute.
\\
\begin{flushright} Answer: 35.09g/mol\end{flushright}
\end{example}%%%%%%%%%%%%%%%%%%%%%%%% EXAMPLE BOX
    
    
    
    
\end{description}




\section{Factors affecting the solubility of solids and gases}
In this section, we will address the impact of temperature on the solubility of solids and gases on liquids and the impact of pressure on the solubility of gases on liquids.
\sloppy 
\begin{description}
\item[\docfilehook{Impact of temperature on the solubility }{}] The solubility of a solute on a solvent is the maximum amount of solute that one can dissolve in a solvent at a given temperature. The temperature has a strong impact on solubility. However, the nature of the impact depends on the nature and state of matter of the solute.
Normally, the solubility of solids increases with temperature. That means the higher temperature the more solute one can dissolve. However, this trend is not true for all solid solutes. Solid solutes like \ce{Ce2(SO4)3} or \ce{Li2SO4} follow the opposite trend the higher temperature the lower solubility. 
The solubility of gas solutes normally decreases with temperature, that is, the higher temperature the less amount of gas will be dissolved in a liquid. If you warm up a can of soda it goes flat as the gas comes out of the liquid--we call this desorption. 
\item[\docfilehook{Impact of pressure on the solubility of gases on liquids}{}] Pressure does not impact the solubility of solids or liquids in a liquid solvent. However, gas pressure impacts the solubility of a gas solute on a liquid. In particular, the higher pressure the higher solubility. The law that related gas solubility with pressure is called Henry's law:
\begin{equation}
\boxed{ s=k\cdot P 	}
\label{\chapterlabel:equation14}
\end{equation}
where:
\begin{where}
 \item $s$  is the solubility of a gas in a liquid solvent in M units
 \item $k$  is Henry's law constant with units of M/atm
  \item $P $ is the gas pressure in atm
\end{where}
An every-day life application of Henry's law is the production of carbonated beverages. These beverages are produced in contact with a liquid with an atmosphere of \ce{CO2} of typical pressures of 5 atm. When you open a can container, the \ce{CO2} in the drink comes out of the liquid in the form of bubbles to equilibrate the high gas pressure on the liquid and the low gas pressure in the atmosphere.
\end{description}
\begin{example} %%%%%%%%%%%%%%%%%%%%%%%% EXAMPLE BOX
For the dissolution of \ce{CO2} in water, Henry's constant is $3.4\times 10^{-2}$ M/atm. Calculate:
\begin{inparaenum}[(a)]	
\item the \ce{CO2} pressure needed to achieve a gas concentration of 0.04M.
\item  If we open a carbonated can until it goes flat and given that the partial pressure of \ce{CO2} in the air is $4\times 10^{-4}$  atm at 25$^{\circ}$C, calculate the final gas concentration in the drink.
\end{inparaenum}\\
 \textlcsc{ \textcolor{dgreen}{\Large \textbf{Solution}} }\\
(a) We will apply Henry's law given that we know solubility and Henry's constant:
  \[	 s=k\cdot P \quad \quad  0.04=3.4\times 10^{-2}\cdot P \]
  Solving for the pressure we have: 1.17atm.
(b)  We will also apply Henry's law, this time knowing the partial pressure of \ce{CO2} and again Henry's constant:
  \[	 s=k\cdot P \quad \quad  s=3.4\times 10^{-2}\cdot 4\times 10^{-4} \]
The solubility is $1.36\times 10^{-5}$M.\\
\faDiamond\ \textlcsc{ \textcolor{dgreen}{\Large \textbf{Study Check}} }\\
Calculate the solubility of nitrogen in water after exposing water to a 5 atm nitrogen pressure, given that Henry's constant is $6.1\times 10^{-4}$M/atm. 
\\
\begin{flushright} Answer: $3.05\times 10^{-3}$M\end{flushright}
\end{example}%%%%%%%%%%%%%%%%%%%%%%%% EXAMPLE BOX
\end{document}