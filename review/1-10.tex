\documentclass[main.tex]{subfiles}
\setlength{\columnsep}{30pt}
\begin{document}
\pagestyle{style2}
\newgeometry{left=0.8in,right=2.8in, top=3.5cm,bottom=2cm}
\setlength{\parskip}{0.5em}
\addcontentsline{toc}{chapter}{Review: Sample Test}
\begin{fullwidth}
\begin{multicols*}{2}\begin{enumerate}  \setlength\itemsep{0.2em}





\item Which of the following measurements has three significant figures? 
\begin{enumerate}[label=(\alph*)]\vspace{-0.5cm}
\begin{multicols*}{3}
\item 0.005 m		
\item  510 m
\item  0.510 m		
\item  0.051 m
\item  5100 m
\end{multicols*}\end{enumerate}\vspace{-0.5cm}

\item Which of the following numbers contains the designated CORRECT number of significant figures? 
\begin{enumerate}[label=(\alph*)]\vspace{-0.5cm}
\begin{multicols*}{3}
\item  0.04300	(5 SF)
\item  0.00302	(2 SF)
\item  156 000	(3 SF)
\item  1.04		(2 SF)
\item  3.0650		(4 SF)
\end{multicols*}\end{enumerate}\vspace{-0.5cm}

\item A nugget of silver with a mass of 200 g is added to 150.0 mL of water.  The water level rises to a volume of 170 mL.  What is the density of the silver in g/mL? 
\begin{enumerate}[label=(\alph*)]\vspace{-0.5cm}
\begin{multicols*}{3}
\item  10 		
\item  6.77  
\item  1.00  		
\item  0.0518  
\item  19.3  
\end{multicols*}\end{enumerate}\vspace{-0.5cm}

\item C is the symbol for 
\begin{enumerate}[label=(\alph*)]\vspace{-0.5cm}
\begin{multicols*}{3}
\item  calcium.		
\item  carbon.
\item  cobalt.		
\item  copper.
\item  cadmium.
\end{multicols*}\end{enumerate}\vspace{-0.5cm}

\item Which of the following is a characteristic of nonmetals?
\begin{enumerate}[label=(\alph*)] 
\item  shiny
\item  malleable
\item  good conductors of heat
\item  low melting points
\item  good conductors of electricity
\end{enumerate} 

\item What is the symbol of the element in Period 3 and Group 13?
\begin{enumerate}[label=(\alph*)]\vspace{-0.5cm}
\begin{multicols*}{3}
\item  Al		
\item  Mg
\item  Ca		
\item  C
\item  Si
\end{multicols*}\end{enumerate}\vspace{-0.5cm}

\item What is the formula for aluminum (III) nitrate?
\begin{enumerate}[label=(\alph*)]\vspace{-0.5cm}
\begin{multicols*}{3}
\item  \ce{Al2NO2} 		
\item  \ce{Al(NO3)3}
\item  \ce{Al(NO2)3} 		
\item  \ce{Al2(NO3)3} 
\item  \ce{Al2(NO2)2 }
\end{multicols*}\end{enumerate}\vspace{-0.5cm}

\item \ce{Ni2(SO4)3} is called
\begin{enumerate}[label=(\alph*)] 
\item  nickel sulfate.
\item  nickel (III) sulfate.
\item  iron (III) sulfate.
\item  dinickel trisulfate.
\item  nickel trisulfate.
\end{enumerate} 


\item Which of the following correctly gives the best coefficients for the reaction below?
	\begin{center}\ce{N2H4  +  H2O2  ->  N2  +  H2O}\end{center}
\begin{enumerate}[label=(\alph*)]\vspace{-0.5cm}
\begin{multicols*}{3}
\item  1, 1, 1 ,1		
\item  1, 2, 1, 4
\item  2, 4, 2, 8		
\item  1, 4, 1, 4
\item  2, 4, 2, 4
\end{multicols*}\end{enumerate}\vspace{-0.5cm}

\item How many moles of iron are present in $3.15 \times 10^{24}$ atoms of iron?
\begin{enumerate}[label=(\alph*)]\vspace{-0.5cm}
\begin{multicols*}{3}
\item  5.23  		
\item  1.90  
\item  292  		
\item  0.523  
\item  $1.90 \times 10^{48}$    
\end{multicols*}\end{enumerate}\vspace{-0.5cm}

\item Calculate the molar mass of potassium chloride, \ce{KCl}.
\begin{enumerate}[label=(\alph*)]\vspace{-0.5cm}
\begin{multicols*}{3}
\item  74.6 g		
\item  54.5 g
\item  6.74 g		
\item  67.4  g
\item  19.0 g
\end{multicols*}\end{enumerate}\vspace{-0.5cm}

\item For the question(s) that follow, consider the following balanced equation.
	\begin{center}\ce{Mg3N2(s) +  6H2O(l)  ->  3Mg(OH)2(s) +  2NH3(g)}\end{center}
How many grams of H2O are needed to produce 150 g of Mg(OH)2? (MW=58g/mol)
\begin{enumerate}[label=(\alph*)]\vspace{-0.5cm}
\begin{multicols*}{3}
\item  46 g		
\item  18 g
\item  130 g		
\item  93 g
\item  23 g   
\end{multicols*}\end{enumerate}\vspace{-0.5cm}

\item When 3 moles of CH4 are mixed with 1 moles of O2 the limiting reactant is  
\begin{center}\ce{CH4  +  2O2  ->  CO2  +  2H2O}\end{center}
\begin{enumerate}[label=(\alph*)]\vspace{-0.5cm}
\begin{multicols*}{3}
\item  \ce{CH4}		
\item   \ce{O2}
\item  \ce{CO2}		
\item  \ce{H2O}
\item  None 
\end{multicols*}\end{enumerate}\vspace{-0.5cm}

\item Consider the following equation. 
	\begin{center}\ce{2Mg  +  O2  -> 2MgO}\end{center}	
Calculate the yield if 6 moles of Mg produce 3.5 moles of MgO
\begin{enumerate}[label=(\alph*)]\vspace{-0.5cm}
\begin{multicols*}{3}
\item  100\%		
\item  150\%
\item  58\%		
\item  25\%
\item  75\%
\end{multicols*}\end{enumerate}\vspace{-0.5cm}

\item A compound contains, by mass, 40.0\% carbon, 6.71\% hydrogen, and 53.3\% oxygen. A 0.320 mole sample of this compound weighs 28.8 g. The molecular formula of this compound is:
\begin{enumerate}[label=(\alph*)]\vspace{-0.5cm}
\begin{multicols*}{3}
\item  \ce{C2H4O2}		
\item  \ce{C3H6O3}
\item  \ce{C2H4O}		
\item  \ce{CH2O}
\item  \ce{C4H7O2}
\end{multicols*}\end{enumerate}\vspace{-0.5cm}

\item What is the concentration, in mass percent, of a solution prepared from 50.0 g \ce{NaCl} and 150.0 g of water? 
\begin{enumerate}[label=(\alph*)]\vspace{-0.5cm}
\begin{multicols*}{3}
\item  0.250\%		
\item  33.3\%
\item  40.0\%		
\item  25.0\%
\item  3.00\%
\end{multicols*}\end{enumerate}\vspace{-0.5cm}

\item What volume of a 1.5 M \ce{KOH} solution is needed to provide 3.0 moles of \ce{KOH}?
\begin{enumerate}[label=(\alph*)]\vspace{-0.5cm}
\begin{multicols*}{3}
\item  3.0 L		
\item  0.50 L
\item  2.0 L		
\item  4.5 L
\item  0.22 L
\end{multicols*}\end{enumerate}\vspace{-0.5cm}

\item All of the following have a $\Delta H^o_f$ value of zero at 25$^{\circ}$C and 1.0 atm except:
\begin{enumerate}[label=(\alph*)]\vspace{-0.5cm}
\begin{multicols*}{3}
\item  \ce{N2(g)}		
\item  \ce{Ne(g)}
\item  \ce{FeO(s)}		
\item  \ce{Fe(s)}
\item  \ce{Hg(l)}
\end{multicols*}\end{enumerate}\vspace{-0.5cm}

\item If $\Delta H^o_f$ for a reaction is positive, ?
\begin{enumerate}[label=(\alph*)]
\item  the reaction rate is generally very fast.
\item  $H^o$(products) is smaller than $H^o$(reactants).
\item  the reaction rate is generally very slow.
\item  the process is endothermic
\item  $H^o$(reactants) is bigger than $H^o$(products). 
\end{enumerate}

\item Calculate $\Delta H^o_f$ for the reaction given the following information:
	\begin{center}\ce{N2(g) + 3H2(g) -> 2NH3 (g)}\end{center}	
$\Delta H^o_f$(\ce{NH3(g)})= -46KJ/mol
\begin{enumerate}[label=(\alph*)]\vspace{-0.5cm}
\begin{multicols*}{3}
\item  +100KJ		
\item  -92KJ 
\item  -920KJ		
\item  -120KJ	
\item  10KJ
\end{multicols*}\end{enumerate}\vspace{-0.5cm}







\item How many valence electrons are in the electron-dot structure of \ce{H2O}?
\begin{enumerate}[label=(\alph*)]\vspace{-0.5cm}
\begin{multicols*}{3}
\item  2			
\item  4
\item  6			
\item  8
\item  10
\end{multicols*}\end{enumerate}\vspace{-0.5cm}

\newcounter{enumTempD}
    \setcounter{enumTempD}{\theenumi}
\end{enumerate}
\end{multicols*}
\end{fullwidth}
\clearpage
\newpage
\thispagestyle{empty}
\newgeometry{left=0.8in,right=2.8in, top=3.5cm,bottom=2cm}
\begin{fullwidth}
\begin{multicols}{2}\begin{enumerate}[resume]  \setlength\itemsep{0.2em}
    \setcounter{enumi}{\theenumTempD}





\item Which of the following substances contains a polar covalent bond?
\begin{enumerate}[label=(\alph*)]\vspace{-0.5cm}
\begin{multicols*}{3}
\item  \ce{H2}			
\item  \ce{CH4}
\item  \ce{NaCl}		
\item  \ce{MgF2}
\item  \ce{N2}
\end{multicols*}\end{enumerate}\vspace{-0.5cm}

\item Which one of the following compounds will NOT be soluble in water?
\begin{enumerate}[label=(\alph*)]\vspace{-0.5cm}
\begin{multicols*}{3}
\item  \ce{NaOH}		
\item   \ce{PbS}
\item   \ce{K2SO4	}	
\item   \ce{LiNO3}
\item   \ce{MgCl2}
\end{multicols*}\end{enumerate}\vspace{-0.5cm}

\item What is the hybridization of the  \ce{S} atom in  \ce{SF4}?
\begin{enumerate}[label=(\alph*)]\vspace{-0.5cm}
\begin{multicols*}{3}
\item  $sp^2$			
\item  $sp^3$
\item  $sp^3d$		
\item  $sp^3d^2$
\item  $spd$
\end{multicols*}\end{enumerate}\vspace{-0.5cm}

\item What is the hybridization of the xenon atom in \ce{XeF4}?
\begin{enumerate}[label=(\alph*)]\vspace{-0.5cm}
\begin{multicols*}{3}
\item  $sp^2$			
\item  $sp^3$
\item  $sp^3d$		
\item  $sp^3d^2$
\item  $spd$
\end{multicols*}\end{enumerate}\vspace{-0.5cm}

\item The molecular orbital diagram for the carbide ion (\ce{C2^{2-}}) would show which of the following molecular orbitals?
\begin{enumerate}[label=(\alph*)] 
\item  $(\sigma_{2s})^2 (\sigma_{2s}*)^2 (\pi 2p)^4 (\sigma_{2p})^2 (\pi_{2p}*)^4 (\sigma_{2p}*)^2$	
\item  $(\sigma_{2s})^2 (\sigma_{2s}*)^2 (\pi_{2p})^2	$		
\item  $(\sigma_{2s})^2 (\sigma_{2s}*)^2 (\pi_{2p})^4 (\sigma_{2p})^2 (\pi_{2p}*)^4$ 
\item  $(\sigma_{2s})^2 (\sigma_{2s}*)^2 (\pi_{2p})^4 (\sigma_{2p})^2$	
\item  $(\sigma_{2s})^2 (\sigma_{2s}*)^2 (\pi_{2p})^4$	
\end{enumerate} 
		
\item When a double bond is formed between two atoms, one of the bonds is a sigma bond and the other is a pi bond. The $\pi$  bond is created by the overlap of...
\begin{enumerate}[label=(\alph*)]\vspace{-0.5cm}
\begin{multicols*}{3}
\item  $sp^2$ hybrid orbitals
\item  $sp^3$ hybrid orbitals
\item  $p$ orbitals
\item  $s$ orbitals
\item  $d$ orbitals
\end{multicols*}\end{enumerate}\vspace{-0.5cm}
		
\item For the N2 molecule, there are .... electrons in the $\sigma_{2p}$  bonding orbital?
\begin{enumerate}[label=(\alph*)]\vspace{-0.5cm}
\begin{multicols*}{3}
\item 	0.		
\item 	1.
\item 	2.		
\item 	3.
\item 	4.
\end{multicols*}\end{enumerate}\vspace{-0.5cm}
	
		
\item	Molecular orbital theory predicts the \ce{O2^-} ion (a minus one charge) has:
\begin{enumerate}[label=(\alph*)] 
\item 	no unpaired electrons.
\item 	one unpaired electron.
\item 	two unpaired electrons.
\item 	three unpaired electrons.
\item 	six unpaired electrons.
\end{enumerate} 
	
\item The \ce{N2} molecule is:
\begin{enumerate}[label=(\alph*)]\vspace{-0.5cm}
\begin{multicols*}{3}
\item 	paramagnetic.
\item 	diamagnetic.
\item 	submagnetic.
\item 	supermagnetic.
\item 	Superbowl magnetic.
\end{multicols*}\end{enumerate}\vspace{-0.5cm}
		
\item	Molecular orbital theory predicts the \ce{F2^{2+}} ion has a bond order of:
\begin{enumerate}[label=(\alph*)]\vspace{-0.5cm}
\begin{multicols*}{3}
\item 	0.0		
\item 	0.5
\item 	1.0		
\item 	1.5
\item 	2.0
\end{multicols*}\end{enumerate}\vspace{-0.5cm}
		
\item Draw the molecular orbital diagram for the molecular ion, \ce{N2^+}. The number of electrons in the $\sigma_{2p}$ molecular orbital is:
\begin{enumerate}[label=(\alph*)]\vspace{-0.5cm}
\begin{multicols*}{3}
\item  0			
\item  1
\item  2			
\item  3
\item  4
\end{multicols*}\end{enumerate}\vspace{-0.5cm}
		
\item What is the bond order in \ce{O2^+}?
\begin{enumerate}[label=(\alph*)]\vspace{-0.5cm}
\begin{multicols*}{3}
\item  3.5			
\item  2.0
\item  1.5			
\item  2.5
\item  0
\end{multicols*}\end{enumerate}\vspace{-0.5cm}

		
\item The number of unpaired electrons in the \ce{B2} molecule is...
\begin{enumerate}[label=(\alph*)]\vspace{-0.5cm}
\begin{multicols*}{3}
\item  zero		
\item  1
\item  2			
\item  3
\item  4
\end{multicols*}\end{enumerate}\vspace{-0.5cm}

		
\item Antibonding molecular orbitals are produced by
\begin{enumerate}[label=(\alph*)] 

\item  constructive interaction of atomic orbitals.
\item  destructive interaction of atomic orbitals.
\item  the overlap of the atomic orbitals of two negative ions
\item  all of these
\item  none of these
 \end{enumerate} 

		
\item The strongest interactions between molecules of ammonia (\ce{NH3}) are
\begin{enumerate}[label=(\alph*)]\vspace{-0.5cm}
\item  ionic bonds.
\item  H bonds.
\item  polar covalent.
\item  dipole-dipole.
\end{enumerate}\vspace{-0.5cm}

		
\item The strongest interactions between molecules of hydrogen (\ce{H2} ) are
\begin{enumerate}[label=(\alph*)]\vspace{-0.5cm}
\begin{multicols*}{3}
\item  ionic bonds.
\item  H bonds.
\item  polar covalent.
\item  dipole-dipole.
\item  dispersion forces.
\end{multicols*}\end{enumerate}\vspace{-0.5cm}
		
\item The strongest interactions between molecules of hydrogen chloride are
\begin{enumerate}[label=(\alph*)]\vspace{-0.5cm}
\begin{multicols*}{3}
\item  ionic bonds.
\item  covalent bonds.
\item  H bonds.
\item  dipole-dipole interactions.
\item  dispersion forces.
\end{multicols*}\end{enumerate}\vspace{-0.5cm}

\item Which of the following boils at the highest temperature?
\begin{enumerate}[label=(\alph*)]\vspace{-0.5cm}
\begin{multicols*}{3}
\item  \ce{CH4}			
\item  \ce{C2H6}	
\item  \ce{C3H8}			
\item  \ce{C4H10}	
\item  \ce{C5H12}	
\end{multicols*}\end{enumerate}\vspace{-0.5cm}

		






\item Which one of the following classifications is incorrect?
\begin{enumerate}[label=(\alph*)]
\item  \ce{H2O(s)}, molecular solid
\item  \ce{C4H10(s)}, molecular solid
\item  \ce{KF(s)}, ionic solid
\item  \ce{SiC(s)}, covalent solid
\item  \ce{S(s)}, metallic solid
 \end{enumerate} 





 \end{enumerate}
\end{multicols}
\end{fullwidth}
\begin{fullwidth}
\par\noindent\rule{0.5\textwidth}{0.4pt}\\
\emph{Answers:}\\
\vspace{-0.5cm}
\begin{tasks}[counter-format={tsk[1].}, label-align=left, label-offset={0mm}, label-width={5mm}, item-indent={1mm}, label-format={\bfseries}](8)
\task (c) 
\task (c) 
\task (a) 
\task (b) 
\task (d) 
\task (a) 
\task (b) 
\task (b) 
\task (b) 
\task (a) 
\task (a) 
\task (d) 
\task (b) 
\task (c) 
\task (b) 
\task (d) 
\task (c) 
\task (c) 
\task (d) 
\task (b) 
\task (d) 
\task (b) 
\task (b) 
\task (c) 
\task (d) 
\task (d) 
\task (c) 
\task (c) 
\task (b) 
\task (b) 
\task (e) 
\task (b) 
\task (b) 
\task (c) 
\task (b) 
\task (b) 
\task (e) 
\task (d) 
\task (e) 
\task (e) 


\end{tasks}






\end{fullwidth}
\restoregeometry
\end{document}