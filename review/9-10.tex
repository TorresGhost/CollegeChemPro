\documentclass[main.tex]{subfiles}
\setlength{\columnsep}{30pt}
\begin{document}
\pagestyle{style4}
\newgeometry{left=0.8in,right=2.8in, top=3.5cm,bottom=2cm}
\setlength{\parskip}{0.5em}
\addcontentsline{toc}{chapter}{Review-Quizz}
\begin{fullwidth}
\begin{multicols*}{2}\begin{enumerate}  \setlength\itemsep{0.2em}








\item How many valence electrons are in the electron-dot structure of \ce{H2O}?
\begin{enumerate}[label=(\alph*)]\vspace{-0.5cm}
\begin{multicols*}{3}
\item  2			
\item  4
\item  6			
\item  8
\item  10
\end{multicols*}\end{enumerate}\vspace{-0.5cm}



\item Which of the following substances contains a polar covalent bond?
\begin{enumerate}[label=(\alph*)]\vspace{-0.5cm}
\begin{multicols*}{3}
\item  \ce{H2}			
\item  \ce{CH4}
\item  \ce{NaCl}		
\item  \ce{MgF2}
\item  \ce{N2}
\end{multicols*}\end{enumerate}\vspace{-0.5cm}

\item Which one of the following compounds will NOT be soluble in water?
\begin{enumerate}[label=(\alph*)]\vspace{-0.5cm}
\begin{multicols*}{3}
\item  \ce{NaOH}		
\item   \ce{PbS}
\item   \ce{K2SO4	}	
\item   \ce{LiNO3}
\item   \ce{MgCl2}
\end{multicols*}\end{enumerate}\vspace{-0.5cm}

\item What is the hybridization of the  \ce{S} atom in  \ce{SF4}?
\begin{enumerate}[label=(\alph*)]\vspace{-0.5cm}
\begin{multicols*}{3}
\item  $sp^2$			
\item  $sp^3$
\item  $sp^3d$		
\item  $sp^3d^2$
\item  $spd$
\end{multicols*}\end{enumerate}\vspace{-0.5cm}

\item What is the hybridization of the xenon atom in \ce{XeF4}?
\begin{enumerate}[label=(\alph*)]\vspace{-0.5cm}
\begin{multicols*}{3}
\item  $sp^2$			
\item  $sp^3$
\item  $sp^3d$		
\item  $sp^3d^2$
\item  $spd$
\end{multicols*}\end{enumerate}\vspace{-0.5cm}

\item The molecular orbital diagram for the carbide ion (\ce{C2^{2-}}) would show which of the following molecular orbitals?
\begin{enumerate}[label=(\alph*)] 
\item  $(\sigma_{2s})^2 (\sigma_{2s}*)^2 (\pi_2p)^4 (\sigma_{2p})^2 (\pi_{2p}*)^4 (\sigma_{2p}*)^2$	
\item  $(\sigma_{2s})^2 (\sigma_{2s}*)^2 (\pi_{2p})^2	$		
\item  $(\sigma_{2s})^2 (\sigma_{2s}*)^2 (\pi_{2p})^4 (\sigma_{2p})^2 (\pi_{2p}*)^4$ 
\item  $(\sigma_{2s})^2 (\sigma_{2s}*)^2 (\pi_{2p})^4 (\sigma_{2p})^2$	
\item  $(\sigma_{2s})^2 (\sigma_{2s}*)^2 (\pi_{2p})^4$	
\end{enumerate} 
		
\item When a double bond is formed between two atoms, one of the bonds is a sigma bond and the other is a pi bond. The $\pi$  bond is created by the overlap of...
\begin{enumerate}[label=(\alph*)]
\item  $sp^2$ hybrid orbitals
\item  $sp^3$ hybrid orbitals
\item  $p$ orbitals
\item  $s$ orbitals
\item  $d$ orbitals
\end{enumerate}
		
\item For the N2 molecule, there are .... electrons in the $\sigma_{2p}$  bonding orbital?
\begin{enumerate}[label=(\alph*)]
\begin{multicols*}{2}
\item 	0.		
\item 	1.
\item 	2.		
\item 	3.
\item 	4.
\end{multicols*}\end{enumerate}\vspace{-0.5cm}
	
		
\item	Molecular orbital theory predicts the \ce{O2^-} ion (a minus one charge) has:
\begin{enumerate}[label=(\alph*)] 
\item 	no unpaired electrons.
\item 	one unpaired electron.
\item 	two unpaired electrons.
\item 	three unpaired electrons.
\item 	six unpaired electrons.
\end{enumerate} 
	
\item The \ce{N2} molecule is:
\begin{enumerate}[label=(\alph*)]\vspace{-0.5cm}
\begin{multicols*}{3}
\item 	paramagnetic.
\item 	diamagnetic.
\item 	submagnetic.
\item 	supermagnetic.
\item 	Superbowl magnetic.
\end{multicols*}\end{enumerate}\vspace{-0.5cm}
		
\item	Molecular orbital theory predicts the \ce{F2^{2+}} ion has a bond order of:
\begin{enumerate}[label=(\alph*)]\vspace{-0.5cm}
\begin{multicols*}{3}
\item 	0.0		
\item 	0.5
\item 	1.0		
\item 	1.5
\item 	2.0
\end{multicols*}\end{enumerate}\vspace{-0.5cm}
		
\item Draw the molecular orbital diagram for the molecular ion, \ce{N2^+}. The number of electrons in the $\sigma_{2p}$ molecular orbital is:
\begin{enumerate}[label=(\alph*)]\vspace{-0.5cm}
\begin{multicols*}{3}
\item  0			
\item  1
\item  2			
\item  3
\item  4
\end{multicols*}\end{enumerate}\vspace{-0.5cm}
		
\item What is the bond order in \ce{O2^+}?
\begin{enumerate}[label=(\alph*)]\vspace{-0.5cm}
\begin{multicols*}{3}
\item  3.5			
\item  2.0
\item  1.5			
\item  2.5
\item  0
\end{multicols*}\end{enumerate}\vspace{-0.5cm}

		
\item The number of unpaired electrons in the \ce{B2} molecule is...
\begin{enumerate}[label=(\alph*)]\vspace{-0.5cm}
\begin{multicols*}{3}
\item  zero		
\item  1
\item  2			
\item  3
\item  4
\end{multicols*}\end{enumerate}\vspace{-0.5cm}

		
\item Antibonding molecular orbitals are produced by
\begin{enumerate}[label=(\alph*)] 

\item  constructive interaction of atomic orbitals.
\item  destructive interaction of atomic orbitals.
\item  the overlap of the atomic orbitals of two negative ions
\item  all of these
\item  none of these
 \end{enumerate} 

		




\item Green light has a wavelength of $5.50\times 10^2$ nm.  The energy of a photon of green light is
\begin{enumerate}[label=(\alph*)]\vspace{-0.5cm}
\begin{multicols*}{2}
\item $3.64\times 10^{-34}$J    		
\item $2.17\times 10^{5}$J    
\item $3.61\times 10^{-19}$J    		
\item $1.09\times 10^{-27}$J    
\item $5.45\times 10^{12}$J   
\end{multicols*}\end{enumerate}\vspace{-0.5cm}


\item The number of electron levels in a magnesium atom is
\begin{enumerate}[label=(\alph*)]\vspace{-0.5cm}
\begin{multicols*}{2}
\item 1.			
\item 2.
\item 3.			
\item 4.
\item 5.
\end{multicols*}\end{enumerate}\vspace{-0.5cm}


\item The maximum number of electrons that may occupy the third energy level is
\begin{enumerate}[label=(\alph*)]\vspace{-0.5cm}
\begin{multicols*}{2}
\item 2.			
\item 8.
\item 10.			
\item 18.
\item 32.
\end{multicols*}\end{enumerate}\vspace{-0.5cm}

\item What is the element with the electron configuration $1s^22s^22p^63s^23p^5$?
\begin{enumerate}[label=(\alph*)]\vspace{-0.5cm}
\begin{multicols*}{2}
\item Be			
\item Cl
\item F			
\item S
\item Ar
\end{multicols*}\end{enumerate}\vspace{-0.5cm}

\item Valence electrons are electrons located
\begin{enumerate}[label=(\alph*)]\vspace{-0.5cm}
\item in the outermost energy level of an atom.
\item in the nucleus of an atom.
\item in the innermost energy level of an atom.
\item throughout the atom.
\item  in the first three shells of an atom.
\end{enumerate}\vspace{-0.5cm}

\item How many f orbitals have n=3?
\begin{enumerate}[label=(\alph*)]\vspace{-0.5cm}
\begin{multicols*}{2}
\item 0			
\item 3
\item 5			
\item 7
\item 1
\end{multicols*}\end{enumerate}\vspace{-0.5cm}

\item Which of the following electron configurations is impossible?
\begin{enumerate}[label=(\alph*)]\vspace{-0.5cm}
\begin{multicols*}{2}
\item $1s^22s^22p^63s^23p^1$	
\item $1s^22s^42p^63s^23p^3$
\item $1s^22s^22p^63s^23p^5$	
\item $1s^22s^22p^63s^23p^6$
\item $1s^22s^22p^63s^23p^3$
\end{multicols*}\end{enumerate}\vspace{-0.5cm}


\item The ionization energy of atoms
\begin{enumerate}[label=(\alph*)]\vspace{-0.5cm}
\item decreases going across a period.
\item decreases going down within a group.
\item increases going down within a group.
\item does not change going down within a group.
\item None of the above.
\end{enumerate}\vspace{-0.5cm}

\item Which of the following elements has the lowest electronegativity?
\begin{enumerate}[label=(\alph*)]\vspace{-0.5cm}
\begin{multicols*}{2}
\item Li				
\item C
\item N				
\item O
\item F
\end{multicols*}\end{enumerate}\vspace{-0.5cm}


 


\item Of the elements:  B, C, F, Li, and Na., the element with the largest atomic radius is
\begin{enumerate}[label=(\alph*)]\vspace{-0.5cm}
\begin{multicols*}{2}
\item B. 				
\item C.
\item F.				
\item Li.
\item Na.
\end{multicols*}\end{enumerate}\vspace{-0.5cm}

\item List the following atoms in order of increasing ionization energy:Li, Na, C, O, F.
\begin{enumerate}[label=(\alph*)]\vspace{-0.5cm}
\begin{multicols*}{2}
\item Li<Na<C<O<F		
\item Na<Li<C<O<F
\item F<O<C<Li<Na		
\item Na<Li<F<O<C
\item Na<Li<C<F<O
\end{multicols*}\end{enumerate}\vspace{-0.5cm}
%









 



\end{enumerate}\end{multicols*}

\end{fullwidth}
\clearpage
\newpage
\thispagestyle{empty}
\newgeometry{left=0.8in,right=2.8in, top=3.5cm,bottom=2cm}
\begin{fullwidth}






\vspace{2cm}\emph{Answers:}\\
\vspace{-0.5cm}
\begin{tasks}[counter-format={tsk[1].}, label-align=left, label-offset={0mm}, label-width={5mm}, item-indent={1mm}, label-format={\bfseries}](4)
\task (d) 
\task (b) 
\task (b) 
\task (c) 
\task (d) 
\task (d) 
\task (c) 
\task (c) 
\task (b) 
\task (b) 
\task (e) 
\task (b) 
\task (b) 
\task (c) 
\task (b)
 
 
 %
\task (c) 
\task (c)
\task (d)
\task (b)
\task (a)
\task (a)
\task (b)
\task (b)
\task (a)
\task (e)
\task (b)

\end{tasks}






\end{fullwidth}
\restoregeometry
\end{document}
