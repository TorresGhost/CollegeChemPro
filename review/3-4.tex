\documentclass[main.tex]{subfiles}
\setlength{\columnsep}{30pt}
\begin{document}
\pagestyle{style4}
\newgeometry{left=0.8in,right=2.8in, top=3.5cm,bottom=2cm}
\setlength{\parskip}{0.5em}
\addcontentsline{toc}{chapter}{Review-Quizz}
\begin{fullwidth}
\begin{multicols*}{2}\begin{enumerate}  \setlength\itemsep{0.2em}









\item The name of \ce{CO_3^{-2}} is:
\begin{enumerate}[label=(\alph*)]\vspace{-0.5cm}
\begin{multicols*}{2}
\item carbon trioxide
\item carbon oxide
\item carbonic 
\item carbonate
\item bicarbonate
\end{multicols*}\end{enumerate}\vspace{-0.5cm}



\item Which of the following chemicals is covalent:
\begin{enumerate}[label=(\alph*)]\vspace{-0.5cm}
\begin{multicols*}{2}
\item \ce{CO}
\item \ce{NO}
\item \ce{NO2}
\item \ce{H2O}
\item all above
\end{multicols*}\end{enumerate}\vspace{-0.5cm}

\item Name the following chemical: \ce{NO2}
\begin{enumerate}[label=(\alph*)]\vspace{-0.5cm}
\begin{multicols*}{2}
\item Nitrogen oxide
\item Nitrogen oxigen
\item Nitrate
\item Nitrite
\item Nitrogen dioxide
\end{multicols*}\end{enumerate}\vspace{-0.5cm}

%\item The bond in \ce{Cl2} is a(n)
%\begin{enumerate}[label=(\alph*)]\vspace{-0.5cm}
%\begin{multicols*}{2}
%\item ionic bond.
%\item nonpolar covalent bond.
%\item metallic bond.
%\item polar ionic bond.
%\item no bond.
%\end{multicols*}\end{enumerate}\vspace{-0.5cm}
%
%\item Ionic bonding is expected in which of these compounds?
%\begin{enumerate}[label=(\alph*)]\vspace{-0.5cm}
%\begin{multicols*}{2}
%\item \ce{Cl2}
%\item \ce{KF}
%\item \ce{OF2}
%\item \ce{HF}
%\item \ce{H2}
%\end{multicols*}\end{enumerate}\vspace{-0.5cm}

\item The correct name for the compound \ce{N2O3} is
\begin{enumerate}[label=(\alph*)]\vspace{-0.5cm}
\begin{multicols*}{2}
\item nitrogen oxide.
\item nitrogen trioxide.
\item dinitride trioxide.
\item dinitrogen oxide.
\item dinitrogen trioxide.
\end{multicols*}\end{enumerate}\vspace{-0.5cm}

%\item The formula for a molecule formed from N and Cl using its valences
% would be
% \begin{enumerate}[label=(\alph*)]\vspace{-0.5cm}
%\begin{multicols*}{2}
%\item \ce{NCl}
%\item \ce{NCl2}
%\item \ce{NCl3}
%\item \ce{N3Cl}
%\item \ce{NCl5}
%\end{multicols*}\end{enumerate}\vspace{-0.5cm}

%\item Which of the following polyatomic ions has a 3- ionic charge?
% \begin{enumerate}[label=(\alph*)]\vspace{-0.5cm}
%\begin{multicols*}{2}
%\item hydroxide
%\item nitrate
%\item sulfate
%\item phosphate
%\item bicarbonate
%\end{multicols*}\end{enumerate}\vspace{-0.5cm}

\item What is the formula for aluminum nitrite?
  \begin{enumerate}[label=(\alph*)]\vspace{-0.5cm}
\begin{multicols*}{2}
\item \ce{Al2NO2} 
\item \ce{AlNO3}
\item \ce{Al(NO2)3} 
\item \ce{Al2(NO3)3} 
\item \ce{Al2(NO2)2} 
\end{multicols*}\end{enumerate}\vspace{-0.5cm}

\item \ce{Fe2(SO4)3} is called 
  \begin{enumerate}[label=(\alph*)]\vspace{-0.5cm}
\begin{multicols*}{2}
\item iron sulfate.
\item iron (II) sulfate.
\item iron (III) sulfate.
\item diiron trisulfate.
\item iron trisulfate.
\end{multicols*}\end{enumerate}\vspace{-0.5cm}

\item What is the formula of a compound that contains \ce{Na+} and \ce{PO4^{3-}} ions?
 \begin{enumerate}[label=(\alph*)]\vspace{-0.5cm}
\begin{multicols*}{2}
\item \ce{Na3PO4 }
\item \ce{NaPO4} 
\item \ce{Na2PO3 }
\item \ce{Na3PO3 }
\item \ce{Na3P}
\end{multicols*}\end{enumerate}\vspace{-0.5cm}

%\item The name of the \ce{HSO4-} ion is
% \begin{enumerate}[label=(\alph*)]\vspace{-0.5cm}
%\begin{multicols*}{2}
%\item sulfate.
%\item hydrogen sulfate.
%\item sulfite.
%\item hydrogen sulfite.
%\item sulfide.
%\end{multicols*}\end{enumerate}\vspace{-0.5cm}

%\item One mole of helium gas weighs
% \begin{enumerate}[label=(\alph*)]\vspace{-0.5cm}
%\begin{multicols*}{2}
%\item 1.00 g.
%\item 2.00 g.
%\item 3.00 g.
%\item 4.00 g.
%\item 8.00 g.
%\end{multicols*}\end{enumerate}\vspace{-0.5cm}

\item Calculate the molar mass of magnesium chloride, \ce{MgCl2}.
 \begin{enumerate}[label=(\alph*)]\vspace{-0.5cm}
\begin{multicols*}{2}
\item 24.3 g
\item 95.2 g
\item 125.9 g
\item 59.8 g
\item 70.0 g
\end{multicols*}\end{enumerate}\vspace{-0.5cm}

\item What is the molar mass of \ce{Mg3(PO4)2}, a substance formerly used in medicine as an antacid?
 \begin{enumerate}[label=(\alph*)]\vspace{-0.5cm}
\begin{multicols*}{2}
\item 71.3 g		
\item 118.3 g
\item 150.3 g		
\item 214.3 g
\item 262.9 g
\end{multicols*}\end{enumerate}\vspace{-0.5cm}

\item How many moles of carbon atoms are there in 0.500 mole of \ce{C2H6}?
 \begin{enumerate}[label=(\alph*)]\vspace{-0.5cm}
\begin{multicols*}{2}
\item 0.500 moles	
\item 1.00 moles
\item 3.00 moles		
\item $6.02 \times 10^{23}$   moles
\item 4.00 moles
\end{multicols*}\end{enumerate}\vspace{-0.5cm}

\item How many grams of \ce{Fe2O3} are there in 0.500 mole of \ce{Fe2O3}?
 \begin{enumerate}[label=(\alph*)]\vspace{-0.5cm}
\begin{multicols*}{2}
\item 79.8 g
\item 35.9 g
\item 63.8 g
\item 51.9 g
\item 160. g
\end{multicols*}\end{enumerate}\vspace{-0.5cm}


\item How many oxygen atoms are present in 75.0 g of \ce{H2O}?
 \begin{enumerate}[label=(\alph*)]\vspace{-0.5cm}
\begin{multicols*}{2}
\item 75.0 atoms
\item 4.17 atoms
\item $7.53 \times 10^{24}$ atoms
\item $2.51 \times 10^{24}$ atoms
\item $5.02 \times10^{24}$ atoms
\end{multicols*}\end{enumerate}\vspace{-0.5cm}

\item Which of the following correctly gives the best coefficients for the reaction below?
	\begin{center}\ce{N2H4  +  H2O2  ->  N2  +  H2O}\end{center}
\begin{enumerate}[label=(\alph*)]\vspace{-0.5cm}
\begin{multicols*}{3}
\item  1, 1, 1 ,1		
\item  1, 2, 1, 4
\item  2, 4, 2, 8		
\item  1, 4, 1, 4
\item  2, 4, 2, 4
\end{multicols*}\end{enumerate}\vspace{-0.5cm}

\item How many moles of iron are present in $3.15 \times 10^{24}$ atoms of iron?
\begin{enumerate}[label=(\alph*)]\vspace{-0.5cm}
\begin{multicols*}{3}
\item  5.23  		
\item  1.90  
\item  292  		
\item  0.523  
\item  $1.90 \times 10^{48}$    
\end{multicols*}\end{enumerate}\vspace{-0.5cm}

%\item An oxide of lead contains 90.65\% Pb, by weight. The empirical formula is:
% \begin{enumerate}[label=(\alph*)]\vspace{-0.5cm}
%\begin{multicols*}{2}
%\item \ce{Pb}			
%\item \ce{PbO}
%\item \ce{Pb3O4}		
%\item \ce{Pb2O3}
%\item \ce{PbO2}
%\end{multicols*}\end{enumerate}\vspace{-0.5cm}


%\item  Which of the following gives the balanced equation for this reaction?
%\begin{center}\ce{K3PO4  +  Ca(NO3)2  ->  Ca3(PO4)2  +  KNO3}\end{center}
% \begin{enumerate}[label=(\alph*)]
%\item  \ce{KPO4  +  CaNO3  +  KNO3}
%\item  \ce{K3PO4  +  Ca(NO3)2 ->  Ca3(PO4)2  +  3KNO3}
%\item  \ce{2K3PO4  +  Ca(NO3)2  ->  Ca3(PO4)2  +  6KNO3}
%\item  \ce{2K3PO4  +  3Ca(NO3)2  ->  Ca3(PO4)2   +  6KNO3}
%\item  \ce{K3PO4  +  Ca(NO3)2  -> Ca3(PO4)2  +  KNO3}
%\end{enumerate}

\item  What coefficient is placed in front of \ce{O2} to complete the balancing of the following equation?
\begin{center}	\ce{C5H8  +  x O2  ->  5CO2  +  4H2O}\end{center}
 \begin{enumerate}[label=(\alph*)]\vspace{-0.5cm}
\begin{multicols*}{2}\item  1			
\item  3
\item  5			
\item  7
\item  9
\end{multicols*}\end{enumerate}\vspace{-0.5cm}

\item  Consider the following equation. 
\begin{center}\ce{	2Mg  +  O2  ->  2MgO}\end{center}	
How many grams of \ce{MgO} are produced when 40.0 grams of \ce{O2} react completely with \ce{Mg}?
 \begin{enumerate}[label=(\alph*)]\vspace{-0.5cm}
\begin{multicols*}{2}\item  30.4 g 		
\item  50.4 g
\item  60.8 g		
\item  101 g
\item  201 g
\end{multicols*}\end{enumerate}\vspace{-0.5cm}

%\item  Find the mass of \ce{AlCl3} that is produced when 10.0 grams of \ce{Al2O3} react with 10.0 g of \ce{HCl} according to the following equation.
%\begin{center}\ce{Al2O3(s)  +  6HCl(aq)  ->  2AlCl3(aq) +  3H2O(aq)}\end{center}
% \begin{enumerate}[label=(\alph*)]\vspace{-0.5cm}
%\begin{multicols*}{2}\item  16.2 g		
%\item  20.0 g
%\item  12.2 g		
%\item  10.0 g
%\item  6.10 g
%\end{multicols*}\end{enumerate}\vspace{-0.5cm}

\item  How many grams of \ce{CO2} are produced from 125 g of \ce{O2} and excess \ce{CH4}?
 \begin{center}\ce{CH4  +  2O2  ->  CO2  +  2H2O }\end{center}
 \begin{enumerate}[label=(\alph*)]\vspace{-0.5cm}
\begin{multicols*}{2}\item  125 g of \ce{CO2}	
\item  62.5 g of \ce{CO2}
\item  172 g of \ce{CO2}	
\item  85.9 g of \ce{CO2}
\item  250. g of \ce{CO2}
\end{multicols*}\end{enumerate}\vspace{-0.5cm}

\item  When 3.05 moles of \ce{CH4} are mixed with 5.03 moles of \ce{O2} the limiting reactant is   
\begin{center}\ce{CH4  +  2O2  ->  CO2  +  2H2O}\end{center}
 \begin{enumerate}[label=(\alph*)]\vspace{-0.5cm}
\begin{multicols*}{2}\item  \ce{CH4}		
\item   \ce{O2}
\item  \ce{CO2}		
\item  \ce{H2O}
\item  None of the above
\end{multicols*}\end{enumerate}\vspace{-0.5cm}

\item  Consider the following equation. 
	\begin{center}\ce{2Mg  +  O2  ->  2MgO	}\end{center}
Calculate the yield if 3 moles of Mg produce 1.5 moles of \ce{MgO}
 \begin{enumerate}[label=(\alph*)]\vspace{-0.5cm}
\begin{multicols*}{2}\item  100\%		
\item  150\%
\item  50\%		
\item  25\%
\item  75\%
\end{multicols*}\end{enumerate}\vspace{-0.5cm}

%\item  Name or formulate the following chemicals:\\
%\begin{tabularx}{0.45\textwidth}{ >{\centering}m{.25\linewidth}  *{3}{Y} }
%  \toprule
%\heading{Formula} & \multicolumn{3}{c}{\textbf{Name}}   \\
%    \midrule
%\ce{SO3} & 	\multicolumn{3}{c}{     }    \\
%\ce{P2O5} & 	\multicolumn{3}{c}{     }    \\
%\ce{FeF2} & 	\multicolumn{3}{c}{     }    \\
%\ce{MnO4} & 	\multicolumn{3}{c}{     }    \\
%\ce{H2O2} & 	\multicolumn{3}{c}{     }    \\
%\ce{Mg(NO3)2} & 	\multicolumn{3}{c}{     }    \\
%\ce{Al2(SO4)3} & 	\multicolumn{3}{c}{     }    \\
%\ce{CoCO3} & 	\multicolumn{3}{c}{     }    \\
%\ce{NaHCO3} & 	\multicolumn{3}{c}{     }    \\
%\ce{KMnO4 \cdot 4 H2O} & 	\multicolumn{3}{c}{     }    \\
%\bottomrule
%\end{tabularx}
%
%\item  Name or formulate the following chemicals:\\
%\begin{tabularx}{0.45\textwidth}{
%    >{\centering}m{.25\linewidth} 
%    *{3}{Y} }
%  \toprule
%\heading{Name} & \multicolumn{3}{c}{\textbf{Formula}}   \\
%    \midrule
% Ammonia& 	\multicolumn{3}{c}{     }    \\
% Chlorine Dioxide& 	\multicolumn{3}{c}{     }    \\
%Cobalt (III) oxide & 	\multicolumn{3}{c}{     }    \\
% Gallium (I) iodide& 	\multicolumn{3}{c}{     }    \\
% Magnesium fluoride& 	\multicolumn{3}{c}{     }    \\
% Sodium permanganate& 	\multicolumn{3}{c}{     }    \\
% Calcium carbonate& 	\multicolumn{3}{c}{     }    \\
% Vanadium (III) sulfate& 	\multicolumn{3}{c}{     }    \\
% Calcium Hydrogencarbonate& 	\multicolumn{3}{c}{     }    \\
% Lithium sulfate tetrahydrate& 	\multicolumn{3}{c}{     }    \\
%\bottomrule
%\end{tabularx}







 


 



\end{enumerate}\end{multicols*}

\end{fullwidth}
\clearpage
\newpage
\thispagestyle{empty}
\newgeometry{left=0.8in,right=2.8in, top=3.5cm,bottom=2cm}
\begin{fullwidth}






\vspace{2cm}\emph{Answers:}\\
\vspace{-0.5cm}
\begin{tasks}[counter-format={tsk[1].}, label-align=left, label-offset={0mm}, label-width={5mm}, item-indent={1mm}, label-format={\bfseries}](4)
\task (d) 
\task (e) 
\task (e) 
%\task (b)
%\task (b) 
\task (e) 
%\task (c) 
%\task (d) 
\task (c) 
\task (c) 
\task (a) 
%\task (b) 
% \task (d)
 \task (b)
 \task (e)
 \task (b)
 \task (a)
 \task (d)
 
 \task (b)
 \task (a)
% \task (c)
% \task (d)
 \task (d)
 \task (b)
% \task (c)
 \task (d)
 \task (b)
 \task (c)
% \task \begin{tasks}[counter-format=itemize, label-align=left, label-offset={0mm}, label-width={5mm}, item-indent={1mm}, label-format={\bfseries}](4)
%  \task  sulphur trioxide (\ce{SO3})
%   \task diphosphorus pentoxide (\ce{P2O5})
%    \task iron(II) fluoride (\ce{FeF2})
%     \task manganese(VIII) oxide (\ce{MnO4})
%      \task dihydrogen dioxide (\ce{H2O2})
%       \task magnesium nitrate (\ce{Mg(NO3)2}) 
%       \task aluminum sulfate(\ce{Al2(SO4)3}) 
%        \task  cobalt(II) carbonate (\ce{CoCO3})
%         \task sodium hydrogen carbonate (\ce{NaHCO3})
%          \task potassium permanganate tetrahydrate(\ce{KMnO4?4H2O})
%\end{tasks}
\end{tasks}






\end{fullwidth}
\restoregeometry
\end{document}
