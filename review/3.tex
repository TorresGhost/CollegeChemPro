\documentclass[main.tex]{subfiles}
\setlength{\columnsep}{30pt}
\begin{document}
\pagestyle{style4}
\newgeometry{left=0.8in,right=2.8in, top=3.5cm,bottom=2cm}
\setlength{\parskip}{0.5em}
\addcontentsline{toc}{chapter}{Review: Sample Test}
\begin{fullwidth}
\begin{multicols*}{2}\begin{enumerate}  \setlength\itemsep{0.2em}

\item  Indicate the redox number of: \ce{H{\underline{N}}O3}
\begin{enumerate}[label=(\alph*)]\vspace{-0.5cm}
\begin{multicols*}{3}
\item $V$			
\item $I$
\item $III$			
\item $IV$
\item $II$
\end{multicols*}\end{enumerate}\vspace{-0.5cm}
\item  Indicate the redox number of: \ce{{\underline{N}}O2}
\begin{enumerate}[label=(\alph*)]\vspace{-0.5cm}
\begin{multicols*}{3}
\item $V$			
\item $I$
\item $III$			
\item $IV$
\item $II$
\end{multicols*}\end{enumerate}\vspace{-0.5cm}

\item  Indicate the redox number of: \ce{{\underline{N}}2H5^+}
\begin{enumerate}[label=(\alph*)]\vspace{-0.5cm}
\begin{multicols*}{3}
\item $V$			
\item $I$
\item $III$			
\item $IV$
\item $II$
\end{multicols*}\end{enumerate}\vspace{-0.5cm}

\item  Classify the following chemical as: \ce{H2SO4}
\begin{enumerate}[label=(\alph*)]
\item non electrolyte			
\item strong electrolyte
\item weak electrolyte			
\end{enumerate}

\item  Classify the following chemical as: \ce{CH3COOH}
\begin{enumerate}[label=(\alph*)]
\item non electrolyte			
\item strong electrolyte
\item weak electrolyte			
\end{enumerate}

\item  Classify the following chemical as: \ce{HF}
\begin{enumerate}[label=(\alph*)]
\item non electrolyte			
\item strong electrolyte
\item weak electrolyte			
\end{enumerate}


%\item  Which measurement describes the pressure of a gas?
%\begin{enumerate}[label=(\alph*)]\vspace{-0.5cm}
%\begin{multicols*}{2}
%\item 315 K
%\item 1.2 g/L
%\item 2.5 L
%\item 725 mmHg
%\item 0.45 mols
%\end{multicols*}\end{enumerate}\vspace{-0.5cm}


\item  The volume of a gas with a pressure of 1.2 atm increases from 1.0 L to 4.0 L. What is the final pressure of the gas, assuming constant temperature?
\begin{enumerate}[label=(\alph*)]\vspace{-0.5cm}
\begin{multicols*}{3}
\item 1.2 atm			
\item 0.30 atm
\item 3.3 atm			
\item 4.8 atm
\item 1.0 atm
\end{multicols*}\end{enumerate}\vspace{-0.5cm}

%\item  The atmospheric pressure in Denver, CO is 633 mmHg.  What is this pressure in atm?
%\begin{enumerate}[label=(\alph*)]\vspace{-0.5cm}
%\begin{multicols*}{2}
%\item 1.20 atm
%\item 633 atm
%\item 0.833 atm
%\item 1.00 atm
%\item 127 atm
%\end{multicols*}\end{enumerate}\vspace{-0.5cm}

%\item  The pressure of 5.0 L of gas increases from 1.50 atm to 1240 mmHg. What is the final volume of the gas, assuming constant temperature?
%\begin{enumerate}[label=(\alph*)]\vspace{-0.5cm}
%\begin{multicols*}{3}
%\item 4100 L			
%\item 5.0 L
%\item 0.0060 L			
%\item 5.4 L
%\item 4.6 L
%\end{multicols*}\end{enumerate}\vspace{-0.5cm}
%
%\item  The volume of a sample of gas, initially at 25 $^{\circ}$C and 158 mL, increases to 450. mL. What is the final temperature of the sample of gas, if the pressure in the container is kept constant?
%\begin{enumerate}[label=(\alph*)]\vspace{-0.5cm}
%\begin{multicols*}{3}
%\item 8.8 $^{\circ}$C			
%\item 71 $^{\circ}$C
%\item 105 $^{\circ}$C			
%\item -168 $^{\circ}$C
%\item 576 $^{\circ}$C
%\end{multicols*}\end{enumerate}\vspace{-0.5cm}

\item  What volume would a 0.250 mol sample of  \ce{H2} gas occupy, if it had a which has a pressure of 1.70 atm, and a temperature of 35 $^{\circ}$C?
\begin{enumerate}[label=(\alph*)]\vspace{-0.5cm}
\begin{multicols*}{3}
\item 0.269 L			
\item 0.423 L
\item 1.25 L			
\item 3.72 L
\item 283 L
\end{multicols*}\end{enumerate}\vspace{-0.5cm}

\item   At STP, what is the volume of 4.50 mols of nitrogen gas?
\begin{enumerate}[label=(\alph*)]\vspace{-0.5cm}
\begin{multicols*}{3}
\item 167 L
\item 3420 L
\item 101 L
\item 60.7 L
\item 1230 L
\end{multicols*}\end{enumerate}\vspace{-0.5cm}

\item  How many L of O2 gas at STP, are needed to react with 15.0 g of Na?
\begin{center}\ce{4Na(s)  +  O2(g )  ->  2Na2O(s)}\end{center}
\begin{enumerate}[label=(\alph*)]\vspace{-0.5cm}
\begin{multicols*}{3}
\item 3.65 L
\item 14.6 L
\item 7.30 L
\item 22.4 L
\item 32.0 L
\end{multicols*}\end{enumerate}\vspace{-0.5cm}

\item  When 3 L of HCl reacts, how many L of \ce{H2} gas are formed at STP conditions?
\begin{center}\ce{Zn(s) + 2 HCl(aq)  ->  H2(g)  + ZnCl2(aq)}\end{center}
\begin{enumerate}[label=(\alph*)]\vspace{-0.5cm}
\begin{multicols*}{3}
\item 1.5 L			
\item 0.120 L
\item 8.32  L			
\item 22.4 L
\item 0.382 L
\end{multicols*}\end{enumerate}\vspace{-0.5cm}

\item  A 1.20-L container contains 1.10 g of an unknown gas at STP.  What is the molcular weight of the unknown gas?
\begin{enumerate}[label=(\alph*)]\vspace{-0.5cm}
\begin{multicols*}{3}
\item 1.32 g/mol	
\item 0.917 g/mol
\item 22.4 g/mol	
\item 20.5 g/mol
\item 1.10 g/mol
\end{multicols*}\end{enumerate}\vspace{-0.5cm}



\item A tank contains a mixture of helium, neon, and argon gases.  If the total pressure in the tank is 490. mmHg and the partial pressures of helium and argon are 215 mmHg and 102 mmHg, respectively, what is the partial pressure of neon?
\begin{enumerate}[label=(\alph*)]\vspace{-0.5cm}
\begin{multicols*}{3}
\item 0.228 mmHg		
\item 603 mmHg
\item 377 mmHg			
\item 807 mmHg
\item 173 mmHg
\end{multicols*}\end{enumerate}\vspace{-0.5cm}

\item  A 2.5 g sample of french fries is placed in a calorimeter with 500.0 g of water at an initial temperature of 21 $^{\circ}$C.  After combustion of the french fries, the water has a temperature of 48 $^{\circ}$C.  What is the combustion energy for the process if the calorimeter factor is negligible?
\begin{enumerate}[label=(\alph*)]\vspace{-0.5cm}
\begin{multicols*}{3}
\item 23 KJ/g			
\item 11 KJ/g
\item 0.14 KJ/g			
\item 4.2 KJ/g
\item 5.4 KJ/g
\end{multicols*}\end{enumerate}\vspace{-0.5cm}

\item  An unknown metal with mass of 100 g absorbs 6 KJ of heat, and its temperature increases from 22 $^{\circ}$C to 23 $^{\circ}$C. Determine the specific heat of this metal in $J/g ^{\circ}C$.
\begin{enumerate}[label=(\alph*)]\vspace{-0.5cm}
\begin{multicols*}{3}
\item 60 		
\item -60 
\item 40 		
\item 160 
\item 10 
\end{multicols*}\end{enumerate}\vspace{-0.5cm}

\item  The specific heat of copper is 0.0920 $cal/g ^{\circ}C$, and the specific heat of silver is 0.0562 $cal/g ^{\circ}C$.  If 100 cal of heat is added to one g of each metal at 25 $^{\circ}$C, what is the expected result?
\begin{enumerate}[label=(\alph*)]
\item The copper will reach a higher temperature.
\item The silver will reach a higher temperature.
\item The two samples will reach the same temperature.
\item The copper will reach a temperature lower than 25 $^{\circ}$C.
\item The silver will soften.
\end{enumerate}

\item  which of the following has a non-zero $\Delta H^0_f$
\begin{enumerate}[label=(\alph*)]\vspace{-0.5cm}
\begin{multicols*}{3}
\item \ce{S(s)}
\item \ce{O2(s)}
\item \ce{NaCl(s)}
\item \ce{Na(s)}
\item \ce{Cl2(g)}
\end{multicols*}\end{enumerate}\vspace{-0.5cm}





\item  At constant temperature and pressure, the heats of formation of \ce{H2O(g)},  \ce{CO2(g)}, and  \ce{C2H6(g)} are given below. Calculate $\Delta H^0_f$ for 1 mol of \ce{C2H6} gas to oxidize to carbon dioxide gas and water vapor?
 
\begin{center}\ce{C2H6 (g) + O2(g) -> 2CO2(g) + 3H2 O(l )}\end{center}
$\Delta H^0_f(\ce{H2O_{(g)}})$=-251KJ/mol; $\Delta H^0_f(\ce{CO2_{(g)}})$=-393KJ/mol; $\Delta H^0_f(\ce{C2H6_{(g)}})$=-84KJ/mol.
\begin{enumerate}[label=(\alph*)]\vspace{-0.5cm}
\begin{multicols*}{3}
\item -8730KJ		
\item -2910KJ
\item -1455KJ		
\item +1455KJ
\item +2910KJ
\end{multicols*}\end{enumerate}\vspace{-0.5cm}

\newcounter{enumTempB}
    \setcounter{enumTempB}{\theenumi}
\end{enumerate}
\end{multicols*}
\end{fullwidth}
\clearpage
\newpage
\thispagestyle{empty}
\newgeometry{left=0.8in,right=2.8in, top=3.5cm,bottom=2cm}
\begin{fullwidth}
\begin{multicols}{2}\begin{enumerate}[resume]  \setlength\itemsep{0.2em}
    \setcounter{enumi}{\theenumTempB}


\item  Given these two standard enthalpies of formation:\\
\begin{tabularx}{\columnwidth}{>{}m{.65\linewidth} *{2}{Y} }
\multicolumn{2}{l}{\hspace{\linewidth} }   \\
\multicolumn{2}{l}{\ce{    S(s)  +  O2(g )  ->  SO2(g)	           }\hspace{0.1cm}$\Delta H_1= -295KJ        $  }   \\
\multicolumn{2}{l}{\ce{    S(s)  +  2/3O2(g )  -> SO3(g)          }\hspace{0.1cm}$\Delta H_2= -395KJ        $ }   \\
\end{tabularx}\\
What is $\Delta H^o_f$ for this reaction:
\begin{center}\ce{O2(g )  +  2SO2(g )  -> 2SO3(g)}\end{center}
\begin{enumerate}[label=(\alph*)]\vspace{-0.5cm}
\begin{multicols*}{3}
\item -1380 KJ/mol		
\item -690KJ/mol
\item -295KJ/mol		
\item -200KJ/mol
\item -100KJ/mol
\end{multicols*}\end{enumerate}\vspace{-0.5cm}




%\item $\Delta H^o_f$ for the following reaction is +1300KJ:
%\begin{center}\ce{2CO2 (g) + H2O(l) -> 5/2 O2(g) +  C2H2(g ) }\end{center}
%Calculate $\Delta H^o_f$ for next reaction:
%\begin{center}\ce{C2H2 (g) -> 2C(s) +  H2(g )}\end{center}
%Given the following information:\\
%$\Delta H^o_f(\ce{H2O_{(l)}})$=-286KJ/mol; 
%$\Delta H^o_f(\ce{CO2_{(g)}})$=-394KJ/mol; 
%\begin{enumerate}[label=(\alph*)]\vspace{-0.5cm}
%\begin{multicols*}{3}
%\item -226KJ		
%\item -113KJ
%\item +113KJ		
%\item +226KJ
%\item +452KJ
%\end{multicols*}\end{enumerate}\vspace{-0.5cm}
%
%
%\item  Find the $\Delta H^o_f$ for the reaction below
%\begin{center}\ce{PCl3(g) +  Cl2 (g ) -> PCl5 (g)}\end{center}
%given the following reactions and subsequent $\Delta H^o_f$ values: 
%\begin{center}\ce{ P4 (s) + 6Cl2(g ) -> 4PCl3(g)}\hspace{0.5cm}$\Delta H_1= -2439KJ        $\end{center}
%\begin{center}\ce{ 4PCl5(g) -> P4 (s) + 10 Cl2(g)  }\hspace{0.5cm}$\Delta H_2= +3438KJ        $\end{center}
%\begin{enumerate}[label=(\alph*)]\vspace{-0.5cm}
%\begin{multicols*}{3}
%\item -250KJ			
%\item -133KJ
%\item+13KJ			
%\item +22KJ
%\item +420KJ
%\end{multicols*}\end{enumerate}\vspace{-0.5cm}

%\item  Chemical reactions always involve
%\begin{enumerate}[label=(\alph*)]\vspace{-0.5cm}
%\begin{multicols*}{2}
%\item release of energy.
%\item absorption of energy.
%\item release or absorption of energy.
%\item release and absorption of energy.
%\item a change in state.
%\end{multicols*}\end{enumerate}\vspace{-0.5cm}


\item What is the concentration, in mass percent, of a solution prepared from 50.0 g NaCl and 150.0 g of water? 
\begin{enumerate}[label=(\alph*)]\vspace{-0.5cm}
\begin{multicols*}{3}
\item 0.250\%		
\item 33.3\%
\item 40.0\%		
\item 25.0\%
\item 3.00\%
\end{multicols*}\end{enumerate}\vspace{-0.5cm}


\item What is the molarity of a solution which contains 58.5 g of sodium chloride dissolved in 0.500 L of solution?
\begin{enumerate}[label=(\alph*)]\vspace{-0.5cm}
\begin{multicols*}{3}
\item 0.500 M		
\item 1.00 M
\item 1.50 M		
\item 2.00 M
\item 4.00 M
\end{multicols*}\end{enumerate}\vspace{-0.5cm}

\item What volume of 0.10 M NaOH can be prepared from 250. mL of 0.30 M NaOH?
\begin{enumerate}[label=(\alph*)]\vspace{-0.5cm}
\begin{multicols*}{3}
\item 0.075 L		
\item 0.25 L
\item 0.75 L		
\item 0.083 L
\item 750 L
\end{multicols*}\end{enumerate}\vspace{-0.5cm}

\item  How many mL of a NaCl 1.7 M solution contains 4g of NaCl (MW=58.44 $g\cdot mol^{-1}$).
\vspace{2cm}


\item What is the final volume, in milliliters, when 5.00 mL of a 3\% NaCl solution is diluted to provide a 1\%  NaCl solution.
\vspace{2cm}

\item Write the net-ionic equations for the following reaction:
\begin{center}\ce{HCl (l) + KOH(aq)  ->  KCl(aq) + H2O(l)}\end{center}
\vspace{2cm}

\item A 15g hot piece of Au (Ce(Au)= 0.13$J/g ^{\circ}C$) at 400 $^{\circ}$C  is dropped on a cold 25$^{\circ}$C contained with 150g of water (Ce(\ce{H2O})=4.184$J/g ^{\circ}C$). Calculate the final temperature of the water+Au piece.
\vspace{2cm}


\item What is the pressure in atm of 3 mole of NH3 gas at 1600K in a 5L container?
(a) Using the ideal gas law; (b) Using the real gas law (a(NH3)=4.17; b(NH3)=0.04).

\vspace{2cm}


 \end{enumerate}
\end{multicols}
\end{fullwidth}
\begin{fullwidth}
\par\noindent\rule{0.5\textwidth}{0.4pt}\\
\emph{Answers:}\\
\vspace{-0.5cm}
\begin{tasks}[counter-format={tsk[1].}, label-align=left, label-offset={0mm}, label-width={5mm}, item-indent={1mm}, label-format={\bfseries}](4)
\task (a) 
\task (b) 
\task (c) 
\task (b)
\task (c)
\task (c)
%\task (a) 
\task (b) 
%\task (c) 
%\task (e) 
%\task (e) 
\task (d) 
\task (c) 
\task (a) 
\task (a) 
\task (d) 

\task (e) 
\task (a) 
\task (a) 
\task (b) 
\task (c) 
\task (c) 
\task (d) 

%\task (a) 
%\task (a) 
%\task (c) 
\task (d) 
\task (d) 
\task (c)
\task 40mL
\task 15mL
 \task \small \ce{H^+_{(aq)} + OH^-_{(aq)}  ->  H2O_{(l)}}
\task 26$^{\circ}$C
\task (a) 78atm; (b) 79.1atm

\end{tasks}






\end{fullwidth}
\restoregeometry
\end{document}