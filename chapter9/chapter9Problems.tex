\documentclass[main.tex]{subfiles}
\begin{document}\newpage
\setdoublesep{0.35700 em}  % 'Bond Spacing'
\setatomsep{1.78500 em}    % 'Fixed Length'
\setbondoffset{0.18265 em} % 'Margin Width'
\newcommand{\bondwidth}{0.06642 em} % 'Line Width'
\setbondstyle{line width = \bondwidth}
\newgeometry{left=0.8in,right=0.8in, top=2.5cm,bottom=2cm}
\fancyhfoffset[E,O]{0pt}
\setlength{\columnsep}{30pt}
\begin{conclusion}
\end{conclusion}
%\setstretch{0.3}
\begin{multicols*}{2}\setcounter{numA}{1}


{\raggedright\textsc{\textbf{Solutions }}\par}
%%%%%PROBLEM
\begin{question}[ID=\the\value{numA}]\SetQuestionProperties{section-title=\nameref{sec:units}}
A solution is prepared by mixing 4 g of \ce{C6H6_{(l)}} and 10 g of \ce{CCl4_{(l)}}. Indicate the true statement:
\begin{inparaenum}[(a)]	
\item \ce{C6H6} is the solute
\item\ce{CCl4} is the solute
\item Both chemicals do not mix
\item The mixture is not a solution
\end{inparaenum} 
\end{question}
\begin{solution}
 \ce{C6H6} is the solute
\hspace{0.1cm}\end{solution}\stepcounter{numA}%%%%%%%%%%%%

%%%%%PROBLEM
\begin{question}[ID=\the\value{numA}]\SetQuestionProperties{section-title=\nameref{sec:units}}
A solution is prepared by mixing 5 g of \ce{Au_{(s)}} and 2 g of \ce{Cu_{(s)}}. Indicate the true statement:
\begin{inparaenum}[(a)]	
\item \ce{Au} is the solute
\item\ce{Cu} is the solute
\item Both elements do not mix
\item The mixture is not a solution
\end{inparaenum} 
\end{question}
\begin{solution}
 \ce{Cu} is the solute
\hspace{0.1cm}\end{solution}\stepcounter{numA}%%%%%%%%%%%%

%%%%%%%PROBLEM
\begin{question}[ID=\the\value{numA}]\SetQuestionProperties{section-title=\nameref{sec:units}}
Which chemicals from the following list will mix with \ce{H2O_{(l)}}:
\begin{inparaenum}[(a)]
\item  \ce{NH3_{(l)}}  % (polar)
\item  \ce{C5H12_{(l)}}  % (nonpolar)
\item \ce{C6H14_{(l)}}  % (nonpolar)
\end{inparaenum}
\end{question}
\begin{solution}
 \ce{NH3_{(l)}}   (polar)
\hspace{0.1cm}\end{solution}\stepcounter{numA}
%%%%%%%%%%%%%%

%%%%%%%PROBLEM
\begin{question}[ID=\the\value{numA}]\SetQuestionProperties{section-title=\nameref{sec:units}}
Which chemicals from the following list will mix with \ce{C5H12_{(l)}}:
\begin{inparaenum}[(a)]
\item  \ce{H2O_{(l)}}  % (polar)
\item   \ce{C6H14_{(l)}}  % (nonpolar)
\item \ce{CH3COOH_{(l)}}  % (polar)
\end{inparaenum}
\end{question}
\begin{solution}
 \ce{C6H14_{(l)}}    (nonpolar)
\hspace{0.1cm}\end{solution}\stepcounter{numA}
%%%%%%%%%%%%%%



%%%%%%%PROBLEM
\begin{question}[ID=\the\value{numA}]\SetQuestionProperties{section-title=\nameref{sec:units}}
Classify the following molecules as polar or nonpolar:
\begin{inparaenum}[(a)]
\item\ce{C6H14_{(l)}}  % (nonpolar)
\item\ce{CH3CH2CH2OH_{(l)}}  % (polar)
\item \ce{C5H10_{(l)}}  % (nonpolar)
\item \ce{C6H5CH3_{(l)}}  % (nonpolar)
\item \ce{CH3CH2OH_{(l)}}  % (polar)
\item \ce{C6H5NH2_{(l)}}  % (polar)
\end{inparaenum}
\end{question}
\begin{solution}
\begin{inparaenum}[(a)]
\item\ce{C6H14_{(l)}}    (nonpolar)
\item\ce{CH3CH2CH2OH_{(l)}}    (polar)
\item \ce{C5H10_{(l)}}    (nonpolar)
\item \ce{C6H5CH3_{(l)}}    (nonpolar)
\item \ce{CH3CH2OH_{(l)}}    (polar)
\item \ce{C6H5NH2_{(l)}}    (polar)
\end{inparaenum}\hspace{0.1cm}\end{solution}\stepcounter{numA}
%%%%%%%%%%%%%%



%%%%%%%PROBLEM
\begin{question}[ID=\the\value{numA}]\SetQuestionProperties{section-title=\nameref{sec:units}}
Classify the following molecules as polar or nonpolar:
\begin{inparaenum}[(a)]
\item \ce{H2O_{(l)}}  % (polar)
\item\ce{C5H12_{(l)}}  % (nonpolar)
\item \ce{CH3COOH_{(l)}}  % (polar)
\item \ce{CH3OH_{(l)}}  % (polar)
\end{inparaenum}
\end{question}
\begin{solution}
\begin{inparaenum}[(a)]
\item \ce{H2O_{(l)}}    (polar)
\item\ce{C5H12_{(l)}}    (nonpolar)
\item \ce{CH3COOH_{(l)}}    (polar)
\item \ce{CH3OH_{(l)}}    (polar)
\end{inparaenum}\hspace{0.1cm}\end{solution}\stepcounter{numA}
%%%%%%%%%%%%%%




%%%%%%%PROBLEM
\begin{question}[ID=\the\value{numA}]\SetQuestionProperties{section-title=\nameref{sec:units}}
Oil and water do not mix due to a polarity difference. Explain why a detergent can help solve oil in water.
\end{question}
\begin{solution}
 Soap are salts of oils and therefore and therefore are overall non-polar molecules. Soaps are produced by mixing an oil with a strong base.
 \hspace{0.1cm}\end{solution}\stepcounter{numA}
%%%%%%%%%%%%%%
{\raggedright\textsc{\textbf{Concentration Units }}\par}

%%%%%%%PROBLEM
\begin{question}[ID=\the\value{numA}]\SetQuestionProperties{section-title=\nameref{sec:units}}
Sodium hydroxide \ce{NaOH}, a very strong base, is a chemical used in drain cleaners. A drain cleaning solution is prepared by mixing 25g of \ce{NaOH} in 250g of water. Calculate the mass percent of solute.
\end{question}
\begin{solution}
9\%
 \hspace{0.1cm}\end{solution}\stepcounter{numA}
%%%%%%%%%%%%%%
%%%%%%%PROBLEM
\begin{question}[ID=\the\value{numA}]\SetQuestionProperties{section-title=\nameref{sec:units}}
A solution is prepared by mixing 15g of \ce{KCl} in 50g of water. Calculate the mass percent of solute.
\end{question}
\begin{solution}
23\%
 \hspace{0.1cm}\end{solution}\stepcounter{numA}
%%%%%%%%%%%%%%
%%%%%%%PROBLEM
\begin{question}[ID=\the\value{numA}]\SetQuestionProperties{section-title=\nameref{sec:units}}
Alcohol-hydroxide is a mixture of a base with an organic alcohol, employed to clean glass. An alcohol-hydroxide mixture is prepared by mixing 60g of \ce{NaOH} with 500g of ethanol. Calculate:
\begin{inparaenum}[(a)]
 \item the mass percent of solvent. %89\%
  \item the mass percent of solute.%11\%
 \end{inparaenum} 
\end{question}
\begin{solution}
\begin{inparaenum}[(a)]
 \item  89\%
  \item 11\%
 \end{inparaenum} 
 \hspace{0.1cm}\end{solution}\stepcounter{numA}
%%%%%%%%%%%%%%
%%%%%%%%PROBLEM
%\begin{question}[ID=\the\value{numA}]\SetQuestionProperties{section-title=\nameref{sec:units}}
%Vanilla extract is a solution vanillin in ethanol. A vanilla solution is made by mixing 15 mL of pure vanillin and 50mL of ethanol.
%\end{question}
%\begin{solution}
%23\%
% \hspace{0.1cm}\end{solution}\stepcounter{numA}
%%%%%%%%%%%%%%%
%%%%%%%PROBLEM
\begin{question}[ID=\the\value{numA}]\SetQuestionProperties{section-title=\nameref{sec:units}}
Vinegar is not a pure chemical, it is indeed a (m/m) 5\% acetic acid solution. How many grams of acetic acid are there in 2g of vinegar.
\end{question}
\begin{solution}
0.1 g
 \hspace{0.1cm}\end{solution}\stepcounter{numA}
%%%%%%%%%%%%%%
%%%%%%%PROBLEM
\begin{question}[ID=\the\value{numA}]\SetQuestionProperties{section-title=\nameref{sec:units}}
An HCl solution is prepared by mixing 4 moles of HCl with water until reaching a volume of 250mL. Calculate the molarity of the solution.
\end{question}
\begin{solution}
16M
 \hspace{0.1cm}\end{solution}\stepcounter{numA}
%%%%%%%%%%%%%%
%%%%%%%PROBLEM
\begin{question}[ID=\the\value{numA}]\SetQuestionProperties{section-title=\nameref{sec:units}}
How many mL of a 3M KCl solution contains 0.06 moles of KCl. 
\end{question}\begin{solution}
20mL
 \hspace{0.1cm}\end{solution}\stepcounter{numA}
%%%%%%%%%%%%%%
%%%%%%%PROBLEM
\begin{question}[ID=\the\value{numA}]\SetQuestionProperties{section-title=\nameref{sec:units}}
How many mL of a 4M NaCl (MW=58$\text{g/mol}$) solution contains 5 grams of NaCl.
\end{question}\begin{solution}
21mL
 \hspace{0.1cm}\end{solution}\stepcounter{numA}
%%%%%%%%%%%%%%
%%%%%%%PROBLEM
\begin{question}[ID=\the\value{numA}]\SetQuestionProperties{section-title=\nameref{sec:units}}
How many grams of solute are there in 100mL of a 0.01M \ce{HNO3} (MW=63$\text{g/mol}$) solution.
\end{question}\begin{solution}
0.062g
 \hspace{0.1cm}\end{solution}\stepcounter{numA}
%%%%%%%%%%%%%%
%%%%%%%PROBLEM
\begin{question}[ID=\the\value{numA}]\SetQuestionProperties{section-title=\nameref{sec:units}}
How many mL of a 0.001M \ce{Ca(OH)2} (MW=74$\text{g/mol}$) solution can be prepared from 5 mg of \ce{Ca(OH)2}.
\end{question}\begin{solution}
67.56mL
 \hspace{0.1cm}\end{solution}\stepcounter{numA}
%%%%%%%%%%%%%%
%%%%%%%PROBLEM
\begin{question}[ID=\the\value{numA}]\SetQuestionProperties{section-title=\nameref{sec:units}}
What is the final volume when 50mL of a 2M NaCl solution is diluted to a 1M.
\end{question}\begin{solution}
100 mL
 \hspace{0.1cm}\end{solution}\stepcounter{numA}
%%%%%%%%%%%%%%
%%%%%%%PROBLEM
\begin{question}[ID=\the\value{numA}]\SetQuestionProperties{section-title=\nameref{sec:units}}
What is the concentration of a solution prepared when 100mL a 4\% HCl solution is diluted to a final volume of 500mL.
\end{question}\begin{solution}
0.8 \%
 \hspace{0.1cm}\end{solution}\stepcounter{numA}
%%%%%%%%%%%%%%
%%%%%%%PROBLEM
\begin{question}[ID=\the\value{numA}]\SetQuestionProperties{section-title=\nameref{sec:units}}
Describe how to prepare 50mL of a 0.5M \ce{H2SO4} solution, starting with a 1M stock \ce{H2SO4} solution.
\end{question}\begin{solution}
25mL
 \hspace{0.1cm}\end{solution}\stepcounter{numA}
%%%%%%%%%%%%%%

{\raggedright\textsc{\textbf{Electrolytes and insoluble compounds}}\par}


%%%%%%%PROBLEM
\begin{question}[ID=\the\value{numA}]\SetQuestionProperties{section-title=\nameref{sec:units}}
Indicate whether solutions of the following chemicals will have ions (I), ions and molecules (I+M), or just molecules (M): 
\vspace{-.5cm}\begin{center}\begin{tabularx}{0.9\columnwidth}{>{ \arraybackslash}p{5em}>{ \arraybackslash}p{5em}>{ \arraybackslash}p{5em}>{ \arraybackslash}p{5em}  }
  \toprule
\heading{Chemical} & \heading{I}  &  \heading{I+M}  &  \heading{M}     \\
    \midrule
 \ce{NaCl} 		&    	&	 &	       \\
  \ce{HCl}			&   	 	&	  &	       \\
 \ce{CaCl2} 		&  	 	&	   &	       \\
 \ce{H2O}			&  	   	&	 	  &     \\
   \ce{NO2} 		&  	  	&	 	&       \\
    \bottomrule
\end{tabularx}\end{center}
\end{question}\begin{solution}
 \ce{NaCl}(I);  \ce{HCl}	(I); \ce{CaCl2} (I); \ce{H2O}	(M+I);\ce{CO2} (M)	
 \hspace{0.1cm}\end{solution}\stepcounter{numA}
%%%%%%%%%%%%%%

%%%%%%%PROBLEM
\begin{question}[ID=\the\value{numA}]\SetQuestionProperties{section-title=\nameref{sec:units}}
Indicate the soluble/insoluble character of the following compounds: 
\begin{center}\begin{tabularx}{0.9\columnwidth}{>{ \arraybackslash}p{5em}>{ \arraybackslash}p{5em}>{ \arraybackslash}p{5em} }
  \toprule
\heading{Chemical} & \heading{Soluble}  &  \heading{Insoluble}      \\
    \midrule
\ce{AgNO3}	&   		&	 	       \\
 \ce{AgBr}		&  	 	& 	       \\
 \ce{CaCO3}	&  	 	&	   	       \\
\ce{Na2CO3}	&  	   	&	 	       \\
    \bottomrule
\end{tabularx}\end{center}

\end{question}\begin{solution}
\ce{AgNO3}(Soluble); \ce{AgBr}	(Insoluble); \ce{CaCO3}(Soluble);\ce{Na2CO3}(Soluble)
\hspace{0.1cm}\end{solution}\stepcounter{numA}
%%%%%%%%%%%%%%


%%%%%%%PROBLEM
\begin{question}[ID=\the\value{numA}]\SetQuestionProperties{section-title=\nameref{sec:units}}
Indicate the soluble/insoluble character of the following compounds: 
\begin{center}\begin{tabularx}{0.9\columnwidth}{>{ \arraybackslash}p{5em}>{ \arraybackslash}p{5em}>{ \arraybackslash}p{5em} }
  \toprule
\heading{Chemical} & \heading{Soluble}  &  \heading{Insoluble}      \\
    \midrule
 \ce{NaCH3COO}	&  		&	 	       \\
 \ce{NaHCO3}	&  	 	&	 	       \\
 \ce{Ag2SO4}	&  		&	   	       \\
 \ce{NaCrO4}	&  	  	&	 	       \\
  \ce{CaS}	&  	 	&	 	       \\
    \bottomrule
\end{tabularx}\end{center}
\end{question}\begin{solution}
 \ce{NaCH3COO}(Soluble); \ce{NaHCO3}(Soluble); \ce{Ag2SO4}(Insoluble); \ce{NaCrO4}(Soluble);  \ce{CaS}(Insoluble);
 \hspace{0.1cm}\end{solution}\stepcounter{numA}
%%%%%%%%%%%%%%




%%%%%%%PROBLEM
\begin{question}[ID=\the\value{numA}]\SetQuestionProperties{section-title=\nameref{sec:units}}
Break down the following compounds into ions:
\begin{inparaenum}[(a)]
\item \ce{Ca(OH)2}		%(\ce{Ca^{2+} +2 OH^{-} })
 \item \ce{K2CrO4}		%(\ce{2K^{+} +CrO4^{2-} })
 \item \ce{Ca(NO3)2}		%(\ce{Ca^{+} +2 NO3^{-} })
\end{inparaenum}
\end{question}\begin{solution}
\begin{inparaenum}[(a)]
\item \ce{Ca(OH)2}		 (\ce{Ca^{2+} +2 OH^{-} })
 \item \ce{K2CrO4}		 (\ce{2K^{+} +CrO4^{2-} })
 \item \ce{Ca(NO3)2}		 (\ce{Ca^{+} +2 NO3^{-} })
\end{inparaenum} \hspace{0.1cm}\end{solution}\stepcounter{numA}
%%%%%%%%%%%%%%


{\raggedright\textsc{\textbf{Precipitation and acid-base reactions}}\par}
%%%%%%%PROBLEM
\begin{question}[ID=\the\value{numA}]\SetQuestionProperties{section-title=\nameref{sec:units}}
Classify the following reaction as acid-base or precipitation:
\noindent
  \begin{enumerate} [topsep=0pt, partopsep=1pt, label=(\alph*), leftmargin=0.5cm]	
\item \ce{ NaCl_{(aq)} + AgNO3 _{(aq)} -> NaNO3_{(aq)} + AgCl_{(s)} v } 
\item \ce{ H2SO4_{(aq)} + 2NaOH_{(aq)} -> 2H2O_{(l)} + Na2SO4_{(aq)}     }
\item	{\raggedleft \ce{  2 Na3PO4_{(aq)} + 3 CaCl2_{(aq)} -> } }  \\ 
	\hspace*{\fill}\ce{6 NaCl_{(aq)} + Ca3(PO4)2_{(s)} v  }\\ 
\end{enumerate}
\end{question}\begin{solution}
\begin{inparaenum}[(a)]
\item Precipitation
 \item Acid-base
\item Precipitation\end{inparaenum}
 \hspace{0.1cm}\end{solution}\stepcounter{numA}
%%%%%%%%%%%%%%


%%%%%%%PROBLEM
\begin{question}[ID=\the\value{numA}]\SetQuestionProperties{section-title=\nameref{sec:units}}
Classify the following reaction as acid-base or precipitation:
\noindent
  \begin{enumerate} [topsep=0pt, partopsep=1pt, label=(\alph*), leftmargin=0.5cm]	
\item \ce{ H2SO4_{(aq)} + Ba(OH)2_{(aq)} -> BaSO4_{(aq)} + 2H2O_{(l)}  } %Acid-base
\item \ce{ Na2CO3_{(aq)} + BaCl2_{(aq)} -> BaCO3_{(s)} + 2NaCl_{(aq)}     }%Precipitation
\item	\ce{4Fe + 3O2 -> 2Fe2O3} %redox
\end{enumerate}
\end{question}\begin{solution}
\begin{inparaenum}[(a)]
\item  Acid-base
\item  Precipitation
\item	 redox
 \end{inparaenum}
 \hspace{0.1cm}\end{solution}\stepcounter{numA}
%%%%%%%%%%%%%%




%%%%%%%PROBLEM
\begin{question}[ID=\the\value{numA}]\SetQuestionProperties{section-title=\nameref{sec:units}}
Obtain the net ionic equation for the following reaction:
 \begin{center}\ce{ NaCl_{(aq)} + AgNO3 _{(aq)} -> NaNO3_{(aq)} + AgCl_{(s)} v    }\end{center}
\end{question}\begin{solution}
\ce{   Ag^+ _{(aq)} + Cl^- _{(aq)} ->  AgCl_{(s)} v    }
 \hspace{0.1cm}\end{solution}\stepcounter{numA}
%%%%%%%%%%%%%%
%%%%%%%PROBLEM
\begin{question}[ID=\the\value{numA}]\SetQuestionProperties{section-title=\nameref{sec:units}}
Obtain the net ionic equation for the following reaction:
\begin{center}\ce{H2SO4_{(aq)} + 2NaOH_{(aq)} -> 2H2O_{(l)} + Na2SO4_{(aq)}       }\end{center}
\end{question}\begin{solution}
 \ce{   2H^+ _{(aq)} + 2OH^- _{(aq)} ->  2 H2O_{(l)}     }  \hspace{0.1cm}\end{solution}\stepcounter{numA}
%%%%%%%%%%%%%%
%%%%%%%PROBLEM
\begin{question}[ID=\the\value{numA}]\SetQuestionProperties{section-title=\nameref{sec:units}}
Obtain the net ionic equation for the following reaction:
\begin{center}\ce{2 Na3PO4_{(aq)} + 3 CaCl2_{(aq)} -> 6 NaCl_{(aq)} + Ca3(PO4)2_{(s)} v       }\end{center}
\end{question}\begin{solution}
  \ce{   3Ca^{2+} _{(aq)} + 2PO4^{3-} _{(aq)} ->  Ca3(PO4)2_{(s)} v    }
   \hspace{0.1cm}\end{solution}\stepcounter{numA}
%%%%%%%%%%%%%%

{\raggedright\textsc{\textbf{Redox}}\par}
%%%%%%%PROBLEM
\begin{question}[ID=\the\value{numA}]\SetQuestionProperties{section-title=\nameref{sec:units}}
Balance the following redox reactions:
\begin{enumerate}[label=(\alph*)]
\item \ce{  Fe_{(s)} + O_2_{(g)} -> Fe^{3+}_{(aq)}  + O^{2-}_{(aq)}}
\item \ce{  Cu_{(s)} + Ag^{+}_{(aq)} -> Ag_{(s)}  + Cu^{2+}_{(aq)}}
\end{enumerate}
\end{question}\begin{solution}
\begin{inparaenum}[(a)]
\item   \ce{  Fe_{(s)} + O_2_{(g)} -> Fe^{3+}_{(aq)}  + O^{2-}_{(aq)}}
\item  Precipitation
 \end{inparaenum}
     \hspace{0.1cm}\end{solution}\stepcounter{numA}
%%%%%%%%%%%%%%
%%%%%%%PROBLEM
\begin{question}[ID=\the\value{numA}]\SetQuestionProperties{section-title=\nameref{sec:units}}
Balance the following redox reactions in acidic medium:
\begin{center}\ce{ I^-_{(aq)} + MnO4^-_{(aq)}   ->  I2_{(s)} + Mn^{2+}_{(aq)}       }\end{center}
\end{question}\begin{solution}
\ce{   10I^-_{(aq)} + 2MnO4^-_{(aq)}+ 16H^+_{(aq)}  ->  5I2_{(s)} + 2Mn^{2+}_{(aq)} + 8H2O_{(l)}    } 
   \hspace{0.1cm}\end{solution}\stepcounter{numA}
%%%%%%%%%%%%%%
%%%%%%%PROBLEM
\begin{question}[ID=\the\value{numA}]\SetQuestionProperties{section-title=\nameref{sec:units}}
Balance the following redox reactions in acidic medium:
\begin{center}\ce{ MnO4^{-}_{(aq)} + SO3^{-2}_{(aq)} -> MnO2_{(s)} + SO4^{-2}_{(aq)}      }\end{center}
\end{question}\begin{solution}
 \ce{    2MnO4^{-}_{(aq)} + 2H^+ +3SO3^{-2}_{(aq)} -> H2O_{(l)} + 2MnO2_{(s)} + 3SO4^{-2}_{(aq)}     }
    \hspace{0.1cm}\end{solution}\stepcounter{numA}
%%%%%%%%%%%%%%

%%%%%%%%PROBLEM
%\begin{question}[ID=\the\value{numA}]\SetQuestionProperties{section-title=\nameref{sec:units}}
%Balance the following redox reactions in acidic medium:
%\begin{center}\ce{  2MnO4^{-}_{(aq)} + 2H^+ +3SO3^{-2}_{(aq)} -> H2O_{(l)} + 2MnO2_{(s)}+ 3SO4^{-2}_{(aq)}     }\end{center}
%\end{question}\begin{solution}
% \ce{     2MnO4^{-}_{(aq)} + H2O_{(l)} +3SO3^{-2}_{(aq)} -> 2OH^-_{(aq)} + 2MnO2_{(s)} + 3SO4^{-2}_{(aq)}    }
%     \hspace{0.1cm}\end{solution}\stepcounter{numA}
%%%%%%%%%%%%%%%

%%%%%%%PROBLEM
\begin{question}[ID=\the\value{numA}]\SetQuestionProperties{section-title=\nameref{sec:units}}
Balance the following redox reactions in acidic medium:
\begin{enumerate}[label=(\alph*)]
 \item \ce{   Zn_{(s)} + NO_3^{-}_{(aq)} ->Zn^{2+}_{(aq)} + NO_2_{(g)}  }
 \item \ce{  O_2_{(g)} + Fe^{2+}_{(aq)} -> Fe^{3+}_{(aq)}}
\end{enumerate}
\end{question}\begin{solution}
 No solution listed
     \hspace{0.1cm}\end{solution}\stepcounter{numA}
%%%%%%%%%%%%%%

%%%%%%%PROBLEM
\begin{question}[ID=\the\value{numA}]\SetQuestionProperties{section-title=\nameref{sec:units}}
Balance the following redox reactions in basic medium:
\begin{enumerate}[label=(\alph*)]
 \item \ce{   Sn^{2+}_{(aq)} + Cr2O7^{2-}_{(aq)} ->Cr^{3+}_{(aq)} + Sn^{4+}_{(aq)}  }
 \item \ce{   FeS_{(s)} + NO3^{-}_{(aq)} ->NO_{(g)} + Fe^{3+}_{(aq)} + SO4^{2-}_{(aq)}  }
 \item \ce{   Sn_{(s)} + NO3^{-}_{(aq)} + Cl^{-}_{(aq)} -> SnCl6^{2-}_{(aq)} + NO2_{(g)}  }
\end{enumerate}
\end{question}\begin{solution}
 No solution listed
     \hspace{0.1cm}\end{solution}\stepcounter{numA}
%%%%%%%%%%%%%%


\end{multicols*}
\newpage
\begin{answersenvironment}
\begin{minipage}[c]{1\textwidth}
\begin{localsize}{10}
{\Large \bf Answers}
\SetupExSheets{
  headings = inline-nr , % numbered and inline
  counter-format = qu) , % numbers 1) 2) ... 
}
%\printsolutions 
\printsolutions[byID={1,3,5,7,9,11,13,15,17,19,21,23,25,27,29, 31, 33 }]
\end{localsize}
\end{minipage}\end{answersenvironment}
\end{document}


