\documentclass[main.tex]{subfiles}
\begin{document}\newpage
\setdoublesep{0.35700 em}  % 'Bond Spacing'
\setatomsep{1.78500 em}    % 'Fixed Length'
\setbondoffset{0.18265 em} % 'Margin Width'
\newcommand{\bondwidth}{0.06642 em} % 'Line Width'
\setbondstyle{line width = \bondwidth}

\begin{fullwidth}





%%%%%%%%%%%%HEADING
\begin{multicols}{2}
\begin{tcolorbox}[enhanced jigsaw,breakable,size=title,
colback=mybrown!05,colframe=black,fonttitle=\bfseries,
title=STUDENT INFO,pad at break=1mm, break at=15cm/0pt ]
\vspace{0.2cm}
\noindent Name: \rule{5cm}{0.4pt}Date:\rule{1cm}{0.4pt}\\
Pre-lab Done: \tikzcheckmark[scale=2,black]{no mark}\quad
\end{tcolorbox}
\end{multicols}
\hfill
\vspace{0.2cm}
\begin{center}
{\large \bfseries 
Pre-lab Questions 
\par
\Huge
Electrolytes and polarity
\\[5pt] \par}
\vspace{0.2cm}
\end{center}
\par
\noindent
\uline{  \hfill \normalsize \hfill       }
%%%%%%%%%%%%HEADING

\begin{enumerate}
% PELAB 1
\item Define electronegativity.
\vspace{3cm}

\item Compare the electronegativity of hydrogen and chlorine. Will HCl be polar or nonpolar?
\vspace{3cm}

\item Compare the electronegativity of two hydrogen atoms. Will \ce{H_2} be polar or nonpolar?
\vspace{3cm}


\item Classify the following chemicals as ionic, covalent, organic chemical, organic acid or organic base: \ce{NaCl},  \ce{CH3-COOH}, \ce{CH3NH2}, \ce{HF}, \ce{CO}.
\vspace{3cm}

\end{enumerate}


\clearpage\mbox{}\clearpage



%%%%%%%%%%%%HEADING
\begin{multicols}{2}
\begin{tcolorbox}[enhanced jigsaw,breakable,size=title,
colback=mybrown!05,colframe=black,fonttitle=\bfseries,
title=STUDENT INFO,pad at break=1mm, break at=15cm/0pt ]
\vspace{0.2cm}
\noindent Name: \rule{5cm}{0.4pt}Date:\rule{1cm}{0.4pt}\\
Pre-lab Done: \tikzcheckmark[scale=2,black]{no mark}\quad
\end{tcolorbox}
\end{multicols}
\hfill
\vspace{0.2cm}
\begin{center}
{\large \bfseries 
Experiment
\par
\Huge
Electrolytes and polarity
\\[5pt] \par}
\vspace{0.2cm}
\end{center}
\par
\noindent
\uline{  \hfill \normalsize \hfill       }
%%%%%%%%%%%%HEADING

\vspace{0.2cm}{\large \bfseries 1. Polarity and miscibility}
The goal of this mini-experiment is to identify the polarity of a given solute by mixing it with a polar and nonpolar solvent. To chemicals with the same polarity will mix with each other due to favorable interactions. Think about oil and soap, both are nonpolar chemicals and hence they mix well. Differently, chemicals with different polarity do not mix well. Think this time about water and oil. Water is polar and oil is nonpolar. Both chemicals do not mix well. By using a polar solvent (water) and a nonpolar solvent (cyclohexane) you will be able to track the polarity of a given solute by studying the miscibility of the solute with both solvents. Ultimately, polarity is due to differences in electronegativity an the existence of a permanent dipolar moment in the molecule. Small molecules (diatomic) made of different elements will always be polar, as the differences of electronegativity will not compensate with each other leading to a permanent dipole moment. Molecules made mainly of C and H will normally be nonpolar as both elements have not appreciable electronegativity differences.
\begin{steps}
    \newstep[] Use eight test tubes in a rack. Four of these will be filled with 3mL of water--a polar solvent--whereas the remaining four tubes will be filled with cyclohexane--a nonpolar solvent.
    \newstep[] Add a few drops of a few crystals of the following solutes both in water and in cyclohexane. If the solute mixes with water that means it will be polar. If the solute mixes with cyclohexane that would mean it is nonpolar.
\end{steps}

\begin{center}\begin{tabular}{ |p{4cm}|p{4cm}|p{4cm}|p{4cm}|  }
\hline
    Solute &  Soluble in Water (\ce{H2O})? &  Soluble in Cyclohexane (\ce{C6H10})? & Polar/Nonpolar        \\
\hline
   \vspace{0cm}\ce{I2}\vspace{.25cm} &     &   &          \\\hline
   \vspace{0cm}Sucrose\vspace{.25cm} &     &   &          \\\hline
   \vspace{0cm}\ce{KMnO4}\vspace{.25cm} &     &   &          \\\hline
   \vspace{0cm}\ce{Vegetable oil}\vspace{.25cm} &     &   &          \\\hline

\hline
\end{tabular}\end{center}

\end{fullwidth}




\newpage
\begin{fullwidth}


\vspace{0.2cm}{\large \bfseries 2. Electrolytes}
Chemicals can be classified as electrolytes or nonelectrolytes depending on whether they conduct the electricity in solution. Electrolytes conduct the electricity as they produce ions, positive and negative, in solution. For example, NaCl is an electrolyte and conducts the electricity when dissolved in water, as NaCl breaks down in water to produce \ce{Na+} and \ce{Cl-}. These ions conduct the electricity.
At the same time, electrolytes can be strong of weak depending of their degree of dissociation. \emph{Strong electrolytes} completely dissociate in water and hence they heavily conduct the electricity in water. In solution, strong electrolytes produce ions. Differently, \emph{weak electrolytes} dissociate only partially in water and hence their conduction character is weak and in solution you will have both molecules and ions. \emph{Nonelectrolytes} do not dissociate in water and hence they do not conduct the electricity and you will only have molecules in solution. 
The following table will help you classify chemicals as electrolytes. In general, ionic compounds (metal+nonmetal) are strong electrolytes and organic compounds (carbon based compounds) are nonelectrolytes. Weak electrolytes are rare. A few examples are \ce{HF}, \ce{H2O}, organic bases like ammonia, and organic acids like acetic acid.


\begin{figure*}[h] % FUL FIGURE
\fontfamily{ppl}\selectfont
\begin{tabular}{llll}
\rowcolor{black!45}
\toprule
    \multicolumn{4}{|c|}{Electrolyte Classification}  \\
Electrolyte Type        & Dissociation                     & Particles in solution& Examples \\
\midrule
Strong     & Fully 								& Ions & Ionic Compounds: \ce{NaCl}, NaOH, HCl, \ce{MgCl2} \\
Weak      & Partially                               & Ions \& molecules & \ce{HF}, \ce{H2O}, Organic bases (e.g. \ce{NH3}), organic acids (e.g. \ce{CH3COOH})   \\
Nonelectrolytes    & No           				& molecules &C-based compounds: \ce{CH3OH}(methanol),  \\
 &  &           				&   \ce{CH3CH2OH}(ethanol), \ce{C12H22O11} (sucrose),\ce{CH4NO2}(urea)  \\
\bottomrule
\end{tabular}
\end{figure*}

In this mini-experiment you will study the electrolyte character of a series of solutes with different nature. By means of two electrodes connected to a lightbulb you will be able to appreciate the degree these chemicals conduct electricity. If the lightbulb glows the chemical will be an electrolyte. Depending on the brightness of the glow the chemical will be a strong or weak electrolyte.



\begin{steps}
    \newstep[] You or the professor will use a setup with two electrodes connected to a lightbulb. Place 20mL of the different solutions in the table below in a beaker.
        \newstep[]  Lower the electrodes to the solution and observe the glow.
        \newstep[]  Observe the glow and classify the chemical as nonelectrolyte, strong electrolyte or weak electrolyte.
\end{steps}

\begin{center}\begin{tabular}{ |p{4cm}|p{4cm}|p{4cm}|p{4cm}|  }
\hline
      \begin{center}Chemical\end{center} &  \begin{center}Light intensity\end{center}  &  \begin{center}Electrolyte type\end{center}  & \begin{center}Particles in solution\end{center}        \\
         &   {\small (No light, weak light, strong light)} &   {\small (Non electrolyte/weak electrolyte/strong electrolyte)} & {\small (Molecules/ions/Molecules+Ions)  }     \\

\hline
   \vspace{0cm}\ce{NaCl}\vspace{.25cm} &     &   &          \\\hline
   \vspace{0cm}Sucrose\vspace{.25cm} &     &   &          \\\hline
   \vspace{0cm}\ce{HCl}\vspace{.25cm} &     &   &          \\\hline
   \vspace{0cm}\ce{CH3-COOH}\vspace{.25cm} &     &   &          \\\hline
      \vspace{0cm}\ce{NH3}\vspace{.25cm} &     &   &          \\\hline
      \vspace{0cm}\ce{CH3-CH2OH}\vspace{.25cm} &     &   &          \\\hline


\end{tabular}\end{center}
\end{fullwidth}


\newpage
\begin{fullwidth}

\vspace{0.2cm}{\large \bfseries PostLab questions }
\begin{enumerate}
\item{} Indicate whether the following diatomic molecules are polar or nonpolar: \ce{Cl2}, \ce{N2}, \ce{O2}.
\vspace{2.5cm}
\item{} Indicate whether the following diatomic molecules are polar or nonpolar: \ce{HCl}, \ce{HI}, \ce{HF}.
\vspace{2.5cm}
\item{} Given the geometry of the following small polyatomic molecules are polar or nonpolar:\\
\begin{center} \hspace{.05in}\chemfig{ H-[:45]O-[:315]H}\hspace{.1in} and  \hspace{.1in}\chemfig{O=C=O}\hspace{.07in} \end{center}
\vspace{2.5cm}

\item{} Given the geometry of the following small polyatomic molecules are polar or nonpolar: \\


\begin{center} \hspace{.05in}\chemfig{B(-[:0]F)(-[:120]F)(-[:240]F)}\hspace{.05in}\hspace{.05in}\chemfig{S(=[:330]O)(-[:210]O)}\hspace{.05in} and \hspace{.05in}\chemfig{C(-[:0]H)(-[:90]H)(-[:180]H)(-[:270]H)}\hspace{.05in} \end{center}
\vspace{2.5cm}

\item{} Indicate whether the following chemicals are nonelectrolytes, weak electrolytes or strong electrolytes: \ce{NaF}, \ce{CH3-CH3},  \ce{CH3-CH2-CH2OH}, \ce{HF}, and \ce{CH3-CH2-COOH}.
\vspace{2.5cm}

\end{enumerate}
\end{fullwidth}

\newpage
\begin{fullwidth}

\vspace{0.2cm}{\large \bfseries 3. Molarity of a solution}
The goal of this mini-experiment is to calculate the molarity of an already prepare solution. In order to calculate molarity, we need the moles of solute and the volume of solution. You will take a given solution volume by using a pipet. At the same time you will learn how to use a pipet--a very common chemistry measuring tool. Then you will evaporate the solution so that only the solute will remain. By weighting this solute and given the molar mass you will convert grams into moles and compute molarity.
\begin{steps}
    \newstep[] Weight an evaporating dish. Write down the mass in the table below.
 \newstep[] Use a small beaker to measure approximately 20mL of the solution. Use a pipet to transfer exactly 10mL of the solution into the evaporating dish. Weight the evaporating dish with the solution.
  \newstep[] Fill a 400mL beaker with 200mL of water. Set the beaker on a hot plate and start heating at medium high heat. Place the evaporating dish on too of the beaker so that it receives indirect heat.
  
   \newstep[] The solution will start to dry. When the evaporating dish is completely dry, stop the heater and wait for the dish to cool down. Weight the evaporating dish with the solute.
    
\end{steps}


\begin{center}\begin{tabular}{ p{4.0cm}p{5.5cm}p{3cm}p{5cm}  }
\hline
 \begin{center}\mycircled{1}\end{center} &\begin{center}Mass of the evaporating dish (g)\end{center}&&\begin{center}\rule{3.0cm}{0.4pt}\end{center}\\
  \begin{center}\mycircled{2}\end{center} &\begin{center}Volume of solution, $v_{solution}$ (mL)\end{center}&&\begin{center}\rule{3.0cm}{0.4pt}\end{center}\\

   \begin{center}\mycircled{3}\end{center} & \begin{center}Mass of the evaporating dish with the solution (g)\end{center}&&\begin{center}\rule{3.0cm}{0.4pt}\end{center}\\
      \begin{center}\mycircled{3}\hspace{0.1cm}$-$\hspace{0.1cm}\mycircled{1}\end{center} & \begin{center}Mass of the solution (g)\end{center}&&\begin{center}\rule{3.0cm}{0.4pt}\end{center}\\
  \begin{center}\mycircled{4}\end{center} & \begin{center}Mass of the evaporating dish with dry solute  \end{center}&&\begin{center}\rule{3.0cm}{0.4pt}\end{center}\\
        \begin{center}\mycircled{4}\hspace{0.1cm}$-$\hspace{0.1cm}\mycircled{1}\end{center} & \begin{center}Mass of solute, $m_{Solute}$ \end{center}&&\begin{center}\rule{3.0cm}{0.4pt}\end{center}\\

  \begin{center}(\hspace{0.1cm}\mycircled{4}\hspace{0.1cm}$-$\hspace{0.1cm}\mycircled{1}\hspace{0.1cm})$\times\frac{\text{1 mol NaCl}}{\text{58 g NaCl}}$\end{center}  & \begin{center}Moles of solute, $n_{solute}$ (mol) \end{center}&&\begin{center}\rule{3.0cm}{0.4pt}\end{center}\\


\hline\end{tabular}\end{center}


Calculate the molarity of the solution by using the following formula:
\begin{equation*}
M=\frac{n_{solute}}{v_{solution}}
\end{equation*}
\hspace{2cm}
\flushright {$M=$\rule{3.0cm}{0.4pt}}

\end{fullwidth}


\newpage
\begin{fullwidth}
\vspace{0.2cm}{\large \bfseries PostLab questions }
\begin{enumerate}
\item Knowing that the density of water is $1g\cdot ml^{-1}$. Calculate the mass of solvent in the pipet. Use this value and the mass of solute you measured to calculate the mass percent of the solution.
\vspace{2.5cm}
\item  Calculate the volume of a 3M solution that contains 4 moles of solute.
\vspace{2.5cm}
\item Calculate the number of moles in 20mL of a 4M solution. 
\vspace{2.5cm}
\end{enumerate}

\end{fullwidth}

\end{document}