\documentclass[main.tex]{subfiles}
\begin{document}\newpage
\setdoublesep{0.35700 em}  % 'Bond Spacing'
\setatomsep{1.78500 em}    % 'Fixed Length'
\setbondoffset{0.18265 em} % 'Margin Width'
\newcommand{\bondwidth}{0.06642 em} % 'Line Width'
\setbondstyle{line width = \bondwidth}
\newgeometry{left=0.8in,right=0.8in, top=2.5cm,bottom=2cm}
\fancyhfoffset[E,O]{0pt}
\setlength{\columnsep}{30pt}
\begin{conclusion}
\end{conclusion}
%\setstretch{0.3}









\begin{multicols*}{2}

\begin{question}[ID=1]\SetQuestionProperties{section-title=\nameref{sec:atoms}}
Select from below the atomic symbol for the element Gold is:
\begin{multicols}{3}
  \noindent
  \begin{enumerate} [topsep=0pt, partopsep=1pt, label=(\alph*), leftmargin=1cm]
\item Go
\item Au
\item G
\item Ca
\item Ol
\end{enumerate}
\end{multicols}    
\end{question}
\begin{solution}
 (b)
\hspace{0.1cm}\end{solution}


\begin{question}[ID=2]\SetQuestionProperties{section-title=\nameref{sec:atoms}}
The atomic symbol for aluminum is:
\begin{multicols}{3}
  \noindent
  \begin{enumerate} [topsep=0pt, partopsep=1pt, label=(\alph*), leftmargin=1cm]
\item Al
\item Am
\item A
\item Sn
\item Ag
\end{enumerate}
\end{multicols}    
\end{question}
\begin{solution}
 (a)
\hspace{0.1cm}\end{solution}




\begin{question}[ID=3]\SetQuestionProperties{section-title=\nameref{sec:atoms}}
The atomic symbol for iron is:
\begin{multicols}{3}
  \noindent
  \begin{enumerate} [topsep=0pt, partopsep=1pt, label=(\alph*), leftmargin=1cm]
\item Ir
\item Fs
\item Fe
\item In
\item Ir
\end{enumerate}
\end{multicols}    
\end{question}
\begin{solution}
(c)
\hspace{0.1cm}\end{solution}


\begin{question}[ID=4]\SetQuestionProperties{section-title=\nameref{sec:atoms}}
Ca is the symbol for:
\begin{multicols}{2}
  \noindent
  \begin{enumerate} [topsep=0pt, partopsep=1pt, label=(\alph*), leftmargin=1cm]
\item Carbon
\item Calcium
\item Cobalt
\item Copper
\item Cadmium
\end{enumerate}
\end{multicols}    
\end{question}
\begin{solution}
(b)
\hspace{0.1cm}\end{solution}


\begin{question}[ID=5]\SetQuestionProperties{section-title=\nameref{sec:atoms}}
Which of the following elements is a metal?
\begin{multicols}{2}
  \noindent
  \begin{enumerate} [topsep=0pt, partopsep=1pt, label=(\alph*), leftmargin=1cm]
\item Nitrogen
\item Lithium
\item Calcium
\item Iron
\item Iodine
\end{enumerate}
\end{multicols}    
\end{question}
\begin{solution}
(d)
\hspace{0.1cm}\end{solution}


\begin{question}[ID=6]\SetQuestionProperties{section-title=\nameref{sec:atoms}}
Which of the following elements is a alkaline metal?
\begin{multicols}{2}
  \noindent
  \begin{enumerate} [topsep=0pt, partopsep=1pt, label=(\alph*), leftmargin=1cm]
\item Nitrogen
\item Lithium
\item Calcium
\item Iron
\item Ruthenium
\end{enumerate}
\end{multicols}    
\end{question}
\begin{solution}
(b)
\hspace{0.1cm}\end{solution}



\begin{question}[ID=7]\SetQuestionProperties{section-title=\nameref{sec:atoms}}
Which of the following elements is a nonmetal?
\begin{multicols}{2}
  \noindent
  \begin{enumerate} [topsep=0pt, partopsep=1pt, label=(\alph*), leftmargin=1cm]
\item Nitrogen
\item Lithium
\item Calcium
\item Iron
\item Iodine
\end{enumerate}
\end{multicols}    
\end{question}
\begin{solution}
 (a)
\hspace{0.1cm}\end{solution}





\begin{question}[ID=8]\SetQuestionProperties{section-title=\nameref{sec:atoms}}
Which of the following elements is a halogen?
\begin{multicols}{2}
  \noindent
  \begin{enumerate} [topsep=0pt, partopsep=1pt, label=(\alph*), leftmargin=1cm]
\item Nitrogen
\item Lithium
\item Calcium
\item Iron
\item Iodine
\end{enumerate}
\end{multicols}    
\end{question}
\begin{solution}
(e)
\hspace{0.1cm}\end{solution}



\begin{question}[ID=9]\SetQuestionProperties{section-title=\nameref{sec:atoms}}
What is the symbol of the element in Period 4 and Group 2?
\begin{multicols}{2}
  \noindent
  \begin{enumerate} [topsep=0pt, partopsep=1pt, label=(\alph*), leftmargin=1cm]
\item Be
\item Mg
\item Ca
\item C
\item Si
\end{enumerate}
\end{multicols}    
\end{question}
\begin{solution}
(c)
\hspace{0.1cm}\end{solution}


{\raggedright\textsc{\textbf{The Atom }}\par}

\begin{question}[ID=10]\SetQuestionProperties{section-title=\nameref{sec:atoms}}
In an atom, the nucleus contains 
  \noindent
  \begin{enumerate} [topsep=0pt, partopsep=1pt, label=(\alph*), leftmargin=1cm]
\item an equal number of protons and electrons.
\item all the protons and neutrons.
\item all the protons and electrons.
\item only neutrons.
\item only protons.
\end{enumerate}
\end{question}
\begin{solution}
(b)
\hspace{0.1cm}\end{solution}

\begin{question}[ID=11]\SetQuestionProperties{section-title=\nameref{sec:atoms}}
The atomic number of an atom is equal to the number of  
  \noindent
  \begin{enumerate} [topsep=0pt, partopsep=1pt, label=(\alph*), leftmargin=1cm]
\item nuclei
\item neutrons
\item neutrons plus protons.
\item electrons plus protons.
\item electrons
\end{enumerate}
\end{question}
\begin{solution}
(e)
\hspace{0.1cm}\end{solution}


\begin{question}[ID=12]\SetQuestionProperties{section-title=\nameref{sec:atoms}}
The mass number of an atom is equal to the number of  
  \noindent
  \begin{enumerate} [topsep=0pt, partopsep=1pt, label=(\alph*), leftmargin=1cm]
\item nuclei
\item neutrons
\item neutrons plus protons.
\item electrons plus protons.
\item electrons
\end{enumerate}
\end{question}
\begin{solution}
(c)
\hspace{0.1cm}\end{solution}




\begin{question}[ID=13]\SetQuestionProperties{section-title=\nameref{sec:atoms}}
The mass number of an atom is equal to the number of  
  \noindent
  \begin{enumerate} [topsep=0pt, partopsep=1pt, label=(\alph*), leftmargin=1cm]
\item electrons
\item neutrons
\item neutrons plus protons.
\item  protons
\end{enumerate}
\end{question}
\begin{solution}
(c)
\hspace{0.1cm}\end{solution}




\begin{question}[ID=14]\SetQuestionProperties{section-title=\nameref{sec:atoms}}
Consider a neutral atom with 30 protons and 34 neutrons. The atomic number of the element is
\begin{multicols}{1}
  \noindent
  \begin{enumerate} [topsep=0pt, partopsep=1pt, label=(\alph*), leftmargin=1cm]
\item 30
\item 32
\item 34
\item  64
\item 94
\end{enumerate}
\end{multicols}    
\end{question}
\begin{solution}
(a)
\hspace{0.1cm}\end{solution}




\begin{question}[ID=15]\SetQuestionProperties{section-title=\nameref{sec:atoms}}
Consider a neutral atom with 30 protons and 34 neutrons. The mass number of the element is
\begin{multicols}{1}
  \noindent
  \begin{enumerate} [topsep=0pt, partopsep=1pt, label=(\alph*), leftmargin=1cm]
\item 30
\item 32
\item 34
\item  64
\item 94
\end{enumerate}
\end{multicols}    
\end{question}
\begin{solution}
(d)
\hspace{0.1cm}\end{solution}



\begin{question}[ID=16]\SetQuestionProperties{section-title=\nameref{sec:atoms}}
The atomic mass of Ga is 69.72 amu. There are only two naturally occurring isotopes of gallium: 69Ga, with a mass of 69.0 amu, and 71Ga, with a mass of 71.0 amu. Calculate the natural abundance of the 69Ga isotope. 
\end{question}
\begin{solution}
64\%
\hspace{0.1cm}\end{solution}


{\raggedright\textsc{\textbf{An introduction to molecules }}\par}

\begin{question}[ID=17]\SetQuestionProperties{section-title=\nameref{sec:atoms}}
Calculate the molecular mass of the following molecule: \ce{CCl2F2}
\end{question}
\begin{solution}
121 amu
\hspace{0.1cm}\end{solution}

\begin{question}[ID=18]\SetQuestionProperties{section-title=\nameref{sec:atoms}}
Calculate the molecular mass of the following molecule: \ce{C4H10}
\end{question}
\begin{solution}
58 amu
\hspace{0.1cm}\end{solution}


\begin{question}[ID=19]\SetQuestionProperties{section-title=\nameref{sec:atoms}}
Calculate the molecular mass of the following molecule: \ce{C6H10O8}
\end{question}
\begin{solution}
210 amu
\hspace{0.1cm}\end{solution}

{\raggedright\textsc{\textbf{Empirical and Molecular Formulas }}\par} 






\begin{question}[ID=20]\SetQuestionProperties{section-title=\nameref{sec:atoms}}
What is the empirical formula of a compound if a sample of this compound contains 2.8 g of nitrogen and 3.2 g of oxygen?
\end{question}
\begin{solution}
\ce{NO}
\hspace{0.1cm}\end{solution}


\begin{question}[ID=21]\SetQuestionProperties{section-title=\nameref{sec:atoms}}
What is the empirical formula and the molecular formula of a compound if a sample contains 3 g of C, 0.5 H and 4 g of oxygen? MW=60amu
\end{question}
\begin{solution}
\ce{CH2O} 
\hspace{0.1cm}\end{solution}



\begin{question}[ID=22]\SetQuestionProperties{section-title=\nameref{sec:atoms}}
What is the empirical and molecular formula of a compound with a percent composition of 49.47\% C, 5.201\% H, 28.84\% N, and 16.48\% O, if its molecular mass is 194.2 amu.
\end{question}
\begin{solution}
\ce{C4H5N2O}
\hspace{0.1cm}\end{solution}

\begin{question}[ID=23]\SetQuestionProperties{section-title=\nameref{sec:atoms}}
A 1.587 g sample of a compound containing N and O was analyzed finding a composition of 0.483 g of Nitrogen and 1.104 g of Oxygen. Calculate the empirical formula of the compound.

\end{question}
\begin{solution}
\ce{NO2} 
\hspace{0.1cm}\end{solution}

\end{multicols*}



\newpage
\begin{answersenvironment}
\begin{minipage}[c]{1\textwidth}
\begin{localsize}{10}
{\Large \bf Answers}
\SetupExSheets{
  headings = inline-nr , % numbered and inline
  counter-format = qu) , % numbers 1) 2) ... 
}
%\printsolutions 
\printsolutions[byID={1,3,5,7,9,11,13,15,17,19,21,23}]
\end{localsize}
\end{minipage}\end{answersenvironment}
\end{document}









