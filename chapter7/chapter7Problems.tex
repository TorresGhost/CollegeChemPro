\documentclass[main.tex]{subfiles}
\begin{document}\newpage
\setdoublesep{0.35700 em}  % 'Bond Spacing'
\setatomsep{1.78500 em}    % 'Fixed Length'
\setbondoffset{0.18265 em} % 'Margin Width'
\newcommand{\bondwidth}{0.06642 em} % 'Line Width'
\setbondstyle{line width = \bondwidth}
\newgeometry{left=0.8in,right=0.8in, top=2.5cm,bottom=2cm}
\fancyhfoffset[E,O]{0pt}
\setlength{\columnsep}{30pt}
\begin{conclusion}
\end{conclusion}
%\setstretch{0.3}
\begin{multicols*}{2}\setcounter{numA}{1}


{\raggedright\textsc{\textbf{The Mole }}\par}

%%%%%PROBLEM
\begin{question}[ID=\the\value{numA}]\SetQuestionProperties{section-title=\nameref{sec:units}}
Calculate the number of molecules in:
\begin{inparaenum}[(a)]	
\item  8 moles of CO
\item	10 moles of \ce{CO2}
\end{inparaenum} 
\end{question}
\begin{solution}
\begin{inparaenum}[(a)]	
\item  $4.8\times 10^{24}$
\item  $6.0\times 10^{24}$
\end{inparaenum} 
\hspace{0.1cm}\end{solution}\stepcounter{numA}%%%%%%%%%%%%
%%%%%PROBLEM
\begin{question}[ID=\the\value{numA}]\SetQuestionProperties{section-title=\nameref{sec:units}}
Calculate the number of molecules in:
\begin{inparaenum}[(a)]	
\item  4 moles of \ce{NH3}
\item	50 moles of \ce{H2SO4}
\end{inparaenum} 
\end{question}
\begin{solution}
\begin{inparaenum}[(a)]	
\item  $2.4\times 10^{24}$
\item  $3\times 10^{25}$
\end{inparaenum} 
\hspace{0.1cm}\end{solution}\stepcounter{numA}%%%%%%%%%%%%

%%%%%PROBLEM
\begin{question}[ID=\the\value{numA}]\SetQuestionProperties{section-title=\nameref{sec:units}}
Calculate the number of moles in:
\begin{inparaenum}[(a)]	
\item $6\times 10^{23}$ molecules of \ce{NO}	%0.99 moles
\item $5\times 10^{15}$ molecules of \ce{NaCl}%$8.3\times 10^{-9}$moles
\item	$3\times 10^{27}$ molecules of \ce{MgO}	%4983.39 moles
\end{inparaenum} 
\end{question}
\begin{solution}
\begin{inparaenum}[(a)]	
\item  0.99 moles of \ce{NO}
\item  $8.3\times 10^{-9}$moles of \ce{NaCl}
\item	4983.39 moles of \ce{MgO}
\end{inparaenum} 
\hspace{0.1cm}\end{solution}\stepcounter{numA}%%%%%%%%%%%%
%%%%%PROBLEM
\begin{question}[ID=\the\value{numA}]\SetQuestionProperties{section-title=\nameref{sec:units}}
Calculate the number of moles in:
\begin{inparaenum}[(a)]	
\item  $3.2\times 10^{21}$ molecules of \ce{H2O}	%$5.3\times 10^{-3}$ moles
\item $2\times 10^{23}$ molecules of \ce{CO2}		%0.33 moles
\end{inparaenum} 
\end{question}
\begin{solution}
\begin{inparaenum}[(a)]	
\item   $5.3\times 10^{-3}$ moles of \ce{H2O}
\item  0.33 moles of \ce{CO2}
\end{inparaenum} 
\hspace{0.1cm}\end{solution}\stepcounter{numA}%%%%%%%%%%%%

%%%%%PROBLEM
\begin{question}[ID=\the\value{numA}]\SetQuestionProperties{section-title=\nameref{sec:units}}
Fill the conversion factor that calculates the final property:
 \begin{equation*}\begin{split}
10^{24}\cancel{\text{ molecules of }\ce{NO2}} \times \dfrac{\hlmath{\hspace{35pt}}\text{ moles of }\ce{NO2}}{\hlmath{\hspace{35pt}}\cancel{\text{ molecules of }\ce{NO2}}}\\
=\hlmath{\hspace{35pt}}\text{ moles of }\ce{NO2}.
\end{split}\end{equation*}
\end{question}
\begin{solution}
$
10^{24}\cancel{\text{ molecules of }\ce{NO2}} \times \dfrac{   1    \text{ moles of }\ce{NO2}}{   6.02\times 10^{23}    \cancel{\text{ molecules of }\ce{NO2}}}
=   1.66   \text{ moles of }\ce{NO2}.
$
\hspace{0.1cm}\end{solution}\stepcounter{numA}%%%%%%%%%%%%
%%%%%PROBLEM
\begin{question}[ID=\the\value{numA}]\SetQuestionProperties{section-title=\nameref{sec:units}}
Fill the conversion factor that calculates the final property:
 \begin{equation*}\begin{split}
3\cancel{\text{ moles of }\ce{NO}} \times \dfrac{\hlmath{\hspace{35pt}}\text{ molecules of }\ce{NO}}{\hlmath{\hspace{35pt}}\cancel{\text{ moles of }\ce{NO}}}\\
=\hlmath{\hspace{35pt}}\text{ molecules of }\ce{NO}.
\end{split}\end{equation*}
\end{question}
\begin{solution}
$
3\cancel{\text{ moles of }\ce{NO}} \times \dfrac{   6.02\times 10^{23}  \text{ molecules of }\ce{NO}}{   1   \cancel{\text{ moles of }\ce{NO}}}
=  1.8\times 10^{24}  \text{ molecules of }\ce{NO}.
$
\hspace{0.1cm}\end{solution}\stepcounter{numA}%%%%%%%%%%%%
%%%%%PROBLEM
\begin{question}[ID=\the\value{numA}]\SetQuestionProperties{section-title=\nameref{sec:units}}
Fill the conversion factor that calculates the final property:
 \begin{equation*}\begin{split}
6\cancel{\text{ moles of }\ce{C6H12O6}} \times \dfrac{\hlmath{\hspace{35pt}}}{\hlmath{\hspace{35pt}}}\\
=\hlmath{\hspace{35pt}}\text{ molecules of }\ce{C6H12O6}.
\end{split}\end{equation*}
\end{question}
\begin{solution}
$
6\cancel{\text{ moles of }\ce{C6H12O6}} \times \dfrac{   6.02\times 10^{23}  \text{ molecules of }\ce{C6H12O6}    }{   1\cancel{\text{ moles of }\ce{C6H12O6}}    }
=  3.6\times 10^{24}   \text{ molecules of }\ce{C6H12O6}.
$
\hspace{0.1cm}\end{solution}\stepcounter{numA}%%%%%%%%%%%%
%%%%%PROBLEM
\begin{question}[ID=\the\value{numA}]\SetQuestionProperties{section-title=\nameref{sec:units}}
Fill the conversion factor that calculates the final property:
 \begin{equation*}\begin{split}
10^{25}\cancel{\text{ molecules of }\ce{CH4N2O}} \times \dfrac{\hlmath{\hspace{35pt}}}{\hlmath{\hspace{35pt}}}\\
=\hlmath{\hspace{35pt}}\text{ moles of }\ce{CH4N2O}.
\end{split}\end{equation*}
\end{question}
\begin{solution}
$
10^{25}\cancel{\text{ molecules of }\ce{CH4N2O}} \times \dfrac{   1 mol of \ce{CH4N2O}}{   6.02\times 10^{23}\cancel{\text{ molecules of }\ce{CH4N2O}}     }
= 16.6\text{ moles of }\ce{CH4N2O}.
$
\hspace{0.1cm}\end{solution}\stepcounter{numA}%%%%%%%%%%%%


{\raggedright\textsc{\textbf{Converting moles into grams and into atoms }}\par}


%%%%%PROBLEM
\begin{question}[ID=\the\value{numA}]\SetQuestionProperties{section-title=\nameref{sec:units}}
Calculate the molar weight of the following molecules:
\begin{inparaenum}[(a)]	
\item  \ce{NH3}	%17 $g\cdot mol^{-1}$
\item 	\ce{O2}	%32 $g\cdot mol^{-1}$
\item 	\ce{CO}	%28 $g\cdot mol^{-1}$
\item 	\ce{H2}	%2 $g\cdot mol^{-1}$
\item 	\ce{Fe2(CO3)3}	%291.71 $g\cdot mol^{-1}$
\end{inparaenum} 
\end{question}
\begin{solution}
\begin{inparaenum}[(a)]	
 \item   	17 $g\cdot mol^{-1}$
\item 	 32 $g\cdot mol^{-1}$
\item 	 28 $g\cdot mol^{-1}$
\item 	 2 $g\cdot mol^{-1}$
\item 	 218 $g\cdot mol^{-1}$
\end{inparaenum} 
\hspace{0.1cm}\end{solution}\stepcounter{numA}%%%%%%%%%%%%


%%%%%PROBLEM
\begin{question}[ID=\the\value{numA}]\SetQuestionProperties{section-title=\nameref{sec:units}}
Fill the conversion factor that calculates the final property:
 \begin{equation*}\begin{split}
4\cancel{\text{ moles of }\ce{CO2}} \times \dfrac{\hlmath{\hspace{35pt}}\text{ g of }\ce{CO2}}{\hlmath{\hspace{35pt}}\cancel{\text{ moles of }\ce{CO2}}}\\
=\hlmath{\hspace{35pt}}\text{ g of }\ce{CO2}.
\end{split}\end{equation*}
\end{question}
\begin{solution}
$
4\cancel{\text{ moles of }\ce{CO2}} \times \dfrac{ 44   \text{ g of }\ce{CO2}}{1\cancel{\text{ moles of }\ce{CO2}}}
=176\text{ g of }\ce{CO2}.
$
\hspace{0.1cm}\end{solution}\stepcounter{numA}%%%%%%%%%%%%
%%%%%PROBLEM
\begin{question}[ID=\the\value{numA}]\SetQuestionProperties{section-title=\nameref{sec:units}}
Fill the conversion factor that calculates the final property:
 \begin{equation*}\begin{split}
10\cancel{\text{ g of }\ce{NO}} \times \dfrac{\hlmath{\hspace{35pt}}\text{ moles of }\ce{NO}}{\hlmath{\hspace{35pt}}\cancel{\text{ g of }\ce{NO}}}\\
=\hlmath{\hspace{35pt}}\text{ moles of }\ce{NO}.
\end{split}\end{equation*}
\end{question}
\begin{solution}
$
10\cancel{\text{ g of }\ce{NO}} \times \dfrac{1\text{ moles of }\ce{NO}}{30\cancel{\text{ g of }\ce{NO}}}
=0.33\text{ moles of }\ce{NO}.
$
\hspace{0.1cm}\end{solution}\stepcounter{numA}%%%%%%%%%%%%
%%%%%PROBLEM
\begin{question}[ID=\the\value{numA}]\SetQuestionProperties{section-title=\nameref{sec:units}}
Fill the conversion factor that calculates the final property:
 \begin{equation*}\begin{split}
5\cancel{\text{ moles of }\ce{C6H12O6}} \times \dfrac{\hlmath{\hspace{35pt}}}{\hlmath{\hspace{35pt}}}\\
=\hlmath{\hspace{35pt}}\text{ g of }\ce{C6H12O6}.
\end{split}\end{equation*}
\end{question}
\begin{solution}
$
5\cancel{\text{ moles of }\ce{C6H12O6}} \times \dfrac{ 180\text{ g of }\ce{C6H12O6}   }{ 1\cancel{\text{ moles of }\ce{C6H12O6}}      }
=    900 \text{ g of }\ce{C6H12O6}.
$
\hspace{0.1cm}\end{solution}\stepcounter{numA}%%%%%%%%%%%%
%%%%%PROBLEM
\begin{question}[ID=\the\value{numA}]\SetQuestionProperties{section-title=\nameref{sec:units}}
Fill the conversion factor that calculates the final property:
 \begin{equation*}\begin{split}
7\cancel{\text{ g of }\ce{CH4N2O}} \times \dfrac{\hlmath{\hspace{35pt}}}{\hlmath{\hspace{35pt}}}\\
=\hlmath{\hspace{35pt}}\text{ moles of }\ce{CH4N2O}.
\end{split}\end{equation*}
\end{question}
\begin{solution}
$
7\cancel{\text{ g of }\ce{CH4N2O}} \times \dfrac{1 \text{mol of }\ce{CH4N2O} }{60\cancel{\text{ g of }\ce{CH4N2O}} }
=0.12\text{ moles of }\ce{CH4N2O}.
$
\hspace{0.1cm}\end{solution}\stepcounter{numA}%%%%%%%%%%%%
%%%%%PROBLEM
\begin{question}[ID=\the\value{numA}]\SetQuestionProperties{section-title=\nameref{sec:units}}
Fill the conversion factor that calculates the final property:
 \begin{equation*}\begin{split}
10^{26}\cancel{\text{ molecules of }\ce{NO2}} \times \dfrac{\hlmath{\hspace{35pt}}\text{ atoms of }\ce{O}}{\hlmath{\hspace{35pt}}\cancel{\text{ molecules of }\ce{NO2}}}\\
=\hlmath{\hspace{35pt}}\text{ atoms of }\ce{O}.
\end{split}\end{equation*}
\end{question}
\begin{solution}
$
10^{26}\cancel{\text{ molecules of }\ce{NO2}} \times \dfrac{2\text{ atoms of }\ce{O}}{1\cancel{\text{ molecules of }\ce{NO2}}}
=2\times10^{26}\text{ atoms of }\ce{O}.
$
\hspace{0.1cm}\end{solution}\stepcounter{numA}%%%%%%%%%%%%
%%%%%PROBLEM
\begin{question}[ID=\the\value{numA}]\SetQuestionProperties{section-title=\nameref{sec:units}}
Fill the conversion factor that calculates the final property:
 \begin{equation*}\begin{split}
10^{22}\cancel{\text{ atoms of }\ce{O}} \times \dfrac{\hlmath{\hspace{35pt}}\text{ molecules of }\ce{H2O}}{\hlmath{\hspace{35pt}}\cancel{\text{ atoms of }\ce{O}}}\\
=\hlmath{\hspace{35pt}}\text{ molecules of }\ce{H2O}
\end{split}\end{equation*}
\end{question}
\begin{solution}
$
10^{22}\cancel{\text{ atoms of }\ce{O}} \times \dfrac{ 1\text{ molecules of }\ce{H2O}}{2\cancel{\text{ atoms of }\ce{O}}}
=5\times 10^{21}\text{ molecules of }\ce{H2O}
$
\hspace{0.1cm}\end{solution}\stepcounter{numA}%%%%%%%%%%%%
%%%%%PROBLEM
\begin{question}[ID=\the\value{numA}]\SetQuestionProperties{section-title=\nameref{sec:units}}
Fill the conversion factor that calculates the final property:
 \begin{equation*}\begin{split}
6\cancel{\text{ molecules of }\ce{C6H12O6}} \times \dfrac{\hlmath{\hspace{35pt}}}{\hlmath{\hspace{35pt}}}\\
=\hlmath{\hspace{35pt}}\text{ atoms of }\ce{C}.
\end{split}\end{equation*}
\end{question}
\begin{solution}
$ 6\cancel{\text{ molecules of }\ce{C6H12O6}} \times \dfrac{ 6\text{ atoms of }\ce{C}     }{1 \cancel{\text{ molecule of }\ce{C6H12O6}}   } 
=   36  \text{ atoms of }\ce{C}$
\hspace{0.1cm}\end{solution}\stepcounter{numA}%%%%%%%%%%%%
%%%%%PROBLEM
\begin{question}[ID=\the\value{numA}]\SetQuestionProperties{section-title=\nameref{sec:units}}
Fill the conversion factor that calculates the final property:
 \begin{equation*}\begin{split}
10^{21}\cancel{\text{ atoms of }\ce{N}} \times \dfrac{\hlmath{\hspace{35pt}}}{\hlmath{\hspace{35pt}}}\\
=\hlmath{\hspace{35pt}}\text{ molecules of }\ce{CH4N2O}.
\end{split}\end{equation*}
\end{question}
\begin{solution}
$ 10^{21}\cancel{\text{ atoms of }\ce{N}} \times \dfrac{1 \text{ molecules of }\ce{CH4N2O}   }{1\cancel{\text{ atoms of }\ce{N}}    }
= 10^{21}   \text{ molecules of }\ce{CH4N2O}$
\hspace{0.1cm}\end{solution}\stepcounter{numA}%%%%%%%%%%%%

%%%%%PROBLEM
\begin{question}[ID=\the\value{numA}]\SetQuestionProperties{section-title=\nameref{sec:units}}
Answer the following questions:
\begin{inparaenum}[(a)]	
\item  Calculate the number of C atoms in 3 moles of \ce{C10H14N2}?	%$1.8\times 10^{25}$ atoms
\item Calculate the number of H atoms in 3 moles of \ce{C10H14N2}?	%	$2.5\times 10^{-25}$ atoms
\item Calculate the number of N atoms in 3 moles of \ce{C10H14N2}		%$3.6\times 10^{-24}$ atoms
\end{inparaenum} 
\end{question}
\begin{solution}
\begin{inparaenum}[(a)]	
 \item   $1.8\times 10^{25}$ atoms
\item  $2.5\times 10^{-25}$ atoms
\item  $3.6\times 10^{-24}$ atoms
\end{inparaenum} 
\hspace{0.1cm}\end{solution}\stepcounter{numA}%%%%%%%%%%%%
%%%%%PROBLEM
\begin{question}[ID=\the\value{numA}]\SetQuestionProperties{section-title=\nameref{sec:units}}
Answer the following questions:
\begin{inparaenum}[(a)]	
\item How many grams are there in 4 moles of \ce{C6H12O6}?	%720.64 grams
\item How many C atoms are there in 3 moles of \ce{C6H12O6}?%$2\times 10^{27}$ atoms
\item How many O atoms are there in 3 moles of \ce{C6H12O6}?%$3.9\times 10^{27}$ atoms
\end{inparaenum} 
\end{question}
\begin{solution}
\begin{inparaenum}[(a)]	
 \item  720.64 grams
\item  $2\times 10^{27}$ atoms
\item  $3.9\times 10^{27}$ atoms
\end{inparaenum} 
\hspace{0.1cm}\end{solution}\stepcounter{numA}%%%%%%%%%%%%

{\raggedright\textsc{\textbf{Chemical Reactions}}\par}

%%%%%%%PROBLEM
\begin{question}[ID=\the\value{numA}]\SetQuestionProperties{section-title=\nameref{sec:units}}
Balance the following reactions:
\noindent
  \begin{enumerate} [topsep=0pt, partopsep=1pt, label=(\alph*), leftmargin=0.5cm]	
\item	\ce{P4(s) + O2(g) -> P4O10(s)} 
\item	\ce{Al(s) + O2(g) -> Al2O3(s)}
\end{enumerate}
\end{question}\begin{solution}
\begin{inparaenum}[(a)]
\item \ce{P4(s) + 5O2(g) -> P4O10(s)}
 \item \ce{4Al(s) + 3O2(g) -> 2Al2O3(s)}
\end{inparaenum}\hspace{0.1cm}\end{solution}\stepcounter{numA}
%%%%%%%%%%%%%%
%%%%%%%PROBLEM
\begin{question}[ID=\the\value{numA}]\SetQuestionProperties{section-title=\nameref{sec:units}}
Balance the following reactions:
\noindent
  \begin{enumerate} [topsep=0pt, partopsep=1pt, label=(\alph*), leftmargin=0.5cm]	
\item \ce{FeS(s) + O2(g) -> Fe2O3(s) + SO2(g)}
\item	\ce{NH3(g) + O2(g) -> NO(g) + H2O(g)}
\end{enumerate}
\end{question}\begin{solution}
\begin{inparaenum}[(a)]
 \item \ce{4FeS(s) + 7O2(g) -> 2Fe2O3(s) + 4SO2(g)}
 \item	\ce{4NH3(g) + 5O2(g) -> 4NO(g) + 6H2O(g)} 
\end{inparaenum}\hspace{0.1cm}\end{solution}\stepcounter{numA}
%%%%%%%%%%%%%%
%%%%%%%PROBLEM
\begin{question}[ID=\the\value{numA}]\SetQuestionProperties{section-title=\nameref{sec:units}}
Classify next reaction as combination, decomposition, single replacement, double replacement, or combustion:
\noindent
  \begin{enumerate} [topsep=0pt, partopsep=1pt, label=(\alph*), leftmargin=0.5cm]	
\item  \ce{Pb_{(s)} + FeSO4_{(s)} -> PbSO4_{(s)} + Fe_{(s)}}
\item	\ce{C6H12_{(g)} + 9O2_{(g)} -> 6CO2_{(g)} + 6H2O_{(g)}}
\item	{\raggedleft \ce{2RbNO3_{(aq)} + BeF2_{(aq)}}}  \\ 
	\hspace*{\fill}\ce{-> Be(NO3)2_{(aq)} + 2RbF_{(aq)}} 
\end{enumerate}
\end{question}\begin{solution}
\begin{inparaenum}[(a)]
 \item single replacement
 \item combustion
 \item double replacement
\end{inparaenum}\hspace{0.1cm}\end{solution}\stepcounter{numA}
%%%%%%%%%%%%%%

{\raggedright\textsc{\textbf{Stoichiometry and mass calculations}}\par}

%%%%%PROBLEM
\begin{question}[ID=\the\value{numA}]\SetQuestionProperties{section-title=\nameref{sec:units}}
Fill the mole ratio for the following reaction:
\begin{center}\ce{ C6H12O6(s) + 6O2(g) -> 6CO2(g) + 6H2O(g)  }\end{center}
\begin{equation*}
\frac{\hlmath{\hspace{35pt}}\text{ moles of }\ce{C6H12O6}}{\hlmath{\hspace{35pt}}\text{ moles of }\ce{O2}} 
\end{equation*}
\end{question}
\begin{solution}
 1/6
\hspace{0.1cm}\end{solution}\stepcounter{numA}%%%%%%%%%%%%
%%%%%PROBLEM
\begin{question}[ID=\the\value{numA}]\SetQuestionProperties{section-title=\nameref{sec:units}}
Fill the mole ratio for the following reaction:
\begin{center}\ce{ C6H12O6(s) + 6O2(g) -> 6CO2(g) + 6H2O(g)  }\end{center}
\begin{equation*}
\frac{\hlmath{\hspace{35pt}}\text{ moles of }\ce{O2}}{\hlmath{\hspace{35pt}}\text{ moles of }\ce{CO2}} 
\end{equation*}
\end{question}
\begin{solution}
 6/6
\hspace{0.1cm}\end{solution}\stepcounter{numA}%%%%%%%%%%%%
%%%%%PROBLEM
\begin{question}[ID=\the\value{numA}]\SetQuestionProperties{section-title=\nameref{sec:units}}
Fill the conversion factor that calculates the moles of oxygen needed to react with 2 moles of Silver producing \ce{AgO}:
\begin{center}\ce{ 2Ag(s) + O2(g) -> 2AgO(s)  }\end{center}
 \begin{equation*}\begin{split}
2\cancel{\text{ moles of }\ce{Ag}} \times \dfrac{\hlmath{\hspace{35pt}}\text{ moles of }\ce{O2}}{\hlmath{\hspace{35pt}}\cancel{\text{ moles of }\ce{Ag}}}\\
=1\text{ moles of }\ce{O2}.
\end{split}\end{equation*}
\end{question}
\begin{solution}
1/2
\hspace{0.1cm}\end{solution}\stepcounter{numA}%%%%%%%%%%%%
%%%%%PROBLEM
\begin{question}[ID=\the\value{numA}]\SetQuestionProperties{section-title=\nameref{sec:units}}
Fill the conversion factor that calculates the moles of \ce{AgO} produced from 2 moles of Silver:
\begin{center}\ce{ 2Ag(s) + O2(g) -> 2AgO(s)  }\end{center}
 \begin{equation*}\begin{split}
2\cancel{\text{ moles of }\ce{Ag}} \times \dfrac{\hlmath{\hspace{35pt}}\text{ moles of }\ce{AgO}}{\hlmath{\hspace{35pt}}\cancel{\text{ moles of }\ce{Ag}}}\\
=2\text{ moles of }\ce{AgO}.
\end{split}\end{equation*}
\end{question}
\begin{solution}
2/2
\hspace{0.1cm}\end{solution}\stepcounter{numA}%%%%%%%%%%%%
%%%%%PROBLEM
\begin{question}[ID=\the\value{numA}]\SetQuestionProperties{section-title=\nameref{sec:units}}
Fill the conversion factor that calculates the moles of \ce{AgO} produced from 10 moles of oxygen:
\begin{center}\ce{ 2Ag(s) + O2(g) -> 2AgO(s)  }\end{center}
 \begin{equation*}\begin{split}
10\cancel{\text{ moles of }\ce{O2}} \times \dfrac{\hlmath{\hspace{35pt}}\text{ moles of }\ce{AgO}}{\hlmath{\hspace{35pt}}\cancel{\text{ moles of }\ce{O2}}}\\
=20\text{ moles of }\ce{AgO}.
\end{split}\end{equation*}
\end{question}
\begin{solution}
20
\hspace{0.1cm}\end{solution}\stepcounter{numA}%%%%%%%%%%%%
%%%%%PROBLEM
\begin{question}[ID=\the\value{numA}]\SetQuestionProperties{section-title=\nameref{sec:units}}
Calculate how many moles of nitrogen are needed to react with 5 moles of hydrogen, to produce ammonia:
\begin{center}  \ce{ $\underset{\text{5 moles}}{\ce{3H2(g)}}$ + $\underset{\hlmath{\hspace{25pt}}\text{ moles}}{\ce{N2(g)}}$ -> 2NH3(g) }\end{center}
\end{question}
\begin{solution}
3.3 moles
\hspace{0.1cm}\end{solution}\stepcounter{numA}%%%%%%%%%%%%
%%%%%PROBLEM
\begin{question}[ID=\the\value{numA}]\SetQuestionProperties{section-title=\nameref{sec:units}}
Calculate the number of grams of nitrogen needed to react with 4 moles of hydrogen, to produce ammonia:
\begin{center}  \ce{ $\underset{\text{4 moles}}{\ce{3H2(g)}}$ + $\underset{\hlmath{\hspace{25pt}}\text{ grams}}{\ce{N2(g)}}$ -> 2NH3(g) }\end{center}
\end{question}
\begin{solution}
37.3 g
\hspace{0.1cm}\end{solution}\stepcounter{numA}%%%%%%%%%%%%
%%%%%PROBLEM
\begin{question}[ID=\the\value{numA}]\SetQuestionProperties{section-title=\nameref{sec:units}}
Calculate the number of grams of hydrogen needed to react with 0.3 moles of nitrogen, to produce ammonia:
\begin{center}  \ce{ $\underset{\hlmath{\hspace{25pt}}\text{ grams}}{\ce{3H2(g)}}$ + $\underset{\text{0.3 moles}}{\ce{N2(g)}}$ -> 2NH3(g) }\end{center}
\end{question}
\begin{solution}
1.8 g
\hspace{0.1cm}\end{solution}\stepcounter{numA}%%%%%%%%%%%%

{\raggedright\textsc{\textbf{Percent yield and limiting reagent}}\par}
%%%%%PROBLEM
\begin{question}[ID=\the\value{numA}]\SetQuestionProperties{section-title=\nameref{sec:units}}
Six moles of nitrogen gas react to produce three moles of ammonia according to the following reaction:
\newcommand\nitro{6 moles}
\newcommand\hydro{}
\newcommand\ammon{3 moles}
\begin{center}  \ce{ $\underset{\text{ \hydro}}{\ce{3H2(g)}}$ + $\underset{\text{\nitro}}{\ce{N2(g)}}$ -> $\underset{\text{\ammon}}{\ce{2NH3(g)}}$ }\end{center}
\end{question}
\begin{solution}
25\%
\hspace{0.1cm}\end{solution}\stepcounter{numA}%%%%%%%%%%%%
%%%%%PROBLEM
\begin{question}[ID=\the\value{numA}]\SetQuestionProperties{section-title=\nameref{sec:units}}
Six moles of nitrogen gas react to produce two moles of ammonia according to the following reaction:
\newcommand\nitro{6 moles}
\newcommand\hydro{}
\newcommand\ammon{2 moles}
\begin{center}  \ce{ $\underset{\text{ \hydro}}{\ce{3H2(g)}}$ + $\underset{\text{\nitro}}{\ce{N2(g)}}$ -> $\underset{\text{\ammon}}{\ce{2NH3(g)}}$ }\end{center}\end{question}
\begin{solution}
16.6\%
\hspace{0.1cm}\end{solution}\stepcounter{numA}%%%%%%%%%%%%
%%%%%PROBLEM
\begin{question}[ID=\the\value{numA}]\SetQuestionProperties{section-title=\nameref{sec:units}}
We mix three moles of hydrogen gas with three moles of nitrogen gas.
\newcommand\nitro{3 moles}
\newcommand\hydro{3 moles}
\newcommand\ammon{ }
\begin{center}  \ce{ $\underset{\text{ \hydro}}{\ce{3H2(g)}}$ + $\underset{\text{\nitro}}{\ce{N2(g)}}$ -> $\underset{\text{\ammon}}{\ce{2NH3(g)}}$ }\end{center}
Calculate the limiting reagent.
\end{question}
\begin{solution}
\ce{H2}
\hspace{0.1cm}\end{solution}\stepcounter{numA}%%%%%%%%%%%%
%%%%%PROBLEM
\begin{question}[ID=\the\value{numA}]\SetQuestionProperties{section-title=\nameref{sec:units}}
We mix three moles of hydrogen gas with half a mole of nitrogen gas.
\newcommand\nitro{0.5 moles}
\newcommand\hydro{3 moles}
\newcommand\ammon{}
\begin{center}  \ce{ $\underset{\text{ \hydro}}{\ce{3H2(g)}}$ + $\underset{\text{\nitro}}{\ce{N2(g)}}$ -> $\underset{\text{\ammon}}{\ce{2NH3(g)}}$ }\end{center}
Calculate the limiting reagent.
\end{question}
\begin{solution}
\ce{N2}
\hspace{0.1cm}\end{solution}\stepcounter{numA}%%%%%%%%%%%%
%%%%%PROBLEM
\begin{question}[ID=\the\value{numA}]\SetQuestionProperties{section-title=\nameref{sec:units}}
We mix two moles of hydrogen gas with five moles of nitrogen gas.
\newcommand\nitro{5 moles}
\newcommand\hydro{2 moles}
\newcommand\ammon{}
\begin{center}  \ce{ $\underset{\text{ \hydro}}{\ce{3H2(g)}}$ + $\underset{\text{\nitro}}{\ce{N2(g)}}$ -> $\underset{\text{\ammon}}{\ce{2NH3(g)}}$ }\end{center}
Calculate the limiting reagent.
\end{question}
\begin{solution}
\ce{H2}
\hspace{0.1cm}\end{solution}\stepcounter{numA}%%%%%%%%%%%%


%%%%%PROBLEM
\begin{question}[ID=\the\value{numA}]\SetQuestionProperties{section-title=\nameref{sec:units}}
Liquid mercury reacts with gas chlorine to produce mercury(II) choride, a white solid:
\begin{center}  \ce{  Hg(l)  +  Cl2(g)  ->  HgCl2(s)  }\end{center}
In an experiment, 5-mL of mercury (AW=200.59g/mol) with density 5g/mL, reactants with 4-g of chlorine to produce 6g of \ce{HgCl2}. What is the percent yield of the reaction.
\end{question}
\begin{solution}
88\%
\hspace{0.1cm}\end{solution}\stepcounter{numA}%%%%%%%%%%%%


%%%%%PROBLEM
\begin{question}[ID=\the\value{numA}]\SetQuestionProperties{section-title=\nameref{sec:units}}
Mercury(II) halides can be converted into mercury (I) halides by combination with metallic mercury. Mercury(I) halides are known as mercurous halides. When chlorine is the halide, the resulting mercury salt is known as calomel:
\begin{center}  \ce{  HgCl2(s)  +  Hg(l)  ->  Hg2Cl2(s) }\end{center}
When 2 grams of mercury(II) chloride reacts to produce 2 grams of calomel, calculate the percent yield of the reaction.
\end{question}
\begin{solution}
47\%
\hspace{0.1cm}\end{solution}\stepcounter{numA}%%%%%%%%%%%%



%%%%%PROBLEM
\begin{question}[ID=\the\value{numA}]\SetQuestionProperties{section-title=\nameref{sec:units}}
The Wurtz reaction results from the reaction of bromomethane (\ce{CH3Br}) with sodium to produce ethylene (\ce{C2H6}) 
\begin{center}  \ce{  2CH3Br_{(g)}  +  2Na_{(s)} ->  C2H6_{(g)} + 2NaBr_{(s)}}\end{center}
How many grams of sodium are need to produce 3g of ethylene given that the yield of the reaction is 30\%.
\end{question}
\begin{solution}
15.55g
\hspace{0.1cm}\end{solution}\stepcounter{numA}%%%%%%%%%%%%

%%%%%PROBLEM
\begin{question}[ID=\the\value{numA}]\SetQuestionProperties{section-title=\nameref{sec:units}}
Nitriles with stannous chloride (\ce{SnCl2}) in the presence of hydrochloric acid produce an imine.

\begin{center}  \ce{ $\underset{nitrile}{\ce{CH3CN(aq)}}$ + SnCl2_{(aq)} + HCl_{(aq)} -> $\underset{imine}{\ce{CH3CHNH(aq)}}$ }\end{center}
How many moles of imine (MW=43g/mol) are produced if we react 3g of nitrile (MW=41g/mol), with 2g of stannous chloride (MW=188g/mol) and 1g of hydrochloric acid (MW=36g/mol).
\end{question}
\begin{solution}
0.45g
\hspace{0.1cm}\end{solution}\stepcounter{numA}%%%%%%%%%%%%







{\raggedright\textsc{\textbf{General problems }}\par}
%%%%%PROBLEM
\begin{question}[ID=\the\value{numA}]\SetQuestionProperties{section-title=\nameref{sec:units}}
Calculate the molar weight of the following molecules:
\begin{inparaenum}[(a)]	
\item  benzene, \ce{C6H6}				%78.114 g\mol
\item  Carbon disulfide, \ce{CS2}		%76.143 g\mol
\item  Nitrogen tetroxide, \ce{N2O4}		%92.011 g\mol
\end{inparaenum} 
\end{question}
\begin{solution}
\begin{inparaenum}[(a)]	
\item   78.114 g\mol
\item   76.143 g\mol
\item   92.011 g\mol
\end{inparaenum} 
\hspace{0.1cm}\end{solution}\stepcounter{numA}%%%%%%%%%%%%
%%%%%PROBLEM
\begin{question}[ID=\the\value{numA}]\SetQuestionProperties{section-title=\nameref{sec:units}}
Calculate the molar weight of the following molecules:
\begin{inparaenum}[(a)]	
\item  Sulfur dioxide, \ce{SO2}		%64.065 g\mol
\item Unsymmetrical dimethyl hydrazine, \ce{(CH3)2NNH2}   %60.0984g\mol
\item Dimethyl sulfide, \ce{(CH3)2S}		%62.136
\end{inparaenum} 
\end{question}
\begin{solution}
\begin{inparaenum}[(a)]	
\item   64.065 g\mol
\item  60.0984g\mol
\item  62.136
\end{inparaenum} 
\hspace{0.1cm}\end{solution}\stepcounter{numA}%%%%%%%%%%%%

%%%%%PROBLEM
\begin{question}[ID=\the\value{numA}]\SetQuestionProperties{section-title=\nameref{sec:units}}
Fill the conversion factor that calculates the final property, given that the molar mass of \ce{C2H6} is 30g/mol:
 \begin{equation*}\begin{split}
7\times 10^{21} \text{ atoms of }\ce{C}
\times 
\dfrac{\hlmath   {\hspace{35pt}}  }
{\hlmath{\hspace{35pt}}     }
\times
\dfrac{\hlmath{\hspace{35pt}}}
{\hlmath{\hspace{35pt}}} \\
=\hlmath{\hspace{35pt}}   \text{ moles of }\ce{C2H6}
\end{split}\end{equation*}
\end{question}
\begin{solution}
$
7\times 10^{21} \text{ atoms of }\ce{C}
\times 
\dfrac{ 1 \text{ molecule of }\ce{C2H6}     }
{  2\text{ atoms of }\ce{C}    }
\times
\dfrac{  1 \text{ mole of }\ce{C2H6}      }
{6.02\times 10^{23} \text{ molecule of }\ce{C2H6}  }  
= 5.8\times 10^{-3}   \text{ moles of }\ce{C2H6}
$
\hspace{0.1cm}\end{solution}\stepcounter{numA}%%%%%%%%%%%%
%%%%%PROBLEM
\begin{question}[ID=\the\value{numA}]\SetQuestionProperties{section-title=\nameref{sec:units}}
Fill the conversion factor that calculates the final property, given that the molar mass of \ce{C2H6} is 30g/mol:
 \begin{equation*}\begin{split}
5\times 10^{25} \text{ atoms of }\ce{H}
\times 
\dfrac{\hlmath   {\hspace{35pt}}  }
{\hlmath{\hspace{35pt}}     }
\times
\dfrac{\hlmath{\hspace{35pt}}}
{\hlmath{\hspace{35pt}}} \\
\times
\dfrac{\hlmath{\hspace{35pt}}}
{\hlmath{\hspace{35pt}}}
=\hlmath{\hspace{35pt}}   \text{ g of }\ce{C2H6}
\end{split}\end{equation*}
\end{question}
\begin{solution}
$
5\times 10^{25} \text{ atoms of }\ce{H}
\times 
\dfrac{ 1 \text{ molecule of }\ce{C2H6} }
{  6\text{ atoms of }\ce{H}   }
\times
\dfrac{1 \text{ mole of }\ce{C2H6}  }
{ 6.02\times 10^{23} \text{ molecule of }\ce{C2H6}} 
\times
\dfrac{30 \text{ g of }\ce{C2H6} }
{1 \text{ mole of }\ce{C2H6} }
= 415  \text{ g of }\ce{C2H6}
$
\hspace{0.1cm}\end{solution}\stepcounter{numA}%%%%%%%%%%%%
%%%%%%%PROBLEM
\begin{question}[ID=\the\value{numA}]\SetQuestionProperties{section-title=\nameref{sec:units}}
Balance the following reactions:
\noindent
  \begin{enumerate} [topsep=0pt, partopsep=1pt, label=(\alph*), leftmargin=0.5cm]	
\item	\ce{H2_{(g)}    +     Br2_{(g)}    ->     HBr_{(g)}}
\item	\ce{C_{(g)}   +     O2_{(g)}    ->     CO_{(g)}}
\item	\ce{O3_{(g)}    ->     O2_{(g)}}
\item	\ce{NH4NO2_{(aq)}    ->     N2_{(g)}   +     H2O_{(l)}}
\item	\ce{Na3PO4_{(aq)} + MgCl2_{(aq)} -> Mg3(PO4)2_{(aq)}   + NaCl_{(aq)}}
\end{enumerate}
\end{question}\begin{solution}
not provided\hspace{0.1cm}\end{solution}\stepcounter{numA}
%%%%%%%%%%%%%%


\end{multicols*}
\newpage
\begin{answersenvironment}
\begin{minipage}[c]{1\textwidth}
\begin{localsize}{10}
{\Large \bf Answers}
\SetupExSheets{
  headings = inline-nr , % numbered and inline
  counter-format = qu) , % numbers 1) 2) ... 
}
%\printsolutions 
\printsolutions[byID={1,3,5,7,9,11,13,15,17,19,21,23,25,27,29,31, 33, 35 }]
\end{localsize}
\end{minipage}\end{answersenvironment}
\end{document}









%
%\item How many grams are there in 3 moles of Silver (AW=$107.9g\cdot mol^{-1}$):
%\begin{enumerate}[label=(\alph*)]
%\begin{multicols*}{2}
%\item 117.3 g  
%\item 176.8 g 
%\item 3 g 
%\item  323.7 g 
%\item  156 g 
%\end{multicols*}\flushright  {\small Ans: 323.7 g }
%\end{enumerate}
%
%\item How many grams are there in 3 moles of Potassium (AW=$39.10g\cdot mol^{-1}$):
%\begin{enumerate}[label=(\alph*)]
%\begin{multicols*}{2}
%\item 117.3 g  
%\item 176.8 g 
%\item 3 g
%\item  323.7 g 
%\item  156 g 
%\end{multicols*}\flushright  {\small Ans: 117.3 g}
%\end{enumerate}
%
%\item How many grams are there in 3 moles of Cromium (AW=$52g\cdot mol^{-1}$):
%\begin{enumerate}[label=(\alph*)]
%\begin{multicols*}{2}
%\item 117.3 g  
%\item 176.8 g 
%\item 3 g 
%\item  323.7 g 
%\item  156 g 
%\end{multicols*}\flushright  {\small Ans: 156 g }
%\end{enumerate}
%
%\item How many grams are there in 3 moles of atomic hydrogen (AW=$1g\cdot mol^{-1}$):
%\begin{enumerate}[label=(\alph*)]
%\begin{multicols*}{2}
%\item 117.3 g  
%\item 176.8 g 
%\item 3 g 
%\item  323.7 g 
%\item  156 g 
%\end{multicols*}\flushright  {\small Ans: 3 g}
%\end{enumerate}
%%\item Calculate the molar mass of the following molecules:
%%\begin{tabularx}{0.45\textwidth}{
%%    >{\centering}m{.25\linewidth} 
%%    *{3}{Y} }
%%  \toprule
%%\heading{Formula} & \multicolumn{3}{c}{\textbf{MW}}   \\
%%    \midrule
%%   \ce{CO2} & 	\multicolumn{3}{c}{     }    \\
%%    \ce{NO } & 	\multicolumn{3}{c}{     }    \\
%%        \ce{C6H12O8} & 	\multicolumn{3}{c}{     }    \\
%%        \ce{CH4N2O} & 	\multicolumn{3}{c}{     }    \\
%%      \bottomrule
%%\end{tabularx}\\
%%
%%\item Calculate the molar weight (MW) of the following molecules:\\
%%\begin{tabularx}{0.45\textwidth}{
%%    >{\centering}m{.25\linewidth} 
%%    *{3}{Y} }
%%  \toprule
%%\heading{Formula} & \multicolumn{3}{c}{\textbf{MW}}   \\
%%    \midrule
%%   \ce{NO2} & 	\multicolumn{3}{c}{     }    \\
%%    \ce{CH3OH } & 	\multicolumn{3}{c}{     }    \\
%%        \ce{CH3COOH} & 	\multicolumn{3}{c}{     }    \\
%%        \ce{HNO3} & 	\multicolumn{3}{c}{     }    \\
%%      \bottomrule
%%\end{tabularx}\\
%
%
%%
%%\item How many grams are there in 3 moles of Cobalt (AW=$58.93g\cdot mol^{-1}$):
%%\begin{enumerate}[label=(\alph*)]
%%\begin{multicols*}{2}
%%\item 117.3 g  
%%\item 176.8 g 
%%\item 3 g 
%%\item  323.7 g 
%%\item  156 g 
%%\end{multicols*}
%%\end{enumerate}
%%
%%
%%\item How many moles are there in 2 grams of carbon (AW=$12g\cdot mol^{-1}$):
%%\begin{enumerate}[label=(\alph*)]
%%\begin{multicols*}{2}
%%\item 117.3 mol  
%%\item 176.8 mol
%%\item 3 mol
%%\item  323.7 mol 
%%\item  156 mol
%%\end{multicols*}
%%\end{enumerate}
%
%
%
%
%%\item Using the giving information and the MW calculate the unknown quantity:\\
%%\begin{tikzpicture}
%%\node [block] (box1) at (0,0) [rectangle,draw=white,fill=red!20!white] {\textcolor{black}{6 Grams of \ce{H2O}}};
%%\node [block] (box2) at  (2,0) [rectangle,draw=white,fill=orange!20!white] {\textcolor{black}{{\Large ?} Moles of \ce{H2O}}};
%%\node [block] (box3) at  (4,0) [rectangle,draw=white,fill=yellow!20!white] {\textcolor{black}{Molec. of \ce{H2O}}};
%%\node [block] (box4) at  (6,0) [rectangle,draw=white,fill=green!20!white] {\textcolor{black}{H Atoms}};
%%\draw[thick,->] (box1.east) -- (box2.west) ;
%%\draw[thick,->] (box2.east) -- (box3.west) ;
%%\draw[thick,->] (box3.east) -- (box4.west) ;
%%\end{tikzpicture}
%%
%%
%%\item Using the giving information and the MW calculate the unknown quantity:\\
%%\begin{tikzpicture}
%%\node [block] (box1) at (0,0) [rectangle,draw=white,fill=red!20!white] {\textcolor{black}{Grams of \ce{H2O}}};
%%\node [block] (box2) at  (2,0) [rectangle,draw=white,fill=orange!20!white] {\textcolor{black}{5 Moles of \ce{H2O}}};
%%\node [block] (box3) at  (4,0) [rectangle,draw=white,fill=yellow!20!white] {\textcolor{black}{{\Large ?} Molec. of \ce{H2O}}};
%%\node [block] (box4) at  (6,0) [rectangle,draw=white,fill=green!20!white] {\textcolor{black}{H Atoms}};
%%\draw[thick,->] (box1.east) -- (box2.west) ;
%%\draw[thick,->] (box2.east) -- (box3.west) ;
%%\draw[thick,->] (box3.east) -- (box4.west) ;
%%\end{tikzpicture}
%%
%%\item Using the giving information and the MW calculate the unknown quantity:\\
%%\begin{tikzpicture}
%%\node [block] (box1) at (0,0) [rectangle,draw=white,fill=red!20!white] {\textcolor{black}{Grams of \ce{CO2}}};
%%\node [block] (box2) at  (2,0) [rectangle,draw=white,fill=orange!20!white] {\textcolor{black}{Moles of \ce{CO2}}};
%%\node [block] (box3) at  (4,0) [rectangle,draw=white,fill=yellow!20!white] {\textcolor{black}{$10^9$ Molec. of \ce{CO2}}};
%%\node [block] (box4) at  (6,0) [rectangle,draw=white,fill=green!20!white] {\textcolor{black}{{\Large ?} O Atoms}};
%%\draw[thick,->] (box1.east) -- (box2.west) ;
%%\draw[thick,->] (box2.east) -- (box3.west) ;
%%\draw[thick,->] (box3.east) -- (box4.west) ;
%%\end{tikzpicture}
%%
%%\item Using the giving information and the MW calculate the unknown quantity:\\
%%\begin{tikzpicture}
%%\node [block] (box1) at (0,0) [rectangle,draw=white,fill=red!20!white] {\textcolor{black}{3 Grams of \ce{HNO2}}};
%%\node [block] (box2) at  (2,0) [rectangle,draw=white,fill=orange!20!white] {\textcolor{black}{Moles of \ce{HNO2}}};
%%\node [block] (box3) at  (4,0) [rectangle,draw=white,fill=yellow!20!white] {\textcolor{black}{{\Large ?} Molec. of \ce{HNO2}}};
%%\node [block] (box4) at  (6,0) [rectangle,draw=white,fill=green!20!white] {\textcolor{black}{O Atoms}};
%%\draw[thick,->] (box1.east) -- (box2.west) ;
%%\draw[thick,->] (box2.east) -- (box3.west) ;
%%\draw[thick,->] (box3.east) -- (box4.west) ;
%%\end{tikzpicture}
%%
%%\item Using the giving information and the MW calculate the unknown quantity:\\
%%\begin{tikzpicture}
%%\node [block] (box1) at (0,0) [rectangle,draw=white,fill=red!20!white] {\textcolor{black}{3 Grams of \ce{NH3}}};
%%\node [block] (box2) at  (2,0) [rectangle,draw=white,fill=orange!20!white] {\textcolor{black}{Moles of \ce{NH3}}};
%%\node [block] (box3) at  (4,0) [rectangle,draw=white,fill=yellow!20!white] {\textcolor{black}{Molec. of \ce{NH3}}};
%%\node [block] (box4) at  (6,0) [rectangle,draw=white,fill=green!20!white] {\textcolor{black}{{\Large ?} H Atoms}};
%%\draw[thick,->] (box1.east) -- (box2.west) ;
%%\draw[thick,->] (box2.east) -- (box3.west) ;
%%\draw[thick,->] (box3.east) -- (box4.west) ;
%%\end{tikzpicture}






%
%

%

%

%

%
%

%
%\restoregeometry
%\end{enumerate}
%\end{multicols*}
%\pagecolor{green!10}\afterpage{\nopagecolor}\newpage
%\end{document}