\documentclass[main.tex]{subfiles}

\begin{document}

\linenumbers


\chapter[The periodic table: atoms and Elements]{The periodic table: atoms and Elements}
%\label{ch:atoms}

\begin{marginfigure}
      \includegraphics{chapter0.5/figure1}
      \label{fig:marginfig1}
   \end{marginfigure}
\lettrine[lines=4]{\color{black!45}M}{atter} is everywhere around you, from the water you drink to the air you inhale. The matter is made of elements and elements are made of atoms. Even the atoms of an elements can be different, having distinct number of proton and neutrons. This chapter covers the principles of atomic and electronic structure. You will learn what makes an atom and will be able to quantify the number of particles inside an atom. %\textquotesingle 
\begin{marginfigure}%LEARNING GOALS BOX
\begin{mytcbox}{GOALS}
\begin{enumerate}[label=\protect\circled{\color{white}\arabic*}]
\item Navigate the periodic table
\item Calculate the number of electrons, protons and neutrons in an atom
\item  Calculate average atomic masses
\item  Calculate empirical formulas from mass
\item  Calculate molecular formulas from empirical formulas
\end{enumerate}
\end{mytcbox}
\end{marginfigure}%LEARNING GOALS BOX





%\section{Classification of the matter}
%Matter makes up all substances. The materials we use such as glass or wood are all made of matter. Because there are so many kinds of materials, we classify matter by the types of components it contains.This section covers the classification of matter according to pure substances and mixtures. It also elaborates on the types of mixtures one can find.
%\sloppy
%\begin{description}
%\item[\docfilehook{Pure Substances}{Pure Substances}] \emph{Pure substances} have a definite composition, that is, are only made of one thing. There are two different types of pure substances: elements and compounds. \emph{Elements} are composed of only one type of atom. Examples are silver, iron, and aluminum that all contain one type of substance, and iron is only made or iron atoms, for example. \emph{Compounds} are combinations of different elements. For example, water, \ce{H2O} is made of a combination of hydrogen and oxygen atoms.
%
%\begin{marginfigure}
%      \includegraphics{chapter0.5/figure1-1}
%      \caption{Matter is made of atoms and molecules}
%      \label{fig:marginfig2}
%   \end{marginfigure}
%   
%\item[\docfilehook{Mixtures}{Mixtures}] \emph{Mixtures} are physical combinations of different substances. The air we breathe is a mixture of oxygen and nitrogen. 
% 
%  \begin{marginfigure}[1cm]
%\begin{tcolorbox}[enhanced,colback=red!5!white,colframe=black!50!red,boxrule=1pt,
%  arc=0pt,outer arc=0pt,drop heavy lifted shadow]
%\faGears\ 
%\docenvdef{Discussion:} Look around your apartment and list a pure substance, a compound, a heterogeneous mixture and a homogeneous mixture?\end{tcolorbox}
% \end{marginfigure}
%
%
%\item[\docfilehook{Types of Mixture}{Types of Mixture}]Mixtures can be homogeneous or heterogeneous. In a \emph{homogeneous mixture}--also known as solutions--the composition is uniform throughout the sample. An example of an homogeneous mixture is air, which contains oxygen and nitrogen or salt water, a solution of salt and water. \emph{Heterogeneous mixtures} are mixtures in which the components are not uniformly distributed throughout the sample. An example would be a chocolate chip cookie in which you can differenciate the dough and the chocolate.
%
%\begin{example} %%%%%%%%%%%%%%%%%%%%%%%% EXAMPLE BOX
%Classify as  element, compound, homogeneous mixture, heterogeneous mixture:
%\begin{multicols}{4}
%\begin{enumerate}[label=(\alph*)]
%\item An iron nail
%\item Milk
%\item Sugar
%\item miso soup 
%\end{enumerate}
%\end{multicols}
%\textlcsc{ \textcolor{dgreen}{\Large \textbf{Solution}} }\\
%(a) An iron nail is an element as it is only made of iron, a single material; (b) Milk is a homogeneous mixture as it is made of water, fat, protein and on the eye you see all the elements of the mixture as a single substance; (c) Sugar is a compound as it is made of carbon and other constituents; (d) miso soup is a mixture of water, fat and other chemicals and therefore is a mixture. As you can see several of the constituents such as the liquid and the miso paste and the tofu it is a heterogeneous mixture.\\
%\faDiamond\ \textlcsc{ \textcolor{dgreen}{\Large \textbf{Study Check}} }\\
%Classify as  element, compound, homogeneous mixture, heterogeneous mixture:
%\begin{multicols}{4}
%\begin{enumerate}[label=(\alph*)]
%\item muscle milk
%\item water
%\item a gold ring
%\item rice \& beans
%\end{enumerate}
%\end{multicols}
%\flushright Answer: (a) homog. mix.; (b) compound; (c) element; (d) heterog. mix.
%\end{example}%%%%%%%%%%%%%%%%%%%%%%%% EXAMPLE BOX
%
%
%
%
%
%
%\begin{figure}[h] % FUL FIGURE
%\includegraphics[width=1\linewidth,scale=0.5]{chapter0.5/figure1-3}
%\caption{Classification of the matter}
%\end{figure}
%
%
%\end{description}







\section{The periodic table}
The periodic table contains all elements that form the matter arranges in columns and rows. Each element has a different given and a symbol that represents their name. The tabular arrangement in the form of rows and columns allow further classification of the elements according to their properties. This section will cover the different features of the periodic table.
\sloppy
\begin{description}
\item[\docfilehook{Elements and Symbols}{Elements and Symbols}] Elements cannot be broken down into more simple substances. For example aluminum is an element and as it is only made of aluminum atoms and if you analyze the composition of a piece of gold you would only find gold atoms. Chemical symbols are one- or two-letter abbreviations that represent the names of the elements. Only the first letter is capitalized and if a second letter exist in the element\textquotesingle s name, the second letter should be lowercase. The chemical symbol for aluminum is Al with capital A and lowercase l. The symbols of all elements can be found in the periodic table (Figure \ref{image3.4}).

\item[\docfilehook{Periods and groups}{Periods and groups}] The periodic table contains elements arranged in rows and columns. The horizontal rows are called periods and the vertical columns are called groups. For example, the  second period contains lithium (Li), beryllium (Be), boron (B), carbon (C), nitrogen (N), oxygen (O), fluorine (F), and neon (Ne), and the second group contains Beryllium (Be), Magnesium (Mg), Calcium (Ca), Strontium (Sr), Barium (Ba) and Radium (Ra). There are eight periods (period 1 to period 8) and 18 groups. Some of the groups are labeled with an A whereas others are labeled with a B. The groups numbers can be written with roman numbers and a letters (A or B) or with a modern group numbering of 1-18 going across the periodic table. For example, the group 2 (Mg-Ra) can also be called IIA, and the group 13 (B-Ti) is also known as IIIA.


%%%%%%%%%%%%%%%%%%% PERIODIC TABLE
\begin{figure*}[h] % FUL FIGURE
\newcommand{\CommonElementTextFormat}[4]
{
  \begin{minipage}{2.5cm}
    \centering
      {\textbf{#1} \hfill #2}%
      \linebreak \linebreak
      {\textbf{#3}}%
      \linebreak \linebreak
      {{#4}}
  \end{minipage}
}

\newcommand{\NaturalElementTextFormat}[4]
{
  \CommonElementTextFormat{#1}{#2}{\Huge {#3}}{#4}
}

\newcommand{\OutlineText}[1]
{
\ifpdf
  % Couldn't find a nicer way of doing an outline font with TikZ
  % other than using pdfliteral 1 Tr
  %
  \pdfliteral direct {0.5 w 1 Tr}{#1}%
  \pdfliteral direct {1 w 0 Tr}%
\else
  % pstricks can do this with \pscharpath from pstricks
  %
  \pscharpath[shadow=false,
    fillstyle=solid,
    fillcolor=white,
    linestyle=solid,
    linecolor=black,
    linewidth=.2pt]{#1} 
\fi
}

\newcommand{\SyntheticElementTextFormat}[4]
{
\ifpdf
  \CommonElementTextFormat{#1}{#2}{\OutlineText{\Huge #3}}{#4}
\else
  % pstricks approach results in slightly larger box
  % that doesn't break, so fudge here
  \CommonElementTextFormat{#1}{#2}{\OutlineText{\Large #3}}{#4}
\fi
}
\begin{tikzpicture}[font=\sffamily, scale=0.30, transform shape]

%% Fill Color Styles
  \tikzstyle{ElementFill} = [fill=yellow!15]
  \tikzstyle{AlkaliMetalFill} = [fill=blue!55]
  \tikzstyle{AlkalineEarthMetalFill} = [fill=blue!40]
  \tikzstyle{MetalFill} = [fill=blue!25]
  \tikzstyle{MetalloidFill} = [fill=orange!25]
  \tikzstyle{NonmetalFill} = [fill=green!25]
  \tikzstyle{HalogenFill} = [fill=green!40]
  \tikzstyle{NobleGasFill} = [fill=green!55]
  \tikzstyle{LanthanideActinideFill} = [fill=purple!25]

%% Element Styles
  \tikzstyle{Element} = [draw=black, ElementFill,
    minimum width=2.75cm, minimum height=2.75cm, node distance=2.75cm]
  \tikzstyle{AlkaliMetal} = [Element, AlkaliMetalFill]
  \tikzstyle{AlkalineEarthMetal} = [Element, AlkalineEarthMetalFill]
  \tikzstyle{Metal} = [Element, MetalFill]
  \tikzstyle{Metalloid} = [Element, MetalloidFill]
  \tikzstyle{Nonmetal} = [Element, NonmetalFill]
  \tikzstyle{Halogen} = [Element, HalogenFill]
  \tikzstyle{NobleGas} = [Element, NobleGasFill]
  \tikzstyle{LanthanideActinide} = [Element, LanthanideActinideFill]
  \tikzstyle{PeriodLabel} = [font={\sffamily\LARGE}, node distance=2.0cm]
  \tikzstyle{GroupLabel} = [font={\sffamily\LARGE}, minimum width=2.75cm, node distance=2.0cm]
  \tikzstyle{TitleLabel} = [font={\sffamily\Huge\bfseries}]

%% Group 1 - IA
  \node[name=H, Element] {\NaturalElementTextFormat{1}{1.0079}{H}{Hydrogen}};
  \node[name=Li, below of=H, AlkaliMetal] {\NaturalElementTextFormat{3}{6.941}{Li}{Lithium}};
  \node[name=Na, below of=Li, AlkaliMetal] {\NaturalElementTextFormat{11}{22.990}{Na}{Sodium}};
  \node[name=K, below of=Na, AlkaliMetal] {\NaturalElementTextFormat{19}{39.098}{K}{Potassium}};
  \node[name=Rb, below of=K, AlkaliMetal] {\NaturalElementTextFormat{37}{85.468}{Rb}{Rubidium}};
  \node[name=Cs, below of=Rb, AlkaliMetal] {\NaturalElementTextFormat{55}{132.91}{Cs}{Caesium}};
  \node[name=Fr, below of=Cs, AlkaliMetal] {\NaturalElementTextFormat{87}{223}{Fr}{Francium}};

%% Group 2 - IIA
  \node[name=Be, right of=Li, AlkalineEarthMetal] {\NaturalElementTextFormat{4}{9.0122}{Be}{Beryllium}};
  \node[name=Mg, below of=Be, AlkalineEarthMetal] {\NaturalElementTextFormat{12}{24.305}{Mg}{Magnesium}};
  \node[name=Ca, below of=Mg, AlkalineEarthMetal] {\NaturalElementTextFormat{20}{40.078}{Ca}{Calcium}};
  \node[name=Sr, below of=Ca, AlkalineEarthMetal] {\NaturalElementTextFormat{38}{87.62}{Sr}{Strontium}};
  \node[name=Ba, below of=Sr, AlkalineEarthMetal] {\NaturalElementTextFormat{56}{137.33}{Ba}{Barium}};
  \node[name=Ra, below of=Ba, AlkalineEarthMetal] {\NaturalElementTextFormat{88}{226}{Ra}{Radium}};

%% Group 3 - IIIB
  \node[name=Sc, right of=Ca, Metal] {\NaturalElementTextFormat{21}{44.956}{Sc}{Scandium}};
  \node[name=Y, below of=Sc, Metal] {\NaturalElementTextFormat{39}{88.906}{Y}{Yttrium}};
  \node[name=LaLu, below of=Y, LanthanideActinide] {\NaturalElementTextFormat{57-71}{}{La-Lu}{Lanthanide}};
  \node[name=AcLr, below of=LaLu, LanthanideActinide] {\NaturalElementTextFormat{89-103}{}{Ac-Lr}{Actinide}};

%% Group 4 - IVB
  \node[name=Ti, right of=Sc, Metal] {\NaturalElementTextFormat{22}{47.867}{Ti}{Titanium}};
  \node[name=Zr, below of=Ti, Metal] {\NaturalElementTextFormat{40}{91.224}{Zr}{Zirconium}};
  \node[name=Hf, below of=Zr, Metal] {\NaturalElementTextFormat{72}{178.49}{Hf}{Halfnium}};
  \node[name=Rf, below of=Hf, Metal] {\SyntheticElementTextFormat{104}{261}{Rf}{Rutherfordium}};

%% Group 5 - VB
  \node[name=V, right of=Ti, Metal] {\NaturalElementTextFormat{23}{50.942}{V}{Vanadium}};
  \node[name=Nb, below of=V, Metal] {\NaturalElementTextFormat{41}{92.906}{Nb}{Niobium}};
  \node[name=Ta, below of=Nb, Metal] {\NaturalElementTextFormat{73}{180.95}{Ta}{Tantalum}};
  \node[name=Db, below of=Ta, Metal] {\SyntheticElementTextFormat{105}{262}{Db}{Dubnium}};

%% Group 6 - VIB
  \node[name=Cr, right of=V, Metal] {\NaturalElementTextFormat{24}{51.996}{Cr}{Chromium}};
  \node[name=Mo, below of=Cr, Metal] {\NaturalElementTextFormat{42}{95.94}{Mo}{Molybdenum}};
  \node[name=W, below of=Mo, Metal] {\NaturalElementTextFormat{74}{183.84}{W}{Tungsten}};
  \node[name=Sg, below of=W, Metal] {\SyntheticElementTextFormat{106}{266}{Sg}{Seaborgium}};

%% Group 7 - VIIB
  \node[name=Mn, right of=Cr, Metal] {\NaturalElementTextFormat{25}{54.938}{Mn}{Manganese}};
  \node[name=Tc, below of=Mn, Metal] {\NaturalElementTextFormat{43}{96}{Tc}{Technetium}};
  \node[name=Re, below of=Tc, Metal] {\NaturalElementTextFormat{75}{186.21}{Re}{Rhenium}};
  \node[name=Bh, below of=Re, Metal] {\SyntheticElementTextFormat{107}{264}{Bh}{Bohrium}};

%% Group 8 - VIIIB
  \node[name=Fe, right of=Mn, Metal] {\NaturalElementTextFormat{26}{55.845}{Fe}{Iron}};
  \node[name=Ru, below of=Fe, Metal] {\NaturalElementTextFormat{44}{101.07}{Ru}{Ruthenium}};
  \node[name=Os, below of=Ru, Metal] {\NaturalElementTextFormat{76}{190.23}{Os}{Osmium}};
  \node[name=Hs, below of=Os, Metal] {\SyntheticElementTextFormat{108}{277}{Hs}{Hassium}};

%% Group 9 - VIIIB
  \node[name=Co, right of=Fe, Metal] {\NaturalElementTextFormat{27}{58.933}{Co}{Cobalt}};
  \node[name=Rh, below of=Co, Metal] {\NaturalElementTextFormat{45}{102.91}{Rh}{Rhodium}};
  \node[name=Ir, below of=Rh, Metal] {\NaturalElementTextFormat{77}{192.22}{Ir}{Iridium}};
  \node[name=Mt, below of=Ir, Metal] {\SyntheticElementTextFormat{109}{268}{Mt}{Meitnerium}};

%% Group 10 - VIIIB
  \node[name=Ni, right of=Co, Metal] {\NaturalElementTextFormat{28}{58.693}{Ni}{Nickel}};
  \node[name=Pd, below of=Ni, Metal] {\NaturalElementTextFormat{46}{106.42}{Pd}{Palladium}};
  \node[name=Pt, below of=Pd, Metal] {\NaturalElementTextFormat{78}{195.08}{Pt}{Platinum}};
  \node[name=Ds, below of=Pt, Metal] {\SyntheticElementTextFormat{110}{281}{Ds}{Darmstadtium}};

%% Group 11 - IB
  \node[name=Cu, right of=Ni, Metal] {\NaturalElementTextFormat{29}{63.546}{Cu}{Copper}};
  \node[name=Ag, below of=Cu, Metal] {\NaturalElementTextFormat{47}{107.87}{Ag}{Silver}};
  \node[name=Au, below of=Ag, Metal] {\NaturalElementTextFormat{79}{196.97}{Au}{Gold}};
  \node[name=Rg, below of=Au, Metal] {\SyntheticElementTextFormat{111}{280}{Rg}{Roentgenium}};

%% Group 12 - IIB
  \node[name=Zn, right of=Cu, Metal] {\NaturalElementTextFormat{30}{65.39}{Zn}{Zinc}};
  \node[name=Cd, below of=Zn, Metal] {\NaturalElementTextFormat{48}{112.41}{Cd}{Cadmium}};
  \node[name=Hg, below of=Cd, Metal] {\NaturalElementTextFormat{80}{200.59}{Hg}{Mercury}};
  \node[name=Uub, below of=Hg, Metal] {\SyntheticElementTextFormat{112}{285}{Uub}{Ununbium}};

%% Group 13 - IIIA
  \node[name=Ga, right of=Zn, Metal] {\NaturalElementTextFormat{31}{69.723}{Ga}{Gallium}};
  \node[name=Al, above of=Ga, Metal] {\NaturalElementTextFormat{13}{26.982}{Al}{Aluminium}};
  \node[name=B, above of=Al, Metalloid] {\NaturalElementTextFormat{5}{10.811}{B}{Boron}};
  \node[name=In, below of=Ga, Metal] {\NaturalElementTextFormat{49}{114.82}{In}{Indium}};
  \node[name=Tl, below of=In, Metal] {\NaturalElementTextFormat{81}{204.38}{Tl}{Thallium}};
  \node[name=Uut, below of=Tl, Metal] {\SyntheticElementTextFormat{113}{284}{Uut}{Ununtrium}};

%% Group 14 - IVA
  \node[name=C, right of=B, Nonmetal] {\NaturalElementTextFormat{6}{12.011}{C}{Carbon}};
  \node[name=Si, below of=C, Metalloid] {\NaturalElementTextFormat{14}{28.086}{Si}{Silicon}};
  \node[name=Ge, below of=Si, Metalloid] {\NaturalElementTextFormat{32}{72.64}{Ge}{Germanium}};
  \node[name=Sn, below of=Ge, Metal] {\NaturalElementTextFormat{50}{118.71}{Sn}{Tin}};
  \node[name=Pb, below of=Sn, Metal] {\NaturalElementTextFormat{82}{207.2}{Pb}{Lead}};
  \node[name=Uuq, below of=Pb, Metal] {\SyntheticElementTextFormat{114}{289}{Uuq}{Ununquadium}};

%% Group 15 - VA
  \node[name=N, right of=C, Nonmetal] {\NaturalElementTextFormat{7}{14.007}{N}{Nitrogen}};
  \node[name=P, below of=N, Nonmetal] {\NaturalElementTextFormat{15}{30.974}{P}{Phosphorus}};
  \node[name=As, below of=P, Metalloid] {\NaturalElementTextFormat{33}{74.922}{As}{Arsenic}};
  \node[name=Sb, below of=As, Metalloid] {\NaturalElementTextFormat{51}{121.76}{Sb}{Antimony}};
  \node[name=Bi, below of=Sb, Metal] {\NaturalElementTextFormat{83}{208.98}{Bi}{Bismuth}};
  \node[name=Uup, below of=Bi, Metal] {\SyntheticElementTextFormat{115}{288}{Uup}{Ununpentium}};

%% Group 16 - VIA
  \node[name=O, right of=N, Nonmetal] {\NaturalElementTextFormat{8}{15.999}{O}{Oxygen}};
  \node[name=S, below of=O, Nonmetal] {\NaturalElementTextFormat{16}{32.065}{S}{Sulphur}};
  \node[name=Se, below of=S, Nonmetal] {\NaturalElementTextFormat{34}{78.96}{Se}{Selenium}};
  \node[name=Te, below of=Se, Metalloid] {\NaturalElementTextFormat{52}{127.6}{Te}{Tellurium}};
  \node[name=Po, below of=Te, Metalloid] {\NaturalElementTextFormat{84}{209}{Po}{Polonium}};
  \node[name=Uuh, below of=Po, Metal] {\SyntheticElementTextFormat{116}{293}{Uuh}{Ununhexium}};

%% Group 17 - VIIA
  \node[name=F, right of=O, Halogen] {\NaturalElementTextFormat{9}{18.998}{F}{Flourine}};
  \node[name=Cl, below of=F, Halogen] {\NaturalElementTextFormat{17}{35.453}{Cl}{Chlorine}};
  \node[name=Br, below of=Cl, Halogen] {\NaturalElementTextFormat{35}{79.904}{Br}{Bromine}};
  \node[name=I, below of=Br, Halogen] {\NaturalElementTextFormat{53}{126.9}{I}{Iodine}};
  \node[name=At, below of=I, Halogen] {\NaturalElementTextFormat{85}{210}{At}{Astatine}};
  \node[name=Uus, below of=At, Element] {\SyntheticElementTextFormat{117}{292}{Uus}{Ununseptium}}; 

%% Group 18 - VIIIA
  \node[name=Ne, right of=F, NobleGas] {\NaturalElementTextFormat{10}{20.180}{Ne}{Neon}};
  \node[name=He, above of=Ne, NobleGas] {\NaturalElementTextFormat{2}{4.0025}{He}{Helium}};
  \node[name=Ar, below of=Ne, NobleGas] {\NaturalElementTextFormat{18}{39.948}{Ar}{Argon}};
  \node[name=Kr, below of=Ar, NobleGas] {\NaturalElementTextFormat{36}{83.8}{Kr}{Krypton}};
  \node[name=Xe, below of=Kr, NobleGas] {\NaturalElementTextFormat{54}{131.29}{Xe}{Xenon}};
  \node[name=Rn, below of=Xe, NobleGas] {\NaturalElementTextFormat{86}{222}{Rn}{Radon}};
  \node[name=Uuo, below of=Rn, Nonmetal] {\SyntheticElementTextFormat{118}{294}{Uuo}{Ununoctium}}; 

%% Period
  \node[name=Period1, left of=H, PeriodLabel] {1};
  \node[name=Period2, left of=Li, PeriodLabel] {2};
  \node[name=Period3, left of=Na, PeriodLabel] {3}; 
  \node[name=Period4, left of=K, PeriodLabel] {4}; 
  \node[name=Period5, left of=Rb, PeriodLabel] {5};
  \node[name=Period6, left of=Cs, PeriodLabel] {6};
  \node[name=Period7, left of=Fr, PeriodLabel] {7};

%% Group
  \node[name=Group1, above of=H, GroupLabel] {1 \hfill IA};
  \node[name=Group2, above of=Be, GroupLabel] {2 \hfill IIA};
  \node[name=Group3, above of=Sc, GroupLabel] {3 \hfill IIIB};
  \node[name=Group4, above of=Ti, GroupLabel] {4 \hfill IVB};
  \node[name=Group5, above of=V, GroupLabel] {5 \hfill VB};
  \node[name=Group6, above of=Cr, GroupLabel] {6 \hfill VIB};
  \node[name=Group7, above of=Mn, GroupLabel] {7 \hfill VIIB};
  \node[name=Group8, above of=Fe, GroupLabel] {8 \hfill VIIIB};
  \node[name=Group9, above of=Co, GroupLabel] {9 \hfill VIIIB};
  \node[name=Group10, above of=Ni, GroupLabel] {10 \hfill VIIIB};
  \node[name=Group11, above of=Cu, GroupLabel] {11 \hfill IB};
  \node[name=Group12, above of=Zn, GroupLabel] {12 \hfill IIB};
  \node[name=Group13, above of=B, GroupLabel] {13 \hfill IIIA};
  \node[name=Group14, above of=C, GroupLabel] {14 \hfill IVA};
  \node[name=Group15, above of=N, GroupLabel] {15 \hfill VA};
  \node[name=Group16, above of=O, GroupLabel] {16 \hfill VIA};
  \node[name=Group17, above of=F, GroupLabel] {17 \hfill VIIA};
  \node[name=Group18, above of=He, GroupLabel] {18 \hfill VIIIA};

%% Lanthanide
  \node[name=La, below of=Rf, LanthanideActinide, yshift=-1cm] {\NaturalElementTextFormat{57}{138.91}{La}{Lanthanum}};
  \node[name=Ce, right of=La, LanthanideActinide] {\NaturalElementTextFormat{58}{140.12}{Ce}{Cerium}};
  \node[name=Pr, right of=Ce, LanthanideActinide] {\NaturalElementTextFormat{59}{140.91}{Pr}{Praseodymium}};
  \node[name=Nd, right of=Pr, LanthanideActinide] {\NaturalElementTextFormat{60}{144.24}{Nd}{Neodymium}};
  \node[name=Pm, right of=Nd, LanthanideActinide] {\NaturalElementTextFormat{61}{145}{Pm}{Promethium}};
  \node[name=Sm, right of=Pm, LanthanideActinide] {\NaturalElementTextFormat{62}{150.36}{Sm}{Samarium}};
  \node[name=Eu, right of=Sm, LanthanideActinide] {\NaturalElementTextFormat{63}{151.96}{Eu}{Europium}};
  \node[name=Gd, right of=Eu, LanthanideActinide] {\NaturalElementTextFormat{64}{157.25}{Gd}{Gadolinium}};
  \node[name=Tb, right of=Gd, LanthanideActinide] {\NaturalElementTextFormat{65}{158.93}{Tb}{Terbium}};
  \node[name=Dy, right of=Tb, LanthanideActinide] {\NaturalElementTextFormat{66}{162.50}{Dy}{Dysprosium}};
  \node[name=Ho, right of=Dy, LanthanideActinide] {\NaturalElementTextFormat{67}{164.93}{Ho}{Holmium}};
  \node[name=Er, right of=Ho, LanthanideActinide] {\NaturalElementTextFormat{68}{167.26}{Er}{Erbium}};
  \node[name=Tm, right of=Er, LanthanideActinide] {\NaturalElementTextFormat{69}{168.93}{Tm}{Thulium}};
  \node[name=Yb, right of=Tm, LanthanideActinide] {\NaturalElementTextFormat{70}{173.04}{Yb}{Ytterbium}};
  \node[name=Lu, right of=Yb, LanthanideActinide] {\NaturalElementTextFormat{71}{174.97}{Lu}{Lutetium}};

%% Actinide
  \node[name=Ac, below of=La, LanthanideActinide, yshift=-1cm] {\NaturalElementTextFormat{89}{227}{Ac}{Actinium}};
  \node[name=Th, right of=Ac, LanthanideActinide] {\NaturalElementTextFormat{90}{232.04}{Th}{Thorium}};
  \node[name=Pa, right of=Th, LanthanideActinide] {\NaturalElementTextFormat{91}{231.04}{Pa}{Protactinium}};
  \node[name=U, right of=Pa, LanthanideActinide] {\NaturalElementTextFormat{92}{238.03}{U}{Uranium}};
  \node[name=Np, right of=U, LanthanideActinide] {\SyntheticElementTextFormat{93}{237}{Np}{Neptunium}};
  \node[name=Pu, right of=Np, LanthanideActinide] {\SyntheticElementTextFormat{94}{244}{Pu}{Plutonium}};
  \node[name=Am, right of=Pu, LanthanideActinide] {\SyntheticElementTextFormat{95}{243}{Am}{Americium}};
  \node[name=Cm, right of=Am, LanthanideActinide] {\SyntheticElementTextFormat{96}{247}{Cm}{Curium}};
  \node[name=Bk, right of=Cm, LanthanideActinide] {\SyntheticElementTextFormat{97}{247}{Bk}{Berkelium}};
  \node[name=Cf, right of=Bk, LanthanideActinide] {\SyntheticElementTextFormat{98}{251}{Cf}{Californium}};
  \node[name=Es, right of=Cf, LanthanideActinide] {\SyntheticElementTextFormat{99}{252}{Es}{Einsteinium}};
  \node[name=Fm, right of=Es, LanthanideActinide] {\SyntheticElementTextFormat{100}{257}{Fm}{Fermium}};
  \node[name=Md, right of=Fm, LanthanideActinide] {\SyntheticElementTextFormat{101}{258}{Md}{Mendelevium}};
  \node[name=No, right of=Md, LanthanideActinide] {\SyntheticElementTextFormat{102}{259}{No}{Nobelium}};
  \node[name=Lr, right of=No, LanthanideActinide] {\SyntheticElementTextFormat{103}{262}{Lr}{Lawrencium}};

%%% Draw dotted lines connecting Lanthanide breakout to main table
%  \draw (LaLu.north west) edge[dotted] (La.north west)
%        (LaLu.north east) edge[dotted] (Lu.north east)
%        (LaLu.south west) edge[dotted] (La.south west)
%        (LaLu.south east) edge[dotted] (Lu.south east);
%%% Draw dotted lines connecting Actinide breakout to main table
%  \draw (AcLr.north west) edge[dotted] (Ac.north west)
%        (AcLr.north east) edge[dotted] (Lr.north east)
%        (AcLr.south west) edge[dotted] (Ac.south west)
%        (AcLr.south east) edge[dotted] (Lr.south east);

%% Legend
  \draw[black, AlkaliMetalFill] ($(La.north -| Fr.west) + (1em,-0.0em)$)
    rectangle +(1em, 1em) node[right, yshift=-1ex]{Alkali Metal};
  \draw[black, AlkalineEarthMetalFill] ($(La.north -| Fr.west) + (1em,-1.5em)$)
    rectangle +(1em, 1em) node[right, yshift=-1ex]{Alkaline Earth Metal};
  \draw[black, MetalFill] ($(La.north -| Fr.west) + (1em,-3.0em)$)
    rectangle +(1em, 1em) node[right, yshift=-1ex]{Metal};
  \draw[black, MetalloidFill] ($(La.north -| Fr.west) + (1em,-4.5em)$)
    rectangle +(1em, 1em) node[right, yshift=-1ex]{Metalloid};
  \draw[black, NonmetalFill] ($(La.north -| Fr.west) + (1em,-6.0em)$)
    rectangle +(1em, 1em) node[right, yshift=-1ex]{Non-metal};
  \draw[black, HalogenFill] ($(La.north -| Fr.west) + (1em,-7.5em)$)
    rectangle +(1em, 1em) node[right, yshift=-1ex]{Halogen};
  \draw[black, NobleGasFill] ($(La.north -| Fr.west) + (1em,-9.0em)$)
    rectangle +(1em, 1em) node[right, yshift=-1ex]{Noble Gas};
  \draw[black, LanthanideActinideFill] ($(La.north -| Fr.west) + (1em,-10.5em)$)
    rectangle +(1em, 1em) node[right, yshift=-1ex]{Lanthanide/Actinide};

%  \node at ($(La.north -| Fr.west) + (5em,-15em)$) [name=elementLegend, Element, fill=white]
%    {\NaturalElementTextFormat{Z}{mass}{Symbol}{Name}};
      \node[yshift=10cm] at ($(Mn.north -| Fe.west) + (5em,-15em)$) [name=elementLegend, Element, fill=white]
    {\NaturalElementTextFormat{Z}{mass}{Symbol}{Name}};
%  \node[Element, fill=white, right of=elementLegend, xshift=1em]
%    {\SyntheticElementTextFormat{}{}{man-made}{}} ;

%%% Diagram Title
%  \node at (H.west -| Fe.north) [name=diagramTitle, TitleLabel]
%    {Periodic Table of Chemical Elements};

\end{tikzpicture}
\caption{The Periodic Table}
\label{image3.4}
\end{figure*}
%%%%%%%%%%%%%%%%%%% PERIODIC TABLE

\item[\docfilehook{Classification of groups}{Classification of groups}] Some of the groups in the periodic table have specific names such as alkali metals, alkaline earth metals, transition metals, halogens or noble gases. Alkali metals are the group 1A elements: lithium (Li), sodium (Na), potassium (K), rubidium (Rb), cesium (Cs), and francium (Fr). Alkali elements are soft and shiny metals. They are also good conductors of heat and electricity and have low melting points. Alkali earth metals are the group 2A (2) elements: beryllium (Be), magnesium (Mg), calcium (Ca), strontium (Sr), barium (Ba), and radium (Ra). Transition metals are the elements from group 3 to 12 and they are in the middle of the table. Halogens are the group 7A (17) elements: fluorine (F), chlorine (Cl), bromine (Br), iodine (I), and astatine (At). Halogens are strongly reactive elements. Finally, noble gases are the group 8A (18) elements: helium (He), neon (Ne), argon (Ar), krypton (Kr), xenon (Xe), and radon (Rn). They are  inert and seldom combine with other elements in the periodic table.
\item[\docfilehook{How to classify Hydrogen}{How to classify Hydrogen}] Hydrogen (H) sometimes seems to be put in the wrong spot at the periodic table. Although it is located at the top of Group 1A (1), it is not an alkali metal, as it has very different properties. Thus hydrogen does not belong to the alkali metals, being a nonmetal. 
\item[\docfilehook{Metals, Nonmetals, and Metalloids}{Metals, Nonmetals, and Metalloids}] The elements of the periodic table can also be classify as metals, nonmetals, and metalloids. Metals are those elements on the left of the table with a blue color and nonmetals are the elements on the right of the table with a green color. The elements between metals and nonmetals are called metalloids and include only B, Si, Ge, As, Sb, Te, Po, and At. Metals are shiny solids and usually melt at higher temperatures. Some examples of metals are Gold (Au) or Iron (Fe). Nonmetals are often poor conductors of heat and electricity with low melting points. They also tend to be not very shiny, malleable, or ductile. Some examples of nonmetals are Carbon (C) or Nitrogen (N). Metalloids are elements that share some properties with metals and others with the nonmetals. For example, they are better conductors of heat and electricity than the nonmetals, but not as good as the metals. The metalloids are semiconductors because they can act as both conductors and insulators under certain conditions. A example of metalloids is Silicon (Si) that should not be confused from silicone, a chemical--and not an element--employed in prosthetics.
 \end{description}
\begin{example} %%%%%%%%%%%%%%%%%%%%%%%% EXAMPLE BOX
Answer the following questions:
\begin{enumerate}[label=(\alph*)]
\item Give the symbol or name the following elements: Au, Iron, Na and Iodine.
\item Give the group and period of the following elements, and give the name: Ca, Ir, and C.
\item Classify as alkali metal, alkali earth metal, transition metal, halogen or noble gas, and give the name: Mg, Li, Co, He, F.
\item Classify as metal, nonmetal or metalloid, and give the name: Ba, N, Si.
\end{enumerate}
\textlcsc{ \textcolor{dgreen}{\Large \textbf{Solution}} }\\
(a) Au is Gold. Iron is Fe and Iodine is I. (b) The period and group of Ca (Calcium) is 2 (2A) and 4. The period and group of Ir (Iridium) is 9 (8B) and 6. The period and group of C (Carbon) is 14 (IVA) and 2. (c) Mg (Magnesium) is a alkali earth metal. Li (Lithium) is a alkali  metal. Co (Cobalt) is a transition metal. He (Helium) is a noble gas. F (Fluorine) is an halogen. (d) Ba (Barium) is an metal. N (Nitrogen) is an nonmetal. Si (Sillicon) is an metalloid.\\
\faDiamond\ \textlcsc{ \textcolor{dgreen}{\Large \textbf{Study Check}} }\\
Answer the following questions:
\begin{enumerate}[label=(\alph*)]
\item Give the symbol or name the following elements: Ni.
\item Give the group and period of the following elements, and give the name: Cl.
\item Classify as alkali metal, alkali earth metal, transition metal, halogen or noble gas, and give the name: Ne.
\item Classify as metal, nonmetal or metalloid, and give the name: W.
\end{enumerate}
\flushright Answer: (a) Nickel; (b) Chlorine: G 17 (VIIA)  P3; (c) Neon Noble gas ; (d) Tungstein metal .
\end{example}%%%%%%%%%%%%%%%%%%%%%%%% EXAMPLE BOX

\section{The atom}
Each of the elements in the periodic table are made of atoms, which is the smallest piece of the element that retains the characteristics of that element. Atoms are also the building blocks of all the matter and the materials you use in your everyday life. This section covers the structure of the atom. You will learn how to calculate the number of particles that made an atom and differentiate the atoms of an element.
\begin{marginfigure}
\begin{tikzpicture}
\tikzset{
    pics/proton/.style={code={\shade[ball color=red] circle (3pt);}},
    pics/neutron/.style={code={\shade[ball color=white] circle (3pt);}},
    pics/nucleussmall/.style={code={%
        \pgfmathdeclarerandomlist{nucleon}{{proton}{proton}{neutron}{neutron}{neutron}}
        \pgfmathsetseed{#1+1}
        \foreach \A/\R in {8/0.2, 5/0.13, 1/0}{
        \pgfmathsetmacro{\S}{360/\A}
            \foreach \B in {0,\S,...,360}{
                \pgfmathrandomitem{\C}{nucleon}
                \pic at ($(\B+2*\A+5*rnd:\R)$) {\C}; } }} },
    pics/nucleusbig/.style={code={%
        \pgfmathdeclarerandomlist{nucleon}{{proton}{proton}{neutron}{neutron}{neutron}}
        \pgfmathsetseed{#1+1}
        \foreach \A/\R in {24/0.4, 24/0.3, 24/0.2, 13/0.35, 11/0.27, 6/0.15, 1/0}{
        \pgfmathsetmacro{\S}{360/\A}
            \foreach \B in {0,\S,...,360}{
                \pgfmathrandomitem{\C}{nucleon}
                \pic at ($(\B+2*\A+5*rnd:\R)$) {\C}; } }} },
    pics/nucleusbiggest/.style={code={%
        \pgfmathdeclarerandomlist{nucleon}{{proton}{proton}{neutron}{neutron}{neutron}}
        \pgfmathsetseed{#1+1}
        \foreach \A/\R in {24/0.5, 24/0.4, 24/0.3, 24/0.2, 13/0.47, 15/0.44, 13/0.37, 11/0.27, 6/0.15, 1/0}{
        \pgfmathsetmacro{\S}{360/\A}
            \foreach \B in {0,\S,...,360}{
                \pgfmathrandomitem{\C}{nucleon}
                \pic at ($(\B+2*\A+5*rnd:\R)$) {\C}; } }} },
    }
\pic at (0,0) {nucleussmall} node at  (1,0) {\ce{^{12}_{6}C}} ;
\pic at (0,2) {nucleusbig=1} node at  (1,2) {\ce{^{44}_{20}Ca}};
\pic at (0,4) {nucleusbiggest=1} node at  (1,4) {\ce{^{197}_{79}Au}} ;    
\end{tikzpicture}
      \caption{The bigger atomic mass the larger the nucleus}
   \end{marginfigure}
   
\sloppy
\begin{description}
\item[\docfilehook{Atomic Structure}{Atomic Structure}] 
Atoms contain three atomic particles: the proton, neutron, and electron. Protons have positive charge (\ce{+}), and electrons carry negative charge (\ce{-}). Neutrons on the other hand are neutral--they have no electrical charge. The protons and neutrons are located in the core of the atom, which is called the nucleus, and account for the mass of the atom. Electrons are located in the exterior part of the atoms. When an atoms is neutral it has no charge and the number of electrons and protons are the same. Some atoms have positive charge, resulting of removing electrons. Others can have negative charge as a result of accepting a negatively charged electron.
\item[\docfilehook{Atomic and mass number}{Atomic and mass number}] 
Elements are made of atoms, and each atom of an element is characterized by the atomic number (Z) and the mass number (A). The atomic number (Z) of an element indicates the number of electrons of an atom. The mass number (A) of an element indicates the combined number of protons and neutrons of an atom. Both A and Z for an atom X are indicated in the following form: 
\begin{center}\ce{^{A}_{Z}X}\end{center} 
This is called the isotope notation. As an example  \ce{^{24}_{12}Mg} means that the atomic number of Mg is Z=12 and the mass number is A=24. The atomic number can be found in the periodic table whereas the mass number A is not on the table. By means of the isotope notation, one can quickly identify the number of protons, neutrons and electrons in an atom. As the atomic number is always indicated on the bottom part, we can quickly identify the number of electrons in an atom. At the same time, the number of electrons and protons in a neutral atom is the same--neutral means an atom without a charge. The number of neutrons, corresponds to the mass number minus the atomic number. 
\begin{example} %%%%%%%%%%%%%%%%%%%%%%%% EXAMPLE BOX
Calculate the number of protons, neutrons and electrons of the following atoms and identify the isotopes:
\begin{multicols}{4}
\begin{enumerate}[label=(\alph*)]
\item \ce{^{27}_{12}Mg} 
\item \ce{^{22}_{10}Ne} 
\item \ce{^{20}_{10}Ne} 
\end{enumerate}
\end{multicols}
\textlcsc{ \textcolor{dgreen}{\Large \textbf{Solution}} }\\
(a)  \ce{^{27}_{12}Mg}  has 12 electrons (Z=12) and 12 protons as well (the number of electrons and protons are the same if the atom is neutral), and 15 neutrons, as 27-12=15. (b) \ce{^{22}_{10}Ne}  has 10 electrons and 10 protons, and 12 neutrons. (c)  \ce{^{20}_{10}Ne} has10 electrons and 10 protons, and 10 neutrons as well.  \ce{^{22}_{10}Ne} and \ce{^{20}_{10}Ne} are both isotopes of Neon.
\\
\faDiamond\ \textlcsc{ \textcolor{dgreen}{\Large \textbf{Study Check}} }\\
Calculate the number of protons, neutrons and electrons of the following atoms:
\begin{multicols}{3}
\begin{enumerate}[label=(\alph*)]
\item \ce{^{32}_{16}S} 
\item \ce{^{34}_{16}S} 
\item \ce{^{36}_{16}S} 
\end{enumerate}\end{multicols}
\flushright Answer: (a) 16p, 16e and 16 n; (b) 16p, 16e and 18 n; (a) 16p, 16e and 20 n. 
\end{example}%%%%%%%%%%%%%%%%%%%%%%%% EXAMPLE BOX


\begin{marginfigure}
    \begin{shadequote}[l]{Democritus}
Nothing exists except atoms and empty space; everything else is opinion.
\end{shadequote}   \end{marginfigure}

\item[\docfilehook{Isotopes }{}] 
All atoms of the same element are not the same. Some are heavier whereas others are lighter. Isotopes are atoms of the same element with different numbers of neutrons. For example:  \ce{^{24}_{12}Mg},  \ce{^{25}_{12}Mg} and  \ce{^{26}_{12}Mg} are three isotopes of Mg.  \ce{^{27}_{12}Mg}  is heavier than  \ce{^{24}_{12}Mg} as it contain more neutrons and protons in the nucleus. Each of the isotopes has a specific abundance. Some are more abundant than other. For example, the abundance of \ce{^{24}_{12}Mg} is 79\%, and the abundance of  \ce{^{25}_{12}Mg} and  \ce{^{26}_{12}Mg} is 10\% and 11\%, respectively. 

\item[\docfilehook{Atomic mass }{}] 
The atomic mass represents the mass of the atoms of an element. The units of atomic mass are called \emph{amu}, which stands for atomic mass units. This value can be simply found at any periodic table. Using the periodic table provided in this manual \ref{image3.4}, you can find the atomic mass of each element on top of the symbol at the right side. For example, the atomic mass of oxygen (\ce{O}) is 15.999 amu and the atomic mass of nitrogen (\ce{N}) is 14.007 amu. As atoms are made of numerous isotopes--this means different atoms of the same element but with different number of neutrons and hence different weight--the atomic mass you find in the periodic table is the result of including the mass of the different isotopes. That is you need to do an average of the mass of each isotope using values of abundance. In another words: the \textit{atomic mass} of an element, expressed in $amu$ (atomic mass units), is the weighted average of the masses of the individual isotopes of the element. For an element with $n$ isotopes with different masses ($A_1$, $A_2$, ..., $A_n$) and different fractional abundances for each isotope ($f_1$, $f_2$, ..,$f_n$ ), the atomic mass is given by
\[\text{Atomic mass}=\sum_{i=1}^{n} A_i\cdot f_i\]
\begin{example} %%%%%%%%%%%%%%%%%%%%%%%% EXAMPLE BOX
Naturally occurring copper (Cu) consists of $69.17\%$ \ce{^{63}Cu} and $30.83\%$ \ce{^{65}Cu}. The mass of \ce{^{63}Cu} is 62.939598 amu, and the mass of \ce{^{65}Cu} is 64.927793 amu. What is the atomic mass of copper?\\
\textlcsc{ \textcolor{dgreen}{\Large \textbf{Solution}} }\\
(The weighted average is the sum of the mass of each isotope times its fractional abundance. We have that the isotope \ce{^{63}Cu} has a mass of 62.939598 amu and an abundance of $69.17\%$, that is the same as $0.6917$. Hence, we need to time these two values. At the same time, the isotope  \ce{^{65}Cu} has a mass of 64.927793 amu and an abundance of $0.3083$. We also need to time these two values as well. After we add both contributions, we have:
\[ 62.939598\:amu\times\frac{69.17}{100}   +
64.927793\:amu\times\frac{30.83 }{100} = 63.55\:amu\]

\faDiamond\ \textlcsc{ \textcolor{dgreen}{\Large \textbf{Study Check}} }\\
Copper is made up of two isotopes, Cu-63 (62.9296 amu) and Cu-65 (64.9278 amu). Calculate the percent abundance of each isotope knowing that copper's atomic weight is 63.546amu.
\flushright Answer: 69.15\% and 30.84\%. 
\end{example}%%%%%%%%%%%%%%%%%%%%%%%% EXAMPLE BOX
\end{description}




\section{An introduction to molecules}
The periodic table contains all elements in nature. Elements can combine to form molecules. For example, in the air you can find traces of Argon--this is an element--but you can also find water (\ce{H2O}), that results from the combination of two elements, hydrogen (\ce{H}) and oxygen (\ce{O}). This section will first introduce you some properties of molecules, without covering their names--this will covered in the following.
\sloppy
\begin{description}
\item[\docfilehook{Molecular weight}{}] 
Here are two example of molecules: molecular oxygen \ce{O2} and molecular nitrigen \ce{N2}. The subscript "2" indicate that each molecule contains two atoms. For example, a \ce{O2} molecular is made of two oxygen atoms. These two molecules have different weight. How do we calculate the weight of a set of molecules? This is actually called the molecular weight (MW), and you can also use different terms to refer to the same property such as: molecular mass, molar mass. All these terms indeed mean the weight a set of molecules. We can calculate the MW by adding the weight of each atom that form the molecule.
\item[\docfilehook{Units of molecular weight}{}] The units of molecular weight are the same as the units of atomic weight: amu, atomic mass units.

\begin{example} %%%%%%%%%%%%%%%%%%%%%%%% EXAMPLE BOX
Calculate: (a) The atomic weight of O; (b) the molecular mass of molecular oxygen, \ce{O2} \\
\textlcsc{ \textcolor{dgreen}{\Large \textbf{Solution}} }\\
(a) According to the periodic table the atomic weight (AW) of Mg is $15.999$ amu. (b) The molar mass of \ce{O2} is the result of adding the atomic masses of 2O atoms, that is $31.998$ amu, close to 32 amu.\\
\faDiamond\ \textlcsc{ \textcolor{dgreen}{\Large \textbf{Study Check}} }\\
Calculate the molar mass of water \ce{H2O} and ammonia, \ce{NH3}\\
\flushright Answer: $18$ and $17$ amu.
\end{example}%%%%%%%%%%%%%%%%%%%%%%%% EXAMPLE BOX

\end{description}

\section{Empirical and molecular formula of a chemical}
You will find two types of formulas: molecular formulas and empirical formulas. A molecular formula (MF) is the real formula of a chemical such as for example hydrogen peroxide \ce{H2O2}. Differently, the empirical formula (EF) is a simplified formula resulting of dividing the molecular formula by the smallest integer number--different than one. So if \ce{H2O2} is the molecular formula of sodium peroxide, \ce{HO} is the empirical formula of the same chemical; as you can see the empirical formula is simplified. The use of empirical formulas comes from the fact that the formulas of all chemicals actually come from experiments, and from experiments one normally can only obtain empirical formulas. The word empirical means "from an experiment".
\sloppy
\begin{description}
\item[\docfilehook{Molecular weight of empirical formulas and molecular formulas }{}] The molecular weight of MF and EF are related by the following formula:
\resizeableyellownote{2.5}{1}{Add this formula to your flashcard.}
\begin{equation*}\begin{split}
\boxed{  n\cdot MW_{EF} = MW_{MF} } 
\end{split}\end{equation*}
where:
\begin{where}
 \item $MW_{EF}$   is the molecular weight of the empirical formula
 \item $MW_{MF}$   is the molecular weight of the molecular formula
\item $n$ is a integer number such as 1, 2, 3...
\end{where}
Empirical formula are just simplified formulas. So when we think about molecular weight we normally have the molecular weight using the real molecular formula in mind. Let us work on an example:
\begin{example} %%%%%%%%%%%%%%%%%%%%%%%% EXAMPLE BOX
The empirical formula of dichloromethane is \ce{ClCH2} and the molecular weigh of the chemical is 98amu. Calculate the molecular formula of dichloromethane.\\
\textlcsc{ \textcolor{dgreen}{\Large \textbf{Solution}} }\\
Given the empirical formula of dichloromethane one can think of many different molecular formulas, for example: \ce{Cl3C3H6} or \ce{Cl2C2H4}. From these, and many other, there is only on real molecular formula. How do we calculate the real molecular formula? By comparing the MW of the molecular and empirical formula we can figure out the number of time we need to multiply the MF to obtain the EF. We know the MW is 98amu. Using the EF we can also calculate a MW: $35+12+2\cdot 1=49$ amu. If we compare both number using the formula:
\[n\cdot MW_{EF}= MW_{MF}\]
we have: $49\cdot n= 98$; solving we have $n=2$. Therefore the MF is: \ce{Cl2C2H4}.
\\
\faDiamond\ \textlcsc{ \textcolor{dgreen}{\Large \textbf{Study Check}} }\\
The empirical formula of dinitrogen tetroxide is \ce{NO2} and the molecular weigh of the chemical is 88amu. Calculate the molecular formula of dinitrogen tetroxide.\\
\flushright Answer: \ce{N2O4}.
\end{example}%%%%%%%%%%%%%%%%%%%%%%%% EXAMPLE BOX
\end{description}

\section{Determining empirical formulas}
 We said that the real formula of a chemical is the molecular formula and therefore the real molecular weight of a chemical comes from these formulas. The empirical formulas are obtained from experiments in which a chemical is fragmented and analyzed so that the elements in the molecule and the percentage of each element is determined. Let us work on an example in order to learn the procedure of obtaining molecular formulas. 
\sloppy
\begin{description}
\item[\docfilehook{Calculating molecular formulas }{}] By means of a experiment, we want to calculate the empirical formula of a chemical given that the chemical contains 2.8 g of nitrogen and 6.4 g of oxygen. In order to calculate the EF we will set up a table. In each column we will add each of the elements that form the molecule. In the first row we will include the grams of each element, in the second we will divide the grams of each element by its atomic weight (AW(N)=14amu,AW(O)=16amu). Among all numbers of the second row, we will select the smallest. Once we have the smallest, we will divide all numbers by the smallest and that will give us round numbers that are the numbers in a empirical formula.
\begin{center}\fontfamily{ppl}\selectfont
\begin{tabular}{lll}
\toprule
\multicolumn{3}{c}{Empirical Formula Table} \\
\midrule
 &  N  &O \\
\midrule
  Grams &  2.8g  &6.4g \\
\midrule
  AW &  14  &16 \\
\midrule
  Grams/AW &  0.2  &0.4 \\
\midrule
  $\div$ by smallest &  1  &2 \\
  \midrule
  Formula & \multicolumn{2}{c}{\ce{N1O2}=\ce{NO2}}     \\
\bottomrule
\end{tabular}\end{center}


\begin{example} %%%%%%%%%%%%%%%%%%%%%%%% EXAMPLE BOX
The mass percentage composition of a compound is: 18.59\% O, 37.25\% S, and 44.16\% F. Calculate its empirical formula.\\
\textlcsc{ \textcolor{dgreen}{\Large \textbf{Solution}} }\\
We will set up the the molecular formula table, knowing that the percentage are mass percentages, that is the mass of each element in the chemical, hence they should go in the grams row. Also the atomic weights of O, S and F are 16, 32 and 19 amu.
\begin{center}\fontfamily{ppl}\selectfont
\begin{tabular}{llll}
\toprule
\multicolumn{4}{c}{Empirical Formula Table} \\
\midrule
 &  O  &S & F\\
\midrule
  Grams &  0.1859g  & 0.3725g & 0.4416g \\
\midrule
   AW &  16 &32 &19 \\
\midrule
  Grams/AW &  0.0116  &0.0116 & 0.0232 \\
\midrule
  $\div$ by smallest &  1  &1 &2 \\
  \midrule
  Formula & \multicolumn{2}{c}{\ce{OSF2}}     \\
\bottomrule
\end{tabular}\end{center}
\faDiamond\ \textlcsc{ \textcolor{dgreen}{\Large \textbf{Study Check}} }\\
What is the empirical formula of a compound if a sample contains 10.28 g of C, 1.71 H and 12.71 g of oxygen?\\
\flushright Answer: \ce{CH2O}.
\end{example}%%%%%%%%%%%%%%%%%%%%%%%% EXAMPLE BOX

\end{description}




\end{document}

