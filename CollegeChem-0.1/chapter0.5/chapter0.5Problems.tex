\documentclass[main.tex]{subfiles}
\begin{document}\newpage
\setdoublesep{0.35700 em}  % 'Bond Spacing'
\setatomsep{1.78500 em}    % 'Fixed Length'
\setbondoffset{0.18265 em} % 'Margin Width'
\newcommand{\bondwidth}{0.06642 em} % 'Line Width'
\setbondstyle{line width = \bondwidth}
\newgeometry{left=0.8in,right=0.8in, top=2.5cm,bottom=2cm}
\fancyhfoffset[E,O]{0pt}
\setlength{\columnsep}{30pt}
\begin{conclusion}
\end{conclusion}
\setstretch{0.3}
\begin{multicols*}{2}

{\raggedright\textsc{\textbf{The Periodic Table }}\par}


\begin{enumerate}



%\item Classify a copper wire as
%\begin{enumerate}[label=(\alph*)]
%\begin{multicols*}{2}
%\item Element
%\item Compound
%\item Homogeneous mixture
%\item Heterogeneous mixture
%\item none of the above
%\end{multicols*}\flushright  {\small Ans: (a)}
%\end{enumerate}
%
%\item Classify a chocolate cookie as
%\begin{enumerate}[label=(\alph*)]
%\begin{multicols*}{2}
%\item Element
%\item Compound
%\item Homogeneous mixture
%\item Heterogeneous mixture
%\item none of the above
%\end{multicols*}\flushright  {\small Ans: (d)}
%\end{enumerate}
%
%\item Classify vinegar as
%\begin{enumerate}[label=(\alph*)]
%\begin{multicols*}{2}
%\item Element
%\item Compound
%\item Homogeneous mixture
%\item Heterogeneous mixture
%\item none of the above
%\end{multicols*}\flushright  {\small Ans: (c)}
%\end{enumerate}
%
%\item Classify ice as
%\begin{enumerate}[label=(\alph*)]
%\begin{multicols*}{2}
%\item Element
%\item Compound
%\item Homogeneous mixture
%\item Heterogeneous mixture
%\item none of the above
%\end{multicols*}\flushright  {\small Ans: (b)}
%\end{enumerate}
%
%\item Classify baking soda as
%\begin{enumerate}[label=(\alph*)]
%\begin{multicols*}{2}
%\item Element
%\item Compound
%\item Homogeneous mixture
%\item Heterogeneous mixture
%\item none of the above
%\end{multicols*}\flushright  {\small Ans: (b)}
%\end{enumerate}
%\item Classify aluminum foil as
%\begin{enumerate}[label=(\alph*)]
%\begin{multicols*}{2}
%\item Element
%\item Compound
%\item Homogeneous mixture
%\item Heterogeneous mixture
%\item none of the above
%\end{multicols*}\flushright  {\small Ans: (a)}
%\end{enumerate}
%
%\item Classify vitamin A as
%\begin{enumerate}[label=(\alph*)]
%\begin{multicols*}{2}
%\item Element
%\item Compound
%\item Homogeneous mixture
%\item Heterogeneous mixture
%\item none of the above
%\end{multicols*}\flushright  {\small Ans: (b)}
%\end{enumerate}

\item The atomic symbol for Gold is:
\begin{enumerate}[label=(\alph*)]
\begin{multicols*}{3}
\item Go
\item Au
\item G
\item Ca
\item Ol
\end{multicols*}\flushright  {\small Ans: (b)}
\end{enumerate}

\item The atomic symbol for aluminum is:
\begin{enumerate}[label=(\alph*)]
\begin{multicols*}{3}
\item Al
\item Am
\item A
\item Sn
\item Ag
\end{multicols*}\flushright  {\small Ans: (a)}
\end{enumerate}

\item The atomic symbol for iron is:
\begin{enumerate}[label=(\alph*)]
\begin{multicols*}{3}
\item Ir
\item Fs
\item Fe
\item In
\item Ir
\end{multicols*}\flushright  {\small Ans: (c)}
\end{enumerate}

\item Ca is the symbol for:
\begin{enumerate}[label=(\alph*)]
\begin{multicols*}{3}
\item Carbon
\item Calcium
\item Cobalt
\item Copper
\item Cadmium
\end{multicols*}\flushright  {\small Ans: (b)}
\end{enumerate}

\item Which of the following elements is a metal?
\begin{enumerate}[label=(\alph*)]
\begin{multicols*}{3}
\item Nitrogen
\item Lithium
\item Calcium
\item Iron
\item Iodine
\end{multicols*}\flushright  {\small Ans: (d)}
\end{enumerate}

\item Which of the following elements is a alkaline metal?
\begin{enumerate}[label=(\alph*)]
\begin{multicols*}{3}
\item Nitrogen
\item Lithium
\item Calcium
\item Iron
\item Iodine
\end{multicols*}\flushright  {\small Ans: (b)}
\end{enumerate}

\item Which of the following elements is a nonmetal?
\begin{enumerate}[label=(\alph*)]
\begin{multicols*}{3}
\item Nitrogen
\item Lithium
\item Calcium
\item Iron
\item Iodine
\end{multicols*}\flushright  {\small Ans: (a)}
\end{enumerate}

\item Which of the following elements is a halogen?
\begin{enumerate}[label=(\alph*)]
\begin{multicols*}{3}
\item Nitrogen
\item Lithium
\item Calcium
\item Iron
\item Iodine
\end{multicols*}\flushright  {\small Ans: (e)}
\end{enumerate}

\item What is the symbol of the element in Period 4 and Group 2?
\begin{enumerate}[label=(\alph*)]
\begin{multicols*}{3}
\item Be
\item Mg
\item Ca
\item C
\item Si
\end{multicols*}\flushright  {\small Ans: (c)}
\end{enumerate}



{\raggedright\textsc{\textbf{The Atom }}\par}

\item In an atom, the nucleus contains 
\begin{enumerate}[label=(\alph*)]
\item an equal number of protons and electrons.
\item all the protons and neutrons.
\item all the protons and electrons.
\item only neutrons.
\item only protons.
\flushright  {\small Ans: (b)}
\end{enumerate}

\item The atomic number of an atom is equal to the number of  
\begin{enumerate}[label=(\alph*)]
\begin{multicols*}{2}
\item nuclei
\item neutrons
\item neutrons plus protons.
\item electrons plus protons.
\item electrons
\end{multicols*}
\flushright  {\small Ans: (e)}
\end{enumerate}

\item The mass number of an atom is equal to the number of  
\begin{enumerate}[label=(\alph*)]
\begin{multicols*}{2}
\item nuclei
\item neutrons
\item neutrons plus protons.
\item electrons plus protons.
\item electrons
\end{multicols*}
\flushright  {\small Ans: (c)}
\end{enumerate}



\item The mass number of an atom is equal to the number of  
\begin{enumerate}[label=(\alph*)]
\begin{multicols*}{2}
\item electrons
\item neutrons
\item neutrons plus protons.
\item  protons
\end{multicols*}
\flushright  {\small Ans: (c)}
\end{enumerate}



\item Consider a neutral atom with 30 protons and 34 neutrons. The atomic number of the element is
\begin{enumerate}[label=(\alph*)]
\begin{multicols*}{3}
\item 30
\item 32
\item 34
\item  64
\item 94
\end{multicols*}\flushright  {\small Ans: (a)}
\end{enumerate}

\item Consider a neutral atom with 30 protons and 34 neutrons. The mass number of the element is
\begin{enumerate}[label=(\alph*)]
\begin{multicols*}{3}
\item 30
\item 32
\item 34
\item  64
\item 94
\end{multicols*}\flushright  {\small Ans: (d)}
\end{enumerate}

\item The atomic mass of Ga is 69.72 amu. There are only two naturally occurring isotopes of gallium: 69Ga, with a mass of 69.0 amu, and 71Ga, with a mass of 71.0 amu. Calculate the natural abundance of the 69Ga isotope. 
\begin{flushright}\small Ans: 64\% \end{flushright}



{\raggedright\textsc{\textbf{An introduction to molecules }}\par}

\item Calculate the molecular mass of the following molecule: \ce{CCI2F2}
\begin{flushright}\small Ans: 121 amu \end{flushright}

\item Calculate the molecular mass of the following molecule: \ce{C4H10}
\begin{flushright}\small Ans: 58 amu \end{flushright}

\item Calculate the molecular mass of the following molecule: \ce{C4H10}
\begin{flushright}\small Ans: 58 amu \end{flushright}
\item Calculate the molecular mass of the following molecule: \ce{C6H10O8}
\begin{flushright}\small Ans: 210 amu \end{flushright}

{\raggedright\textsc{\textbf{Empirical and Molecular Formulas }}\par}

\item What is the empirical formula of a compound if a sample contains 2.8 g of nitrogen and 3.2 g of oxygen?
\begin{flushright}\small Ans: \ce{NO} \end{flushright}


\item What is the empirical formula and the molecular formula of a compound if a sample contains 3 g of C, 0.5 H and 4 g of oxygen? MW=60amu
\begin{flushright}\small Ans: \ce{CH2O} \end{flushright}


\item What is the empirical and molecular formula of a compound with a percent composition of 49.47\% C, 5.201\% H, 28.84\% N, and 16.48\% O, if its molecular mass is 194.2 amu.
\begin{flushright}\small Ans: \ce{C4H5N2O} \end{flushright}

\item A 1.587 g sample of a compound containing N and O was analyzed finding a composition of 0.483 g of Nitrogen and 1.104 g of Oxygen. Calculate the empirical formula of the compound.
\begin{flushright}\small Ans: \ce{NO2} \end{flushright}

\restoregeometry
\end{enumerate}
\end{multicols*}
\pagecolor{green!10}\afterpage{\nopagecolor}\newpage
\end{document}

