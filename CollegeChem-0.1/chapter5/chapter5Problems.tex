\documentclass[main.tex]{subfiles}
\begin{document}\newpage
\setdoublesep{0.35700 em}  % 'Bond Spacing'
\setatomsep{1.78500 em}    % 'Fixed Length'
\setbondoffset{0.18265 em} % 'Margin Width'
\newcommand{\bondwidth}{0.06642 em} % 'Line Width'
\setbondstyle{line width = \bondwidth}
\newgeometry{left=0.8in,right=0.8in, top=2.5cm,bottom=2cm}
\fancyhfoffset[E,O]{0pt}
\setlength{\columnsep}{30pt}
\begin{conclusion}
\end{conclusion}
\setstretch{0.3}
\begin{multicols*}{2}

{\raggedright\textsc{\textbf{Ions \& Ionic charges}}\par}

\begin{enumerate}

\item \ce{Fe^{2+}} is:
\begin{enumerate}[label=(\alph*)]
\begin{multicols*}{2}
\item an atom
\item an element
\item an anion
\item a cation
\item none of the above
\end{multicols*}\flushright  {\small Ans: (d)}
\end{enumerate}

\item \ce{F^{-}} is:
\begin{enumerate}[label=(\alph*)]
\begin{multicols*}{2}
\item an atom
\item an element
\item an anion
\item a cation
\item none of the above
\end{multicols*}\flushright  {\small Ans: (c)}
\end{enumerate}

\item \ce{Cl} is:
\begin{enumerate}[label=(\alph*)]
\begin{multicols*}{2}
\item an atom
\item an ion
\item an anion
\item a cation
\item none of the above
\end{multicols*}\flushright  {\small Ans: (a)}
\end{enumerate}



\item Identify the correct ionic state of Mg :
\begin{enumerate}[label=(\alph*)]
\begin{multicols*}{2}
\item \ce{Mg^{2+}}
\item \ce{Mg^{2-}}
\item \ce{Mg^{+}}
\item \ce{Mg^{2-}}
\item \ce{Mg}
\end{multicols*}\flushright  {\small Ans: (a)}
\end{enumerate}

\item Identify the correct ionic state of O:
\begin{enumerate}[label=(\alph*)]
\begin{multicols*}{2}
\item \ce{O^{2+}}
\item \ce{O^{2-}}
\item \ce{O}
\item \ce{O^{+}}
\item \ce{O^{-}}
\end{multicols*}\flushright  {\small Ans: (b)}
\end{enumerate}


\item Identify the correct ionic state of N:
\begin{enumerate}[label=(\alph*)]
\begin{multicols*}{2}
\item \ce{N^{2+}}
\item \ce{N^{2-}}
\item \ce{N^{3-}}
\item \ce{N}
\item \ce{N^{-}}
\end{multicols*}\flushright  {\small Ans: (c)}
\end{enumerate}



{\raggedright\textsc{\textbf{Covalent Compounds}}\par}

\item Name or formulate the following compound: \ce{NO}
\begin{enumerate}[label=(\alph*)]
\begin{multicols*}{2}
\item nitrogen monoxide
\item nitrogen(I) oxide
\item nitrogen(II) oxide
\item mononitrogen monoxide
\item oxygen nitride
\end{multicols*}\flushright  {\small Ans: (a)}
\end{enumerate}

\item Name or formulate the following compound: phosphorus trichloride
\begin{enumerate}[label=(\alph*)]
\begin{multicols*}{2}
\item \ce{PCl5}
\item \ce{PCl3}
\item \ce{PCl2}
\item \ce{P3Cl}
\item \ce{PCl}
\end{multicols*}\flushright  {\small Ans: (b)}
\end{enumerate}


\item Name or formulate the following compound: \ce{SO3}
\begin{enumerate}[label=(\alph*)]
\begin{multicols*}{2}
\item sulfur(III) oxide
\item sulfur(VI) oxide
\item sulfur trioxide
\item trisulfur monoxide
\item trisulfur oxide
\end{multicols*}\flushright  {\small Ans: (c)}
\end{enumerate}

\item Name or formulate the following compound: sulfur hexafluoride
\begin{enumerate}[label=(\alph*)]
\begin{multicols*}{2}
\item \ce{SF6}
\item \ce{S6F}
\item \ce{SF}
\item \ce{SF2}
\item \ce{S6F3}
\end{multicols*}\flushright  {\small Ans: (a)}
\end{enumerate}


{\raggedright\textsc{\textbf{Ionic Compounds}}\par}

\item Name or formulate the following compound: \ce{Li3N}
\begin{enumerate}[label=(\alph*)]
\begin{multicols*}{2}
\item Trilithium nitride
\item Lithium(I) nitride
\item Lithium nitride
\item Lithium Dinitride
\item Lithium trinitride
\end{multicols*}\flushright  {\small Ans: (c)}
\end{enumerate}

\item Name or formulate the following compound: \ce{MgCl2}
\begin{enumerate}[label=(\alph*)]
\begin{multicols*}{2}
\item Magnesium(II) chloride
\item Magnesium chloride(II)
\item Magnesium Dichloride
\item Dimagnesium chloride
\item Magnesium chloride
\end{multicols*}\flushright  {\small Ans: (e)}
\end{enumerate}

\item Name or formulate the following compound: calcium sulfide
\begin{enumerate}[label=(\alph*)]
\begin{multicols*}{2}
\item \ce{Ca2S2}
\item \ce{CaS2}
\item \ce{Ca2S}
\item \ce{CaS}
\item \ce{Ca2S1}
\end{multicols*}\flushright  {\small Ans: (d)}
\end{enumerate}

\item Combine the following ions: \ce{Ba^{2+}} and \ce{P^{3-}} 
\begin{enumerate}[label=(\alph*)]
\begin{multicols*}{2}
\item \ce{BaP}
\item \ce{Ba2P3}
\item \ce{Ba3P2}
\item \ce{BaP2}
\item \ce{Ba3P3}
\end{multicols*}\flushright  {\small Ans: (c)}
\end{enumerate}


\item Name or formulate the following compound: \ce{VO}
\begin{enumerate}[label=(\alph*)]
\begin{multicols*}{2}
\item Vanadium monoxide
\item Vanadium(I) oxide
\item Vanadium(II) oxide
\item Vanadium oxide
\end{multicols*}\flushright  {\small Ans: (c)}
\end{enumerate}




\item Name or formulate the following compound: \ce{Ni2O3}
\begin{enumerate}[label=(\alph*)]
\begin{multicols*}{2}
\item Nickel(II) oxide
\item Nickel(I) oxide
\item Nickel oxide
\item Nickel monoxide
\end{multicols*}\flushright  {\small Ans: (a)}
\end{enumerate}

\item Name or formulate the following compound: Iron(III) chloride
\begin{enumerate}[label=(\alph*)]
\begin{multicols*}{2}
\item \ce{FeCl3}
\item \ce{FeCl2}
\item \ce{Fe3Cl}
\item \ce{Fe2Cl3}
\item \ce{FeCl}
\end{multicols*}\flushright  {\small Ans: (a)}
\end{enumerate}



{\raggedright\textsc{\textbf{Acids and Bases Naming}}\par}


\item  Name or formulate the following compound: hydrochloric Acid
\begin{enumerate}[label=(\alph*)]
\begin{multicols*}{2}
\item \ce{HI}
\item \ce{HClO4}
\item \ce{HCl}
\item \ce{HClO3}
\item \ce{ClH}
\end{multicols*}\flushright  {\small Ans: (c)}
\end{enumerate}



\item  Name or formulate the following compound: hydroiodic Acid
\begin{enumerate}[label=(\alph*)]
\begin{multicols*}{2}
\item \ce{HI}
\item \ce{HClO4}
\item \ce{HCl}
\item \ce{HClO3}
\item \ce{ClH}
\end{multicols*}\flushright  {\small Ans: (a)}
\end{enumerate}


\item  Name or formulate the following compound: perchloric acid
\begin{enumerate}[label=(\alph*)]
\begin{multicols*}{2}
\item \ce{HMnO4}
\item \ce{H2CO3}
\item \ce{HNO3}
\item \ce{H2SO4}
\item \ce{HClO4}
\end{multicols*}\flushright  {\small Ans: (e)}
\end{enumerate}

\item  Name or formulate the following compound: nitric acid
\begin{enumerate}[label=(\alph*)]
\begin{multicols*}{2}
\item \ce{HMnO4}
\item \ce{H2CO3}
\item \ce{HNO3}
\item \ce{H2SO4}
\item \ce{HClO4}
\end{multicols*}\flushright  {\small Ans: (c)}
\end{enumerate}

\item  Name or formulate the following compound: permanganic acid
\begin{enumerate}[label=(\alph*)]
\begin{multicols*}{2}
\item \ce{HMnO4}
\item \ce{H2CO3}
\item \ce{HNO3}
\item \ce{H2SO4}
\item \ce{HClO4}
\end{multicols*}\flushright  {\small Ans: (a)}
\end{enumerate}

\item  Name or formulate the following compound: sulfuric acid
\begin{enumerate}[label=(\alph*)]
\begin{multicols*}{2}
\item \ce{HMnO4}
\item \ce{H2CO3}
\item \ce{HNO3}
\item \ce{H2SO4}
\item \ce{HClO4}
\end{multicols*}\flushright  {\small Ans: (d)}
\end{enumerate}


\item  Name or formulate the following compound: \ce{Ca(OH)2}
\begin{enumerate}[label=(\alph*)]
\begin{multicols*}{2}
\item calcium hydroxyde
\item calcium(II) hydroxyde
\item calcium(I) hydroxyde
\item Calcium dihydroxide
\item Calcium hydrate
\end{multicols*}\flushright  {\small Ans: (a)}
\end{enumerate}



{\raggedright\textsc{\textbf{Naming of oxosalts, hydrosalts, hydrates  \& common chemicals}}\par}
\item  Combine the chemical: \ce{SO4^{2-}} and \ce{Ca^{2+}}
\begin{enumerate}[label=(\alph*)]
\begin{multicols*}{2}
\item \ce{CaSO4}
\item \ce{Ca2(SO4)2}
\item \ce{Ca2SO4}
\item \ce{CaSO3}
\item \ce{CaS4}
\end{multicols*}\flushright  {\small Ans: (a)}
\end{enumerate}

\item  Give the ions forming the chemical: \ce{KNO3}
\begin{enumerate}[label=(\alph*)]
\begin{multicols*}{2}
\item \ce{NO4^{-}} and \ce{K^{+}}
\item  \ce{NO3^{-2}} and \ce{K^{+}}
\item \ce{NO3^{-2}} and \ce{K^{+2}}
\item  \ce{NO3^{-}} and \ce{K^{+}}
\item \ce{NO3^{-}} and \ce{K^{2+}}
\end{multicols*}\flushright  {\small Ans: (d)}
\end{enumerate}


%\item  Name or formulate the following compound: \ce{LiNO3}
%\begin{enumerate}[label=(\alph*)]
%\begin{multicols*}{2}
%\item Lithium nitrate
%\item Lithium nitrogen trioxide 
%\item Lithium(I) nitrate
%\item  Lithium nitrogen oxide
%\item Lithium nitrite
%\end{multicols*}\flushright  {\small Ans: (a)}
%\end{enumerate}

\item  Name or formulate the following compound: \ce{Na2CO3}
\begin{enumerate}[label=(\alph*)]
\begin{multicols*}{2}
\item Sodium bicarbonate
\item  Sodium(I) carbonate
\item Sodium(II) carbonate
\item  Sodium carbonate
\item Sodium carbon oxide
\end{multicols*}\flushright  {\small Ans: (d)}
\end{enumerate}

\item  Name or formulate the following compound: \ce{FeCO3}
\begin{enumerate}[label=(\alph*)]
\begin{multicols*}{2}
\item Iron(II) carbonate
\item  Iron(I) carbonate
\item Iron carbonate
\item Iron carbon oxide
\item Iron(III) carbonate
\end{multicols*}\flushright  {\small Ans: (a)}
\end{enumerate}


\item  Name or formulate the following compound: potassium permanganate
\begin{enumerate}[label=(\alph*)]
\begin{multicols*}{2}
\item \ce{KMnO}
\item  \ce{K2Mn}
\item \ce{K2MnO4}
\item \ce{KMnO3}
\item \ce{KMnO4}
\end{multicols*}\flushright  {\small Ans: (e)}
\end{enumerate}

\item  Name or formulate the following compound: sodium bicarbonate(common name)
\begin{enumerate}[label=(\alph*)]
\begin{multicols*}{2}
\item  \ce{NaCO4}
\item  \ce{NaHCO3}
\item  \ce{NaCO3}
\item  \ce{Na2CO3}
\item \ce{NaCO2}
\end{multicols*}\flushright  {\small Ans: (b)}
\end{enumerate}

%\item  Name or formulate the following compound: table salt(common name)
%\begin{enumerate}[label=(\alph*)]
%\begin{multicols*}{2}
%\item  \ce{CH4}
%\item  \ce{NH3}
%\item  \ce{NaCO3}
%\item  \ce{Mg(OH)2}
%\item \ce{NaCl}
%\end{multicols*}\flushright  {\small Ans: (e)}
%\end{enumerate}

\item  Name or formulate the following compound: ammonia(common name)
\begin{enumerate}[label=(\alph*)]
\begin{multicols*}{2}
\item  \ce{CH4}
\item  \ce{NH3}
\item  \ce{NaCO3}
\item  \ce{Mg(OH)2}
\item \ce{NaCl}
\end{multicols*}\flushright  {\small Ans: (b)}
\end{enumerate}






\item  Name or formulate the following compound:
\begin{enumerate}[label=(\alph*)]
\item  \ce{MgSO4}
\item  \ce{Ni(SO4)3}
\item  Cobalt(II) nitrate
\item  Cobalt(II) sulfate dihydrate
\item \ce{KHCO3}
\flushright  {\small Ans: magnesium sulfate, nicke(III) sulfate, \ce{Co(NO3)2}, \ce{CoSO4 . 2H2O}, potassium hydrogencarbonate. }
\end{enumerate}

\item  Name or formulate the following compound:
\begin{enumerate}[label=(\alph*)]
\item  \ce{Ca(NO3)2}
\item  \ce{Na(HCO3)2}
\item  Nickel(III) sulfate
\item  Nicke(III) sulfate tetrahydrate
\item \ce{Na2H2PO4}
\flushright  {\small Ans: calcium nitrate, sodium hydrogen carbonate, \ce{NiSO4}, \ce{NiSO4 . 4H2O}, sodium dihydrogenphosphate. }
\end{enumerate}




\restoregeometry
\end{enumerate}
\end{multicols*}
\pagecolor{green!10}\afterpage{\nopagecolor}\newpage
\end{document}
