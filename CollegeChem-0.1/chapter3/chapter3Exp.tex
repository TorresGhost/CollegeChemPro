\documentclass[main.tex]{subfiles}
\begin{document}\newpage
\setdoublesep{0.35700 em}  % 'Bond Spacing'
\setatomsep{1.78500 em}    % 'Fixed Length'
\setbondoffset{0.18265 em} % 'Margin Width'
\newcommand{\bondwidth}{0.06642 em} % 'Line Width'
\setbondstyle{line width = \bondwidth}

\begin{fullwidth}





%%%%%%%%%%%%HEADING
\begin{multicols}{2}
\begin{tcolorbox}[enhanced jigsaw,breakable,size=title,
colback=mybrown!05,colframe=black,fonttitle=\bfseries,
title=STUDENT INFO,pad at break=1mm, break at=15cm/0pt ]
\vspace{0.2cm}
\noindent Name: \rule{5cm}{0.4pt}Date:\rule{1cm}{0.4pt}\\
Pre-lab Done: \tikzcheckmark[scale=2,black]{no mark}\quad
\end{tcolorbox}
\end{multicols}
\hfill
\vspace{0.2cm}
\begin{center}
{\large \bfseries 
Pre-lab Questions 
\par
\Huge
Specific heat and food
\\[5pt] \par}
\vspace{0.2cm}
\end{center}
\par
\noindent
\uline{  \hfill \normalsize \hfill       }
%%%%%%%%%%%%HEADING

\begin{enumerate}
% PELAB 1
\item What is the formula to calculate the energy needed to warm up a metal? Explain the meaning of each variable.
\vspace{3cm}


\item The specific heat of Al is 0.215 cal/g/$^\circ$C whereas the one for brass is 0.09 cal/g/$^\circ$C. Explain the implications.
\vspace{3cm}

\item How many calories are absorbed by a 45.2g piece of aluminum ($C_e=$0.215 cal/g/$^\circ$C)
if its temperature rises from 10$^\circ$C to 40$^\circ$C.
\vspace{3cm}

\item What is the difference between cal and Cal?
\vspace{3cm}

\item A pepperoni pizza slice contains 10g of fat (9kcal/g), 36g of carbs (4kcal/g) and 14 g of protein (4kcal/g), where the caloric values are indicated in parenthesis. What is the total caloric intake of the slice? 

\end{enumerate}


\clearpage\mbox{}\clearpage



%%%%%%%%%%%%HEADING
\begin{multicols}{2}
\begin{tcolorbox}[enhanced jigsaw,breakable,size=title,
colback=mybrown!05,colframe=black,fonttitle=\bfseries,
title=STUDENT INFO,pad at break=1mm, break at=15cm/0pt ]
\vspace{0.2cm}
\noindent Name: \rule{5cm}{0.4pt}Date:\rule{1cm}{0.4pt}\\
Pre-lab Done: \tikzcheckmark[scale=2,black]{no mark}\quad
\end{tcolorbox}
\end{multicols}
\hfill
\vspace{0.2cm}
\begin{center}
{\large \bfseries 
Experiment
\par
\Huge
Specific heat and food
\\[5pt] \par}
\vspace{0.2cm}
\end{center}
\par
\noindent
\uline{  \hfill \normalsize \hfill       }
%%%%%%%%%%%%HEADING

\vspace{0.2cm}{\large \bfseries 1. Specific heat of a metal}
The goal of this mini experiment is to calculate the specific heat of an unknown metal. You will do this by warming up the metal in a hot water bath and by using a calorimeter to cool down the metal.
\begin{steps}
    \newstep[] Obtain a metallic object. Record its mass. 
    \newstep[] Place a 250mL beaker (or a 400mL) on top of a hot plate. Place a thermometer in the beaker so that it does not touch the walls of the beaker and secure it with a clam. Start warming up the water at high heat so that the water boils. Add some boiling chips.
    \newstep[] Tie a string to the object and submerge it in the how bath. Let it there for 10 min.
    \newstep[] Obtain a double styrofoam cup and weight it. Record its mass.
    \newstep[] Add 50mL of water to the cup and weight again. Record the new mass. Make sure the water is enough to fully cover the metal. If not add some more.
    \newstep[] Measure the temperature of the hot bath after the metals it's been there for 10 min. Record the value.
    \newstep[] Using the string and being careful not to drop the object, transfer the metal object from the hot bath to the cup with water. Cover the cup quickly and stir.
    \newstep[] Using the thermometer in the calorimeter, measure the highest temperature reached by the water in the cup after you drop the object.
    \newstep[] You might have to replicate the experiment.

\end{steps}
\end{fullwidth}






 






\newpage
\begin{fullwidth}
\begin{center}\begin{tabular}{ p{2.0cm}p{6.5cm}p{3cm}p{5cm}  }
\hline
 \begin{center}\mycircled{1}\end{center} &\begin{center}Mass of the metal, $m_{metal}$ (g)\end{center}&&\begin{center}\rule{2.0cm}{0.4pt}\end{center}\\
   \begin{center}\mycircled{2}\end{center} & \begin{center}Mass of the calorimeter (g)\end{center}&&\begin{center}\rule{2.0cm}{0.4pt}\end{center}\\
      \begin{center}\mycircled{3}\end{center} & \begin{center}Mass of the calorimeter + water (g)\end{center}&&\begin{center}\rule{2.0cm}{0.4pt}\end{center}\\
      \begin{center}\mycircled{3}\hspace{0.1cm}$-$\hspace{0.1cm}\mycircled{2}\end{center} & \begin{center}Mass of the water,$m_{water}$ (g)\end{center}&&\begin{center}\rule{2.0cm}{0.4pt}\end{center}\\
  \begin{center}\mycircled{4}\end{center} & \begin{center}Temperature of boiling water ($^\circ$C) \end{center}&&\begin{center}\rule{2.0cm}{0.4pt}\end{center}\\
  \begin{center}\mycircled{5}\end{center}& \begin{center}Initial temperature of water in calorimeter ($^\circ$C) \end{center}&&\begin{center}\rule{2.0cm}{0.4pt}\end{center}\\
    \begin{center}\mycircled{6}\end{center}& \begin{center}Final temperature of water in calorimeter  ($^\circ$C) \end{center}&&\begin{center}\rule{2.0cm}{0.4pt}\end{center}\\
        \begin{center}\mycircled{6}\hspace{0.1cm}$-$\hspace{0.1cm}\mycircled{5}\end{center} & \begin{center}Temperature change , $\Delta T$ ($^\circ$C)\end{center}&&\begin{center}\rule{3.0cm}{0.4pt}\end{center}\\

\hline\end{tabular}\end{center}
Calculate the specify heat of the metal by means of the following formula in which $C_{e, water}$ is the specific heat of water (1cal/g/$^\circ$C):

\begin{equation*}
m_{metal}C_{e, metal}\times \Delta T+m_{water}\times C_{e, water}\times \Delta T=0
\end{equation*}
\hspace{2cm}
\flushright {$C_{e, metal}=$\rule{3.0cm}{0.4pt}}









\end{fullwidth}


\newpage
\begin{fullwidth}
\vspace{0.2cm}{\large \bfseries 2. Food value in food}
The goal of this mini experiment is to calculate the number of Calories, or Kcal, in different food products using the energy values of different food ingredients--carbs, fats and protein.
\begin{steps}
    \newstep[] Obtain a food product labels.
        \newstep[]  Write down the name of the product in the table below.
        \newstep[]  Write down the mass of a serving in the table below.
        \newstep[]  List the grams of carbohydrates, fats and protein in your product.
                \newstep[]  calculate the number of Calories (kcal) for each food type in a serving. The accepted energy values of carbohydrates, fats and proteins are 4, 9 and 4 Cal/g.
                \newstep[] Calculate the total number of calories in a serving and compare the value with the one in the food label. 
\end{steps}

\begin{center}\begin{tabular}{ p{2.0cm}p{7.5cm}p{3cm}p{5cm}  }
\hline
 &\begin{center}Name of food product \end{center}&&\begin{center}\rule{3.0cm}{0.4pt}\end{center}\\
    & \begin{center}Mass of a serving (g)\end{center}&&\begin{center}\rule{3.0cm}{0.4pt}\end{center}\\
      & \begin{center}Mass of carbohydrate(g)\end{center}&&\begin{center}\rule{3.0cm}{0.4pt}\end{center}\\
            & \begin{center}Mass of fats(g)\end{center}&&\begin{center}\rule{3.0cm}{0.4pt}\end{center}\\
                  & \begin{center}Mass of protein(g)\end{center}&&\begin{center}\rule{3.0cm}{0.4pt}\end{center}\\
            & \begin{center}Calories from carbohydrates (Cal, or Kcal)\end{center}&&\begin{center}\rule{3.0cm}{0.4pt}\end{center}\\
                        & \begin{center}Calories from fat (Cal, or Kcal)\end{center}&&\begin{center}\rule{3.0cm}{0.4pt}\end{center}\\
            & \begin{center}Calories from protein (Cal, or Kcal)\end{center}&&\begin{center}\rule{3.0cm}{0.4pt}\end{center}\\

\hline\end{tabular}\end{center}
\vspace{0.2cm}{\large \bfseries 6. PostLab questions }
\begin{enumerate}
\item What percentage of the total calories in your food product is coming from fats?
\vspace{2.cm}
\item What percentage of the total calories in your food product is coming from carbs?
\vspace{2.cm}
\end{enumerate}
\end{fullwidth}




\end{document}