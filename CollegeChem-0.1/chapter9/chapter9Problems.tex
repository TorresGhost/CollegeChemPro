\documentclass[main.tex]{subfiles}
\begin{document}\newpage
\setdoublesep{0.35700 em}  % 'Bond Spacing'
\setatomsep{1.78500 em}    % 'Fixed Length'
\setbondoffset{0.18265 em} % 'Margin Width'
\newcommand{\bondwidth}{0.06642 em} % 'Line Width'
\setbondstyle{line width = \bondwidth}
\newgeometry{left=0.8in,right=0.8in, top=2.5cm,bottom=2cm}
\fancyhfoffset[E,O]{0pt}
\setlength{\columnsep}{30pt}
\begin{conclusion}
\end{conclusion}
\setstretch{0.3}
\begin{multicols*}{2}
{\raggedright\textsc{\textbf{Solutions }}\par}
\begin{enumerate}

\item A solution is prepared by mixing 4 g of \ce{C6H6(l)} and 5 g of \ce{CCl4(l)}. Indicate the true statement.
\begin{enumerate}[label=(\alph*)]
\begin{multicols*}{2}
\item \ce{C6H6} is the solute
\item\ce{CCl4} is the solute
\item Both do not mix
\item This is not a solution
\end{multicols*}\flushright  {\small Ans: (a)}
\end{enumerate}


\item Which of the following chemicals will mix with \ce{CH3Cl}
\begin{enumerate}[label=(\alph*)]
\begin{multicols*}{2}
\item \ce{C6H6}
\item \ce{CCl4} 
\item  \ce{H2O}
\item  \ce{CH2Cl2}
\end{multicols*}\flushright  {\small Ans: (c)}
\end{enumerate}

\item Which of the following chemicals will mix with \ce{CCl4}
\begin{enumerate}[label=(\alph*)]
\begin{multicols*}{2}
\item \ce{C6H6}
\item \ce{CHCl3} 
\item  \ce{H2O}
\item  \ce{CH3Cl}
\end{multicols*}\flushright  {\small Ans: (a)}
\end{enumerate}

\item Air is :
\begin{enumerate}[label=(\alph*)]
\begin{multicols*}{2}
\item A solid solution %A
\item A liquid solution %B
\item A gas solution %C
\item A heterogeneous mixture %D
\end{multicols*}\flushright  {\small Ans: (c)}
\end{enumerate}

\item Ammonia and water:
\begin{enumerate}[label=(\alph*)]
\begin{multicols*}{2}
\item A solid solution %A
\item A liquid solution %B
\item A gas solution %C
\item A heterogeneous mixture %D
\end{multicols*}\flushright  {\small Ans: (c)}
\end{enumerate}

\item Oil and water do not mix due to a polarity difference. Explain why a detergent can help solve oil in water.


{\raggedright\textsc{\textbf{Concentration Units }}\par}

\item Sodium hydroxide \ce{NaOH} is a chemical used in drain cleaners. A solution is prepared by mixing 25g of \ce{NaOH} in 250g of water.
\begin{enumerate}[label=(\alph*)]
\item Water is the solute %A
\item The percent (m/m) of solute is 9\% %B
\item The percent (m/m) of solvent is 80\% %C
\item The percent (m/m) of solvent is 10\% %D
\flushright  {\small Ans: (b)}
\end{enumerate}

\item A solution is prepared by mixing 15g of \ce{NaOH} in 50g of water.
\begin{enumerate}[label=(\alph*)]
\item The percent (m/m) of solute is 30\% %A
\item The percent (m/m) of solute is 0.3\% %B
\item The percent (m/m) of solute is 23\% %C
\item The percent (m/m) of solvent is 10\% %D
\flushright  {\small Ans: (c)}
\end{enumerate}


\item Alcohol-hydroxide is a mixture employed to clean glass. A mixture is prepared by mixing 60g of \ce{NaOH} with 500g of ethanol.
\begin{enumerate}[label=(\alph*)]
\item The percent (m/m) of solute is 8\% %A
\item The percent (m/m) of solute is 0.08\% %B
\item The percent (m/m) of solvent is 70\% %C
\item The percent (m/m) of solvent is 89\% %D
\flushright  {\small Ans: (d)}
\end{enumerate}

%\item Vanilla extract is a solution vanillin in ethanol. A vanilla solution is made by mixing 15 mL of pure vanillin and 50mL of ethanol.
%\begin{enumerate}[label=(\alph*)]
%\item The percent (v/v) of solute is 80\% %A
%\item The percent (v/v) of solute is 23\% %B
%\item The percent (v/v) of solvent is 70\% %C
%\item The percent (v/v) of solvent is 60\% %D
%\flushright  {\small Ans: (b)}
%\end{enumerate}
%
%\item Vanilla extract is a solution vanillin in ethanol. A vanilla solution is made by mixing 15 mL of pure vanillin and 50mL of ethanol.
%\begin{enumerate}[label=(\alph*)]
%\item The percent (v/v) of solute is 80\% %A
%\item The percent (v/v) of solute is 23\% %B
%\item The percent (v/v) of solvent is 70\% %C
%\item The percent (v/v) of solvent is 60\% %D
%\flushright  {\small Ans: (b)}
%\end{enumerate}

\item Vinegar is a (m/m) 5\% acetic acid solution. How many grams of acetic acid are there in 2g of vinegar:
\begin{enumerate}[label=(\alph*)]
\begin{multicols*}{2}
\item 10g %A
\item 3g %B
\item 0.01g %C
\item 0.1 g %D
\end{multicols*}\flushright  {\small Ans: (d)}
\end{enumerate}

\item An HCl solution is prepared by mixing 4 moles of HCl with water reaching a volume of 250mL. Calculate the molarity of the solution
\begin{enumerate}[label=(\alph*)]
\begin{multicols*}{2}
\item 16M %A
\item 0.016M %B
\item 0.16M %C
\item 0.014 M %D
\end{multicols*}\flushright  {\small Ans: (a)}
\end{enumerate}


\item How many mL of a 3M KCl solution contains 0.06 KCl moles.
\begin{flushright}\small Ans: 20mL\end{flushright}


\item How many mL of a 4M NaCl (MW=58$g\cdot mol^{-1}$) solution contains 5 grans of NaCl.
\begin{flushright}\small Ans: 21mL\end{flushright}

\item How many grams of solute are there in 100mL of a 0.01M \ce{HNO3} (MW=63$g\cdot mol^{-1}$) solution.
\begin{flushright}\small Ans: 0.062g\end{flushright}


\item How many mL of a 0.001M \ce{Ca(OH)2} (MW=74$g\cdot mol^{-1}$) solution can be prepared from 5 mg of \ce{Ca(OH)2}.
\begin{flushright}\small Ans: 67.56mL\end{flushright}



\item What is the final volume when 50mL of a 2M NaCl solution is diluted to a 1M.
\begin{enumerate}[label=(\alph*)]
\begin{multicols*}{2}
\item 25 mL %A
\item 50 mL %B
\item 125 mL %C
\item 100 mL %D
\end{multicols*}
\flushright  {\small Ans: d}
\end{enumerate}


\item What is the molarity of a solution prepared when 100mL a 4\% HCl solution is diluted to a final volume of 500mL.
\begin{enumerate}[label=(\alph*)]
\begin{multicols*}{2}
\item 0.8 \% %A
\item 8 \% %B
\item 20 \% %C
\item 10 \% %D
\end{multicols*}
\flushright  {\small Ans: a}
\end{enumerate}

\item Describe how to prepare 50mL of a 0.5M \ce{H2SO4} solution, starting with a 1M stock \ce{H2SO4} solution.
\enumeratext{\flushright  {\small Ans: 25mL}}


{\raggedright\textsc{\textbf{Electrolytes and insoluble compounds}}\par}

\item Indicate the composition of an aqueous solution of \ce{NaCl}:
\begin{enumerate}[label=(\alph*)]
\item \ce{Na^{+}} anions and \ce{Cl^{-}} cations %A
\item Ions and molecules %B
\item \ce{Na^{+}} cations and \ce{Cl^{-}} anions %C
\item \ce{NaCl} molecules %D
\flushright  {\small Ans: c}
\end{enumerate}

\item Indicate the composition of an aqueous solution of \ce{HCl}:
\begin{enumerate}[label=(\alph*)]
\item Ions and molecules %A
\item \ce{H^{+}} anions and \ce{Cl^{-}} cations  %B
\item  \ce{HCl} molecules %C
\item \ce{H^{+}} cations and \ce{Cl^{-}} anions  %D
\flushright  {\small Ans: d}
\end{enumerate}

\item Indicate the composition of an aqueous solution of \ce{CaCl2}:
\begin{enumerate}[label=(\alph*)]
\item \ce{CaCl2} molecules  %A
\item \ce{Ca^{2+}} cations and \ce{Cl^{-}} anions   %B
\item   Ions and molecules %C
\item   \ce{Ca^{+2}} anions and \ce{Cl^{-}} cations %D
\flushright  {\small Ans: b}
\end{enumerate}

\item Indicate the composition of an aqueous solution of \ce{H2O}:
\begin{enumerate}[label=(\alph*)]
\item \ce{H2O} molecules  %A
\item \ce{H^{+}} cations and \ce{OH^{-}} anions   %B
\item   Ions and molecules %C
\item   \ce{H^{+}} anions and \ce{OH^{-}} cations %D
\flushright  {\small Ans: c}
\end{enumerate}


\item Predict the soluble character of the following compound: \ce{AgNO3}.
\begin{flushright}\small Ans: soluble\end{flushright}


\item Predict the soluble character of the following compound: \ce{AgBr}.
\begin{flushright}\small Ans:  insoluble\end{flushright}

\item Predict the soluble character of the following compound: \ce{CaCO3}.
\begin{flushright}\small Ans:  soluble\end{flushright}

\item Predict the soluble character of the following compound: \ce{Na2CO3}.
\begin{flushright}\small Ans:  soluble\end{flushright}

\item Break down the following chemical into ions if possible: \ce{Ca(OH)2}.
\begin{flushright}\small Ans:  \ce{Ca^{2+} +2 OH^{-} }\end{flushright}


\item Break down the following chemical into ions if possible: \ce{NH3}.
\begin{flushright}\small Ans:  \ce{NH3_(aq) }\end{flushright}


\item Break down the following chemical into ions if possible: \ce{K2CrO4}.
\begin{flushright}\small Ans: \ce{2K^{+} +CrO4^{-} }\end{flushright}

\item Break down the following chemical into ions if possible: \ce{Ca(NO3)2}.
\begin{flushright}\small Ans: \ce{Ca^{+} +2 NO3^{-} }\end{flushright}

{\raggedright\textsc{\textbf{Precipitation and acid-base reactions}}\par}

\item Classify the following reaction as acid-base or precipitation:
\begin{center}\ce{ NaCl_{(aq)}+AgNO3 _{(aq)} -> NaNO3_{(aq)} + AgCl_{(s)} ^     }\end{center}
\begin{flushright}\small Ans:  Precipitation \end{flushright}

\item Classify the following reaction as acid-base or precipitation:
\begin{center}\ce{ H2SO4_{(aq)} + 2NaOH_{(aq)} -> 2H2O_{(l)} + Na2SO4_{(aq)}     }\end{center}
\begin{flushright}\small Ans:  Acid-base \end{flushright}

\item Classify the following reaction as acid-base or precipitation:
\begin{center}\ce{ 2 Na3PO4_{(aq)} + 3 CaCl2_{(aq)} -> 6 NaCl_{(aq)} + Ca3(PO4)2_{(s)} ^     }\end{center}
\begin{flushright}\small Ans:  Precipitation \end{flushright}


\item Obtain the net ionic equation for the following reaction:
\begin{center}\ce{ NaCl_{(aq)}+AgNO3 _{(aq)} -> NaNO3_{(aq)}+AgCl_{(s)} ^    }\end{center}
\begin{flushright}\small Ans:  \ce{   Ag^+ _{(aq)} + Cl^- _{(aq)} ->  AgCl_{(s)} ^    } \end{flushright}

\item Obtain the net ionic equation for the following reaction:
\begin{center}\ce{H2SO4_{(aq)} + 2NaOH_{(aq)} -> 2H2O_{(l)} + Na2SO4_{(aq)}       }\end{center}
\begin{flushright}\small Ans:  \ce{   2H^+ _{(aq)} + 2OH^- _{(aq)} ->  2 H2O_{(l)}     } \end{flushright}

\item Obtain the net ionic equation for the following reaction:
\begin{center}\ce{2 Na3PO4_{(aq)} + 3 CaCl2_{(aq)} -> 6 NaCl_{(aq)} + Ca3(PO4)2_{(s)} ^       }\end{center}
\begin{flushright}\small Ans:  \ce{   3Ca^{2+} _{(aq)} + 2PO4^{2-} _{(aq)} ->  Ca3(PO4)2_{(s)} ^    } \end{flushright}


{\raggedright\textsc{\textbf{Redox}}\par}

\item Balance the following redox reactions in acidic medium:
\begin{center}\ce{ I^-_{(aq)} + MnO4^-_{(aq)}   ->  I2_{(s)} + Mn^{2+}_{(aq)}       }\end{center}
\begin{flushright}\small Ans:  \ce{   10I^-_{(aq)} + 2MnO4^-_{(aq)}+ 16H^+_{(aq)}  ->  5I2_{(s)} + 2Mn^{2+}_{(aq)} + 8H2O_{(l)}    } \end{flushright}


\item Balance the following redox reactions in acidic medium:
\begin{center}\ce{ MnO4^{-}_{(aq)} + SO3^{-2}_{(aq)} -> MnO2_{(s)}+SO4^{-2}_{(aq)}      }\end{center}
\begin{flushright}\small Ans:  \ce{    2MnO4^{-}_{(aq)} + 2H^+ +3SO3^{-2}_{(aq)} -> H2O_{(l)} + 2MnO2_{(s)}+ 3SO4^{-2}_{(aq)}     } \end{flushright}

\item Convert the following acidic redox into basic redox:
\begin{center}\ce{  2MnO4^{-}_{(aq)} + 2H^+ +3SO3^{-2}_{(aq)} -> H2O_{(l)} + 2MnO2_{(s)}+ 3SO4^{-2}_{(aq)}     }\end{center}
\begin{flushright}\small Ans:  \ce{     2MnO4^{-}_{(aq)} + H2O_{(l)} +3SO3^{-2}_{(aq)} -> 2OH^-_{(aq)} + 2MnO2_{(s)}+ 3SO4^{-2}_{(aq)}    } \end{flushright}



\restoregeometry
\end{enumerate}
\end{multicols*}
\pagecolor{green!10}\afterpage{\nopagecolor}\newpage
\end{document}
