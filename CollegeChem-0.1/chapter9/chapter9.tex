\documentclass[main.tex]{subfiles}

\begin{document}

\linenumbers


\chapter[Reactions in solution  ]{Reactions in solution }

\begin{marginfigure}
      \includegraphics{chapter9/figure1}
   \end{marginfigure}
\lettrine[lines=4]{\color{black!45}T}{he} most common reactions happen in solution. Think for example when you add salt to your soup or when a metal exposed to air rusts when it gets wet. The first is a dissociation reaction, whereas the second is a redox reaction. This chapter first covers the properties of solutions and you will learn how to quantify the amount of solute in a solution. This chapter also covers some important types of reactions happening in solution. Another important concept covered in this chapter is the idea of electrolytes. Most of you will be surprised to know that water do not conduct the electricity. This is because pure water is a weak electrolyte. The importance of electrolytes is well know among the sports community. If you have ever played a sport, you have probably chugged an sports drink. These are actually electrolyte solutions with extra sugar added. However, few know the specifics of their function. Electrolytes are actually salts that conduct electricity in water by separating into positive and negative ions. Here you will be able to identify different types of electrolytes. Finally, here we will also cover some important reactions happening in solution. In particular reactions between acids and base and reaction that result in a solid.
\begin{marginfigure}%LEARNING GOALS BOX
\begin{mytcbox}{GOALS}
\begin{enumerate}[label=\protect\circled{\color{white}\arabic*}]
\item Carry composition calculations
\item Classify electrolytes
\item Identify insoluble compounds
\item Write down net ionic equations
\item Balance redox reactions
\end{enumerate}
\end{mytcbox}
\end{marginfigure}%LEARNING GOALS BOX




\section{\color{blue!30!black}{Solutions and composition}}
This section covers the basic of solutions. First, solutions are not just a simple mixture of two components. The state of the matter of both components of the mixture or the polarity affect the formation of a solution. For example, a solution does not result from mixing oil and water. 
\sloppy
 
     \begin{marginfigure}
\begin{tcolorbox}[enhanced,colback=red!5!white,colframe=black!50!red,boxrule=1pt,
  arc=0pt,outer arc=0pt,drop heavy lifted shadow]
\faGears\ 
\docenvdef{Discussion:} List three solutions in your household containing just a single solute. Give the chemical formula of the solute and the name of the solvent. \end{tcolorbox}
 \end{marginfigure}
 
\begin{description}
\item[\docfilehook{What makes a solution?}{What makes a solution?}] Solutions are homogeneous mixtures of solute and solvent. Homogeneous means that if you look at the mixture you will not be able to differentiate both components and you will only see the mixture as whole. In a solution, the solute is the component of the mixture in less amount, whereas the solvent is the component in a larger amount. Think about mixing a small amount of sugar with water. Sugar is sweet and water tasteless. When you mix both, you form a solution of sugar and you will not see the sugar in the solution. In this particular example, sugar will be the solute in the solution, as the sugar is in less amount than water. A solution is the result or mixing a solute and a solvent:
\resizeableyellownote{2.5}{1}{Add this relation into your flashcard.}
\begin{equation*}
\boxed{ \text{Solution}=\text{Solute}+\text{Solvent} }   
\end{equation*}



%\begin{example} %%%%%%%%%%%%%%%%%%%%%%%% EXAMPLE BOX
%Identify the solute and the solvent in the following mixtures: (a) 10g of \ce{H2O(l)} mixed with 3g of \ce{KCl(s)}; (b) 10g \ce{Cu(s)} mixed with 3g \ce{Zn(s)}
%\\
%\textlcsc{ \textcolor{dgreen}{\Large \textbf{Solution}} }\\
%(a) Potassium chloride is in less amount and hence will be the solute. Water, in larger quantity, will be the solvent. This is an example of an aquose-based solution. (b) Zn is in less amount and hence will be the solute, and Cu the solvent. This is a solid solution.
%\\
%\faDiamond\ \textlcsc{ \textcolor{dgreen}{\Large \textbf{Study Check}} }\\
%Identify solute and solvent in the following mixture: (a) 10g of \ce{H2O(l)} and 20g of \ce{CH3OH(l)}; (b) 1g of \ce{I2(s)} and 100g of \ce{CH3CH2OH(l)}
%\\
%\flushright  {\small Answer: (a) water(solute); (b) iodine(solute).}
%\end{example}%%%%%%%%%%%%%%%%%%%%%%%% EXAMPLE BOX
\item[\docfilehook{Types of solutions}{Types of solutions}] You can prepare different types of solutions by mixing a solid and a liquid, like when you mix sugar and water, or salt and water. You can create solutions as well by mixing two liquids or two solids. Examples are vinegar--a liquid solution of acetic acid (liquid) in water (liquid)--or steel-- a solid solution that contains iron and carbon. 
\begin{figure*}[h]%%%%%%%%  FIGURE
\includegraphics{chapter9/figure3}
\caption{A solution results from dissolving a solute into a solvent}
\end{figure*}%%%%%%% FIGURE
\item[\docfilehook{Mixing and polarity}{Mixing and polarity}] A solution is formed when both the solute and the solvent mix. However, they will only mix if they have the same polarity. As an example, water (\ce{H2O}) is a polar molecule and methanol (\ce{CH3-OH}) too. Hence they will both mix and form a solution. If the elements of a mixture have different polarity they will not mix. An example is benzene (\ce{C6H6}, nonpolar) and water, or for example oil (nonpolar) and water (polar). 

\begin{marginfigure}%%%%%%%MARGIN FIGURE
\begin{tcolorbox}[tab2,tabularx={X|Y|Y}]%%%% FANCY COLOR TABLE
 Solvent     & Solute     & Mixing?      \\\hline\hline
Polar &   Polar & Yes          \\\hline
Polar &   Nonpolar & No          \\\hline
Nonpolar &   polar & No          \\\hline
Nonpolar &   Nonpolar & Yes                  
\end{tcolorbox}%%%% FANCY COLOR TABLE
 \end{marginfigure}%%%%%%%MARGIN FIGURE
 
 


\begin{example} %%%%%%%%%%%%%%%%%%%%%%%% EXAMPLE BOX
Use polarity arguments to indicate if the following substances will mix: (a) \ce{H2O(g)} and \ce{CH4(g)}; (b) \ce{H2O(g)} and \ce{HCl(g)}
\\
\textlcsc{ \textcolor{dgreen}{\Large \textbf{Solution}} }\\
(a) Water and methane (\ce{CH4}) will not mix, as water is a polar molecule and \ce{CH4} (methane) is non-polar. (b) They will mix as HCl is a polar molecule and so is water.
\\
\faDiamond\ \textlcsc{ \textcolor{dgreen}{\Large \textbf{Study Check}} }\\
Use polarity arguments to indicate if the following substances will mix: (a) \ce{H2O(l)} and \ce{CH3Cl(l)}; (b) \ce{CH3Cl(l)} and \ce{CCl4(l)}
\\
\flushright  {\small Answer: (a) will mix; (b) will not mix.}
\end{example}%%%%%%%%%%%%%%%%%%%%%%%% EXAMPLE BOX
\end{description}




                            
One of the most important properties of a solution is its concentration. In this section you will learn how to calculate solute concentrations. The larger the concentration of a solution the more solute in the solution. There are many different concentration units, such as molarity or mass percent concentration. This section will introduce you to some of the most important concentration units.
\sloppy 
\begin{description}
\item[\docfilehook{Meaning of concentration}{Meaning of concentration}] The concentration of a solution refers to the amount of solute with respect to the amount of solution. The larger concentration the larger number of solute particles with respect to the solution. 
\item[\docfilehook{Mass percent concentration}{Mass percent concentration}] The mass percent (m/m) is the amount of solute in grams per grams of solution in percent form
\resizeableyellownote{2.5}{1}{Add this relation into your flashcard.}
\begin{equation*}
\boxed{ m/m=\frac{\text{g of solute}}{\text{g of solution}}\times 100}   
\end{equation*}



%\begin{example} %%%%%%%%%%%%%%%%%%%%%%%% EXAMPLE BOX
%A NaCl solution is prepared by mixing 4g of \ce{NaCl} with 50g of \ce{H2O}. Calculate the percent (m/m) of the solution.
%\\
%\textlcsc{ \textcolor{dgreen}{\Large \textbf{Solution}} }\\
%We need the grams of solute and the grams of solution. The grams of solute are given (4g of \ce{NaCl}), whereas the grams of solution result from adding the grams of solvent and solute: 54 g of solution. Using the formula for the percent (m/m), he have:
%\begin{equation*}
% m/m=\frac{\text{4 g of solute}}{\text{54 g of solution}}\times 100=7.4\% 
%\end{equation*}
%This means that by mixing 4g of \ce{NaCl} with 50g of \ce{H2O} you prepare a $7.4\%$ (m/m) solution.
%\\
%\faDiamond\ \textlcsc{ \textcolor{dgreen}{\Large \textbf{Study Check}} }\\
%A KCl solution is prepared by mixing 5g of \ce{KCl} with 200g of \ce{H2O}. Calculate the percent (m/m) of the solution.
%\\
%\flushright  {\small Answer: 2.4 \%.}
%\end{example}%%%%%%%%%%%%%%%%%%%%%%%% EXAMPLE BOX




%\item[\docfilehook{Volume percent concentration}{Volume percent concentration}] The volume percent concentration (v/v) is the volume of solute per volume of solution in percent form
%\resizeableyellownote{2.5}{1}{Add this relation into your flashcard.}
%\begin{equation*}
%\boxed{ v/v=\frac{\text{volume of solute}}{\text{volume of solution}}\times 100}   
%\end{equation*}
%\item[\docfilehook{Mass/volume  percent concentration}{Mass/volume  percent concentration}] The mass/volume  percent concentration (m/v) is the mass of solute per mL of solution in percent form.
%\resizeableyellownote{2.5}{1}{Add this relation into your flashcard.}
%\begin{equation*}
%\boxed{ m/v=\frac{\text{g of solute}}{\text{mL of solution}}\times 100}   
%\end{equation*}

\item[\docfilehook{Molarity concentration}{Molarity concentration}] The molarity (M) is the moles of solute per L of solution.
\resizeableyellownote{2.5}{1}{Add this relation into your flashcard.}
\begin{equation*}
\boxed{ M=\frac{\text{moles of solute}}{\text{L of solution}}}   
\end{equation*}
\begin{example} %%%%%%%%%%%%%%%%%%%%%%%% EXAMPLE BOX
A NaCl solution is prepared by mixing 4g of \ce{NaCl} (MW=58.4$g\cdot mol^{-1}$) with 50 g of water until a final volume of 52mL of solution. Calculate:  (a) the mass percent (m/m) concentration; (b) the molarity.
\\
\textlcsc{ \textcolor{dgreen}{\Large \textbf{Solution}} }\\
(a) to calculate the mass percent (m/m) we just need the grams of solute and the grams of solution--that is four plus fifty. Both numbers are already given:
\begin{equation*}
 m/m=\frac{\text{g of solute}}{\text{g of solution}}\times 100 =\frac{\text{4 g of solute}}{\text{54 g of solution}}\times 100
\end{equation*}
The result is $9.2\%$. (b) To calculate molarity we need the moles of solute and the liters of solution. We have the mL of solution, that can be converted to L: 52mL$=5.2\cdot 10^{-2}L$. To calculate the moles of solute, we will use the grams of solute and the molar mass to convert this value into moles: $4g/58.4 g\cdot mol^{-1}=$0.068moles. Plugging all values into the molarity formula:
\begin{equation*}
M=\frac{\text{moles of solute}}{\text{L of solution}}
=\frac{\text{0.068 moles of solute}}{5.2\cdot 10^{-2}\text{L of solution}=1.31M}
%% 4  \cancel{\text{g of \ce{NaCl}}} \times \dfrac{\text{1 mole of \ce{NaCl}} } { \cancel{\text{58.4 g of \ce{NaCl}}} } 
\end{equation*}
\\
\faDiamond\ \textlcsc{ \textcolor{dgreen}{\Large \textbf{Study Check}} }\\
(a) A KCl solution is prepared by mixing 8g of \ce{NaCl} (MW=74$g\cdot mol^{-1}$) with 250mL of \ce{H2O}. Calculate the molarity; (b) A KCl solution is prepared by mixing 5g of \ce{KCl} with 200g of \ce{H2O}. Calculate the percent (m/m) of the solution.
\\
\flushright  {\small Answer: 0.43M; 2.4 \%.}
\end{example}%%%%%%%%%%%%%%%%%%%%%%%% EXAMPLE BOX

\begin{marginfigure}%%%%%%%% MARGIN FIGURE
\includegraphics{chapter9/figure4}
\caption{Champagne is a solution of gas in a liquid}
\end{marginfigure}%%%%%%% MARGIN FIGURE

\item[\docfilehook{Concentration units as conversion factors}{Concentration units as conversion factors}] Each of the different concentration units--molarity, mass percent, volume percent, mass/volume percent--can be used in a conversion factor form. For example, if the molarity of a solution is 3M, this means that in the solution there is 3 moles of solute every one litter of solution. 
\begin{equation*}
\boxed{   \text{3M}\quad\text{ or } \frac{3\text{ mol of solute}}{1\text{ L of solution}}\quad\text{ or }   \frac{1\text{ L of solution}}{3\text{ mol of solute}}}   
\end{equation*}

Similarly, if the mass percent of a solution is 5\% this means that there is 5 grams of solute every 100 grams of solution. We often use concentration units as conversion factors when we need to transform between on unit on top (bottom) of the conversion factor and the unit on the bottom (top). 

\begin{marginfigure}%%%%%%%MARGIN FIGURE
\begin{tcolorbox}[tab2,tabularx={X|Y|Y}]%%%% FANCY COLOR TABLE
Concentration &  Conversion     \\\hline\hline
3M &  $\frac{\text{3 moles of solute}}{\text{1 L of solution}}$     \\\hline
5\% (m/m) &  $\frac{\text{5 grams of solute}}{\text{100 g of solution}}$     
%2\% (v/v) &  $\frac{\text{2 L of solute}}{\text{100 L of solution}}$       \\\hline
%6\% (m/v) &  $\frac{\text{6 g of solute}}{\text{100 mL of solution}}$     
\end{tcolorbox}%%%% FANCY COLOR TABLE
\caption{Conversion factors from concentration units}        
 \end{marginfigure}%%%%%%%MARGIN FIGURE


\begin{example} %%%%%%%%%%%%%%%%%%%%%%%% EXAMPLE BOX
How much volume of a 4M solution do you need to provide 5 moles of solute.
\\
\textlcsc{ \textcolor{dgreen}{\Large \textbf{Solution}} }\\
We will use the conversion factor of Molarity using the volume on top and the moles on the bottom in order to cancel the units:
\begin{equation*}
5   \cancel{\text{moles of solute}} \times 
\dfrac{\text{1 L of solution}  } {  4\cancel{\text{moles of solute }}} =1.25\text{L}
\end{equation*}
This means that 1.25L of a 4M solution will provide 5 moles of solute.
\\
\faDiamond\ \textlcsc{ \textcolor{dgreen}{\Large \textbf{Study Check}} }\\
How many liters of a 6\% (m/v)solution do you need to provide 5 grams of solute.
\\
\flushright  {\small Answer: $8.3\times 10^{-2}$L.}
\end{example}%%%%%%%%%%%%%%%%%%%%%%%% EXAMPLE BOX



\begin{marginfigure}%%%%%%%% MARGIN FIGURE
\includegraphics{chapter9/figure2}
\caption{We dilute by adding water to a stock solution}
\end{marginfigure}%%%%%%% MARGIN FIGURE
   
%   \begin{marginfigure}
%\begin{tcolorbox}[enhanced,colback=red!5!white,colframe=black!50!red,boxrule=1pt,
%  arc=0pt,outer arc=0pt,drop heavy lifted shadow]
%\faGears\ 
%\docenvdef{Discussion:} Oil spills on your shirt on diner. List three chemicals than can remove the stain
%\end{tcolorbox}
%\end{marginfigure} 
   
\item[\docfilehook{Dilution}{}] Dilution is the process for preparing a diluted solution from a more concentrated solution. Solutions are often times stored in a stock room in concentrated form. These stocks should be diluted before use. This section covers the dilution process in order to estimate the amount of concentrated solution needed to prepare a more diluted solution. In order to dilute a solution we need to take a certain amount of the concentrated solution and add water. When adding water, the number of moles of solute does not change, and the concentration always decreases. We have a concentrated solution ($c_1$) and we need to prepare a certain volume ($V_2$) of a more diluted solution ($c_2$). The question is how much volume of the concentrated solution ($V_1$) we need to take. In order to answer this we should use the following formula:
 \begin{equation*}
\boxed{ c_1\cdot V_1=c_2\cdot V_2 }   
\end{equation*}

\begin{example} %%%%%%%%%%%%%%%%%%%%%%%% EXAMPLE BOX
How many liters of a 3M NaCl solution are required to prepare 2L of a 1M diluted NaCl solution.
\\
\textlcsc{ \textcolor{dgreen}{\Large \textbf{Solution}} }\\
We have a concentrated solution of 3M molarity and we want to prepare a more dilute solution. In particular 2L of a 1M. Hence: $c_1=3$ and $c_2=1M$ and $V_2=2L$. Using the dilution formula:
 \begin{equation*}
 3M \cdot V_1=1M \cdot  2L 
\end{equation*}
Solving for $V_1$ we have a volume of 0.66L.
\\
\faDiamond\ \textlcsc{ \textcolor{dgreen}{\Large \textbf{Study Check}} }\\
How many liters of a 5M NaCl solution are required to prepare 3L of a 3M diluted NaCl solution.
  \\
\flushright  {\small Answer: 1.8 L.}
\end{example}%%%%%%%%%%%%%%%%%%%%%%%% EXAMPLE BOX

\end{description}



\begin{marginfigure}%%%%%%%% MARGIN FIGURE
\includegraphics{chapter9/figure5}
\caption{Common ingredients found in energy drinks are caffeine or sugars}
\end{marginfigure}%%%%%%% MARGIN FIGURE
\section{\color{blue!30!black}{Electrolytes and insoluble compounds}}
Some chemicals when dissolved in water conduct the electricity, other do not. Electrolytes are solutes that once dissolved in water conduct the electricity. Differently, nonelectrolytes do not conduct the electricity in water. This section covers electrolytes, its properties and characteristics. You will be able to tell whether a solute conduct the electricity or not. Some other chemicals are insoluble. This means they do not dissolve in water. Here you will be able to identify insoluble compounds.
\sloppy 
\begin{description}
\item[\docfilehook{Strong electrolytes}{Strong electrolytes}] These are a type of electrolytes that completely dissociate in water. Hence in a solution of a strong electrolyte you will only have ions. Strong electrolytes are typically ionic compounds such as \ce{MgCl2}:
\begin{center}\ce{MgCl2(s)  ->[H2O] Mg^{2+}(aq) + 2Cl^{-}(aq) }.\end{center}
\item[\docfilehook{Weak electrolytes}{Weak electrolytes}] These electrolytes partially dissociate in water, and this is indicated by means of a chemical reaction with a double arrow. Hence in a solution of a weak electrolyte you will have ions as well as molecules. Examples of weak electrolytes are hydrofluoric acid, water, ammonia or acetic acid. The dissociation of \ce{HF} proceeds as:
\begin{center}\ce{HF(g)  <=>[H2O] H^{+}(aq) + F^{-}(aq) }.\end{center}
\item[\docfilehook{Nonelectrolytes}{Nonelectrolytes}] Nonelectrolytes do not dissociate in water. Hence a nonelectrolyte solution only contains molecules. Examples of weak electrolytes are carbon-based chemicals such as methanol, ethanol, urea or sucrose. The dissociation of urea for example \ce{CH4N2O} proceeds as:
\begin{center}\ce{CH4N2O(s)  ->[H2O] CH4N2O(aq) }.\end{center}

\begin{example} %%%%%%%%%%%%%%%%%%%%%%%% EXAMPLE BOX
For the following chemicals indicate whether you will have (a) only ions on solution, (b) ions and some molecules, or (c) molecules:\\
\begin{center}\fontfamily{ppl}\selectfont
\begin{tabular}{ll}
\rowcolor{black!45}
\toprule
Chemical &  Particles in solution  \\
\midrule
\ce{NH3} &\hspace{3cm}  \\
\ce{KOH} & \hspace{3cm}   \\
\ce{C12H22O11} &\hspace{1cm}    \\
\bottomrule
\end{tabular}\end{center}

\textlcsc{ \textcolor{dgreen}{\Large \textbf{Solution}} }\\
Ammonia is a weak electrolyte and a solution of ammonia will contain ions and well as ammonia molecules. Differently \ce{KOH} is a strong electrolyte and a solution of this chemical will contain only ions (\ce{K^+} and \ce{OH^-}). Sucrose (\ce{C12H22O11}) is a nonelectrolyte and sucrose solution will only contain molecules. In table format:
\begin{center}\fontfamily{ppl}\selectfont
\begin{tabular}{ll}
\rowcolor{black!45}
\toprule
Chemical &  Particles in solution  \\
\midrule
\ce{NH3_{(g)}} &  \ce{NH3_{(aq)}} + \ce{NH4^+} + \ce{OH^-}\\
\ce{KOH_{(s)}} & \ce{K^+}+ \ce{OH^-}    \\
\ce{C12H22O11_{(s)}} & \ce{C12H22O11_{(aq)}}     \\
\bottomrule
\end{tabular}\end{center}

\faDiamond\ \textlcsc{ \textcolor{dgreen}{\Large \textbf{Study Check}} }\\
For the following chemicals indicate whether you will have only ions on solution, ions and some molecules, or  molecules: (a) \ce{H2SO4}, \ce{HNO3} and \ce{CH3OH}. 
  \\
\flushright  {\small Answer: (a) ions, (b) ions and (c) molecules.}
\end{example}%%%%%%%%%%%%%%%%%%%%%%%% EXAMPLE BOX

\begin{figure*} % FUL FIGURE
\fontfamily{ppl}\selectfont
%\begin{tabular}{llll}
\begin{tabularx}{\linewidth}{c|c|c|L}
\rowcolor{black!45}
%\toprule
\hline
Electrolyte Type        & Dissociation                     & Particles in solution& Examples \\
%\midrule
\hline
Strong     & Fully 								& Most ions & Ionic Compounds: \ce{NaCl}, NaOH, HCl, \ce{MgCl2} \\
\hline
Weak      & Partially                               & Ions \& molecules & \ce{NH3}, \ce{CH3COOH}, \ce{HF}, \ce{H2O}   \\
\hline
Nonelectrolytes    & No           				& molecules &Most covalent compounds: \ce{CH3OH}(methanol), \ce{CH3CH2OH}(ethanol), \ce{C12H22O11} (sucrose), \ce{CH4NO2}(urea)  \\
%\bottomrule
\hline
\end{tabularx}
\caption{Different types of electrolytes}
\end{figure*}


\item[\docfilehook{Breaking down electrolytes into ions}{}] Electrolytes--in particular strong electrolytes--dissociate forming ions. This way, a solution of for example \ce{NaCl} does not contain \ce{NaCl} molecules but \ce{Na^+_{(aq)}} cations and \ce{Cl^-_{(aq)}} anions. Hence it is important to break down electrolytes into ions correctly. In order to do this, you need to revert what you did while combining ions to name chemicals. First, it is important to point that you will only be able to break down strong electrolytes into ions, as weak electrolytes dissociate only partially and nonelectrolytes do not dissociate at all.
For example, let us beak magnesium chloride \ce{MgCl2_{(aq)}} into ions. We can break down this chemical as it is a strong electrolyte. This is a strong electrolytes formed by magnesium and chloride. The valence of magnesium is II and the valence of chlorine is I. The \ce{MgCl2} formula also tells us we have one magnesium and two chlorines. The overall process is:
\begin{center}\ce{MgCl2_{(aq)} -> Mg^{2+}_{(aq)} + 2Cl^{-}_{(aq)} }\end{center}
Another example if magnesium nitrite \ce{Mg(NO3)2}. This strong electrolyte--as this is an ionic salt--is made of lithium with valence I and nitrate with valence negative one. The formula indicated we have one \ce{Mg^{2+}_{(aq)}} and two \ce{NO3^{-}_{(aq)}}. Hence:
\begin{center}\ce{Mg(NO3)2_{(aq)} -> Mg^{2+}_{(aq)} + 2NO3^{-}_{(aq)} }\end{center}

\begin{example} %%%%%%%%%%%%%%%%%%%%%%%% EXAMPLE BOX
Break down the following chemicals into ions, if possible:\\
\begin{center}\fontfamily{ppl}\selectfont
\begin{tabular}{ll}
\rowcolor{black!45}
\toprule
Chemical &  Particles in solution  \\
\midrule
\ce{K2CrO4_{(aq)}} &\hspace{3cm}  \\
\ce{Ba(NO3)2_{(aq)}} & \hspace{3cm}   \\
\ce{Ba(CrO4)2_{(s)}} &\hspace{1cm}    \\
 \ce{KNO3_{(aq)} } &\hspace{1cm}    \\
\bottomrule
\end{tabular}\end{center}
\textlcsc{ \textcolor{dgreen}{\Large \textbf{Solution}} }\\
We can only break down into ions ionic compounds and oxosalts that are not solid. From the list of chemicals in the example, we will not be able to break down \ce{Ba(CrO4)2_{(s)}} into ions as it is a solid. From the other chemicals, \ce{K2CrO4_{(aq)}} is named potassium chromate and contains 2\ce{K^+_{(aq)}} and \ce{CrO4^{2-}_{(aq)}} ions. Barium nitrate--\ce{Ba(NO3)2_{(aq)}}--will produce \ce{Ba^{2+}_{(aq)}} and 2\ce{NO3^{-}_{(aq)}}. Finally, potassium nitrate-- \ce{KNO3_{(aq)} }--will produce \ce{K^{+}_{(aq)}} and \ce{NO3^{-}_{(aq)}}. In the table:
\begin{center}\fontfamily{ppl}\selectfont
\begin{tabular}{ll}
\rowcolor{black!45}
\toprule
Chemical &  Particles in solution  \\
\midrule
\ce{K2CrO4_{(aq)}} & 2\ce{K^+_{(aq)}} + \ce{CrO4^{2-}_{(aq)}}  \\
\ce{Ba(NO3)2_{(aq)}} &\ce{Ba^{2+}_{(aq)}} + 2\ce{NO3^{-}_{(aq)}}\\
\ce{Ba(CrO4)2_{(s)}} &\ce{Ba(CrO4)2_{(s)}}    \\
 \ce{KNO3_{(aq)} } &\ce{K^{+}_{(aq)}} + \ce{NO3^{-}_{(aq)}}   \\
\bottomrule
\end{tabular}\end{center}
\faDiamond\ \textlcsc{ \textcolor{dgreen}{\Large \textbf{Study Check}} }\\
Break down the following chemicals into ions, if possible: \ce{H2O_{l}}, \ce{NH3_{l}}, \ce{AgNO3_{(aq)}}.
  \\
\flushright  {\small Answer: \ce{H2O_{(l)}}, \ce{NH3_{(l)}}, \ce{Ag^+_{(aq)}}, \ce{NO3^-_{(aq)}}.}
\end{example}%%%%%%%%%%%%%%%%%%%%%%%% EXAMPLE BOX


\item[\docfilehook{Soluble and insoluble salts}{Soluble and insoluble salts}] How do we know that \ce{Ba(CrO4)2_{(s)}} is an insoluble salt? Soluble compounds are ionic compounds that dissolve in water. Insoluble salts, differently, will not solve in water. The following table will help you predict the solubility of a salt. In order to predict the solubility of salt you need to start by the right ion of the molecule, and look for it on the left column of the table. After that you need to check if the ion on the left is part of the exceptions in the same row but on the right column of the table. Let us predict for example the soluble/insoluble nature if \ce{CaSO4}. We start by looking for \ce{SO4^{2-}} in the left column to find out is soluble. Next we continue in the same line as \ce{SO4^{2-}} and look for the ion in the left \ce{Ca^{2+}}. In conclusion, even when \ce{SO4^{2-}}  is soluble, when combined with \ce{Ca^{2+}}, we have that \ce{CaSO4} is insoluble.




\begin{example} %%%%%%%%%%%%%%%%%%%%%%%% EXAMPLE BOX
Predict the soluble/insoluble nature of the following ionic compounds: (a) \ce{K2CO3}, (b) \ce{NaNO3} and (c) \ce{Ca(OH)2}.
\\
\textlcsc{ \textcolor{dgreen}{\Large \textbf{Solution}} }\\
(a) \ce{K2CO3} is soluble, as \ce{CO3^{-2}} is insoluble but when combined with \ce{K+} the salt becomes soluble. (b) All nitrates are soluble, and there are no exceptions. (c) \ce{Ca(OH)2} is soluble.
\\
\faDiamond\ \textlcsc{ \textcolor{dgreen}{\Large \textbf{Study Check}} }\\
Predict the soluble/insoluble nature of the following ionic compounds: (a) \ce{Li3PO4}, (b) \ce{Na2S} and (c) \ce{AgCl}.
  \\
\flushright {\small Answer: (a) soluble, (b) soluble and (c) insoluble.}
\end{example}%%%%%%%%%%%%%%%%%%%%%%%% EXAMPLE BOX



\begin{figure*} % FUL FIGURE
\fontfamily{ppl}\selectfont
\begin{tabular}{ll}
\rowcolor{black!45}
\toprule
Ions that form \textit{soluble} compounds... & ... \begin{bf}except\end{bf}  when combined with        \\
 \midrule
Group I ions (\ce{Na+}, \ce{Li+}, \ce{K+}, etc) & no exceptions   \\
Ammonium (\ce{NH4+}) &  no exceptions  \\
Nitrate (\ce{NO3-}) &   no exceptions  \\
Acetate (\ce{CH3COO-}) &  no exceptions   \\
Hydrogen carbonate (\ce{HCO3-}) & no exceptions   \\
Chlorate (\ce{ClO3-})  &   no exceptions \\
Halide (\ce{F-}, \ce{Cl-}, \ce{Br-})  & \ce{Pb^2+}, \ce{Ag+} and \ce{Hg2^{2+}}   \\
Sulfate (\ce{SO4^{2-}})  & \ce{Ag^{+}}, \ce{Ca^{2+}}, \ce{Sr^{2+}}, \ce{Ba^{2+}}, \ce{Hg2^{2+}} and  \ce{Pb^{2+}} \\
 \rowcolor{black!45}
\midrule
Ions that form \textit{insoluble} compounds... & ... \begin{bf}except\end{bf} when combined with     \\
 \midrule
Carbonates (\ce{CO3^2-}) & group I ions (\ce{Na+}, \ce{Li+}, \ce{K+}, etc) or ammonium (\ce{NH4+})      \\
 Chromates (\ce{CrO4^2-})    &  group I ions (\ce{Na+}, \ce{Li+}, \ce{K+}, etc) or \ce{Ca^2+}, \ce{Mg^2+}\newline  or ammonium (\ce{NH4+})  \\
Phosphates (\ce{PO4^3-})   &  group I ions (\ce{Na+}, \ce{Li+}, \ce{K+}, etc) or ammonium (\ce{NH4+})  \\
Sulfides (\ce{S^2-})    &  group I ions (\ce{Na+}, \ce{Li+}, \ce{K+}, etc) or  ammonium (\ce{NH4+})   \\
Hydroxides (\ce{OH-})     & group I ions (\ce{Na+}, \ce{Li+}, \ce{K+}, etc) or \ce{Ca^2+}, \ce{Mg^2+},  \ce{Sr^2+} \newline or ammonium (\ce{NH4+})   \\
\bottomrule
\end{tabular}
\end{figure*}
\end{description}



\section{\color{blue!30!black}{An introduction to reactions in solution}}
There are three different reactions in solution: acid-base reactions, precipitation reaction and redox reactions. The key to identify acid-base reactions is in the reactants, as an acid-base reaction results from the reaction between and acid and a base. Precipitation reactions are reactions that produce a precipitate. Hence, the key to identify a precipitation reaction is in the products. Precipitation reactions always contains a solid as a product. Redox reactions contain two elements with different redox number in the reactants and products. The key to identify redox reactions is to be able to spot elements with different oxidation state, for example: \ce{Cu} and \ce{Cu^{2+}} or \ce{H^+} and \ce{H2}. In the following we will describe more about the three different types of reactions in solution. The goal of this section is for you to be able to identify each type.
\sloppy 
\begin{description}
\item[\docfilehook{Acid-base reactions}{}] Acid-base reactions result from the reaction of an acid with a base. Both they produce water and another chemical. An example is:
\begin{center}\ce{HBr_{(aq)} +  KOH_{(aq)}  ->  KBr_{(aq)}  +  H2O_{(l)} \textcolor{red}{ (acid-base reaction) }}\end{center}
Hydrobromic acid (\ce{HBr}) is an acid and potassium hydroxide (\ce{KOH}) a base.
The result of an acid-base reaction is always water and an ionic compound, in this case \ce{KBr}.

\item[\docfilehook{Precipitation reactions}{}] Precipitation reactions result in a insoluble chemical, that is, results in a solid chemical. An example would be:\\
\ce{K2CrO4_{(aq)} + Ba(NO3)2_{(aq)} -> Ba(CrO4)2_{(s)} v + 2KNO3_{(aq)}    \textcolor{red}{ (precipitation reaction) }      }\\
The chemical \ce{Ba(CrO4)2_{(s)}} is a solid that precipitates in the solution, hence the name of the type of reaction. The symbol on \ce{Ba(CrO4)2_{(s)} v } represents the precipitation process. The solubility of a given solute such as \ce{Ba(CrO4)2_{(s)}} is the amount of solute (in grams) that can be dissolve in a given mass of solvent (in particular 100 g of solvent). A solute with a low solubility will be hard to dissolve. Think about cacao and water. The solubility of cacao is low and hence by simply adding cacao powder to water you will not be able to make a solution. However, solubility depends on the solute and solvent combination, but also on the temperature and by warming up a solvent you can increase solubility and fit more solute in the same amount of solvent. This section covers different aspects of solubility.

\item[\docfilehook{Redox reactions}{}] Redox reaction are different than acid-base or precipitation reaction. They contain the same chemical element in two different states resulting from the loss or win of electrons. Look for example:
\begin{center}\ce{Al_{(s)} + Cu_{(aq)}^{+2} -> Al_{(aq)}^{+3} + Cu_{(s)}       }\textcolor{red}{ (redox reaction) }\end{center}
We have that neither \ce{Al_{(s)}} or \ce{Cu_{(aq)}^{+2}} are an acid or a base, therefore this is not an acid-base reaction. Also there is no product precipitate, hence this is not a precipitation reaction. Differently, this is a redox reaction, as we have \ce{Al} in two different states: as metallic \ce{Al_{(s)}} and as ionic  \ce{Al_{(aq)}^{+3}}, which result from the loss of three electron. Therefore in redox reaction there is always elements in the chemicals that lose electrons. In redox reactions there is also an element that wins electrons. For example, \ce{Cu_{(s)}} and \ce{Cu_{(aq)}^{+2}} have different redox number. In particular, \ce{Cu_{(aq)}^{+2}} is the result of removing three electrons from \ce{Cu_{(s)}}. At this point, we have that this reaction is redox as it contains an element that gains electrons and an element that loses electrons. Sometimes, the redox state of the elements is not that obvious. Look at this example:
\begin{center}\ce{Fe_{(s)} +   CuSO4_{(aq)} -> FeSO4_{(aq)} +   Cu_{(s)}      }\textcolor{red}{ (redox reaction) }\end{center}
This is a redox reaction as you can find iron and copper in two states, metallic and also ionic. Therefore, these two metals have two different redox numbers in the reaction.
\begin{example} %%%%%%%%%%%%%%%%%%%%%%%% EXAMPLE BOX
Classify the following reactions as acid-base, redox or precipitation.
(a) \ce{ Fe_{(s)} +   Cu_{(aq)}^{+2} -> Fe_{(aq)}^{+2} +   Cu_{(s)} } \\
(b) \ce{ AgNO3_{(aq)} + NaCl_{(aq)} -> AgCl_{(s)} + NaNO3_{(aq)} }  \\
(c) \ce{ 2HCl_{(aq)} +   Ca(OH)2_{(aq)} -> CaCl2_{(aq)}   2H2O_{(l)} } \\
\textlcsc{ \textcolor{dgreen}{\Large \textbf{Solution}} }\\
The first reaction is a redox reaction. This is because we can find two different oxidation states for Cu and also for Fe. That means one of these elements lost electrons and the other won electrons. The second reaction is a precipitation reaction as it produces a solid. The last reaction is an acid base, as the reactants are an acid and a base.\\
\faDiamond\ \textlcsc{ \textcolor{dgreen}{\Large \textbf{Study Check}} }\\
Classify the following reactions as acid-base, redox or precipitation.
(a) \ce{ HNO2_{(aq)}  + NaOH_{(aq)} -> NaNO2_{(aq)} +  H2O_{(l)}  } 
(b) \ce{ 2Na_{(s)} +  Cl2_{(g)} -> 2NaCl_{(s)} }   
(c) \ce{ MgCl2_{(aq)}  +   2AgNO3_{(aq)} -> 2AgCl_{(s)} +   Mg(NO3)2_{(aq)} }   
\flushright  {\small Answer:  acid-base; redox; precipitation   }
\end{example}%%%%%%%%%%%%%%%%%%%%%%%% EXAMPLE BOX




\end{description}




\section{\color{blue!30!black}{Precipitation reactions and acid-base reactions}}
This section deal with two important types of reactions in solution. Precipitation reactions are characterized by the products and acid-base by the reactants. In an acid-base reaction, the reactants are an acid and a base, and they react to produce water and other chemical. Precipitation reactions produce a precipitate, that is, a solid.
\sloppy 
\begin{description}
%\item[\docfilehook{Solubility formula}{Solubility formula}] Solubility ($s$) is the grams of a solute per 100 g of solvent:
%\begin{equation*}
%\boxed{ s=\frac{\text{g of solute}}{\text{100 g of solvent}} }   
%\end{equation*}
%A saturated solution can be achieved when you fit the maximum amount of solute in the solvent. If you continue adding solute to a saturated solution it will precipitate and solid will form. 

\item[\docfilehook{Acid-base reactions}{}] Acid-base reactions result from the reaction of an acid with a base. Both they produce water and another chemical. An example is:
\begin{center}\ce{HBr_{(aq)} +  KOH_{(aq)}  ->  KBr_{(aq)}  +  H2O_{(l)} \textcolor{red}{ (acid-base reaction) }}\end{center}
Hydrobromic acid \ce{HBr_{(aq)}} is an acid and potassium hydroxide a base.

\item[\docfilehook{Precipitation reactions}{}] Precipitation reactions result in a insoluble chemical. An example would be:\\
\ce{K2CrO4_{(aq)} + Ba(NO3)2_{(aq)} -> Ba(CrO4)2_{(s)} v + 2KNO3_{(aq)}    \textcolor{red}{ (precipitation reaction) }      }\\
The chemical \ce{Ba(CrO4)2_{(s)}} is a solid that precipitates in the solution.
The solubility of a given solute such as \ce{Ba(CrO4)2_{(s)}} is the amount of solute (in grams) that can be dissolve in a given mass of solvent (in particular 100 g of solvent). A solute with a low solubility will be hard to dissolve. Think about cacao and water. The solubility of cacao is low and hence by simply adding cacao powder to water you will not be able to make a solution. However, solubility depends on the solute and solvent combination, but also on the temperature and by warming up a solvent you can increase solubility and fit more solute in the same amount of solvent. This section covers different aspects of solubility.

\item[\docfilehook{Formula equations, ionic equations and net ionic equations}{}] Electrolytes in solutions contains ions--cations and anions--however, when we write chemical formulas we barely show those ions. Differently, we just write the formulas and that is the reason that chemical equations are referred as \emph{formula equation}. Look for example:
\begin{center}\ce{K2CrO4_{(aq)} + Ba(NO3)2_{(aq)} -> Ba(CrO4)2_{(s)} + 2KNO3_{(aq)}    \textcolor{red}{ (formula equation) }      }\end{center}
In this equation, \ce{K2CrO4_{(aq)}} is actually in the form of ions: 2\ce{K^+_{(aq)}} and \ce{CrO4^{2-}_{(aq)}}. At the same time, \ce{Ba(NO3)2_{(aq)}} in the form of ions results in \ce{Ba^{2+}_{(aq)}} and 2\ce{NO3^{-}_{(aq)}}. Also, 2\ce{KNO3_{(aq)}} contains 2\ce{K^{+}_{(aq)}} and 2\ce{NO3^{-}_{(aq)}}. Finally, \ce{Ba(CrO4)_{(s)}} does not produce any ions in solution, as it is a solid.
Ionic equations result from writing all ions in a formula equation:
\\
{\raggedleft \ce{   2K^+_{(aq)} + CrO4^{2-}_{(aq)} + Ba^{2+}_{(aq)} + 2NO3^{-}_{(aq)}  -> } }  \\ 
\hspace*{\fill}
\ce{Ba(CrO4)2_{(s)} + 2K^{+}_{(aq)} + 2NO3^{-}_{(aq)}    \textcolor{red}{ (ionic equation) }      } \\
However, the ionic equation contains repeated ions. Look for example the previous equation with \ce{   2K^+_{(aq)}} on the left and on the right of the equation. If we simplify the repeated ions \\
{\raggedleft \ce{   \Cancel{2K^+_{(aq)}} + CrO4^{2-}_{(aq)} + Ba^{2+}_{(aq)} + \Cancel{2NO3^{-}_{(aq)}}  -> } }  \\ 
\hspace*{\fill}
\ce{Ba(CrO4)2_{(s)} + \Cancel{2K^{+}_{(aq)}} + \Cancel{2NO3^{-}_{(aq)}}       }\\ 
we obtain what is called as the \emph{net ionic equation}:
\begin{center} \ce{    CrO4^{2-}_{(aq)} + Ba^{2+}_{(aq)}   ->  Ba(CrO4)2_{(s)}     \textcolor{red}{ (net ionic equation) }      } \end{center}
Overall, we have that the formula equation, ionic equation and net ionic equation are just three different ways to express the same chemical equation. The first form includes only molecules whereas the second included all ions produced by each chemical. The last form, includes only ions that are not repeated in both sides of the equation.

\begin{example} %%%%%%%%%%%%%%%%%%%%%%%% EXAMPLE BOX
Write down the ionic equation and net ionic for the following formula equation:
\begin{center}\ce{HBr_{(aq)} +  KOH_{(aq)}  ->  KBr_{(aq)}  +  H2O_{(l)}}\end{center}
\textlcsc{ \textcolor{dgreen}{\Large \textbf{Solution}} }\\
Mind we can only break down strong electrolytes. Hence, water will not be expressed in the form or ions as it is a weak electrolyte. If we break down the other chemicals we have the ionic equation:
\begin{center}\ce{H^+_{(aq)} + Br^-_{(aq)} +  K^+_{(aq)}  + OH^-_{(aq)} ->  K^+_{(aq)}  + Br^-_{(aq)} +  H2O_{(l)}}\end{center}
If we eliminate the ions that are repeated in both sides:
\begin{center}\ce{H^+_{(aq)} + \Cancel{Br^-_{(aq)}} +  \Cancel{K^+_{(aq)}}  + OH^-_{(aq)} ->  \Cancel{K^+_{(aq)} } + \Cancel{Br^-_{(aq)}} +  H2O_{(l)}}\end{center}
we have the net ionic equation:
\begin{center}\ce{H^+_{(aq)}  + OH^-_{(aq)} ->   H2O_{(l)}}\end{center}
\faDiamond\ \textlcsc{ \textcolor{dgreen}{\Large \textbf{Study Check}} }\\
Write down the ionic equation and net ionic for the following formula equation:
\begin{center}\ce{ AgNO3_{(aq)} + NaBr_{(aq)}  -> AgBr_{(s)}  +  NaNO3_{(aq)}   }\end{center}
\flushright  {\small Answer:     \ce{ Ag^{+}_{(aq)} + NO3^{-}_{(aq)} + Na^{+}_{(aq)} + Br^{-}_{(aq)}  -> AgBr_{(s)}  +  Na^{+}_{(aq)} + NO3^{-}_{(aq)}   } ;  \ce{ Ag^{+}_{(aq)} + Br^{-}_{(aq)}  -> AgBr_{(s)}     }     }
\end{example}%%%%%%%%%%%%%%%%%%%%%%%% EXAMPLE BOX





\section{\color{blue!30!black}{Redox reactions}}
Redox reaction are different than acid-base or precipitation reaction. They contain the same chemical element in two different states resulting from the loss or win of electrons. Look for example:
\begin{center}\ce{2Na_{(s)} + Cl2_{(g)} -> 2NaCl_{(aq)} }    \textcolor{red}{ (redox reaction) }      \end{center}
We have that neither \ce{Na_{(s)}} or \ce{Cl2_{(g)}} are an acid or a base, therefore this is not an acid-base reaction. Also there is no product precipitate, hence this is not a precipitation reaction. Differently, this is a redox reaction, as we have \ce{Na} in two different states: as metallic \ce{Na_{(s)}} and as ionic  \ce{Na^+_{(aq)}}, which result from the loss of an electron. Therefore in redox reaction there is always elements in the chemicals that lose electrons and chemicals winning electrons.
\sloppy 
\begin{description}
\item[\docfilehook{Oxidation state or redox number}{}] How do we know if a chemical has lost of won electrons? The answer is by means of a number called redox number or oxidation state. There is four rules to identify the redox number of an element. First, single atoms or elements have zero redox number. Examples are \ce{Na} or \ce{H2}, both with redox zero. Second, monoatomic ions have redox number equal to their charge. Examples are \ce{Na^+} or \ce{Cl^-} with redox $+1$ and $-1$, respectively. Third, the redox number of fluorine is $-1$ and hydrogen on its covalent compounds $+1$. Fourth, the redox number of oxygen in normal oxides is normally $-2$, with the exception of peroxides (e.g. \ce{H2O2}) in which is $-1$.
We indicate redox numbers with roman number on top of the element. For example the redox number of manganese in this compound is $+7$: \ce{{\underline{Mn\textsuperscript{VII}}}O4^-}.
\item[\docfilehook{Calculating the redox number}{}] How do we calculate the redox number for example of manganese in this chemical: \ce{{\underline{Mn}}O4^-}, permanganate. IN order to do this, we need to set up a formula so that the redox numbers of all elements in the molecule--taking into account the number of atoms in the molecule--equals to the charge. In the case of permanganate, let us call $x$ to the redox number of manganese. We know the redox of oxygen is $-2$ and we have four oxygens in the molecule. We also know the charge of the ion is $-1$. Therefore we have:
\[x+4\cdot (-2)=-1\]
If we solve for $x$ we obtain a redox number of manganese of $\text{VII}$.
\begin{example} %%%%%%%%%%%%%%%%%%%%%%%% EXAMPLE BOX
Calculate the redox number of the elements underlined in the following molecules: (a) \ce{K2{\underline{C}}O3} and (b) \ce{H2{\underline{C}}O}.
\\
\textlcsc{ \textcolor{dgreen}{\Large \textbf{Solution}} }\\
Let us set up the redox equation for the first compound, knowing that the redox of oxygen is $-2$ and potassium $+1$. The unknown variable $x$ represents the redox number of the underlined element. We have:
\[2\cdot (+1)+x+3\cdot (-2)=0\]
Mind we have two potassium and three oxygens hence we need to time the redox of \ce{K} by two and the redox of \ce{O} by three. If we solve for $x$ we obtain a redox number for carbon of $\text{IV}$. The redox equation for the second example is:
\[2\cdot (+1) + x+ (-2)=0\]
Mind that according to the redox rules, the redox number of oxygen is $+1$. Solving for $x$ we have a redox number of zero.
\\
\faDiamond\ \textlcsc{ \textcolor{dgreen}{\Large \textbf{Study Check}} }\\
Calculate the redox number of the elements underlined in the following molecules: (a) \ce{{\underline{Cr}}2O7^{2-}} and (b) \ce{{\underline{Cr}}2O3} 
  \\
\flushright {\small Answer: $\text{VI}$; $\text{III}$.}
\end{example}%%%%%%%%%%%%%%%%%%%%%%%% EXAMPLE BOX
\end{description}



\item[\docfilehook{Redox means oxidation and reduction}{}] By comparing the redox number of the same element in two different compounds we can figure out in what compound the element has lost or gained electrons. Look for example the case of \ce{{\underline{Cr\textsuperscript{VI}}}2O7^{2-}} and  \ce{{\underline{Cr\textsuperscript{III}}}2O3}. The same element in two different molecules has two different redox numbers. In the case of dichromate, the redox of \ce{Cr} is  $\text{VI}$, whereas in the case of chromium(III) oxide the redox of \ce{Cr} is $\text{III}$. The larger the redox number the more oxidized is an element, and that means the element has lost electrons. The smaller the redox number the more reduced is the element and that means it has gained electrons. If we compare both case, we have that \ce{Cr} in dichromate is oxidized--it lost electrons--and \ce{Cr} in chromium(III) oxide is reduced--it gained electrons.

\item[\docfilehook{Redox numbers in chemical reactions}{}] The goal is to identify the element that undergoes oxidation and reduction in a chemical reaction. 
We can reach this goal by using the half-reaction method. Every redox reaction is composed of two process, a reduction and the oxidation. These two processes can be separated into two half-reactions so that the combination of both half-reactions lead to the balanced redox. Let us work on a simple unbalanced redox reaction:
\begin{center}\ce{Al_{(s)} + Cu_{(aq)}^{+2} -> Al_{(aq)}^{+3} + Cu_{(s)}       }\end{center}
Solid \ce{Al_{(s)}} on the reactant side has zero redox number, whereas ionic \ce{Al_{(aq)}^{+3}} on the product side has redox number equal to $\text{III}$. Al has undergone oxidation as its redox number increases from zero to three. Al has lost three electrons. We can write the oxidation half-reaction:
\begin{center}\ce{Al_{(s)}  -> Al_{(aq)}^{+3} + 3e^-     \textcolor{red}{ (oxidation) }   }\end{center}
Mind that electrons have negative charge and we add electrons to compensate the charge of \ce{Al_{(aq)}^{+3}}. 
Now let us compare the redox number of \ce{Cu}. In the reactant side we have \ce{Cu_{(aq)}^{+2}} with redox of $\text{II}$. In the product side we have \ce{Cu_{(s)}} with zero redox. Cu has undergone reduction as its redox number has decreases. This means it has gained electrons, in particular two electrons: 
\begin{center}\ce{Cu_{(aq)}^{+2} + 2e^- ->Cu_{(s)}      \textcolor{red}{ (reduction) }   }\end{center}

 \item[\docfilehook{Balancing simple redox reactions}{}] The goal here is to balance a redox chemical reaction by combining two half-reactions. In the example above the oxidation and reaction involve different number of electrons. Hence in order to be able to add both redox we need to time each half-reaction by a number so that the number of electrons cancel out. As the first reaction involved three electrons and the second two, we will do:
 \begin{center}
$2\cdot \big($ \ce{ Al_{(s)}  -> Al_{(aq)}^{+3} + 3e^- } $\big)$ \hspace*{0pt}\hfill  \textcolor{red}{ (oxidation) }\\
$3\cdot \big($  \ce{ Cu_{(aq)}^{+2} + 2e^- ->Cu_{(s)} }$\big)$ \hspace*{0pt}\hfill  \textcolor{red}{ (reduction) } \\
\rule{12cm}{0.4pt}$+$\\
  \ce{ 2Al_{(s)}  + 3Cu_{(aq)}^{+2} + \Cancel{6e^-} ->2Al_{(aq)}^{+3} + 3Cu_{(s)}  + \Cancel{6e^-}}\hspace*{0pt}\hfill  \textcolor{green}{ (redox) } \\
 \end{center}
The overall balanced redox equation is:

\vspace{0.5cm}
 \begin{reaction*}
   2 "\OX{o1,\ox*{0,Al}}" \sld{} + 3 "\OX{r1,\ox*{+2,Cu^2+}}" \aq{} 
    ->
  2 "\OX{o2,\ox*{+3,Al^3+}}" \aq{} + 3 "\OX{r2,\ox*{0,Cu}}" \sld{}
  "\redox(o1,o2)[->]{\small Oxidation: $- 3\el$}"
  "\redox(r1,r2)[->][-1]{\small Reduction: $+ 2\el$}"
\end{reaction*}
\vspace{0.5cm}
 \item[\docfilehook{Balancing redox reactions in acidic medium}{}] Redox reactions happen in either basic or acidic medium. Here we will go over how to balance redox reactions in acidic medium. In order to do this we will first separate the reaction in two half-reactions. In each semi-reaction we will balance all elements but hydrogen and oxygen. When all elements are balanced, we will proceed to balance \ce{O} by adding \ce{H2O} molecules and we will balance \ce{H} by adding \ce{H^+}. Finally, we all add electrons to compensate the charge of the reaction. Let us work on an example:
\begin{center}\ce{MnO4_{(aq)}^{-} + Fe_{(aq)}^{2+} -> Mn_{(aq)}^{2+}  + Fe_{(aq)}^{3+}       }\end{center}
One of the semi-reactions involve Manganese whereas the other involves Iron. The redox number of \ce{Mn} in permanganate is $\text{VII}$ hence Manganese is being reduced, as its redox number decreases from $\text{VII}$ to $\text{II}$, whereas Iron is being oxidized as its redox number increases from $\text{II}$ to $\text{III}$. The oxidation half-reaction does not contain hydrogen or oxygen hence we will only have to balance the charger with one electron:
\begin{center}\ce{ Fe_{(aq)}^{2+} ->  Fe_{(aq)}^{3+} + e^-    \textcolor{red}{ (oxidation) }   }\end{center}
The reduction half-reaction contains oxygen. Hence, we will have to add \ce{H2O} molecules to balance oxygen and \ce{H^+} to balance hydrogen. In particular, we will need two water molecules--as \ce{MnO4^-} has four oxygens and we will have to add four protons as we are adding two molecules of water. Finally, we need to add three electrons to equalize the number of electrons:

\begin{center}\ce{ MnO4_{(aq)}^{-}  + 4H_{(aq)}^{+}  + 3e^- -> Mn_{(aq)}^{2+}  + 2H2O_{(l)}    \textcolor{red}{ (reduction) }   }\end{center}
As the oxidation involves one electron and the reduction three, we need to time the oxidation by three:
 \begin{center}
$3\cdot \big($ \ce{ Fe_{(aq)}^{2+} ->  Fe_{(aq)}^{3+} + e^-  } $\big)$ \hspace*{0pt}\hfill  \textcolor{red}{ (oxidation) }\\
 \ce{ MnO4_{(aq)}^{-}  + 4H_{(aq)}^{+}  + 3e^- -> Mn_{(aq)}^{2+}  + 2H2O_{(l)}  }  \hspace*{0pt}\hfill  \textcolor{red}{ (reduction) } \\
\rule{12cm}{0.4pt}$+$\\
{\raggedleft \ce{  3Fe_{(aq)}^{2+} + MnO4_{(aq)}^{-}  + 4H_{(aq)}^{+}  + \Cancel{3e^-} -> } \hspace*{0pt}\hfill  \textcolor{green}{ (redox) } }  \\ 
\hspace*{\fill}
\ce{3Fe_{(aq)}^{3+} + Mn_{(aq)}^{2+}  + 2H2O_{(l)} + \Cancel{3e^-}     }
 \end{center}

\vspace{0.5cm}
 \begin{reaction*}
  4 H^+\aq{} + "\OX{r01,\ox*{+7,Mn}}" O4 \aq{} + 3 "\OX{o01,\ox*{+2,Fe^2+}}" \aq{} 
    ->
  3 "\OX{o02,\ox*{+3,Fe^3+}}" \aq{} +  "\OX{r02,\ox*{+2,Mn^2+}}" \aq{} + 2 H2O\lqd{}
  "\redox(o01,o02)[->]{\small Oxidation: $- 1\el$}"
  "\redox(r01,r02)[->][-1]{\small Reduction: $+ 3\el$}"
\end{reaction*}\vspace{0.5cm}
\begin{example} %%%%%%%%%%%%%%%%%%%%%%%% EXAMPLE BOX
Balance the following redox in acidic medium:
\begin{center}\ce{ Cr2O7^{2-} + SO3^{2-}  -> Cr^{3+}  + SO4^{2-}}\end{center}
\textlcsc{ \textcolor{dgreen}{\Large \textbf{Solution}} }\\
We locate the same element in both sides of the reaction with different redox number. We found Chromium in the form of dichromate (\ce{Cr2O7^{2-}}) with redox number $\text{VI}$ and Chromium in the product side with redox $\text{III}$. Therefore Chromium is being reduced, as its redox number decreases. Differently, we found sulphur in the reactants side in the form of sulfite (\ce{SO3^{2-}}) with redox number of $\text{IV}$ and in the product side with redox number of $\text{VI}$. Therefore Sulphur is being oxidized. We will first set up the oxidation half-reaction knowing that we have different amounts of \ce{Cr} in both side and that we will have to add water molecules to balance \ce{O} and protons to balance  \ce{H}. As we have seven oxygens in \ce{Cr2O7^{2-}} we will have to add seven water molecules. Also, as we add seven water molecules we will have to add fourteen protons. We will need six electrons to compensate the charge:
\begin{center}\ce{ Cr2O7^{2-} + 14 H^+ + 6 e^- ->  2Cr^{3+} + 7 H2O     \textcolor{red}{ (oxidation) }   }\end{center}
For the reduction half-reaction, we have a difference of one oxygen atoms and hence we will need one water molecule and two protons; we will need two electrons to compensate the charge:
\begin{center}\ce{ SO3^{2-} + H2O   ->  SO4^{2-} + 2 H^+ + 2 e^-    \textcolor{red}{ (reduction) }   }\end{center}
In order to add both half-reactions as the reduction involves two electrons and the oxidation six, we will have to multiply the reduction half-reaction by three:
 \begin{center}
$1\cdot \big($ \ce{ Cr2O7^{2-} + 14 H^+ + 6 e^- ->  2 Cr^{3+} + 7 H2O  } $\big)$ \hspace*{0pt}\hfill  \textcolor{red}{ (oxidation) }\\
 $3\cdot \big($\ce{ SO3^{2-} + H2O   ->  SO4^{2-} + 2 H^+ + 2 e^- }$\big)$  \hspace*{0pt}\hfill  \textcolor{red}{ (reduction) } \\
\rule{10.5cm}{0.4pt}$+$\\
{\raggedleft \ce{  Cr2O7^{2-} + 14 H^+ + 3 SO3^{2-} + 3 H2O + \Cancel{6 e^-}   -> } \hspace*{0pt}\hfill  \textcolor{green}{ (redox) } }  \\ 
\hspace*{\fill}
\ce{2 Cr^{3+} + 7 H2O  + 3 SO4^{2-} + 6 H^+ + \Cancel{6 e^-}    }
 \end{center}
After we cancel the electrons and eliminate protons and water molecules, the balanced reaction is:
\vspace{0.5cm}
 \begin{reaction*}
    "\OX{r11,\ox*{+6,Cr2}}" O7^2-  \aq{} + 8 H^+\aq{}  + 3 "\OX{o11,\ox*{+4,S}}" O3^2- \aq{} 
    ->
  3 "\OX{o12,\ox*{+6,S}}" O4^2- \aq{} + 2 "\OX{r12,\ox*{+3,Cr^3+}}" \aq{} + 7 H2O\lqd{}
  "\redox(o11,o12)[->]{\small Oxidation: $- 1\el$}"
  "\redox(r11,r12)[->][-1]{\small Reduction: $+ 3\el$}"
\end{reaction*}\vspace{0.5cm}\\
\faDiamond\ \textlcsc{ \textcolor{dgreen}{\Large \textbf{Study Check}} }\\
Balance the following redox in acidic medium:
\begin{center}\ce{  Cr2O2^{-7}_{(aq)} + HNO2_{(aq)} -> Cr^{3+}_{(aq)} + NO3^-_{(aq)}    }\end{center}
\flushright {\small Answer: \ce{3HNO2_{(aq)} + 5H^+_{(aq)} + Cr2O2^{7-}_{(aq)} -> 3NO3^-_{(aq)} + 2Cr^{3+}_{(aq)} + 4H2O_{(l)}}.}
\end{example}%%%%%%%%%%%%%%%%%%%%%%%% EXAMPLE BOX

 \item[\docfilehook{Balancing redox reactions in basic medium}{}] In order to balance a redox in basic medium we need first to balance the reaction in acidic medium. After, we will compensate all \ce{H^+} with \ce{OH^-} in both sides of the reaction. Mind that when combining \ce{H^+} with \ce{OH^-} we obtain \ce{H2O}. For example, in order to balance the following reaction in basic medium:
\begin{center}\ce{ 3Fe_{(aq)}^{2+} + MnO4_{(aq)}^{-}  + 4H_{(aq)}^{+} -> 3Fe_{(aq)}^{3+} + Mn_{(aq)}^{2+}  + 2H2O_{(l)}      }\end{center}
we will add four \ce{OH^-} in both sides:
\begin{center}\ce{ 3Fe_{(aq)}^{2+} + MnO4_{(aq)}^{-}  + 4H_{(aq)}^{+} +  4OH_{(aq)}^{-} -> 3Fe_{(aq)}^{3+} + Mn_{(aq)}^{2+}  + 2H2O_{(l)} +  4OH_{(aq)}^{-}     }\end{center}
And after cancelling the four protons with the four hydroxyls, we have:
\begin{center}\ce{ 3Fe_{(aq)}^{2+} + MnO4_{(aq)}^{-}  + 4H2O_{(l)}    -> 3Fe_{(aq)}^{3+} + Mn_{(aq)}^{2+}  + 2H2O_{(l)} +  4OH_{(aq)}^{-}     }\end{center}
Now we have four water molecules in the left and two in the right. We will cancel two molecules from each side in order not to report water molecules twice:
\begin{center}\ce{ 3Fe_{(aq)}^{2+} + MnO4_{(aq)}^{-}  + 2H2O_{(l)}    -> 3Fe_{(aq)}^{3+} + Mn_{(aq)}^{2+}  +  4OH_{(aq)}^{-}     }\end{center}





\end{description}




\end{document}
