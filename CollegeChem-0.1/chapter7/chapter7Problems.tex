\documentclass[main.tex]{subfiles}
\begin{document}\newpage
\setdoublesep{0.35700 em}  % 'Bond Spacing'
\setatomsep{1.78500 em}    % 'Fixed Length'
\setbondoffset{0.18265 em} % 'Margin Width'
\newcommand{\bondwidth}{0.06642 em} % 'Line Width'
\setbondstyle{line width = \bondwidth}
\newgeometry{left=0.8in,right=0.8in, top=2.5cm,bottom=2cm}
\fancyhfoffset[E,O]{0pt}
\setlength{\columnsep}{30pt}
\begin{conclusion}
\end{conclusion}
\setstretch{0.3}
\begin{multicols*}{2}

{\raggedright\textsc{\textbf{The Mole }}\par}



\begin{enumerate}

\item Calculate the number of atoms in 8 moles of Carbon:
\begin{enumerate}[label=(\alph*)]
\begin{multicols*}{2}
\item $2.4\times 10^{24}$
\item $4.8\times 10^{24}$
\item $6.0\times 10^{24}$
\item $9\times 10^{24}$
\item $6\times 10^{23}$
\end{multicols*}\flushright  {\small Ans: (b)}
\end{enumerate}

%\item Calculate the number of atoms in 10 moles of Carbon:
%\begin{enumerate}[label=(\alph*)]
%\begin{multicols*}{2}
%\item $2.4\times 10^{24}$
%\item $4.8\times 10^{24}$
%\item $6.0\times 10^{24}$
%\item $9\times 10^{24}$
%\item $6\times 10^{23}$
%\end{multicols*}\flushright  {\small Ans: (c)}
%\end{enumerate}
%\item Calculate the number of atoms in 15 moles of Carbon:
%\begin{enumerate}[label=(\alph*)]
%\begin{multicols*}{2}
%\item $2.4\times 10^{24}$
%\item $4.8\times 10^{24}$
%\item $6.0\times 10^{24}$
%\item $9\times 10^{24}$
%\item $6\times 10^{23}$
%\end{multicols*}\flushright  {\small Ans: (d)}
%\end{enumerate}
%
%\item Calculate the number of atoms in 4 moles of Carbon:
%\begin{enumerate}[label=(\alph*)]
%\begin{multicols*}{2}
%\item $2.4\times 10^{24}$
%\item $4.8\times 10^{24}$
%\item $6.0\times 10^{24}$
%\item $9\times 10^{24}$
%\item $6\times 10^{23}$
%\end{multicols*}\flushright  {\small Ans: (a)}
%\end{enumerate}
%\item Calculate the number of atoms in 1 moles of Carbon:
%\begin{enumerate}[label=(\alph*)]
%\begin{multicols*}{2}
%\item $2.4\times 10^{24}$
%\item $4.8\times 10^{24}$
%\item $6.0\times 10^{24}$
%\item $9\times 10^{24}$
%\item $6\times 10^{23}$
%\end{multicols*}
%\end{enumerate}

\item How many moles equal to $3.2\times 10^{21}$ atoms:
\begin{enumerate}[label=(\alph*)]
\begin{multicols*}{2}
\item 4983.39 moles
\item 0.33 moles
\item $5.3\times 10^{-3}$ moles
\item $8.3\times 10^{-9}$moles
\item 0.99 moles
\end{multicols*}\flushright  {\small Ans: (c)}
\end{enumerate}

\item How many moles equal to $2\times 10^{23}$ atoms:
\begin{enumerate}[label=(\alph*)]
\begin{multicols*}{2}
\item 4983.39 moles
\item 0.33 moles
\item $5.3\times 10^{-3}$ moles
\item $8.3\times 10^{-9}$moles
\item 0.99 moles
\end{multicols*}\flushright  {\small Ans: (b)}
\end{enumerate}


%\item How many moles equal to $6\times 10^{23}$ atoms:
%\begin{enumerate}[label=(\alph*)]
%\begin{multicols*}{2}
%\item 4983.39 moles
%\item 0.33 moles
%\item $5.3\times 10^{-3}$ moles
%\item $8.3\times 10^{-9}$moles
%\item 0.99 moles
%\end{multicols*}\flushright  {\small Ans: (e)}
%\end{enumerate}
%
%
%%\item How many moles equal to $5\times 10^{15}$ atoms:
%%\begin{enumerate}[label=(\alph*)]
%%\begin{multicols*}{2}
%%\item 4983.39 moles
%%\item 0.33 moles
%%\item $5.3\times 10^{-3}$ moles
%%\item $8.3\times 10^{-9}$moles
%%\item 0.99 moles
%%\end{multicols*}\flushright  {\small Ans: (d)}
%%\end{enumerate}
%
%\item How many moles equal to $3\times 10^{27}$ atoms:
%\begin{enumerate}[label=(\alph*)]
%\begin{multicols*}{2}
%\item 4983.39 moles
%\item 0.33 moles
%\item $5.3\times 10^{-3}$ moles
%\item $8.3\times 10^{-9}$moles
%\item 0.99 moles
%\end{multicols*}\flushright  {\small Ans: (a)}
%\end{enumerate}

\item Calculate the number of C atoms in 3 moles of \ce{C10H14N2}:
\begin{enumerate}[label=(\alph*)]
\begin{multicols*}{2}
\item $1\times 10^{-20}$ atoms
\item $3.6\times 10^{-22}$ atoms
\item $2.5\times 10^{-25}$ atoms
\item $1.8\times 10^{25}$ atoms
\item 0.99 moles
\flushright  {\small Ans: (d)}
\end{multicols*}
\end{enumerate}
\item Calculate the number of H atoms in 3 moles of \ce{C10H14N2}:
\begin{enumerate}[label=(\alph*)]
\begin{multicols*}{2}
\item $1\times 10^{-20}$ atoms
\item $3.6\times 10^{-22}$ atoms
\item $2.5\times 10^{-25}$ atoms
\item $1.8\times 10^{-25}$ atoms
\item 0.99 moles
\flushright  {\small Ans: (c)}
\end{multicols*}
\end{enumerate}
%\item Calculate the number of N atoms in 3 moles of \ce{C10H14N2}:
%\begin{enumerate}[label=(\alph*)]
%\begin{multicols*}{2}
%\item $1\times 10^{-20}$ atoms
%\item $3.6\times 10^{-24}$ atoms
%\item $2.5\times 10^{-25}$ atoms
%\item $1.8\times 10^{-25}$ atoms
%\item 0.99 moles
%\flushright  {\small Ans: (b)}
%\end{multicols*}
%\end{enumerate}


{\raggedright\textsc{\textbf{The Molecular Mass }}\par}


\item Compute the molar mass of \ce{NH3}:
\begin{enumerate}[label=(\alph*)]
\begin{multicols*}{2}
\item 17 $g\cdot mol^{-1}$  
\item 32 $g\cdot mol^{-1}$  
\item 28 $g\cdot mol^{-1}$  
\item 291.71 $g\cdot mol^{-1}$   
\item 2 $g\cdot mol^{-1}$  
\end{multicols*}\flushright  {\small Ans: (a)}
\end{enumerate}

%\item Compute the molar mass of \ce{O2}:
%\begin{enumerate}[label=(\alph*)]
%\begin{multicols*}{2}
%\item 17 $g\cdot mol^{-1}$  
%\item 32 $g\cdot mol^{-1}$  
%\item 28 $g\cdot mol^{-1}$  
%\item 291.71 $g\cdot mol^{-1}$   
%\item 2 $g\cdot mol^{-1}$  
%\end{multicols*}\flushright  {\small Ans: (b)}
%\end{enumerate}

\item Compute the molar mass of \ce{CO}:
\begin{enumerate}[label=(\alph*)]
\begin{multicols*}{2}
\item 17 $g\cdot mol^{-1}$  
\item 32 $g\cdot mol^{-1}$  
\item 28 $g\cdot mol^{-1}$  
\item 291.71 $g\cdot mol^{-1}$   
\item 2 $g\cdot mol^{-1}$  
\end{multicols*}\flushright  {\small Ans: (c)}
\end{enumerate}

%\item Compute the molar mass of \ce{H2}:
%\begin{enumerate}[label=(\alph*)]
%\begin{multicols*}{2}
%\item 17 $g\cdot mol^{-1}$  
%\item 32 $g\cdot mol^{-1}$  
%\item 28 $g\cdot mol^{-1}$  
%\item 291.71 $g\cdot mol^{-1}$   
%\item 2 $g\cdot mol^{-1}$  
%\end{multicols*}\flushright  {\small Ans: (e)}
%\end{enumerate}



\item Compute the molar mass of \ce{Fe2(CO3)3}:
\begin{enumerate}[label=(\alph*)]
\begin{multicols*}{2}
\item 17 $g\cdot mol^{-1}$  
\item 32 $g\cdot mol^{-1}$  
\item 116 $g\cdot mol^{-1}$  
\item 291.71 $g\cdot mol^{-1}$   
\item 2 $g\cdot mol^{-1}$  
\end{multicols*}\flushright  {\small Ans: (c)}
\end{enumerate}



\item How many grams are there in 3 moles of Silver (AW=$107.9g\cdot mol^{-1}$):
\begin{enumerate}[label=(\alph*)]
\begin{multicols*}{2}
\item 117.3 g  
\item 176.8 g 
\item 3 g 
\item  323.7 g 
\item  156 g 
\end{multicols*}\flushright  {\small Ans: (d)}
\end{enumerate}

\item How many grams are there in 3 moles of Potassium (AW=$39.10g\cdot mol^{-1}$):
\begin{enumerate}[label=(\alph*)]
\begin{multicols*}{2}
\item 117.3 g  
\item 176.8 g 
\item 3 g
\item  323.7 g 
\item  156 g 
\end{multicols*}\flushright  {\small Ans: (b)}
\end{enumerate}

\item How many grams are there in 3 moles of Cromium (AW=$52g\cdot mol^{-1}$):
\begin{enumerate}[label=(\alph*)]
\begin{multicols*}{2}
\item 117.3 g  
\item 176.8 g 
\item 3 g 
\item  323.7 g 
\item  156 g 
\end{multicols*}\flushright  {\small Ans: (e)}
\end{enumerate}

\item How many grams are there in 3 moles of atomic hydrogen (AW=$1g\cdot mol^{-1}$):
\begin{enumerate}[label=(\alph*)]
\begin{multicols*}{2}
\item 117.3 g  
\item 176.8 g 
\item 3 g 
\item  323.7 g 
\item  156 g 
\end{multicols*}\flushright  {\small Ans: (c)}
\end{enumerate}
\item Calculate the molar mass of the following molecules:
\begin{tabularx}{0.45\textwidth}{
    >{\centering}m{.25\linewidth} 
    *{3}{Y} }
  \toprule
\heading{Formula} & \multicolumn{3}{c}{\textbf{MW}}   \\
    \midrule
   \ce{CO2} & 	\multicolumn{3}{c}{     }    \\
    \ce{NO } & 	\multicolumn{3}{c}{     }    \\
        \ce{C6H12O8} & 	\multicolumn{3}{c}{     }    \\
        \ce{CH4N2O} & 	\multicolumn{3}{c}{     }    \\
      \bottomrule
\end{tabularx}\\

\item Calculate the molar weight (MW) of the following molecules:\\
\begin{tabularx}{0.45\textwidth}{
    >{\centering}m{.25\linewidth} 
    *{3}{Y} }
  \toprule
\heading{Formula} & \multicolumn{3}{c}{\textbf{MW}}   \\
    \midrule
   \ce{NO2} & 	\multicolumn{3}{c}{     }    \\
    \ce{CH3OH } & 	\multicolumn{3}{c}{     }    \\
        \ce{CH3COOH} & 	\multicolumn{3}{c}{     }    \\
        \ce{HNO3} & 	\multicolumn{3}{c}{     }    \\
      \bottomrule
\end{tabularx}\\


%
%\item How many grams are there in 3 moles of Cobalt (AW=$58.93g\cdot mol^{-1}$):
%\begin{enumerate}[label=(\alph*)]
%\begin{multicols*}{2}
%\item 117.3 g  
%\item 176.8 g 
%\item 3 g 
%\item  323.7 g 
%\item  156 g 
%\end{multicols*}
%\end{enumerate}
%
%
%\item How many moles are there in 2 grams of carbon (AW=$12g\cdot mol^{-1}$):
%\begin{enumerate}[label=(\alph*)]
%\begin{multicols*}{2}
%\item 117.3 mol  
%\item 176.8 mol
%\item 3 mol
%\item  323.7 mol 
%\item  156 mol
%\end{multicols*}
%\end{enumerate}


\item How many grams are there in 4 moles of \ce{C6H12O6}:
\begin{enumerate}[label=(\alph*)]
\begin{multicols*}{2}
\item 100.10 grams  
\item 834.02 grams
\item 720.64 grams
\item 602.50 grams
\item 55.20 grams
\end{multicols*}\flushright  {\small Ans: (c)}
\end{enumerate}


\item How many C atoms are there in 3 moles of \ce{C6H12O6}:
\begin{enumerate}[label=(\alph*)]
\begin{multicols*}{2}
\item $2\times 10^{27}$ atoms  
\item $2.13\times 10^{26}$ atoms
\item $1.08\times 10^{25}$ atoms
\item $2.17\times 10^{25}$ atoms
\item $5\times 10^{25}$ atoms
\end{multicols*}\flushright  {\small Ans: (a)}
\end{enumerate}

\item How many N atoms are there in 3 moles of \ce{C6H12O6}:
\begin{enumerate}[label=(\alph*)]
\begin{multicols*}{2}
\item $2\times 10^{27}$ atoms  
\item $2.13\times 10^{26}$ atoms
\item $1.08\times 10^{25}$ atoms
\item $3.9\times 10^{27}$ atoms
\item $5\times 10^{25}$ atoms
\end{multicols*}\flushright  {\small Ans: (d)}
\end{enumerate}

\item Fill the conversion factor that calculates the final property:
 \begin{equation*}\begin{split}
4\cancel{\text{ moles of }\ce{CO2}} \times \dfrac{\hlmath{\hspace{35pt}}\text{ g of }\ce{CO2}}{\hlmath{\hspace{35pt}}\cancel{\text{ moles of }\ce{CO2}}}\\
=\hlmath{\hspace{35pt}}\text{ g of }\ce{CO2}.
\end{split}\end{equation*}


\item Fill the conversion factor that calculates the final property:
 \begin{equation*}\begin{split}
10\cancel{\text{ g of }\ce{NO}} \times \dfrac{\hlmath{\hspace{35pt}}\text{ moles of }\ce{NO}}{\hlmath{\hspace{35pt}}\cancel{\text{ g of }\ce{NO}}}\\
=\hlmath{\hspace{35pt}}\text{ moles of }\ce{NO}.
\end{split}\end{equation*}


\item Fill the conversion factor that calculates the final property:
 \begin{equation*}\begin{split}
5\cancel{\text{ moles of }\ce{C6H12O6}} \times \dfrac{\hlmath{\hspace{35pt}}}{\hlmath{\hspace{35pt}}}\\
=\hlmath{\hspace{35pt}}\text{ g of }\ce{C6H12O6}.
\end{split}\end{equation*}


\item Fill the conversion factor that calculates the final property:
 \begin{equation*}\begin{split}
7\cancel{\text{ g of }\ce{CH4N2O}} \times \dfrac{\hlmath{\hspace{35pt}}}{\hlmath{\hspace{35pt}}}\\
=\hlmath{\hspace{35pt}}\text{ moles of }\ce{CH4N2O}.
\end{split}\end{equation*}


\item Fill the conversion factor that calculates the final property:
 \begin{equation*}\begin{split}
10^{24}\cancel{\text{ molecules of }\ce{NO2}} \times \dfrac{\hlmath{\hspace{35pt}}\text{ moles of }\ce{NO2}}{\hlmath{\hspace{35pt}}\cancel{\text{ molecules of }\ce{NO2}}}\\
=\hlmath{\hspace{35pt}}\text{ moles of }\ce{NO2}.
\end{split}\end{equation*}


\item Fill the conversion factor that calculates the final property:
 \begin{equation*}\begin{split}
3\cancel{\text{ moles of }\ce{NO}} \times \dfrac{\hlmath{\hspace{35pt}}\text{ molecules of }\ce{NO}}{\hlmath{\hspace{35pt}}\cancel{\text{ moles of }\ce{NO}}}\\
=\hlmath{\hspace{35pt}}\text{ molecules of }\ce{NO}.
\end{split}\end{equation*}

\item Fill the conversion factor that calculates the final property:
 \begin{equation*}\begin{split}
6\cancel{\text{ moles of }\ce{C6H12O6}} \times \dfrac{\hlmath{\hspace{35pt}}}{\hlmath{\hspace{35pt}}}\\
=\hlmath{\hspace{35pt}}\text{ molecules of }\ce{C6H12O6}.
\end{split}\end{equation*}


\item Fill the conversion factor that calculates the final property:
 \begin{equation*}\begin{split}
10^{25}\cancel{\text{ molecules of }\ce{CH4N2O}} \times \dfrac{\hlmath{\hspace{35pt}}}{\hlmath{\hspace{35pt}}}\\
=\hlmath{\hspace{35pt}}\text{ moles of }\ce{CH4N2O}.
\end{split}\end{equation*}



\item Fill the conversion factor that calculates the final property:
 \begin{equation*}\begin{split}
10^{26}\cancel{\text{ molecules of }\ce{NO2}} \times \dfrac{\hlmath{\hspace{35pt}}\text{ atoms of }\ce{O}}{\hlmath{\hspace{35pt}}\cancel{\text{ molecules of }\ce{NO2}}}\\
=\hlmath{\hspace{35pt}}\text{ atoms of }\ce{O}.
\end{split}\end{equation*}


\item Fill the conversion factor that calculates the final property:
 \begin{equation*}\begin{split}
10^{22}\cancel{\text{ atoms of }\ce{O}} \times \dfrac{\hlmath{\hspace{35pt}}\text{ molecules of }\ce{H2O}}{\hlmath{\hspace{35pt}}\cancel{\text{ atoms of }\ce{O}}}\\
=\hlmath{\hspace{35pt}}\text{ molecules of }\ce{H2O}.
\end{split}\end{equation*}


\item Fill the conversion factor that calculates the final property:
 \begin{equation*}\begin{split}
6\cancel{\text{ molecules of }\ce{C6H12O6}} \times \dfrac{\hlmath{\hspace{35pt}}}{\hlmath{\hspace{35pt}}}\\
=\hlmath{\hspace{35pt}}\text{ atoms of }\ce{C}.
\end{split}\end{equation*}


\item Fill the conversion factor that calculates the final property:
 \begin{equation*}\begin{split}
10^{21}\cancel{\text{ atoms of }\ce{N}} \times \dfrac{\hlmath{\hspace{35pt}}}{\hlmath{\hspace{35pt}}}\\
=\hlmath{\hspace{35pt}}\text{ molecules of }\ce{CH4N2O}.
\end{split}\end{equation*}

%\item Using the giving information and the MW calculate the unknown quantity:\\
%\begin{tikzpicture}
%\node [block] (box1) at (0,0) [rectangle,draw=white,fill=red!20!white] {\textcolor{black}{6 Grams of \ce{H2O}}};
%\node [block] (box2) at  (2,0) [rectangle,draw=white,fill=orange!20!white] {\textcolor{black}{{\Large ?} Moles of \ce{H2O}}};
%\node [block] (box3) at  (4,0) [rectangle,draw=white,fill=yellow!20!white] {\textcolor{black}{Molec. of \ce{H2O}}};
%\node [block] (box4) at  (6,0) [rectangle,draw=white,fill=green!20!white] {\textcolor{black}{H Atoms}};
%\draw[thick,->] (box1.east) -- (box2.west) ;
%\draw[thick,->] (box2.east) -- (box3.west) ;
%\draw[thick,->] (box3.east) -- (box4.west) ;
%\end{tikzpicture}
%
%
%\item Using the giving information and the MW calculate the unknown quantity:\\
%\begin{tikzpicture}
%\node [block] (box1) at (0,0) [rectangle,draw=white,fill=red!20!white] {\textcolor{black}{Grams of \ce{H2O}}};
%\node [block] (box2) at  (2,0) [rectangle,draw=white,fill=orange!20!white] {\textcolor{black}{5 Moles of \ce{H2O}}};
%\node [block] (box3) at  (4,0) [rectangle,draw=white,fill=yellow!20!white] {\textcolor{black}{{\Large ?} Molec. of \ce{H2O}}};
%\node [block] (box4) at  (6,0) [rectangle,draw=white,fill=green!20!white] {\textcolor{black}{H Atoms}};
%\draw[thick,->] (box1.east) -- (box2.west) ;
%\draw[thick,->] (box2.east) -- (box3.west) ;
%\draw[thick,->] (box3.east) -- (box4.west) ;
%\end{tikzpicture}
%
%\item Using the giving information and the MW calculate the unknown quantity:\\
%\begin{tikzpicture}
%\node [block] (box1) at (0,0) [rectangle,draw=white,fill=red!20!white] {\textcolor{black}{Grams of \ce{CO2}}};
%\node [block] (box2) at  (2,0) [rectangle,draw=white,fill=orange!20!white] {\textcolor{black}{Moles of \ce{CO2}}};
%\node [block] (box3) at  (4,0) [rectangle,draw=white,fill=yellow!20!white] {\textcolor{black}{$10^9$ Molec. of \ce{CO2}}};
%\node [block] (box4) at  (6,0) [rectangle,draw=white,fill=green!20!white] {\textcolor{black}{{\Large ?} O Atoms}};
%\draw[thick,->] (box1.east) -- (box2.west) ;
%\draw[thick,->] (box2.east) -- (box3.west) ;
%\draw[thick,->] (box3.east) -- (box4.west) ;
%\end{tikzpicture}
%
%\item Using the giving information and the MW calculate the unknown quantity:\\
%\begin{tikzpicture}
%\node [block] (box1) at (0,0) [rectangle,draw=white,fill=red!20!white] {\textcolor{black}{3 Grams of \ce{HNO2}}};
%\node [block] (box2) at  (2,0) [rectangle,draw=white,fill=orange!20!white] {\textcolor{black}{Moles of \ce{HNO2}}};
%\node [block] (box3) at  (4,0) [rectangle,draw=white,fill=yellow!20!white] {\textcolor{black}{{\Large ?} Molec. of \ce{HNO2}}};
%\node [block] (box4) at  (6,0) [rectangle,draw=white,fill=green!20!white] {\textcolor{black}{O Atoms}};
%\draw[thick,->] (box1.east) -- (box2.west) ;
%\draw[thick,->] (box2.east) -- (box3.west) ;
%\draw[thick,->] (box3.east) -- (box4.west) ;
%\end{tikzpicture}
%
%\item Using the giving information and the MW calculate the unknown quantity:\\
%\begin{tikzpicture}
%\node [block] (box1) at (0,0) [rectangle,draw=white,fill=red!20!white] {\textcolor{black}{3 Grams of \ce{NH3}}};
%\node [block] (box2) at  (2,0) [rectangle,draw=white,fill=orange!20!white] {\textcolor{black}{Moles of \ce{NH3}}};
%\node [block] (box3) at  (4,0) [rectangle,draw=white,fill=yellow!20!white] {\textcolor{black}{Molec. of \ce{NH3}}};
%\node [block] (box4) at  (6,0) [rectangle,draw=white,fill=green!20!white] {\textcolor{black}{{\Large ?} H Atoms}};
%\draw[thick,->] (box1.east) -- (box2.west) ;
%\draw[thick,->] (box2.east) -- (box3.west) ;
%\draw[thick,->] (box3.east) -- (box4.west) ;
%\end{tikzpicture}
{\raggedright\textsc{\textbf{Chemical Reactions}}\par}

\item Indicate the stoichiometric coefficients that balance next reaction:
\begin{center}\ce{P4(s) + O2(g) -> P4O10(s)}\end{center}
\begin{enumerate}[label=(\alph*)]
\begin{multicols*}{2}
\item 1,1,1
\item 1,1,10
\item 1,1,5
\item 1,5,1
\item 1,2,5
\end{multicols*}\flushright  {\small Ans: (d)}
\end{enumerate}

\item Balance the following reaction:
\begin{center}\ce{Al(s) + O2(g) -> Al2O3(s)}\end{center}
\begin{flushright}\small Ans: 4,3,2    \end{flushright}



\item Balance the following reaction:
\begin{center}\ce{FeS(s) + O2(g) -> Fe2O3(s) + SO2(g)}\end{center}
\enumeratext{\flushright  {\small Ans: 4,7,2,4}}
\item Balance the following reaction:
\begin{center}\ce{NH3(g) + O2(g) -> NO(g) + H2O(g)}\end{center}
\begin{flushright}\small Ans: 4,5,4,6    \end{flushright}


%\item Classify next reaction as combination, decomposition, single replacement, double replacement, or combustion:
%\begin{center}\ce{Pb(s) + FeSO4(s) -> PbSO4(s) + Fe(s)}\end{center}
%\enumeratext{\flushright  {\small Ans: S.R.}}
%
%\item Classify next reaction as combination, decomposition, single replacement, double replacement, or combustion:
%\begin{center}\ce{C6H12(g) + 9O2(g) -> 6CO2(g) + 6H2O(g)}\end{center}
%\enumeratext{\flushright  {\small Ans: C}}
%\item Classify next reaction as combination, decomposition, single replacement, double replacement, or combustion:
%\begin{center}\ce{2RbNO3(aq) + BeF2(aq) -> Be(NO3)2(aq) + 2RbF(aq)}\end{center}
%\enumeratext{\flushright  {\small Ans: D.R.}}

{\raggedright\textsc{\textbf{Mole-Mole Relationships}}\par}

\item Fill the mole ratio for the following reaction:
\begin{center}\ce{ C6H12O6(s) + 6O2(g) -> 6CO2(g) + 6H2O(g)  }\end{center}
\begin{equation*}
\frac{\hlmath{\hspace{35pt}}\text{ moles of }\ce{C6H12O6}}{\hlmath{\hspace{35pt}}\text{ moles of }\ce{O2}} 
\end{equation*}
\begin{flushright}\small Ans: 1/6    \end{flushright}





\item Fill the mole ratio for the following reaction:
\begin{center}\ce{ C6H12O6(s) + 6O2(g) -> 6CO2(g) + 6H2O(g)  }\end{center}
\begin{equation*}
\frac{\hlmath{\hspace{35pt}}\text{ moles of }\ce{O2}}{\hlmath{\hspace{35pt}}\text{ moles of }\ce{CO2}} 
\end{equation*}
\begin{flushright}\small Ans: 6/6    \end{flushright}


\item Fill the conversion factor that calculates the moles of oxygen needed to react with 2 moles of Silver producing \ce{AgO}:
\begin{center}\ce{ 2Ag(s) + O2(g) -> 2AgO(s)  }\end{center}
 \begin{equation*}\begin{split}
2\cancel{\text{ moles of }\ce{Ag}} \times \dfrac{\hlmath{\hspace{35pt}}\text{ moles of }\ce{O2}}{\hlmath{\hspace{35pt}}\cancel{\text{ moles of }\ce{Ag}}}\\
=1\text{ moles of }\ce{O2}.
\end{split}\end{equation*}
\begin{flushright}\small Ans: 1/2   \end{flushright}


\item Fill the conversion factor that calculates the moles of \ce{AgO} produced from 2 moles of Silver:
\begin{center}\ce{ 2Ag(s) + O2(g) -> 2AgO(s)  }\end{center}
 \begin{equation*}\begin{split}
2\cancel{\text{ moles of }\ce{Ag}} \times \dfrac{\hlmath{\hspace{35pt}}\text{ moles of }\ce{AgO}}{\hlmath{\hspace{35pt}}\cancel{\text{ moles of }\ce{Ag}}}\\
=2\text{ moles of }\ce{AgO}.
\end{split}\end{equation*}
\begin{flushright}\small Ans:  2/2  \end{flushright}



\item Fill the conversion factor that calculates the moles of \ce{AgO} produced from 10 moles of oxygen:
\begin{center}\ce{ 2Ag(s) + O2(g) -> 2AgO(s)  }\end{center}
 \begin{equation*}\begin{split}
10\cancel{\text{ moles of }\ce{O2}} \times \dfrac{\hlmath{\hspace{35pt}}\text{ moles of }\ce{AgO}}{\hlmath{\hspace{35pt}}\cancel{\text{ moles of }\ce{O2}}}\\
=5\text{ moles of }\ce{AgO}.
\end{split}\end{equation*}
\begin{flushright}\small Ans: 2/1   \end{flushright}


\item Calculate how many moles of nitrogen are needed to react with 5 moles of hydrogen, to produce ammonia:
\begin{center}  \ce{ $\underset{\text{5 moles}}{\ce{3H2(g)}}$ + $\underset{\hlmath{\hspace{25pt}}\text{ moles}}{\ce{N2(g)}}$ -> 2NH3(g) }\end{center}
\begin{flushright}\small Ans: 3.3 moles    \end{flushright}






{\raggedright\textsc{\textbf{Percent yield and limiting reagent}}\par}



\item Six moles of nitrogen gas react to produce three moles of ammonia according to the following reaction:
\begin{center}  \ce{ $\underset{\text{ }}{\ce{3H2(g)}}$ + $\underset{\text{6 moles}}{\ce{N2(g)}}$ -> $\underset{\text{3 moles}}{\ce{2NH3(g)}}$ }\end{center}
Calculate the percent yield.
\begin{flushright}\small Ans: 50\%.\end{flushright}

\item Six moles of nitrogen gas react to produce two moles of ammonia according to the following reaction:
\begin{center}  \ce{ $\underset{\text{ }}{\ce{3H2(g)}}$ + $\underset{\text{6 moles}}{\ce{N2(g)}}$ -> $\underset{\text{2 moles}}{\ce{2NH3(g)}}$ }\end{center}
Calculate the percent yield.
\begin{flushright}\small Ans: 16.6\%.\end{flushright}

\item We mix three moles of hydrogen gas with three moles of nitrogen gas.
\begin{center}  \ce{ $\underset{\text{ }}{\ce{3H2(g)}}$ + $\underset{\text{6 moles}}{\ce{N2(g)}}$ -> $\underset{\text{2 moles}}{\ce{2NH3(g)}}$ }\end{center}
Calculate the percent limiting reagent.
\begin{flushright}\small Ans: \ce{H2}.\end{flushright}

\item We mix three moles of hydrogen gas with half a mole of nitrogen gas.
\begin{center}  \ce{ $\underset{\text{ }}{\ce{3H2(g)}}$ + $\underset{\text{6 moles}}{\ce{N2(g)}}$ -> $\underset{\text{2 moles}}{\ce{2NH3(g)}}$ }\end{center}
Calculate the percent limiting reagent.
\begin{flushright}\small Ans: \ce{N2}.\end{flushright}

\item We mix two moles of hydrogen gas with five moles of nitrogen gas.
\begin{center}  \ce{ $\underset{\text{ }}{\ce{3H2(g)}}$ + $\underset{\text{6 moles}}{\ce{N2(g)}}$ -> $\underset{\text{2 moles}}{\ce{2NH3(g)}}$ }\end{center}
Calculate the percent limiting reagent.
\begin{flushright}\small Ans: \ce{H2}.\end{flushright}


{\raggedright\textsc{\textbf{Mass calculations}}\par}

\item Calculate the number of grams of nitrogen needed to react with 4 moles of hydrogen, to produce ammonia:
\begin{center}  \ce{ $\underset{\text{4 moles}}{\ce{3H2(g)}}$ + $\underset{\hlmath{\hspace{25pt}}\text{ grams}}{\ce{N2(g)}}$ -> 2NH3(g) }\end{center}
\begin{flushright}\small Ans: 37.3 g\end{flushright}


\item Calculate the number of grams of hydrogen needed to react with 0.3 moles of nitrogen, to produce ammonia:
\begin{center}  \ce{ $\underset{\hlmath{\hspace{25pt}}\text{ grams}}{\ce{3H2(g)}}$ + $\underset{\text{0.3 moles}}{\ce{N2(g)}}$ -> 2NH3(g) }\end{center}
\begin{flushright}\small Ans: 37.3 g\end{flushright}


\item Calculate the number of grams of hydrogen needed to react with 0.3 moles of nitrogen, to produce ammonia:
\begin{center}  \ce{ $\underset{\hlmath{\hspace{25pt}}\text{ grams}}{\ce{3H2(g)}}$ + $\underset{\text{0.3 moles}}{\ce{N2(g)}}$ -> 2NH3(g) }\end{center}
\begin{flushright}\small Ans: 37.3 g\end{flushright}

\restoregeometry
\end{enumerate}
\end{multicols*}
\pagecolor{green!10}\afterpage{\nopagecolor}\newpage
\end{document}