\documentclass[main.tex]{subfiles}
\setlength{\columnsep}{30pt}
\begin{document}
\pagestyle{style4}
\newgeometry{left=0.8in,right=2.8in, top=3.5cm,bottom=2cm}
\setlength{\parskip}{0.5em}
\addcontentsline{toc}{chapter}{Review-Quizz}
\begin{fullwidth}
\begin{multicols*}{2}\begin{enumerate}  \setlength\itemsep{0.2em}












\item  A 2.5 g sample of french fries is placed in a calorimeter with 500.0 g of water at an initial temperature of 21 $^{\circ}$C.  After combustion of the french fries, the water has a temperature of 48 $^{\circ}$C.  What is the combustion energy for the process if the calorimeter factor is negligible?
\begin{enumerate}[label=(\alph*)]\vspace{-0.5cm}
\begin{multicols*}{3}
\item 23 KJ/g			
\item 11 KJ/g
\item 0.14 KJ/g			
\item 4.2 KJ/g
\item 5.4 KJ/g
\end{multicols*}\end{enumerate}\vspace{-0.5cm}

\item  An unknown metal with mass of 100 g absorbs 6 KJ of heat, and its temperature increases from 22 $^{\circ}$C to 23 $^{\circ}$C. Determine the specific heat of this metal in $J/g ^{\circ}C$.
\begin{enumerate}[label=(\alph*)]\vspace{-0.5cm}
\begin{multicols*}{3}
\item 60 		
\item -60 
\item 40 		
\item 160 
\item 10 
\end{multicols*}\end{enumerate}\vspace{-0.5cm}

\item  The specific heat of copper is 0.0920 $cal/g ^{\circ}C$, and the specific heat of silver is 0.0562 $cal/g ^{\circ}C$.  If 100 cal of heat is added to one g of each metal at 25 $^{\circ}$C, what is the expected result?
\begin{enumerate}[label=(\alph*)]
\item The copper will reach a higher temperature.
\item The silver will reach a higher temperature.
\item The two samples will reach the same temperature.
\item The copper will reach a temperature lower than 25 $^{\circ}$C.
\item The silver will soften.
\end{enumerate}

\item  which of the following has a non-zero $\Delta H^0_f$
\begin{enumerate}[label=(\alph*)]\vspace{-0.5cm}
\begin{multicols*}{3}
\item \ce{S(s)}
\item \ce{O2(s)}
\item \ce{NaCl(s)}
\item \ce{Na(s)}
\item \ce{Cl2(g)}
\end{multicols*}\end{enumerate}\vspace{-0.5cm}





\item  At constant temperature and pressure, the heats of formation of \ce{H2O(g)},  \ce{CO2(g)}, and  \ce{C2H6(g)} are given below. Calculate $\Delta H^0_f$ for 1 mol of \ce{C2H6} gas to oxidize to carbon dioxide gas and water vapor?
 
\begin{center}\ce{C2H6 (g) + O2(g) -> 2CO2(g) + 3H2 O(l )}\end{center}
$\Delta H^0_f(\ce{H2O_{(g)}})$=-251KJ/mol; $\Delta H^0_f(\ce{CO2_{(g)}})$=-393KJ/mol; $\Delta H^0_f(\ce{C2H6_{(g)}})$=-84KJ/mol.
\begin{enumerate}[label=(\alph*)]\vspace{-0.5cm}
\begin{multicols*}{3}
\item -8730KJ		
\item -2910KJ
\item -1455KJ		
\item +1455KJ
\item +2910KJ
\end{multicols*}\end{enumerate}\vspace{-0.5cm}


\item  Given these two standard enthalpies of formation:\\
\begin{tabularx}{\columnwidth}{>{}m{.65\linewidth} *{2}{Y} }
\multicolumn{2}{l}{\hspace{\linewidth} }   \\
\multicolumn{2}{l}{\ce{    S(s)  +  O2(g )  ->  SO2(g)	           }\hspace{0.1cm}$\Delta H_1= -295KJ        $  }   \\
\multicolumn{2}{l}{\ce{    S(s)  +  2/3O2(g )  -> SO3(g)          }\hspace{0.1cm}$\Delta H_2= -395KJ        $ }   \\
\end{tabularx}\\
What is $\Delta H^o_f$ for this reaction:
\begin{center}\ce{O2(g )  +  2SO2(g )  -> 2SO3(g)}\end{center}
\begin{enumerate}[label=(\alph*)]\vspace{-0.5cm}
\begin{multicols*}{3}
\item -1380 KJ/mol		
\item -690KJ/mol
\item -295KJ/mol		
\item -200KJ/mol
\item -100KJ/mol
\end{multicols*}\end{enumerate}\vspace{-0.5cm}

\item If $\Delta H^o_f$ for a reaction is positive?
\begin{enumerate}[label=(\alph*)]
\item  the reaction rate is generally very fast.
\item  $H^o$(products) is smaller than $H^o$(reactants).
\item  the reaction rate is generally very slow.
\item  the process is endothermic
\item  $H^o$(reactants) is bigger than $H^o$(products). 
\end{enumerate}

\item Calculate $\Delta H^o_f$ for the reaction given the following information:
	\begin{center}\ce{N2(g) + 3H2(g) -> 2NH3 (g)}\end{center}	
$\Delta H^o_f$(\ce{NH3(g)})= -46KJ/mol
\begin{enumerate}[label=(\alph*)]\vspace{-0.5cm}
\begin{multicols*}{3}
\item  +100KJ		
\item  -92KJ 
\item  -920KJ		
\item  -120KJ	
\item  10KJ
\end{multicols*}\end{enumerate}\vspace{-0.5cm}





\item Green light has a wavelength of $5.50\times 10^2$ nm.  The energy of a photon of green light is
\begin{enumerate}[label=(\alph*)]\vspace{-0.5cm}
\begin{multicols*}{2}
\item $3.64\times 10^{-34}$J    		
\item $2.17\times 10^{5}$J    
\item $3.61\times 10^{-19}$J    		
\item $1.09\times 10^{-27}$J    
\item $5.45\times 10^{12}$J   
\end{multicols*}\end{enumerate}\vspace{-0.5cm}


\item The number of electron levels in a magnesium atom is
\begin{enumerate}[label=(\alph*)]\vspace{-0.5cm}
\begin{multicols*}{2}
\item 1.			
\item 2.
\item 3.			
\item 4.
\item 5.
\end{multicols*}\end{enumerate}\vspace{-0.5cm}


\item The maximum number of electrons that may occupy the third energy level is
\begin{enumerate}[label=(\alph*)]\vspace{-0.5cm}
\begin{multicols*}{2}
\item 2.			
\item 8.
\item 10.			
\item 18.
\item 32.
\end{multicols*}\end{enumerate}\vspace{-0.5cm}

\item What is the element with the electron configuration $1s^22s^22p^63s^23p^5$?
\begin{enumerate}[label=(\alph*)]\vspace{-0.5cm}
\begin{multicols*}{2}
\item Be			
\item Cl
\item F			
\item S
\item Ar
\end{multicols*}\end{enumerate}\vspace{-0.5cm}

\item Valence electrons are electrons located
\begin{enumerate}[label=(\alph*)]\vspace{-0.5cm}
\begin{multicols*}{2}
\item in the outermost energy level of an atom.
\item in the nucleus of an atom.
\item in the innermost energy level of an atom.
\item throughout the atom.
\item  in the first three shells of an atom.
\end{multicols*}\end{enumerate}\vspace{-0.5cm}

\item How many f orbitals have n=3?
\begin{enumerate}[label=(\alph*)]\vspace{-0.5cm}
\begin{multicols*}{2}
\item 0			
\item 3
\item 5			
\item 7
\item 1
\end{multicols*}\end{enumerate}\vspace{-0.5cm}

\item Which of the following electron configurations is impossible?
\begin{enumerate}[label=(\alph*)]\vspace{-0.5cm}
\begin{multicols*}{2}
\item $1s^22s^22p^63s^23p^1$	
\item $1s^22s^42p^63s^23p^3$
\item $1s^22s^22p^63s^23p^5$	
\item $1s^22s^22p^63s^23p^6$
\item $1s^22s^22p^63s^23p^3$
\end{multicols*}\end{enumerate}\vspace{-0.5cm}


\item The ionization energy of atoms
\begin{enumerate}[label=(\alph*)]\vspace{-0.5cm}
\begin{multicols*}{2}
\item decreases going across a period.
\item decreases going down within a group.
\item increases going down within a group.
\item does not change going down within a group.
\item None of the above.
\end{multicols*}\end{enumerate}\vspace{-0.5cm}

\item Which of the following elements has the lowest electronegativity?
\begin{enumerate}[label=(\alph*)]\vspace{-0.5cm}
\begin{multicols*}{2}
\item Li				
\item C
\item N				
\item O
\item F
\end{multicols*}\end{enumerate}\vspace{-0.5cm}


\newcounter{enumTempD}
    \setcounter{enumTempD}{\theenumi}
\end{enumerate}
\end{multicols*}
\end{fullwidth}
\clearpage
\newpage
\thispagestyle{empty}
\newgeometry{left=0.8in,right=2.8in, top=3.5cm,bottom=2cm}
\begin{fullwidth}
\begin{multicols}{2}\begin{enumerate}[resume]  \setlength\itemsep{0.2em}
    \setcounter{enumi}{\theenumTempD}


\item Of the elements:  B, C, F, Li, and Na., the element with the largest atomic radius is
\begin{enumerate}[label=(\alph*)]\vspace{-0.5cm}
\begin{multicols*}{2}
\item B. 				
\item C.
\item F.				
\item Li.
\item Na.
\end{multicols*}\end{enumerate}\vspace{-0.5cm}

\item List the following atoms in order of increasing ionization energy:Li, Na, C, O, F.
\begin{enumerate}[label=(\alph*)]\vspace{-0.5cm}
\begin{multicols*}{2}
\item Li<Na<C<O<F		
\item Na<Li<C<O<F
\item F<O<C<Li<Na		
\item Na<Li<F<O<C
\item Na<Li<C<F<O
\end{multicols*}\end{enumerate}\vspace{-0.5cm}









\item Calculate $\Delta H^o_f$ for the reaction given the following information:
	\begin{center}\ce{N2(g) + 3H2(g) -> 2NH3 (g)}\end{center}	
$\Delta H^o_f$(\ce{NH3(g)})= -46KJ/mol
\begin{enumerate}[label=(\alph*)]\vspace{-0.5cm}
\begin{multicols*}{3}
\item  +100KJ		
\item  -92KJ 
\item  -920KJ		
\item  -120KJ	
\item  10KJ
\end{multicols*}\end{enumerate}\vspace{-0.5cm}












 \end{enumerate}
\end{multicols}
\end{fullwidth}
\begin{fullwidth}
\par\noindent\rule{0.5\textwidth}{0.4pt}\\
\emph{Answers:}\\
\vspace{-0.5cm}
\begin{tasks}[counter-format={tsk[1].}, label-align=left, label-offset={0mm}, label-width={5mm}, item-indent={1mm}, label-format={\bfseries}](8)
\task (e) 
\task (a) 
\task (a) 
\task (b) 
\task (c) 
\task (c) 
\task (d)
\task (d) 
\task (b) 
\task (c)
\task (d)
\task (b)
\task (a)
\task (a)
\task (b)
\task (b)
\task (a)
\task (e)
\task (a)
\task (b)





\end{tasks}






\end{fullwidth}
\restoregeometry
\end{document}