\documentclass[main.tex]{subfiles}
\begin{document}\newpage
\setdoublesep{0.35700 em}  % 'Bond Spacing'
\setatomsep{1.78500 em}    % 'Fixed Length'
\setbondoffset{0.18265 em} % 'Margin Width'
\newcommand{\bondwidth}{0.06642 em} % 'Line Width'
\setbondstyle{line width = \bondwidth}
\newgeometry{left=0.8in,right=0.8in, top=2.5cm,bottom=2cm}
\fancyhfoffset[E,O]{0pt}
\setlength{\columnsep}{30pt}
\begin{conclusion}
\end{conclusion}
\setstretch{0.3}
\begin{multicols*}{2}


{\raggedright\textsc{\textbf{Gases And Its Properties }}\par}



\begin{enumerate}

\item The pressure of a gas is 2 atm. This value in mmHg is:
\begin{enumerate}[label=(\alph*)]
\begin{multicols*}{2}
\item 760 mmHg
\item 1520 mmHg
\item $1.52\times 10^{-3}$ mmHg
\item $1.6\times 10^{3}$ mmHg
\end{multicols*}\flushright  {\small Ans: (b)}
\end{enumerate}

\item The pressure of a gas is 900 mmHg. This value in torr is:
\begin{enumerate}[label=(\alph*)]
\begin{multicols*}{2}
\item 900 torr
\item 760 torr
\item $1\times 10^{-3}$ torr
\item $7\times 10^{5}$ torr
\end{multicols*}\flushright  {\small Ans: (d)}
\end{enumerate}

\item The pressure of a gas is 3000 Pa. This value in atm is:
\begin{enumerate}[label=(\alph*)]
\begin{multicols*}{2}
\item 3000 atm
\item 2 atm
\item $2.96\times 10^{-2}$ atm
\item $3\times 10^{8}$ atm
\end{multicols*}\flushright  {\small Ans: (c)}
\end{enumerate}


\item The kinetic molecular theory of gases assumes that:
\begin{enumerate}[label=(\alph*)]
\item gas particles interact with each other
\item gas particles have large sizes
\item particles move slowly
\item gas particles move randomly
\flushright  {\small Ans: (d)}
\end{enumerate}

{\raggedright\textsc{\textbf{Ideal Gas Law }}\par}


\item A gas contained in a 3L tank has a pressure of 5 atm at a temperature of 400 K.  Calculate the number of moles in the talk.
\begin{enumerate}[label=(\alph*)]
\begin{multicols*}{2}
\item 45 moles
\item 1 moles
\item 5 moles
\item 0.45 moles
\end{multicols*}\flushright  {\small Ans: (c)}
\end{enumerate}

\item A 4 moles sample of gas at 400K has a pressure of 10 atm. Calculate the volume of the sample:
\begin{enumerate}[label=(\alph*)]
\begin{multicols*}{2}
\item 10 L
\item 3 L
\item 13.12 L
\item 1915 L
\end{multicols*}\flushright  {\small Ans: (d)}
\end{enumerate}

%\item A 3 grams sample of Ar at 40$^\circ$C is placed in a 3mL container. Calculate the pressure inside the container.
%\begin{enumerate}[label=(\alph*)]
%\begin{multicols*}{2}
%\item 182 atm
%\item 3.28 atm
%\item 12 atm
%\item 1425 atm
%\end{multicols*}\flushright  {\small Ans: (b)}
%\end{enumerate}

\item Eighteen moles of a gas in a 11L container at 400K exert a pressure of 3 atm. Calculate the molar mass of the gas.
\begin{enumerate}[label=(\alph*)]
\begin{multicols*}{2}
\item 1 $g\cdot mol^{-1}$
\item 18 $g\cdot mol^{-1}$
\item 161 $g\cdot mol^{-1}$
\item 6 $g\cdot mol^{-1}$
\end{multicols*}\flushright  {\small Ans: (b)}
\end{enumerate}
\item Calculate the volume of a 4 moles of Ar at STP conditions.
\enumeratext{\flushright  {\small Ans: 89.6L}}

\item Calculate the volume of a 4 moles of Ne at STP conditions.
\enumeratext{\flushright  {\small Ans: 89.6L}}

\item Calculate the moles of gas in 3L  of Ar at STP conditions.
\enumeratext{\flushright  {\small Ans: 0.13mol}}

{\raggedright\textsc{\textbf{Change of Gas Properties }}\par}

\item A sample of a gas at 400K and 12 atm is cooled in the same container to 200K. Calculate the new pressure
\begin{enumerate}[label=(\alph*)]
\begin{multicols*}{2}
\item 15 atm
\item 6 atm
\item 24 atm
\item 0.2 atm
\end{multicols*}\flushright  {\small Ans: (b)}
\end{enumerate}

\item A sample of Ne in a closed, expandable container, has a volume of 3L at 40$^\circ$C. Calculate the new volume if the container is cooled to 25$^\circ$C.
\begin{enumerate}[label=(\alph*)]
\begin{multicols*}{2}
\item 1.8 L
\item 3.15 L
\item 2.8 L
\item 1 L
\end{multicols*}\flushright  {\small Ans: (c)}
\end{enumerate}


\item If the pressure of a gas increases, at fixed temperature and moles, its volume....
\begin{enumerate}[label=(\alph*)]
\begin{multicols*}{2}
\item increases
\item decreases
\item does not change
\end{multicols*}\flushright  {\small Ans: (b)}
\end{enumerate}

%\item If the temperature of a gas increases, at fixed volume and moles, its pressure....
%\begin{enumerate}[label=(\alph*)]
%\begin{multicols*}{2}
%\item increases
%\item decreases
%\item does not change
%\end{multicols*}\flushright  {\small Ans: (a)}
%\end{enumerate}




{\raggedright\textsc{\textbf{Mixture of Gases and gas stoichiometry }}\par}


\item  A tank contains Ne gas at 700 mmHg, Ar at 2 atm, and Kr at 700 torr. What is the total pressure of the mixture in atm?
\begin{flushright}\small Ans: 1402torr\end{flushright}


\item The atmospheric pressure on a hot day is 780 mmHg. Given that the air is made of 78\% of nitrogen and 22\% of oxygen, calculate the partial pressure of each gas in the air.
\begin{flushright}\small Ans: \ce{N2} 608 mmHg, \ce{O2} 171.6 mmHg\end{flushright}


\item Phosphorus reacts with oxygen gas to produce tetraphosphorus decaoxide according to the following equation:
\begin{center}\ce{  P4(s)  + $\underset{\text{\large 2L}}{5\ce{O2(g)   }}$  -> $\underset{\text{\large x mol}}{\ce{P4O10(g)}}$  } \end{center}
Calculate the number of moles of phosphorus that react with 2L of oxygen at STP conditions.
\begin{flushright}\small Ans: 0.017 mol\end{flushright}


{\raggedright\textsc{\textbf{Real gases and the kinetic molecular theory of gases }}\par}


\item What is the pressure in atm of 1 mol of He at 600K in a 1L container: (a) Using the ideal gas law and (b) Using the real gas law given $a=$ 0.0342$ atm\cdot L^2\cdot mol^{-2}$ and $b=$ 0.0237$L\cdot mol$.
\begin{flushright}\small Ans: 49.2 atm; 50.36 atm\end{flushright}



\item What is the rms speed of O2 at STP?.
\begin{flushright}\small Ans: Ans: 481.9 m/s\end{flushright}


\restoregeometry
\end{enumerate}
\end{multicols*}
\pagecolor{green!10}\afterpage{\nopagecolor}\newpage
\end{document}