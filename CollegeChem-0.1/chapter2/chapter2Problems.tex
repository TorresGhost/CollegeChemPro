\documentclass[main.tex]{subfiles}
\begin{document}\newpage
\setdoublesep{0.35700 em}  % 'Bond Spacing'
\setatomsep{1.78500 em}    % 'Fixed Length'
\setbondoffset{0.18265 em} % 'Margin Width'
\newcommand{\bondwidth}{0.06642 em} % 'Line Width'
\setbondstyle{line width = \bondwidth}
\newgeometry{left=0.8in,right=0.8in, top=2.5cm,bottom=2cm}
\fancyhfoffset[E,O]{0pt}
\setlength{\columnsep}{30pt}
\begin{conclusion}
\end{conclusion}
\setstretch{0.3}
\begin{multicols*}{2}


{\raggedright\textsc{\textbf{Energy and temperature }}\par}

\begin{enumerate}

\item The energy associated with the motion of particles in a substance is called
\begin{enumerate}[label=(\alph*)]
\begin{multicols*}{2}
\item kinetic energy
\item potential energy
\item heat
\item chemical energy
\item none of the above
\end{multicols*}\flushright  {\small Ans: (a)}
\end{enumerate}
\item The energy stored in the chemical bonds of a carbohydrate molecule is  
\begin{enumerate}[label=(\alph*)]
\begin{multicols*}{2}
\item kinetic energy
\item potential energy
\item heat
\item chemical energy
\item none of the above
\end{multicols*}\flushright  {\small Ans: (d)}
\end{enumerate}
\item The energy stored in heigh  is  
\begin{enumerate}[label=(\alph*)]
\begin{multicols*}{2}
\item kinetic energy
\item potential energy
\item heat
\item chemical energy
\item none of the above
\end{multicols*}\flushright  {\small Ans: (b)}
\end{enumerate}

\item 650J is the same amount of energy as   
\begin{enumerate}[label=(\alph*)]
\begin{multicols*}{2}
\item 155 cal
\item 2720 cal
\item 650 cal
\item 1550 cal
\item 2.72 cal
\end{multicols*}\flushright  {\small Ans: (a)}
\end{enumerate}

\item 3.25 kcal is the same amount of energy as   
\begin{enumerate}[label=(\alph*)]
\begin{multicols*}{2}
\item 3.25 J
\item 0.777 J
\item 777 J
\item 13598 J
\item 13.6 J
\end{multicols*}\flushright  {\small Ans: (d)}
\end{enumerate}
\item A temperature of 41$^{\circ}F$ is the same as   
\begin{enumerate}[label=(\alph*)]
\begin{multicols*}{2}
\item 5$^{\circ}C$
\item 310$^{\circ}C$
\item -9$^{\circ}C$
\item 16$^{\circ}C$
\item 42$^{\circ}C$
\end{multicols*}\flushright  {\small Ans: (a)}
\end{enumerate}
\item A temperature of 20$^{\circ}C$ is the same as   
\begin{enumerate}[label=(\alph*)]
\begin{multicols*}{2}
\item -22$^{\circ}F$
\item 68$^{\circ}F$
\item 43$^{\circ}F$
\item 239$^{\circ}F$
\item 94$^{\circ}F$
\end{multicols*}\flushright  {\small Ans: (b)}
\end{enumerate}
\item A temperature of 300K is the same as   
\begin{enumerate}[label=(\alph*)]
\begin{multicols*}{2}
\item 27$^{\circ}C$
\item 20$^{\circ}C$
\item 45$^{\circ}C$
\item 50$^{\circ}C$
\item 90$^{\circ}C$
\end{multicols*}\flushright  {\small Ans: (a)}
\end{enumerate}



{\raggedright\textsc{\textbf{From Energy To Temperature }}\par}

\item The specific heat of a substance is the amount of heat needed to
\begin{enumerate}[label=(\alph*)]
\item change 1 g of the substance from the solid to the liquid state.
\item raise the temperature of 1 g of the substance by 1$^{\circ}C$.
\item change 1 g of the substance from the liquid to the solid state.
\item convert 1 g of a liquid to gas.
\item convert 1 g of a solid to a gas.
\flushright  {\small Ans: (b)}
\end{enumerate}

\item How many calories are required to raise the temperature of a 35 g sample of iron from 25$^{\circ}C$ to 35$^{\circ}C$?  Iron has a specific heat of $0.108cal/g^{\circ}C$.
\begin{enumerate}[label=(\alph*)]\begin{multicols*}{2}
\item 38 cal
\item 1.1 cal
\item 3.8 cal
\item 93 cal
\item 130 cal
\end{multicols*}\flushright  {\small Ans: (a)}\end{enumerate}
\item What is the final temperature of a 35 g sample of iron at 25$^{\circ}C$ after receiving 50cal?  Iron has a specific heat of $0.108cal/g^{\circ}C$.
\begin{enumerate}[label=(\alph*)]\begin{multicols*}{2}
\item 25$^{\circ}C$
\item 35$^{\circ}C$
\item 38$^{\circ}C$
\item 50$^{\circ}C$
\item 27$^{\circ}C$
\end{multicols*}\flushright  {\small Ans: (c)}\end{enumerate}

\item What is the initial temperature of a 50 g sample of aluminum that after receiving 50cal reaches a temperature of 50$^{\circ}C$?  Al has a specific heat of $0.2cal/g^{\circ}C$.
\begin{enumerate}[label=(\alph*)]\begin{multicols*}{2}
\item 55$^{\circ}C$
\item 25$^{\circ}C$
\item 40$^{\circ}C$
\item 45$^{\circ}C$
\item 50$^{\circ}C$
\end{multicols*}\flushright  {\small Ans: (d)}\end{enumerate}

\item What is the specific heat of a metal if a 100 g sample at 25$^{\circ}C$ warms up until 50$^{\circ}C$ after receiving 100cal?  
\begin{enumerate}[label=(\alph*)]\begin{multicols*}{2}
\item $0.04cal/g^{\circ}C$
\item $0.14cal/g^{\circ}C$
\item $1.04cal/g^{\circ}C$
\item $2.0cal/g^{\circ}C$
\item $0.34cal/g^{\circ}C$
\end{multicols*}\flushright  {\small Ans: (a)}\end{enumerate}


{\raggedright\textsc{\textbf{Calorimetry }}\par}

\item A 3 moles sample of \ce{C(s)} is burned in a constant-volume calorimeter containing 40g of water. The temperature inside the calorimeter increases from 25.0$^{\circ}C$ to 25.89 $^{\circ}C$. The calorimeter constant is 9.90 $KJ/^{\circ}C$.  Calculate the molar enthalpy of combustion of the sample. 
\begin{flushright}\small Ans: -3KJ/mol\end{flushright}

\item A 10 grams sample of fructose (MW=180$g\cdot mol^{-1}$) is burned in a constant-volume calorimeter containing 50g of water. The temperature inside the calorimeter increases 7$^{\circ}C$ . The calorimeter constant is 10.8 $KJ/^{\circ}C$.  Calculate the molar enthalpy of combustion of the sample. 
\begin{flushright}\small Ans: -1387KJ/mol\end{flushright}

\item When a 0.09-g sample of trinitrotoluene (TNT, MW=213$g\cdot mol^{-1}$), is burned in a bomb calorimeter, the temperature increases from 23.5 $^{\circ}C$ to 27.1$^{\circ}C$. The heat capacity of the calorimeter is 400 J/$^{\circ}C$, and it contains 100 mL of water. How much heat was produced by the combustion of the TNT sample?
\begin{flushright}\small Ans: -6973KJ/mol\end{flushright}

\item You warm up 10 g of a solid to 200$^{\circ}C$ and add it to 60 g of water in a coffee-cup calorimeter. The water temperature changes from 24$^{\circ}C$ to 29$^{\circ}C$. Calculate the specific heat of the solid. You can consider negligible the heat capacity of the coffee-cup calorimeter.
\begin{flushright}\small Ans: 0.17J/$^{\circ}C$\end{flushright}


{\raggedright\textsc{\textbf{Enthalpy }}\par}
\item The following reaction:
\begin{center}\ce{ C6H12O6(s) + 6O2(g) -> 6CO2(g) + 6H2O(g) \text{+ 2800KJ}  }\end{center}
\begin{enumerate}[label=(\alph*)]
\begin{multicols*}{2}
\item is endothermic
\item is exothermic
\item uses energy
\item  Consumes energy 
\item  None of the above
\end{multicols*}\flushright  {\small Ans: (b)}
\end{enumerate}


\item The following reaction:
\begin{center}\ce{ B2O3(s) + 3H2O(g) \text{+ 2035KJ} -> 3O2(g) + B2H6(g)  }\end{center}
\begin{enumerate}[label=(\alph*)]
\begin{multicols*}{2}
\item is endothermic
\item is exothermic
\item produces energy
\item  None of the above
\end{multicols*}\flushright  {\small Ans: (a)}
\end{enumerate}


\item The following reaction:
\begin{center}\ce{ C6H12O6(s) + 6O2(g) -> 6CO2(g) + 6H2O(g) \text{+ 2800KJ}  }\end{center}
Fill the conversion factor:
\begin{equation*}
\frac{\hlmath{\hspace{35pt}}\text{ moles of }\ce{O2}}{\text{ -2800 KJ }} 
\end{equation*}
\enumeratext{\flushright  {\small Ans: 6}}
\item In the following combustion reaction:
\begin{center}\ce{ C6H12O6(s) + 6O2(g) -> 6CO2(g) + 6H2O(g) \text{+ 2800KJ}  }\end{center}
glucose (\ce{C6H12O6}) burns to produce carbon dioxide and water. Calculate the heat involved in the combustion of 3 moles of glucose.
\begin{flushright}  {\small Ans: 8400KJ}\end{flushright} 

\item Calculate the enthalpy of reaction for:
\begin{center}\ce{ 2OF2(g)    -> O2(g)  +  2F2(g)}\end{center}
given:
\begin{center}
$\Delta H^{\circ}_f(OF2(g))=24.5KJ$
\end{center}
\begin{flushright}\small Ans: -49KJ/mol\end{flushright}

\item Calculate the enthalpy of reaction for:
\begin{center}\ce{ 2ClF(g)  +  O2(g)    ->  Cl2O(g)  +  OF2(g) }\end{center}
given:
\begin{center}
$\Delta H^{\circ}_f(\ce{ClF(g)})=-56KJ$\\
$\Delta H^{\circ}_f(\ce{Cl2O(g)})=88KJ$\\
$\Delta H^{\circ}_f(\ce{OF2(g)})=25KJ$
\end{center}
\begin{flushright}\small Ans: 225KJ/mol\end{flushright}

\item Calculate the enthalpy of reaction for:
\begin{center}\ce{ ClF3(g)  +  O2(g)    ->  Cl2O(g)  +  3/2OF2(g) }\end{center}
given:
\begin{center}
$\Delta H^{\circ}_f(\ce{ClF3(g)})=-156KJ$\\
$\Delta H^{\circ}_f(\ce{Cl2O(g)})=88KJ$\\
$\Delta H^{\circ}_f(\ce{OF2(g)})=25KJ$
\end{center}
\begin{flushright}\small Ans: 324KJ/mol\end{flushright}


{\raggedright\textsc{\textbf{Hess's Law }}\par}
\item Using the following reactions:\\
\begin{tabularx}{\columnwidth}{>{}m{.65\linewidth} *{2}{Y} }
\multicolumn{2}{l}{\hspace{\linewidth} }   \\
\multicolumn{2}{l}{\ce{2OF2(g)    -> O2(g)  +  2F2(g) } }   \\
\multicolumn{2}{r}{ $\Delta H_1=-49KJ$  }    \\
\multicolumn{2}{l}{\ce{2ClF(g)  +  O2(g)    ->  Cl2O(g)  +  OF2(g) } }   \\
\multicolumn{2}{r}{ $\Delta H_2=225KJ$   }    \\
\multicolumn{2}{l}{\ce{ClF3(g)  +  O2(g)    ->  Cl2O(g)  +  3/2OF2(g) } }   \\
\multicolumn{2}{r}{ $\Delta H_3=324KJ$   }    \\
\end{tabularx}\\
Determine the enthalpy change for:
\begin{center}\ce{ ClF(g) + F2(g) -> ClF3(g)}\end{center}
\begin{flushright}\small Ans: -178KJ/mol\end{flushright}

\item Using the following reactions:\\
\begin{tabularx}{\columnwidth}{>{}m{.65\linewidth} *{2}{Y} }
\multicolumn{2}{l}{\hspace{\linewidth} }   \\
\multicolumn{2}{l}{\ce{N2(g) + 3H2(g) -> 2NH3(g) } }   \\
\multicolumn{2}{r}{ $\Delta H_1=-92KJ$  }    \\
\multicolumn{2}{l}{\ce{C(s) + 2H2(g) -> CH4(g) } }   \\
\multicolumn{2}{r}{ $\Delta H_2=-75KJ$   }    \\
\multicolumn{2}{l}{\ce{H2(g) + 2C(s) + N2(g) -> 2HCN(g) } }   \\
\multicolumn{2}{r}{ $\Delta H_3=270KJ$   }    \\
\end{tabularx}\\
Determine the enthalpy change for:
\begin{center}\ce{ CH4(g) + NH3(g) -> HCN(g) + 3H2(g)}\end{center}
\begin{flushright}\small Ans: 260KJ/mol\end{flushright}


\item Using the following reactions:\\
\begin{tabularx}{\columnwidth}{>{}m{.65\linewidth} *{2}{Y} }
\multicolumn{2}{l}{\hspace{\linewidth} }   \\
\multicolumn{2}{l}{\ce{3C(s) + 3H2(g) + 1/2 O2(g) -> C3H6O(l) } }   \\
\multicolumn{2}{r}{ $\Delta H_1=-285KJ$  }    \\
\multicolumn{2}{l}{\ce{C(s) + O2(g) -> CO2(g) } }   \\
\multicolumn{2}{r}{ $\Delta H_2=-394KJ$   }    \\
\multicolumn{2}{l}{\ce{H2(g) + 1/2 O2(g) -> H2O(l) } }   \\
\multicolumn{2}{r}{ $\Delta H_3=-286KJ$   }    \\
\end{tabularx}\\
Determine the enthalpy change for:
\begin{center}\ce{ C3H6O(l) + 4O2(g) -> 3CO2(g) + 3H2O(l)}\end{center}
\begin{flushright}\small Ans: -1755KJ/mol\end{flushright}

%\item Using the following reactions:\\
%\begin{tabularx}{\columnwidth}{>{}m{.65\linewidth} *{2}{Y} }
%\multicolumn{2}{l}{\hspace{\linewidth} }   \\
%\multicolumn{2}{l}{\ce{2C2H6 + 7O2 -> 4CO2 + 6H2O	 } }   \\
%\multicolumn{2}{r}{ $\Delta H_1=-3120KJ$  }    \\
%\multicolumn{2}{l}{\ce{2H2 + O2 -> 2H2O } }   \\
%\multicolumn{2}{r}{ $\Delta H_2=-479KJ$   }    \\
%\multicolumn{2}{l}{\ce{2CO + O2 -> 2CO2 } }   \\
%\multicolumn{2}{r}{ $\Delta H_3=-566KJ$   }    \\
%\end{tabularx}\\
%Determine the enthalpy change for:
%\begin{center}\ce{ C2H6 + O2 -> 3H2 + 2CO}\end{center}
%\begin{flushright}\small Ans: -276KJ/mol\end{flushright}


\restoregeometry
\end{enumerate}
\end{multicols*}
\pagecolor{green!10}\afterpage{\nopagecolor}\newpage
\end{document}
