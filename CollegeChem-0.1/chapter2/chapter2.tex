\documentclass[main.tex]{subfiles}

\begin{document}
\linenumbers



\chapter[Thermochemistry]{Thermochemistry}
\label{ch:energy}

\begin{marginfigure}
      \includegraphics{chapter2/figure1}
   \end{marginfigure}
\lettrine[lines=4]{\color{black!45}E}{nergy}  involves many aspects of our everyday life. The chemical reactions in our body consume or release energy as we walk, study, and even breath. We also use energy in our homes to warm our food or turn on the lights, and to drive our cars and go to work. The energy needed for our body to function comes from food. If we do not eat for a while, we run low of energy. Similarly, the burning of fossil fuels such as oil, propane, or gasoline provides enough energy to maintain our homes. Some reactions produce energy whereas others release energy. On the other hand, how do we measure the energy release or consumed in a reaction? This chapter will answer this and other questions as it covers different aspects of thermochemistry that involves the interaction between chemistry and energy. You will learn about temperature, heat and how to compute the energy exchanged during a chemical reaction.

%\textquotesingle 
\begin{marginfigure}%LEARNING GOALS BOX
\begin{mytcbox}{GOALS}
\begin{enumerate}[label=\protect\circled{\color{white}\arabic*}]
\item Convert heat to temperature rise
\item Carry calorimetric calculations
\item Use the enthalpy table
\item Compute enthalpy changes
\item Apply Hess's Law to compute enthalpy
\end{enumerate}
\end{mytcbox}
\end{marginfigure}%LEARNING GOALS BOX





\section{Energy \& temperature}
When you are running, walking, dancing, or thinking, you are using energy to do work. In fact, energy is defined as the ability to do work. Suppose you are climbing a steep hill. Perhaps you become too tired to go on; you do not have sufficient energy to do any more work. Now suppose you sit down and have lunch. In a while you will have obtained energy from the food, and you will be able to do more work and complete the climb.
\sloppy
\begin{description}
\begin{marginfigure}
      \includegraphics{chapter2/figure1-1}
      \caption{water on a dam has potential energy}
      \label{fig:marginfig}
   \end{marginfigure}
\item[\docfilehook{Potential \& Kinetic Energy}{Potential \& Kinetic Energy}] 
Energy can be classified as potential energy or kinetic energy. 
Kinetic energy is the energy of motion and any object that is moving has kinetic energy. Think about a bullet coming out of a gun; as the bullet moves very fast it contains kinetic energy that can be releases when it collide with a target. Potential energy is energy stored in objects located at a certain height. A boulder resting on top of a mountain has potential energy because of its location. If the boulder rolls down the mountain, the potential energy becomes kinetic energy. Water stored in a reservoir has potential energy. When the water goes over the dam, the potential energy is converted to kinetic energy. 
\begin{marginfigure}
      \includegraphics{chapter2/figure1-2}
      \caption{a bullet has kinetic energy}
      \label{fig:marginfig}
   \end{marginfigure}
     \begin{marginfigure}
\begin{tcolorbox}[enhanced,colback=red!5!white,colframe=black!50!red,boxrule=1pt,
  arc=0pt,outer arc=0pt,drop heavy lifted shadow]
\faGears\ 
\docenvdef{Discussion:} What do you think about renewable energy? List three benefits and three inconvenients of renewable sources of energy. \end{tcolorbox}
 \end{marginfigure}
\item[\docfilehook{Heat}{Heat}] Heat refers to thermal energy, which is associated with the motion of particles in a substance. A frozen pizza feels cold because the particles in the pizza are moving very slowly. As the pizza receives heat, the motions of the particles increase, and the pizza becomes warm. Eventually the particles have enough energy to make the pizza hot and ready to eat. When a substance receives heat it gets warmer and it raises its temperature, whereas if it looses heat it gets cooler and its temperature decreases. 

   
\item[\docfilehook{Energy units}{Energy units}] Two different units of energy are often employed: calories (cal) and joules (J). One can transform calories to joules and joules to calories using the following conversion factor: 
\begin{equation}
\boxed{    1 cal=4.184 J   } \qquad\text{or}\qquad  \boxed{\frac{1cal}{4.184 J}}\qquad\text{or}\qquad  \boxed{\frac{4.184 J}{1 cal}}
\label{formula2:1}
\end{equation}
\resizeableyellownote{2.5}{1}{Add Equation \ref{formula2:1} to your flashcard.}



\begin{example} %%%%%%%%%%%%%%%%%%%%%%%% EXAMPLE BOX
Convert the following energy values:
\begin{multicols}{2}
\begin{enumerate}[label=(\alph*)]
\item 50000 cal to Kcal
\item 48000 J to cal
\end{enumerate}
\end{multicols}
\textlcsc{ \textcolor{dgreen}{\Large \textbf{Solution}} }\\
(a) $50000 cal\times\frac{Kcal}{1000 cal}=50Kcal  $; (b) $ 48000 J\times\frac{1 cal}{4.184 J}=11472.2 cal $.\\
\faDiamond\ \textlcsc{ \textcolor{dgreen}{\Large \textbf{Study Check}} }\\
Convert the following energy units:
\begin{multicols}{2}
\begin{enumerate}[label=(\alph*)]
\item 200 cal to Kcal
\item 7000 J to cal
\end{enumerate}
\end{multicols}
\flushright Answer: (a) 0.2Kcal; (b) 1673 cal.
\end{example}%%%%%%%%%%%%%%%%%%%%%%%% EXAMPLE BOX


\item[\docfilehook{Temperature }{Temperature }]
\begin{marginfigure}[-2cm]
      \includegraphics{chapter2/figure1-3}
      \caption{Some thermometers has different temperature scales.}
      \label{fig:marginfig}
   \end{marginfigure}
Temperature indicates how hot or cold a substance is compared to another substance. Heat always flows from a substance with a higher temperature to a substance with a lower temperature until the temperatures of both are the same. When you drink hot coffee or touch a hot pan, heat flows to your mouth or hand, which is at a lower temperature. When you touch an ice cube, it feels cold because heat flows from your hand to the colder ice cube. Three units of temperature often employed are celsius ($^{\circ}$C), Fahrenheit ($^{\circ}$F) or Kelvin (K). If you need to convert temperature units from Fahrenheit  to celsius or from celsius to Fahrenheit you need to use the following formula on the left. Differently, if you need to convert between Kelvin and celsius you need to use the formula on the right:
\resizeableyellownote{2.5}{1}{Add Equation \ref{formula2:2} to your flashcard.}
\begin{equation}
\boxed{   T_F = 1.8T_C  + 32   }
\qquad
\boxed{     T_K = T_C  + 273   }
\label{formula2:2}
\end{equation}

\begin{example} %%%%%%%%%%%%%%%%%%%%%%%% EXAMPLE BOX
Convert 25 $^{\circ}$C to $^{\circ}$F.\\
\textlcsc{ \textcolor{dgreen}{\Large \textbf{Solution}} }\\
\begin{enumerate}[label=\protect\circled{\color{white}\arabic*}]
\item \begin{bf}Step one:\end{bf} list of the given variables.
%%% DATA BOX
\begin{tcbitemize}[raster columns=3, raster rows=3, enhanced, sharp corners, raster equal height=rows, raster force size=false, raster column skip=0pt, raster row skip = 0pt]
%Empty corner and two headers
\tcbitem[blankest, width=1cm]
\tcbitem[header = helpful]
\texta
\tcbitem[header = harmful]
\textb
%First row
\tcbitem[firstcol = internal]
\textcn
\tcbitem[swotbox = G]
$T_c=25^{\circ}C$\\
\tcbitem[swotbox = A]
$T_F$
\end{tcbitemize}%%% DATA BOX
\break\item \begin{bf}Step two:\end{bf} use the formula $T_F = 1.8T_C  + 32$ to convert from $^{\circ}$C to $^{\circ}$F.
 %%%% COMMENTED EQUATION
\begin{equation*}
    T_F = 1.8\,\,\,\,\tikzmark{I}{$T_c$} \,\,\,+ 32
\end{equation*}
\begin{tikzpicture}[overlay, remember picture,node distance =1.0cm]
    \node (Idescr) [below left=of I ]{$25^{\circ}$};\draw[,->,thick] (Idescr) to [in=-90,out=90] (I);
\end{tikzpicture}\vspace{5mm} %%%% COMMENTED EQUATION
\item \begin{bf}Step three:\end{bf} solve for $T_F=1.8\times 25+32=77^{\circ}F$.
\end{enumerate}
\faDiamond\ \textlcsc{ \textcolor{dgreen}{\Large \textbf{Study Check}} }\\
Convert  $200^{\circ}$C to K.\\
\flushright Answer: 473K.
\end{example}%%%%%%%%%%%%%%%%%%%%%%%% EXAMPLE BOX









\end{description}

   \begin{marginfigure}[-2cm]
      \includegraphics{chapter2/figure1-8}
      \caption{Metals have low specific heat, which means they heat up very fast.}
      \label{fig:marginfig}
   \end{marginfigure}

\section{From energy  to temperature}
Materials can absorb heat. Think about a pizza in your over, or a cup of milk in the microwaves. These substances receive heat from the oven or in form of microwaves and they become hot. Heat transforms in an increase of temperature. Some substances like metals are able to increase its temperature very quickly with a small amount of heat received, whereas others need a larger amount of heat to rise up its temperature. Think about why you use oil to deep fried food? Why not using water? First of all, oil can rise its temperature very quickly and on top of that it does not boil easily.
\sloppy
   
   \begin{marginfigure}
\begin{tcolorbox}[tab2,tabularx={XY}]%%%% FANCY COLOR TABLE
\begin{center}Material\end{center} & \begin{center}Specific Heat $J/g^{\circ}C$  \end{center}        \\\hline\hline
\ce{H2O(l)}& 4.184       \\\hline
\ce{H2O(s)}& 2.010       \\\hline
ethyl alcohol& 2.460       \\\hline
vegetable oil& 2.010       \\\hline
\ce{Cu}& 0.385       \\\hline
\ce{Au}& 0.129       \\\hline
\ce{Fe(s)}& 0.444       
\end{tcolorbox}%%%% FANCY COLOR TABLE
  \end{marginfigure}


\begin{description}
\item[\docfilehook{Specific Heat}{Specific Heat}] 
Specific heat $c_e$ is a property of each material that indicates the energy required to rise its temperature. For example, the specific heat of water is $1 cal/g^{\circ}C$ that is the same as $4.184 J/g^{\circ}C$. That means that we need to give 1 calorie in order to warm up one gram of water $1^{\circ}C$. Similarly, the specific heat of aluminum, a metal, is $0.2 cal/g^{\circ}C$ or $0.89J/g^{\circ}C$; that means the energy needed to rise the temperature of an aluminum gram is  0.2 calories of 0.89 J. Mind the difference between these two values: we need to give 1 cal in order to increase the temperature of a gram of water in $1^{\circ}C$, whereas we need to give 0.2 cal in order to increase the temperature of a gram of aluminum in $1^{\circ}C$. Why are these two numbers so different? The answer is because water and aluminum are different materials. Normally metals warp up very easily, that is, they need less heat to increase their temperature, whereas liquids need more heat to increase their temperature. That is why pans and cooking pots tend to be metallic. Mind the specific heat if water is a well know value that you need to be familiar with:
\begin{equation}
\boxed{   c_e^{\ce{H2O}} = 4.184 J/g^{\circ}C   }
\qquad
\boxed{     c_e^{\ce{H2O}} = 1 cal/g^{\circ}C   }
\label{formula2:9}
\end{equation}

     \begin{marginfigure}[-1cm]
      \includegraphics{chapter2/figure1-5}
      \caption{Oil has low specific heat. This allows achieving hight temperatures with small heat intakes.}
      \label{fig:marginfig}
   \end{marginfigure}
\item[\docfilehook{From heat to temperature}{From heat to temperature}] 
When a materials receives heat, that heat normally becomes temperature and the temperature of the material increases. For example, if you warm milk in a microwaves, that milk warms up from room temperature until a higher temperature. How to estimate the temperature increase given the heat received? We can use the following formula:
\resizeableyellownote{2.5}{1}{Add Equation \ref{formula2:3} to your flashcard.}
\begin{equation}
\boxed{    Q=m\cdot c_e\cdot  (T_{Final}-T_{Initial})   } 
\label{formula2:3}
\end{equation}
where $m$ is the mass of material in grams, $c_e$ is the specific heat of the material which can either be given in $cal/g^{\circ}C$ or $J/g^{\circ}C$, $T_{Initial}$ is the initial temperature and $T_{Final}$ is the final temperature in $^{\circ}C$. $Q$ is the amount of heat received, either in cal or J.


\begin{example} %%%%%%%%%%%%%%%%%%%%%%%% EXAMPLE BOX
How many calories are absorbed by a 45.2g piece of aluminum ($c_e=0.214\frac{cal}{g\cdot^{\circ}C}$) if its temperature rises from 25$^{\circ}$C to 50$^{\circ}$C. \\
\textlcsc{ \textcolor{dgreen}{\Large \textbf{Solution}} }\\
\begin{enumerate}[label=\protect\circled{\color{white}\arabic*}]
\item \begin{bf}Step one:\end{bf} list of the given variables.
%%% DATA BOX
\begin{tcbitemize}[raster columns=3, raster rows=3, enhanced, sharp corners, raster equal height=rows, raster force size=false, raster column skip=0pt, raster row skip = 0pt]
%Empty corner and two headers
\tcbitem[blankest, width=1cm]
\tcbitem[header = helpful]
\texta
\tcbitem[header = harmful]
\textb
%First row
\tcbitem[firstcol = internal]
\textcn
\tcbitem[swotbox = G]
$c_e=0.214\frac{cal}{g\cdot^{\circ}C}$\\
$m=45.2g$\\
$T_{Initial}=25^{\circ}C$\\
$T_{Final}=50^{\circ}C$\\
\tcbitem[swotbox = A]
$Q$
\end{tcbitemize}%%% DATA BOX
\item \begin{bf}Step two:\end{bf} use the formula $Q=m\cdot c_e\cdot  (T_{Final}-T_{Initial})$ to temperature increase to heat absorbed:
 %%%% COMMENTED EQUATION
\vspace{10mm} \begin{equation*}
     Q\,\,= \, \tikzmark{A}{m}\,\,\,\cdot \,\,\, \tikzmark{B}{$c_e\cdot$}\,\,\,  (\,\,\,\,\,\,\,\,\,\tikzmark{C}{$T_{Final}$}\,\,\,\,\,\,\,\,\,-\,\,\,\,\,\,\,\,\,\tikzmark{D}{$T_{Initial}$}\,\,\,\,\,\,\,\,\,)
\end{equation*}
\begin{tikzpicture}[overlay, remember picture,node distance =1.5cm]
    \node (Adescr) [below left=of A ]{$45.2g$};\draw[,->,thick] (Adescr) to [in=-90,out=90] (A);
   \node[red] (Bdescr) [below =of B]{$0.214\frac{cal}{g\cdot^{\circ}C}$}; \draw[red,->,thick] (Bdescr) to [in=-90,out=90] (B);
   \node[purple] (Cdescr) [below right =of C]{$50^{\circ}C$};\draw[purple,->,thick] (Cdescr) to [in=-90,out=90] (C.south);
   \node[blue,xshift=1cm] (Ddescr) [above right =of D]{$25^{\circ}C$};\draw[blue,->,thick] (Ddescr) to [in=45,out=-90] (D.north);
\end{tikzpicture}\vspace{12mm} %%%% COMMENTED EQUATION
\item \begin{bf}Step three:\end{bf} solve $Q=45.2\cdot 0.214\cdot  (50-25)=241.82cal$.
\end{enumerate}
\faDiamond\ \textlcsc{ \textcolor{dgreen}{\Large \textbf{Study Check}} }\\
How many calories are absorbed by 100g of Gold ($c_e=0.0308\frac{cal}{g\cdot^{\circ}}$) if its temperature rises from 25$^{\circ}$C to 100$^{\circ}$C. \\
\flushright Answer: $Q=231cal$.
\end{example}%%%%%%%%%%%%%%%%%%%%%%%% EXAMPLE BOX


In the previous example you needed to convert temperature into heat. In the next example, the heat is given and you need to calculate the final temperature of an object after it receives a certain amount of heat.
\begin{example} %%%%%%%%%%%%%%%%%%%%%%%% EXAMPLE BOX
A 50g piece of aluminum ($c_e=0.214\frac{cal}{g\cdot^{\circ}C}$) initially at 25$^{\circ}$C absorbs 100cal. Calculate the final temperature of the aluminum piece.\\
\textlcsc{ \textcolor{dgreen}{\Large \textbf{Solution}} }\\
\begin{enumerate}[label=\protect\circled{\color{white}\arabic*}]
\item \begin{bf}Step one:\end{bf} list of the given variables.
%%% DATA BOX
\begin{tcbitemize}[raster columns=3, raster rows=3, enhanced, sharp corners, raster equal height=rows, raster force size=false, raster column skip=0pt, raster row skip = 0pt]
%Empty corner and two headers
\tcbitem[blankest, width=1cm]
\tcbitem[header = helpful]
\texta
\tcbitem[header = harmful]
\textb
%First row
\tcbitem[firstcol = internal]
\textcn
\tcbitem[swotbox = G]
$c_e=0.214\frac{cal}{g\cdot^{\circ}C}$\\
$m=50g$\\
$T_{Initial}=25^{\circ}C$\\
$Q=100cal$
\tcbitem[swotbox = A]
$T_{Final}$\\
\end{tcbitemize}%%% DATA BOX
\item \begin{bf}Step two:\end{bf} use the formula $Q=m\cdot c_e\cdot  (T_{Final}-T_{Initial})$ that converts temperature increase to heat absorbed:
 %%%% COMMENTED EQUATION
\vspace{10mm} \begin{equation*}
     \tikzmark{A}{Q}\,\,= \, \tikzmark{B}{m}\,\,\,\cdot \,\,\, \tikzmark{C}{$c_e\cdot$}\,\,\,  (T_{Final}-\,\,\,\,\,\,\,\,\,\tikzmark{D}{$T_{Initial}$}\,\,\,\,\,\,\,\,\,)
\end{equation*}
\begin{tikzpicture}[overlay, remember picture,node distance =1.5cm]
    \node (Adescr) [below left=of A ]{$100 cal$};\draw[,->,thick] (Adescr) to [in=-90,out=90] (A);
   \node[red] (Bdescr) [below =of B]{$50g$}; \draw[red,->,thick] (Bdescr) to [in=-90,out=90] (B);
   \node[purple] (Cdescr) [below right =of C]{$0.214\frac{cal}{g\cdot^{\circ}C}$};\draw[purple,->,thick] (Cdescr) to [in=-90,out=90] (C.south);
   \node[blue,xshift=0.7cm] (Ddescr) [above right =of D]{$25^{\circ}C$};\draw[blue,->,thick] (Ddescr) to [in=45,out=-90] (D.north);
\end{tikzpicture}\vspace{12mm} %%%% COMMENTED EQUATION
\item \begin{bf}Step three:\end{bf} solve $100=50\cdot 0.214\cdot  (T_{Final}-25)$ for $T_{Final}$:
\begin{align}
  100=50\cdot 0.214\cdot  (T_{Final}-25)                   \tag*{divide by 50 in both sides}
    \\ \frac{100}{50}=0.214\cdot  (T_{Final}-25)   \tag*{divide by 0.214 in both sides}    
        \\ \frac{100}{50\cdot 0.214}= (T_{Final}-25)   \tag*{}             
                 \\ 9.34= (T_{Final}-25)   \tag*{add 25 in both sides}             
                 \\ 9.34+25= T_{Final}   \tag*{}             
                 \\ 34.34= T_{Final}   \notag            
\end{align}
The final temperature of the aluminum piece is $34.34^{\circ}C$.
\end{enumerate}
\faDiamond\ \textlcsc{ \textcolor{dgreen}{\Large \textbf{Study Check}} }\\
A 200g piece of iron ($c_e=0.1\frac{cal}{g\cdot^{\circ}C}$) initially at 15$^{\circ}$C absorbs 1000cal. Calculate the final temperature of the metal piece.\\
\flushright Answer: $T_{Final}=65^{\circ}C$.
\end{example}%%%%%%%%%%%%%%%%%%%%%%%% EXAMPLE BOX
\end{description}






\section{Calorimetry}
How do we measure the heat exchanged in chemical reactions? And more importantly, how do we know if a chemical reaction produces or consumes energy? The answer to these questions is: by means of a tool called calorimeter. Some calorimeters are very fancy and expensive, whereas others are as simple as a coffee cup. And both are used to measure the energy exchanges--sometimes produced, other consumed--in chemical reactions. This section will show you how to carry calorimetric calculations. 
\sloppy
\begin{description}
\item[\docfilehook{The calorimeter}{The calorimeter}] 
A calorimeter is device used to measure the energy exchanged in chemical reactions. Calorimetry is the science that measures heat exchange by using calorimeters. In essence is a closed system such as a coffee cup that does not let the heat come though its walls. There are tow different types of calorimeters. A constant-pressure calorimeter is the simplest of all calorimeters and is called constant-pressure as the pressure inside the calorimeter is constant and equal to the atmospheric pressure. Think of a double coffee cup covered with a lit. Inside this cup a chemical reaction occurs in a liquid phase. If the reaction produces any gases as the cup if just covered with a lit, the pressure will always be equal to the atmospheric pressure. Differently, a constant-volume calorimeter--also know as a bomb calorimeter--is a more complex and costly calorimeter in which normally gas phase reactions occur. Examples of the use of a constant-volume calorimeter are the calculation of the energy value of food--that is how they know how many calories are there in a bag of chips. In constant-volume calorimeter, the pressure is not constant but the volume of the calorimeter is.
   \begin{marginfigure}[-4cm]
      \includegraphics{chapter2/figure1-6}
      \caption{A constant-volume calorimeter is also called bomb calorimeter.}
      \label{fig:marginfig}
   \end{marginfigure}
    
\item[\docfilehook{Calorimetry formula}{Calorimetry formula}] 
The overall use of a calorimeter is to calculate the energy exchanges in chemical reactions--this is called the heat of reaction $\Delta Q_{reaction}$. In essence, there is a single formula to cary calorimetric calculations and all calorimeters function in the same way. Let us use a reaction that produced heat as an example. Inside the calorimeter, you introduced a given sample of a compound and a reaction happens producing heat (first contribution), the heat generated goes from the reaction to a container with water that warms up (second contribution). At the same time, the walls of the calorimeter also absorb certain amount of heat (third contribution). The formula used in calorimetry is the following: 
\begin{equation*}\begin{split}
 0=n\cdot\Delta Q_{reaction}+\Delta Q_{water}+\Delta Q_{walls} 
\end{split}\end{equation*}
where:
\begin{where}
 \item $n\cdot\Delta Q_{reaction}$   is the hear exchanged due to a chemical reaction
\item $\Delta Q_{water}$    is the heat received or released by water 
 \item $\Delta Q_{walls}$   is the heat absorbed by the walls
\end{where}

    
The water contribution is given by the heat formula given above and the walls contribution is the result of the effective heat capacity of the calorimeter, $c_{e,Cal}$. The larger the effective heat capacity of the calorimeter the more heat it absorbs from the reaction. Heat capacity is the same idea as specific heat but in different units. After we plug these two contribution into the formula above we arrive to the calorimetry formula:\resizeableyellownote{2.5}{1}{Add this formula to your flashcard.}
\begin{equation*}\begin{split}
\boxed{  0=n\cdot\Delta Q_{reaction}+m\cdot c_{e, Water}\cdot  (T_{F}-T_{I})   +c_{e,Cal}\cdot  (T_{F}-T_{I})   } \quad \textcolor{blue}{\text{Calorimetry formula}}
\end{split}\end{equation*}
where:
\begin{where}
 \item $\Delta Q_{reaction}$   is the hear exchanged due to a chemical reaction in $J/mol$
  \item $n$   is the number of moles reacted inside the calorimeter
\item $m$ is the mas of water contained in the calorimeter 
 \item $c_{e, Water}$  is the specific heat absorbed of water:$4.184 J/g^{\circ}C$
  \item $c_{e, Cal}$  is the heat capacity of the calorimeter also known as calorimeter factor
  \item $T_{F}$  is the final temperature of water in the calorimeter
  \item $T_{I}$  is the initial temperature of water in the calorimeter
\end{where}
\item[\docfilehook{Exothermic and endothermic reactions}{Exothermic and endothermic reactions}] 
Some reactions release heat and are called exothermic. Others absorb heat and are called endothermic. Think for example the combustion of the gas in a cooking stove, it produces gas and hence the chemical reaction happening is exothermic. Differently, if you cook bread, it needs heat to rise. Of if you melt an ice cube you need to give energy to the cube so that it becomes water. These are examples of endothermic reactions. Endothermic reactions have positive  $\Delta Q_{reaction}$ whereas exothermic reactions have negative $\Delta Q_{reaction}$.
   \begin{marginfigure}[-0cm]
      \includegraphics{chapter2/figure1-7}
      \caption{A constant-pressure calorimeter is also referred as a coffee-cup calorimeter and it consists simply in two cups with a lit and a thermometer.}
      \label{fig:marginfig}
   \end{marginfigure}
\begin{example} %%%%%%%%%%%%%%%%%%%%%%%% EXAMPLE BOX
A 3 mol-sample of a chemical is combusted in a calorimeter with 10g of water and a heat capacity of 10$KJ/^{\circ}C$. Calculate the heat of reaction knowing that the initial temperature of the water inside the calorimeter is $25^{\circ}C$ and the final $40^{\circ}C$.
\\
\textlcsc{ \textcolor{dgreen}{\Large \textbf{Solution}} }\\
%%% DATA BOX
\begin{tcbitemize}[raster columns=3, raster rows=3, enhanced, sharp corners, raster equal height=rows, raster force size=false, raster column skip=0pt, raster row skip = 0pt]
%Empty corner and two headers
\tcbitem[blankest, width=1cm]
\tcbitem[header = helpful]
\texta
\tcbitem[header = harmful]
\textb
%First row
\tcbitem[firstcol = internal]
\textcn
\tcbitem[swotbox = G]
$n=3mol$\\
$m=10g$\\
$T_{F}=40^{\circ}C$\\
$T_{I}=25^{\circ}C$\\
$c_{e, C}=10000 J/^{\circ}C$\\
$c_{e, W}=4.184 J/g^{\circ}C$\\
\tcbitem[swotbox = A]
$\Delta Q_{reaction}$\\
\end{tcbitemize}%%% DATA BOX
We have all date needed to solve the calorimetry formula. We have the moles of chemical inside the calorimeter, the heat capacity of the calorimeter, the initial and final temperature of water, and the amount of water. Mind that the specific heat of water is always given and you need to remember the value. Also and more importantly mind that the units of the heat capacity of the calorimeter are $KJ/^{\circ}C$, whereas the units of the specific heat of water are $J/g^{\circ}C$ and hence, we need to convert $KJ$ into $J$; that is the reason we use $10000 J/^{\circ}C$ as the heat capacity of the calorimeter. Plugging all values into the calorimetry formula we have:\\
\begin{equation*}\begin{split}
0=3mol\cdot\Delta Q_{reaction}+10g\cdot 4.184 J/g^{\circ}C\cdot  (40^{\circ}C-25^{\circ}C)  \\
+10000 J/^{\circ}C\cdot  (40^{\circ}C-25^{\circ}C) \end{split}\end{equation*}
Solving for $\Delta Q_{reaction}$ we obtain $-50209J/mol$ that is the same as $-50.209KJ/mol$. As the value is negative, it means that the reaction produced energy and hence is exothermic.
\\
\faDiamond\ \textlcsc{ \textcolor{dgreen}{\Large \textbf{Study Check}} }\\
A 2 mol-sample of a chemical reacts in a  bomb calorimeter with 20g of water and a heat capacity of 11$KJ/^{\circ}C$. Calculate the heat of reaction knowing that the temperature of water inside the calorimeter rises $10^{\circ}C$.
\\
\flushright Answer: $-55KJ/mol$.
\end{example}%%%%%%%%%%%%%%%%%%%%%%%% EXAMPLE BOX

\end{description}
  
  \section{Enthalpy}
In the last section we have seen that when a chemical reaction proceeds it exchanges energy with the surroundings. This energy can me measured in many different conditions. When it is measured at constant pressure--these are regular conditions in chemistry, think about a reaction happening at a beaker--this energy change has a different name: it is called enthalpy and is represented with the symbol $\Delta H_f^{\circ}$. In this section we will cover the different types of enthalpies depending on the type of reaction--formation or reaction--and we will find out how to compute the enthalpy change for a reaction using tables of standard enthalpies.
 \begin{marginfigure}
\begin{tcolorbox}[tab2,tabularx={XY}]%%%% FANCY COLOR TABLE
\begin{center}Standard state\end{center} & \begin{center}$\Delta H_f^{\circ}$ $(KJ/mol)$  \end{center}        \\\hline\hline
\ce{H2(g)}& 0       \\\hline
\ce{O2(g)}& 0       \\\hline
\ce{N2(g)}& 0       \\\hline
\ce{Cl2(g)}& 0       \\\hline
\ce{Fe(s)}& 0       \\\hline
\ce{Al(s)}& 0       \\\hline
\ce{C(graphite)}& 0       \\\hline
\ce{P4(s)}& 0       
\end{tcolorbox}%%%% FANCY COLOR TABLE
  \end{marginfigure}

\sloppy
\begin{description}

  
  
  
\item[\docfilehook{Table of standard enthalpies}{Table of standard enthalpies}] 
Enthalpies are tabulated in tables of standard enthalpies. The term standard refers to standard pressure conditions (1 atm) and is indicates by a degree sign on the top right side. Let us learn how to use this table. If you look for the standard enthalpy of C--an element--from the table you might find several values. The values of graphite carbon is $\Delta H_f^{\circ}=0KJ/mol$. Differently, the values for diamond carbon is different than zero, being $\Delta H_f^{\circ}=1.0KJ/mol$. Similarly, the value for gas carbon is not zero also, being $\Delta H_f^{\circ}=716.67KJ/mol$. This is because the nature state of carbon is in the form of graphite. That is, the most common way in which we find carbon in nature is in the form of graphite and not diamond or gas. Let us find the standard enthalpy for molecular nitrogen, \ce{N2(g)}--another element. If you look into the table you will find a value of $\Delta H_f^{\circ}=0KJ/mol$, again because the natural state of nitrogen is in the form of gas \ce{N2}. What is the standard enthalpy of gas hydrogen, \ce{H2}? If you look in the table, the value is also $\Delta H_f^{\circ}=0KJ/mol$. The rule of thumb is: elements on its natural state have zero $H_f^{\circ}$. Below we will explain more about the meaning of natural state.
Now, look for the standard entropy of carbon monoxide gas, \ce{CO(g)}. The value should not be zero, as carbon dioxide is not an element and is made of two different types of atoms. Indeed, in the table we find $\Delta H_f^{\circ}(\ce{CO})=-110.5KJ/mol$.
\item[\docfilehook{Natural state of an element}{Natural state of an element}]
The natural state of an element is the most stable state in which we find this element in nature. For example, Aluminum, it's natural state is not as a liquid or as a gas, is a s metallic solid. That is the reason why $\Delta H_f^{\circ}(\ce{Al(g)})=314KJ/mol$, whereas $\Delta H_f^{\circ}(\ce{Al(s)})=0KJ/mol$. In general, for metals, its natural state is solid. For non-metallic elements, such as hydrogen, oxygen, nitrogen, fluorine, chlorine, its natural state is in the form of a diatomic gas molecule. For example, $\Delta H_f^{\circ}(\ce{H2(s)})=0KJ/mol$, $\Delta H_f^{\circ}(\ce{N2(s)})=0KJ/mol$ or $\Delta H_f^{\circ}(\ce{O2(s)})=0KJ/mol$. For the case of carbon, its natural state is graphite, $\Delta H_f^{\circ}(\ce{C(s)_{graphite}})=0KJ/mol$.
\item[\docfilehook{Standard enthalpy of molecules}{Standard enthalpy of molecules}] 
Molecules such as \ce{H2O} or \ce{NO} have standard entropy different than zero. Mind that molecules are not elements, and hence are made of different elements.
\item[\docfilehook{Standard enthalpy change for a reaction}{Standard enthalpy change for a reaction}] 
In order to calculate the standard enthalpy for a reaction you need to use the following formula:
\resizeableyellownote{2.5}{1}{Add this formula to your flashcard.}
\begin{equation*}\begin{split}
\boxed{  \Delta H^{\circ}_R=\Delta H^{\circ}_{products}-\Delta H^{\circ}_{reactants}  } \quad \textcolor{blue}{\text{Enthalpy change}}
\end{split}\end{equation*}
where:
\begin{where}
 \item $\Delta H^{\circ}_R$   is the standard enthalpy change of the reaction
  \item $\Delta H^{\circ}_{products}$   is the standard enthalpy  of all products
\item $\Delta H^{\circ}_{reactants} $ is the standard enthalpy  of all reactants
 \end{where}
 
Now, imagine we need to calculate the change of standard enthalpy for the following reaction:
\begin{center}\ce{N2(g) + O2(g) -> 2NO(g)}\end{center}
In the case of the reaction above, we need to locate three enthalpies from the table: $\Delta H_f^{\circ}(\ce{N2(g)})$, $\Delta H_f^{\circ}(\ce{O2(g)})$, $\Delta H_f^{\circ}(\ce{NO(g)})$. If we look in the tables, you will find the values: $\Delta H_f^{\circ}(\ce{N2(g)})=0KJ/mol$ and $\Delta H_f^{\circ}(\ce{O2(g)})=0KJ/mol$. This makes sense, as these are the natural states of nitrogen and oxygen. Differently, $\Delta H_f^{\circ}(\ce{NO(g)})=90.29KJ/mol$. Now, in order to compute $\Delta H^{\circ}_R$ we need to take into account that the reaction involves two \ce{NO} molecules. Let us set up the formula:
\begin{equation*}\begin{split}
  \Delta H^{\circ}_R= \Delta H^{\circ}_{products}-\Delta H^{\circ}_{reactants}= \Big(2\cdot \Delta H_f^{\circ}(\ce{NO(g)})    \Big)-\Big(\Delta H_f^{\circ}(\ce{N2(g)})+ \Delta H_f^{\circ}(\ce{O2(g)}) \Big)      \\
  =     \Big(2\cdot90.29  \Big)-\Big( 0+0 \Big)= 181KJ
\end{split}\end{equation*}
This reaction is endothermic, that means it consumes energy.
Let's work on another example and calculate $\Delta H^{\circ}_R$ for the reaction:
\begin{center}\ce{2NO(g) + O2(g) -> 2NO2(g)}\end{center}
We need to locate three enthalpies from the table: $\Delta H_f^{\circ}(\ce{NO(g)})$, $\Delta H_f^{\circ}(\ce{O2(g)})$, $\Delta H_f^{\circ}(\ce{NO2(g)})$. If you locate these values in the table you will see $\Delta H_f^{\circ}(\ce{O2(g)})=0KJ/mol$, whereas $\Delta H_f^{\circ}(\ce{NO(g)})=90.29KJ/mol$ and $\Delta H_f^{\circ}(\ce{NO2(g)})=33.2KJ/mol$. Using the formula for $\Delta H^{\circ}_R$ we have:
\begin{equation*}\begin{split}
  \Delta H^{\circ}_R= \Delta H^{\circ}_{products}-\Delta H^{\circ}_{reactants}= \Big(2\cdot \Delta H_f^{\circ}(\ce{NO2(g)})    \Big)-\Big(2\cdot \Delta H_f^{\circ}(\ce{NO(g)})+ \Delta H_f^{\circ}(\ce{O2(g)}) \Big)      \\
  =     \Big(2\cdot 33.2  \Big)-\Big( 2\cdot 90.29+0 \Big)= -114KJ
\end{split}\end{equation*}
This reaction is exothermic and releases heat.

\item[\docfilehook{How to indicate $\Delta H^{\circ}_R$ in a reaction }{How to indicate $\Delta H^{\circ}_R$ in a reaction}]
Normally in chemical reaction the value of $\Delta H^{\circ}_R$ can be written in two different ways. You can see the enthalpy added in the reaction as a reactant or product or you can find the entropy written on the right side of the reaction. For example, in the first example above--the endothermic reaction--you can write:
\ce{ N2(g) + O2(g) -> 2NO(g) + 181KJ} 
or you can find:
\ce{ N2(g) + O2(g) -> 2NO(g) } \hspace*{0pt}\hfill $\Delta H^{\circ}_R=181KJ$.
For the second example above, the exothermic reaction, you can indicate the enthalpy like:
\ce{ 2NO(g) + O2(g) + 114KJ -> 2NO2(g)} 
or you can find:
\begin{center}\ce{ N2(g) + O2(g) -> 2NO(g) } \hspace*{0pt}\hfill $\Delta H^{\circ}_R=-114KJ$.\end{center}
Note that in exothermic reaction, the enthalpy value is indicated as a product as the reaction produces heat, whereas in endothermic reaction it is indicated as a reactant, as these reactions consume heat.

  \item[\docfilehook{Heat-Mole conversions}{Heat-Mole conversions}] 
Remember that a chemical reaction can be translated into a series of conversion factors that relate the moles of reactants with the products or with other reactants. At the same time, a chemical reaction involving heat can be converted into a series of conversion factors that related energy and the moles of reactants and products. \\
For the exothermic reaction:
\begin{center}\ce{ 2H2(g) + O2(g)  ->  2H2O(g) + 572KJ} \hspace*{0pt}\hfill $\Delta H=-572KJ$.\end{center}
the moles of hydrogen are related to heat as:
\begin{equation*}
\boxed{   \frac{\text{2 moles of }\ce{H2}}{\text{-572 KJ }}\ \text{ or  } \frac{ \text{-572 KJ } }{ \text{2 moles of } \ce{H2} }\   }
\end{equation*}
Similarly, we can relate energy with moles of \ce{O2}  or moles of water:

\begin{equation*}
\boxed{   \frac{\text{1 moles of }\ce{O2}}{\text{-572 KJ }}\ \text{ or  } \frac{\text{-572 KJ } }{ \text{1 moles of } \ce{O2} }\   }\quad
\boxed{   \frac{\text{2 moles of }\ce{H2O}}{\text{-572 KJ }}\ \text{ or  } \frac{ \text{-572 KJ }}{  \text{2 moles of } \ce{H2O} }\   }
\end{equation*}


We will use these relationships to convert moles of reactant or products into heat.
  
  \begin{example} %%%%%%%%%%%%%%%%%%%%%%%% EXAMPLE BOX
Hydrogen reacts with nitrogen to produce ammonia (\ce{NH3}) according to the following reaction
\begin{center}\ce{3H2(g) + N2(g) -> 2NH3(g) + 92KJ} \end{center}
Calculate: (a) the enthalpy of reaction; (b) indicate whether the reaction is endo or exothermic; (c) calculate the heat produced when produced 5 moles of ammonia.\\
\textlcsc{ \textcolor{dgreen}{\Large \textbf{Solution}} }\\
(a) the heat of reaction is -92KJ, and (b) the reaction is exothermic as the heat appears as a product. This means the reaction produces heat.
(c) We will use the conversion factor that relates ammonia with heat and will set up the moles of ammonia on the bottom of the conversion factor so that the units will cancel and energy will remain
\begin{equation*}
5\cancel{\text{ moles of }\ce{NH3}} \times \dfrac{\text{-92KJ}}{2\cancel{\text{ moles of }\ce{NH3}}}=-230KJ,
\end{equation*}
that is: 5 moles of ammonia produce -230KJ. The fact that this value is negative means that heat will be released.
\\
\faDiamond\ \textlcsc{ \textcolor{dgreen}{\Large \textbf{Study Check}} }\\
Calculate the number of hydrogen moles needed to generate -200KJ. \\
\flushright Answer: 6.5 moles.
\end{example}%%%%%%%%%%%%%%%%%%%%%%%% EXAMPLE BOX
  
  \clearpage\thispagestyle{empty}\mbox{}

  
%%%%%%%%%%%%%%%ENTHALPY TABLES%%%%%%%%%%%
\newpage\begin{fullwidth}
\begin{figure*}[h] % FUL FIGURE
\centering
\fontfamily{ppl}\selectfont
\begin{tabular}{llllllll}
\toprule
\multicolumn{8}{l}{Standard entropy table at 1atm and 298K.}   \\
\toprule
\rowcolor{black!45}Substance & $\Delta H_f^{\circ}$ &   & Substance  & $\Delta H_f^{\circ}$ && Substance & $\Delta H_f^{\circ}$  \\
\midrule
\rowcolor{black!15}Aluminum &         &      &      & & &      &       \\
\ce{Al(s)} &	0 && \ce{AlCl3(s)}	&-705.63&& \ce{Al2O3(s)}&	-1675.5 \\
\ce{Al(OH)3(s)}&	-1277&& \ce{Al2(SO4)3(s)}&	-3440&& \ce{NH3(Aq)} &	-80.8 \\
\ce{NH3(g)}&	-46.1&& \ce{NH4NO3(s)}& -365.6 & &\ce{Al(g)} &314  \\
\rowcolor{black!15}Barium &         &      &      & & &      &       \\
\ce{BaCl2(s)}	&-858.6 && \ce{BaCO3(s)}	&-1213&& \ce{Ba(OH)2(s)}	&-944.7\\
	\ce{BaO(s)}	 &-548.1&& \ce{BaSO4(s)}	&-1473.2&& \ce{BaSO4(s)}	&-1473.2\\


\rowcolor{black!15}Boron&         &      &      & & &      &       \\
	Solid	 \ce{BCl3(s)}&	-402.96& &
&&&
&\\

\rowcolor{black!15}Bromine&         &      &      & & &      &       \\
	 \ce{Br2(l)}&	0& &
	Aqueous	 \ce{Br-}&	-121& &
	 \ce{Br(g)}&	111.884\\
	 \ce{Br2(g)}	&30.91& &
	 \ce{BrF3	(g)}&-255.60& &
	 \ce{HBr(g)}&	-36.29\\
	 %

\rowcolor{black!15}Cadmium&         &      &      & & &      &       \\
	 \ce{Cd(s)}&	0& &
	 \ce{CdO(s)}&	-258& &
	 \ce{Cd(OH)2(s)}&	-561\\
	 \ce{CdS(s)}&	-162& &
	 \ce{CdSO4(s)}&	-935& &
&\\


\rowcolor{black!15}Calcium&         &      &      & & &      &       \\
	 \ce{Ca(s)}&	0& &
	 \ce{Ca(g)}&	178.2& &
	 \ce{Ca2+(g)}&	1925.90\\
	 \ce{CaC2(s)}	&-59.8& &
	 \ce{CaCO3(s)}&	-1206.9& &
	 \ce{CaCl2(s)}	&-795.8\\
	 \ce{CaCl2(aq)}&	-877.3& &
	 \ce{Ca3(PO4)2(s)}&	-4132& &
	 \ce{CaF2(s)}&	-1219.6\\
	 \ce{CaH2(s)}&	-186.2& &
	 \ce{Ca(OH)2(s)}&	-986.09& &
	 \ce{Ca(OH)2(aq)}	&-1002.82\\
	 \ce{CaO(s)}&	-635.09& &
	 \ce{CaSO4(s)}&	-1434.52& &
	 \ce{CaS(s)}&	-482.4\\
	 \ce{CaSiO3(s)}&	-1630& &
&&&
&\\

\rowcolor{black!15}Caesium&         &      &      & & &      &       \\
	 \ce{Cs(s)}&	0& &
	 \ce{Cs(g)}&	76.50& &
 \ce{Cs(l)}&	2.09\\
	 \ce{Cs+(g)}&	457.964& &
	 \ce{CsCl(s)}&	-443.04& &
&\\
\rowcolor{black!15}Carbon&         &      &      & & &      &       \\
	 \ce{C_{graphite}(s)}&	0& &
d	 \ce{C_{diamond}(s)}&	1.9& &
 \ce{C(g)}&	716.67\\
	 \ce{CO2(g)}&	-393.509& &
	 \ce{CS2(l)}	&89.41& &
 \ce{CS2(g)	}&116.7\\
	 \ce{CO(g)}&	-110.525& &
	 \ce{COCl2(g)}&	-218.8& &
	 \ce{CO2(aq)}	&-419.26\\
 \ce{HCO3-(aq)}	&-689.93& &
	 \ce{CO3^2-(aq)}&	-675.23& &
&\\

\rowcolor{black!15}Chlorine&         &      &      & & &      &       \\
	 \ce{Cl(g)}&	121.7& &
 \ce{Cl-(aq)}&	-167.2& &
	 \ce{Cl2(g)}	&0\\

\rowcolor{black!15}Chromium&         &      &      & & &      &       \\
	 \ce{Cr(s)}&	0& &
&&&
&\\

\rowcolor{black!15}Copper&         &      &      & & &      &       \\
	 \ce{Cu(s)}&	0& &
	 \ce{CuO(s)}&	-155.2& &
	 \ce{CuSO4(aq)}&	-769.98\\

\rowcolor{black!15}Fluorine&         &      &      & & &      &       \\
	 \ce{F2(g)}&	0& &
&&&
&\\

\rowcolor{black!15}Hydrogen&         &      &      & & &      &       \\
	 \ce{H(g)}&	218& &
	 \ce{H2(g)}&	0& &
 \ce{H2O(g)}&	-241.818\\
	 \ce{H2O(l)}&	-285.8& &
	 \ce{H+(aq)}&	0& &
	 \ce{OH-(aq)}&	-230\\
	 \ce{H2O2}&	-187.8& &
	 \ce{H3PO4(l)}&	-1288& &
	 \ce{HCN(g)}&	130.5\\
	 \ce{HBr(l)}&	-36.3& &
	 \ce{HCl(g)}&	-92.30& &
	 \ce{HCl(aq)}&	-167.2\\
	 \ce{HF(g)}&	-273.3& &
	 \ce{HI(g)}&	26.5& &
&\\
\rowcolor{black!15}Iodine&         &      &      & & &      &       \\
	 \ce{I2(s)}&	0& &
	 \ce{I2(g)}&	62.438& &
	 \ce{I2(aq)}&	23\\
	 \ce{I-(aq)}&	-55& &
&&&
&\\
\bottomrule
\end{tabular}
\label{tab:H}
\end{figure*} % FUL FIGURE
\end{fullwidth}







\newpage\begin{fullwidth}
\begin{figure*}[h] % FUL FIGURE
\centering
\fontfamily{ppl}\selectfont
\begin{tabular}{llllllll}
\toprule
\multicolumn{8}{l}{(cont.) Standard entropy table at 1atm and 298K.}   \\
\toprule
\rowcolor{black!45}Substance & $\Delta H_f^{\circ}$ &   & Substance  & $\Delta H_f^{\circ}$ && Substance & $\Delta H_f^{\circ}$  \\
\midrule
\rowcolor{black!15}Iron&         &      &      & & &      &       \\
	\ce{Fe}&	0& &
	 \ce{Fe3C(s)}&	5.4& &
	 \ce{FeCO3(s)}&	-750.6\\
	 \ce{FeCl3(s)}&	-399.4& &
 \ce{FeO(s)}&	-272& &
	 \ce{Fe3O4(s)}&	-1118.4\\
	 \ce{Fe2O3(s)}&	-824.2& &
	 \ce{FeSO4(s)}&	-929& &
	 \ce{Fe2(SO4)3(s)}&	-2583\\
	 \ce{FeS(s)}&	-102& &
	 \ce{FeS2(s)}&	-178& &
&\\

\rowcolor{black!15}Lead&         &      &      & & &      &       \\
	 \ce{Pb(s)}&	0& &
 \ce{PbO2(s)}&	-277& &
	 \ce{PbS(s)}&	-100\\
	 \ce{PbSO4(s)}&	-920& &
	 \ce{Pb(NO3)2(s)}&	-452& &
	 \ce{PbSO4(s)}&	-920\\
	 %
\rowcolor{black!15}Magnesium&         &      &      & & &      &       \\
 \ce{Mg(s)}&	0& &
 \ce{Mg2+(aq)}&	-466.85& &
	 \ce{MgCO3(s)}&	-1095.7\\
	 \ce{MgO(s)	}&-601.6& &
	 \ce{MgSO4(s)}&	-1278.2& &
\ce{MgCl2(s)}&	-641.8\\

\rowcolor{black!15}Manganese&         &      &      & & &      &       \\
 \ce{Mn(s)}	&0& &
	 \ce{MnO	(s)}&-384.9& &
	 \ce{MnO2(s)}	&-519.7\\
	 \ce{Mn2O3(s)}&	-971& &
	 \ce{Mn3O4(s)}&	-1387& &
 \ce{MnO4-(aq)}	&-543\\

\rowcolor{black!15}Mercury&         &      &      & & &      &       \\
	 \ce{HgO(s)}&	-90.83& &
	 \ce{HgS(s)}&	-58.2& &
&\\

\rowcolor{black!15}Nitrogen&         &      &      & & &      &       \\
	 \ce{N2(g)}&	0& &
 \ce{NH3(aq)}	&-80.8& &
 \ce{NH3(g)}&	-45.90\\
	 \ce{NH4Cl}&	-314.55& &
	 \ce{NO2(g)}&	33.2& &
	 \ce{N2O(g)}&	82.05\\
	 \ce{NO(g)}&	90.29& &
	 \ce{N2O4(g)}	&9.16& &
	 \ce{N2O5(s)}&	-43.1\\

\rowcolor{black!15}Oxygen&         &      &      & & &      &       \\
	 \ce{O(g)}&	249& &
	 \ce{O2(g)}&	0& &
 \ce{O3(g)}	&143\\

\rowcolor{black!15}Phosphorus&         &      &      & & &      &       \\
	 \ce{P4(s)}&	0& &
	 \ce{P_{red}(s)}	&-17.4& &
	 \ce{P_{black}(s)}&	-39.3\\
	 \ce{PCl3(l)}&	-319.7& &
 \ce{PCl3(g)	}&-278& &
	 \ce{PCl5(s)}&	-440\\
	 \ce{PCl5(g)}& 	-321& &
	 \ce{P2O5(s)}&	-1505.5& &
&\\

\rowcolor{black!15}Potassium&         &      &      & & &      &       \\
	 \ce{KBr(s)}&	-392.2& &
	 \ce{K2CO3(s)}&	-1150& &
	 \ce{KClO3(s)}&	-391.4\\
	 \ce{KCl(s)}	&-436.68& &
	 \ce{KF(s)}&	-562.6& &
	 \ce{K2O(s)}&	-363\\
	 \ce{KClO4(s)}&	-430.12& &
&&&
&\\


\rowcolor{black!15}Silicon&         &      &      & & &      &       \\
	 \ce{Si(g)	}& 368.2& &
	 \ce{SiC(s)}	&-74.4 & &
	 \ce{SiCl4(l)}&	-640.1\\
	 \ce{SiO2(s)}&	-910.86& &
&&&
&\\
\rowcolor{black!15}Silver&         &      &      & & &      &       \\
 \ce{AgBr(s)}&	-99.5& &
	 \ce{AgCl(s)	}&-127.01& &
	 \ce{AgI(s)}	&-62.4\\
	 \ce{Ag2O(s)}&	-31.1& &
	 \ce{Ag2S(s)}&	-31.8& &
&\\




\rowcolor{black!15}Sodium&         &      &      & & &      &       \\
	 \ce{Na(s)}&	0& &
\ce{Na(g)}&	+107.5& &
	 \ce{NaHCO3(s)}&	-950.8\\
	 \ce{Na2CO3(s)}&	-1130.77& &
 \ce{NaCl(aq)}&	-407.27& &
	 \ce{NaCl(s)}&	-411.12\\
 
 \ce{NaF(s)}&	-569.0& &
	 \ce{NaOH(aq)}&	-469.15& &
	 \ce{NaOH(s)}	&-425.93\\

	 \ce{Na2O(s)}	&-414.2& &
		&& &
	&\\





\rowcolor{black!15}Sulfur&         &      &      & & &      &       \\
	 \ce{S8_{monoclinic}(s)}&	0.3& &
	 \ce{S8_{rhombic}(s)}&	0& &
	 \ce{H2S(g)}&	-20.63\\
 \ce{SO2(g)}&	-296.84& &
 \ce{SO3(g)}&	-395.7& &
\ce{H2SO4(l)}&	-814\\

\rowcolor{black!15}Titanium&         &      &      & & &      &       \\

	 \ce{Ti(s)}&	0& &
 \ce{Ti(g)}&	468& &
 \ce{TiCl4(g)}&	-763.2  \\
 \ce{TiCl4(l)}&	-804.2& &
	 \ce{TiO2(s)}&	-944.7& &
&	\\

\rowcolor{black!15}Zinc&         &      &      & & &      &       \\
	 \ce{Zn(g)}&	130.7& &
	 \ce{ZnCl2(s)}&	-415.1& &
	 \ce{ZnO(s)}&	-348.0  \\

\bottomrule
\end{tabular}
\end{figure*} % FUL FIGURE
\end{fullwidth}
%%%%%%%%%%%%%%%ENTHALPY TABLES%%%%%%%%%%%





\begin{example} %%%%%%%%%%%%%%%%%%%%%%%% EXAMPLE BOX
Using the enthalpy table, calculate $\Delta H^{\circ}_R$ for the following reactions: (a)\ce{ 4H2(g) + O2(g) -> 2H2O(l) }; (b)\ce{ 3H2(g) + N2(g) -> 2NH3(g) }; (c)\ce{ 2Al(s) + 3Cl2(g) -> 2AlCl3(s) }.
\\
\textlcsc{ \textcolor{dgreen}{\Large \textbf{Solution}} }\\
In order to answer all questions, we need a set of $\Delta H^{\circ}_f$ values: $\Delta H^{\circ}_f(\ce{H2(g)})$, $\Delta H^{\circ}_f(\ce{O2(g)})$, $\Delta H^{\circ}_f(\ce{N2(g)})$, $\Delta H^{\circ}_f(\ce{Al(s)})$ are all zero, whereas $\Delta H^{\circ}_f(\ce{H2O(l)})=-285.8KJ/mol$, $\Delta H^{\circ}_f(\ce{NH3(g)})=-45.0KJ/mol$ and $\Delta H^{\circ}_f(\ce{AlCl3(s)})=-705.63KJ/mol$. For the first example, we have:
\begin{equation*}\begin{split}
  \Delta H^{\circ}_R= \Big(2\cdot \Delta H_f^{\circ}(\ce{H2O(l)})    \Big)-\Big(4\cdot \Delta H_f^{\circ}(\ce{H2(g)})+ \Delta H_f^{\circ}(\ce{O2(g)}) \Big)      \\
  =     \Big(2\cdot -285.8  \Big)-\Big( 4\cdot 0+0 \Big)= -572KJ
\end{split}\end{equation*}
For the second example:
\begin{equation*}\begin{split}
  \Delta H^{\circ}_R= \Big(2\cdot \Delta H_f^{\circ}(\ce{NH3(g)})    \Big)-\Big(2\cdot \Delta H_f^{\circ}(\ce{Al(s)})+ 3\cdot \Delta H_f^{\circ}(\ce{Cl2(g)}) \Big)      \\
  =     \Big(2\cdot -45 \Big)-\Big( 2\cdot 0+3\cdot0 \Big)= -90KJ
\end{split}\end{equation*}
Finally, for the last reaction:
\begin{equation*}\begin{split}
  \Delta H^{\circ}_R= \Big(2\cdot \Delta H_f^{\circ}(\ce{AlCl3(s)})    \Big)-\Big(3\cdot \Delta H_f^{\circ}(\ce{H2(g)})+ \Delta H_f^{\circ}(\ce{N2(g)}) \Big)      \\
  =     \Big(2\cdot -705.63  \Big)-\Big( 3\cdot 0+0 \Big)= -1411KJ
\end{split}\end{equation*}
\\
\faDiamond\ \textlcsc{ \textcolor{dgreen}{\Large \textbf{Study Check}} }\\
Using the enthalpy table, calculate $\Delta H^{\circ}_R$ for the following reaction: \ce{ Fe2O3(s) +  3CO(g) -> 2Fe(s) + 3CO2(g) }.   
\\
\flushright Answer: $-25KJ$.
\end{example}%%%%%%%%%%%%%%%%%%%%%%%% EXAMPLE BOX


\item[\docfilehook{ $\Delta H^{\circ}_R$ and $\Delta H^{\circ}_f$ }{$\Delta H^{\circ}_R$ and $\Delta H^{\circ}_f$ }]
Consider the following two reactions:
\begin{center}\ce{ N2(g) + O2(g) -> 2NO(g) } \hspace*{0pt}\hfill $\Delta H^{\circ}_f=-114KJ$.\end{center}
\begin{center}\ce{ Fe2O3(s) +  3CO(g) -> 2Fe(s) + 3CO2(g) } \hspace*{0pt}\hfill $\Delta H^{\circ}_R=-25KJ$.\end{center}
The first example represents a formation reaction and thus the enthalpy is labeled as $\Delta H^{\circ}_f$. This is called standard enthalpy of formation. A formation reaction always have the elements on its standard state as reactants. Think of \ce{N2(g)} and \ce{H2(g)}. The enthalpy of formation for both is zero as they represent nitrogen and hydrogen on their natural states. Differently, the second reaction is not a formation reaction, as the reactants are not elements on its natural state. Think of \ce{CO(g)} or \ce{Fe2O3(s)}, their enthalpy is not zero as they do not represent any element on its natural state. That is the reason, the second enthalpy is labeled as $\Delta H^{\circ}_R$. This is called standard enthalpy of reaction.
  \end{description}



 \section{Hess's Law: Manipulating reaction enthalpies }
In the previous section we relied on a table el standard enthalpies of formation in order to compute enthalpy changes in general reaction. This enthalpy change $\Delta H_R^{\circ}$ is related to the heat exchanged in the reaction. In this section we will not use the tables of enthalpy anymore. Imagine you do not have access to this table. And we will find alternative ways to predict $\Delta H_f^{\circ}$ given a series of reactions with know enthalpies. In short you will have to identify the enthalpies that are zero--the enthalpies corresponding to an element on its natural state-- and set up an equation that helps you find out the missing enthalpy. 
\sloppy
\begin{description}
\item[\docfilehook{Reverting reactions}{Reverting reactions}] 
Imagine they give you the following reaction:
\begin{center}\ce{ N2(g) + O2(g) -> 2NO(g) } \hspace*{0pt}\hfill $\Delta H^{\circ}_1=-114KJ$\end{center}
and you need to calculate the enthalpy change for this other reaction:
\begin{center}\ce{2NO(g)   -> N2(g) + O2(g) } \hspace*{0pt}\hfill $\Delta H^{\circ}_2$\end{center}
If you compare both reaction you will see the second reaction equals to the first reaction but reverted. If you revert a reaction, the enthalpy change changes sign. Therefore, $\Delta H^{\circ}_2=114KJ$.
\item[\docfilehook{Timing reactions by a number}{Timing reactions by a number}] 
Imagine they give you the following reaction:
\begin{center}\ce{ N2(g) + O2(g) -> 2NO(g) } \hspace*{0pt}\hfill $\Delta H^{\circ}_1=-114KJ$\end{center}
and you need to calculate the enthalpy change for this other reaction:
\begin{center}\ce{2N2(g) + 2O2(g) -> 4NO(g) } \hspace*{0pt}\hfill $\Delta H^{\circ}_2$\end{center}
If you compare both reaction you will see the second reaction equals to the first reaction timed by two. If you time a reaction by two, the enthalpy change should also be timed by two. Therefore, $\Delta H^{\circ}_2=2\cdot -114=-228KJ$.
\item[\docfilehook{Combining reactions}{Combining reactions}] 
Imagine they give you the following two reactions:
\begin{center}
\begin{tabular}{ r l r }
\ce{C(s) + O2(g)  & -> \: CO2(g)}&$\qquad \Delta H_1$ \\
\ce{H2(g) + 1/2 O2(g) & -> \: H2O(l)}&$\qquad \Delta H_2$ \\
 \end{tabular}
 \end{center}
 and the ask the enthalpy change for the following reaction:
\begin{center}\ce{ C(s) + H2(g) +3/2 O2(g)  -> CO2(g) + H2O(l) } \hspace*{0pt}\hfill $\Delta H^{\circ}_3$\end{center}
If you look closely to the last reaction, you will see it results from adding the first two reactions, so that:
\begin{center}
\begin{tabular}{ r l r }
\ce{C(s) + O2(g)  & -> \: CO2(g)}&$\qquad \Delta H_1=-393KJ$ \\
\ce{H2(g) + 1/2 O2(g) & -> \: H2O(l)}&$\qquad \Delta H_2=-286KJ$ \\
\multicolumn{3}{r}{} \rule{13cm}{0.4pt}\\
Sum: \ce{C(s) + H2(g) +3/2 O2(g)  &-> \: CO2(g) + H2O(l)} &$\qquad \Delta H_3$\\
 \end{tabular}
 \end{center}
Therefore, $\Delta H_3=\Delta H_1+\Delta H_2=-679KJ$. When adding two chemical reactions the resulting enthalpy is the result of adding the enthalpy of both reactions.

\begin{example} %%%%%%%%%%%%%%%%%%%%%%%% EXAMPLE BOX
 Calculate the enthalpy for this reaction:
 \begin{center}\ce{ 2C(s) + H2(g) -> C2H2(g) } \hspace*{0pt}\hfill $\Delta H^{\circ}_4$\end{center}
Given the following thermochemical equations:
\begin{center}
\begin{tabular}{ r l r }
\ce{C2H2(g) + 5/2 O2(g)  & -> \: 2CO2(g) + H2O(l)}&$\qquad \Delta H_1=-1299.5KJ$ \\
\ce{C(s) + O2(g)& -> \:  CO2(g)}&$\qquad \Delta H_2=-393.5KJ$ \\
\ce{H2(g) + 1/2 O2(g)& -> \:  H2O(l)}&$\qquad \Delta H_3=-285.8KJ$ \\
 \end{tabular}
 \end{center}
\textlcsc{ \textcolor{dgreen}{\Large \textbf{Solution}} }\\
In order to get the enthalpy for reaction (4) we will have to combine reactions (1), (2) and (3), by adding, subtracting, or multiplying by a number so that the results adds up to reaction (4). A trick to do this is compare molecule by molecule in reaction (4) and see in which reaction we can find the same one. For example, reaction (4) contains 2\ce{C(s)} in the reactant side. \ce{C(s)}  can also be found in (2) also as reactant. However, in (2) \ce{C(s)} is not timed by 2. There we will use two times reaction (2):
 \begin{center}\ce{ 2C(s) + 2O2(g) ->   2CO2(g)} \hspace*{0pt}\hfill $2\cdot \Delta H^{\circ}_2= -787$\end{center}
Reaction (4) also contains  \ce{H2(g)}, which can be found in  (3). There we will use (3) as it is:
 \begin{center}\ce{ H2(g) + 1/2 O2(g) ->   H2O(l)} \hspace*{0pt}\hfill $\Delta H^{\circ}_3=-285.8KJ$\end{center}
Reaction (4) also contains \ce{C2H2(g)} as a product. We can find the same chemical in (1) but as a reactant. There we will have to invert (1):
 \begin{center}\ce{  2CO2(g) + H2O(l)  -> C2H2(g) + 5/2 O2(g)  } \hspace*{0pt}\hfill $-1\cdot \Delta H_1=1299.5KJ$\end{center}
If we add the three previous reactions, we have:
\begin{center}
\begin{tabular}{ l l l }
\ce{2C(s) + 2O2(g) 	& -> \: 	2CO2(g)}&$\enskip  	2\cdot \Delta H_2= -787KJ$ \\
\ce{H2(g) + 1/2 O2(g)   	& -> \: 	H2O(l)}&$\enskip  	\Delta H_3=-285.8KJ$ \\
\ce{2CO2(g) + H2O(l)  	& -> \: 	C2H2(g) + 5/2 O2(g)}&$\enskip 	-1\cdot \Delta H_1=1299.5KJ$ \\
\multicolumn{2}{l}{} \rule{5cm}{0.4pt}&\\
Sum: \ce{2C(s) + H2(g)	&-> \: 	2CO2(g) + C2H2(g) } &$\enskip 	\Delta H_4$\\
 \end{tabular}
 \end{center}
 Therefore in the enthalpy for the reaction (4) will be: \begin{center}$\Delta H^{\circ}_4=2\cdot \Delta H_2+\Delta H_3-1\cdot \Delta H_1=226.7 KJ$ \end{center}
\faDiamond\ \textlcsc{ \textcolor{dgreen}{\Large \textbf{Study Check}} }\\
Calculate the enthalpy for this reaction:
\begin{center}\ce{ 3NO2(g) + H2O(l) -> 2HNO3(aq) + NO(g) }\end{center}
Given the following thermochemical equations:
\begin{center}
\begin{tabular}{ r l r }
\ce{2NO(g) + O2(g)  & -> \: 2NO2(g)}&$\qquad \Delta H_1=-116 KJ$ \\
\ce{2N2(g) + 5O2(g) + 2H2O(l) & -> \:  4HNO3(aq)}&$\qquad \Delta H_2=-256KJ$ \\
\ce{N2(g) + O2(g)& -> \:  2NO(g)}&$\qquad \Delta H_3=183KJ$ \\
 \end{tabular}
 \end{center}
\flushright Answer: $-137 KJ$.
\end{example}%%%%%%%%%%%%%%%%%%%%%%%% EXAMPLE BOX






\end{description}



\end{document}
%\textquotesingle

%\section{The atom}\marginnote{ \faEnvelope\myemail{dtorresrangel@bmcc.cuny.edu}{Error in the Book}{ Send Me typos!}}
%XXXXX
%\sloppy
%\begin{description}
%\item[\docfilehook{Elements and Symbols}{Elements and Symbols}] 
%\end{description}
%

%\begin{marginfigure}%%%%%%%QUOTES
%    \begin{shadequote}[l]{Democritus}
%Nothing exists except atoms and empty space; everything else is opinion.
%\end{shadequote}   \end{marginfigure}%%%%%%QUOTES
%\begin{marginfigure}%%%%%%%MARGIN FIGURE
%      \includegraphics{chapter2/figure1}
%      \label{fig:marginfig}
%   \end{marginfigure}%%%%%%%MARGIN FIGURE

%\begin{definition}[Political Factors]%%%%%%%%%ADITIONAL INFO BOX
%\begin{minipage}{0.25\linewidth}
%\includegraphics[width = \linewidth]{example-image-a}
%\end{minipage}%
%\hfill
%\begin{minipage}{0.7\linewidth}
%Analyses to what degree the government intervenes in the
%economy. It includes regulations and legal issues and defines
%both formal and informal rules under which the firm must
%operate. Political factors include: tax policy, employment laws,
%environmental regulations, trade restriction tariffs and political
%stability.
%\end{minipage}%
%\end{definition}%%%%%%%%%ADITIONAL INFO BOX
%
%\begin{example} %%%%%%% EXAMPLE BOX with explanation
%Obtain the electronic configuration of C.\\
%\textlcsc{ \textcolor{dgreen}{\Large Solution} }\\
%The atomic number of C is Z=6 and that means C has 6 electrons. The orbital order from Figure \ref{fig:orbitaltable} is: $1s$,$2s$, $2p$, $3s$, etc. Each $s$ orbital can fit two electrons, whereas the occupancy of  the $p$ orbitals is six electrons. Hence the electronic configuration of C is: $1s^2 2s^2 2p^2$. The $s$ orbitals are all filled, whereas the $p$ orbital is only occupied with two electrons.
%\\
%\faDiamond\ \textlcsc{ \textcolor{dgreen}{\Large \textbf{Study Check}} }\\Obtain the electronic configuration of Ni.
%\flushright Answer: $1s^2 2s^2 2p^6 3s^2 3p^6 4s^2 3d^8$. 
%\end{example}
%\begin{marginfigure}
%\begin{work} % 
%\textlcsc{ \textcolor{olive}{\Large Get the Answer by: } }
%\begin{enumerate}
%\item Get the electrons
%\item Check the orbital order table
%\item Fill each orbital following the order
%\end{enumerate}
%\end{work}% 
%\end{marginfigure}%%%%%%% EXAMPLE BOX with explanation

% \begin{marginfigure}%%%%% DISCUSSION
%\begin{tcolorbox}[enhanced,colback=red!5!white,colframe=black!50!red,boxrule=1pt,
%  arc=0pt,outer arc=0pt,drop heavy lifted shadow]
%\faGears\ 
%\docenvdef{Discussion:} Look around your apartment and list a pure substance, a compound, a heterogeneous mixture and a homogeneous mixture?\end{tcolorbox}
% \end{marginfigure} %%%%% DISCUSSION
%Table \ref{tab:units} % TABLE OR FIGURE REFERENCE




%\vspace{6mm} \begin{equation*}%%%% COMMENTED EQUATION
%    \mathcal{A} = (\,\tikzmark{identity}{\texttt{I}} -\tikzmark[red]{G}{\texttt{G}}\,\,\, 
%    \tikzmark[blue]{L}{\texttt{L}} - \tikzmark[purple]{C}{\texttt{C }}\,)
%\end{equation*}
%\begin{tikzpicture}[overlay, remember picture,node distance =1.5cm]
%    \node (identitydescr) [below left=of identity ]{words};
%    \draw[,->,thick] (identitydescr) to [in=-90,out=90] (identity);
%    \node[red] (Gdescr) [below =of G]{other words};
%    \draw[red,->,thick] (Gdescr) to [in=-90,out=90] (G);
%    \node[blue,xshift=1cm] (Ldescr) [above right =of L]{some words};
%    \draw[blue,->,thick] (Ldescr) to [in=45,out=-90] (L.north);
%    \node[purple] (Cdescr) [below right =of C]{more words};
%    \draw[purple,->,thick] (Cdescr) to [in=-90,out=90] (C.south);
%\end{tikzpicture}\vspace{10mm} %%%% COMMENTED EQUATION


%\begin{tcolorbox}[tab2,tabularx={X||Y|Y|Y|Y||Y}]%%%% FANCY COLOR TABLE
%Group & One     & Two     & Three    & Four     & Sum      \\\hline\hline
%Red   & 1000.00 & 2000.00 &  3000.00 &  4000.00 & 10000.00 \\\hline
%Green & 2000.00 & 3000.00 &  4000.00 &  5000.00 & 14000.00 \\\hline
%Blue  & 3000.00 & 4000.00 &  5000.00 &  6000.00 & 18000.00 \\\hline\hline
%Sum   & 6000.00 & 9000.00 & 12000.00 & 15000.00 & 42000.00
%\end{tcolorbox}%%%% FANCY COLOR TABLE
