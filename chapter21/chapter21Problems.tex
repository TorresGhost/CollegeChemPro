\documentclass[main.tex]{subfiles}
\begin{document}\newpage
\setdoublesep{0.35700 em}  % 'Bond Spacing'
\setatomsep{1.78500 em}    % 'Fixed Length'
\setbondoffset{0.18265 em} % 'Margin Width'
\newcommand{\bondwidth}{0.06642 em} % 'Line Width'
\setbondstyle{line width = \bondwidth}
\newgeometry{left=0.8in,right=0.8in, top=2.5cm,bottom=2cm}
\fancyhfoffset[E,O]{0pt}
\setlength{\columnsep}{30pt}
\begin{conclusion}
\end{conclusion}
%\setstretch{0.3}
\begin{multicols*}{2}\setcounter{numA}{1}

{\raggedright\textsc{\textbf{The nature of acids and bases}}\par}


\begin{question}[ID=\the\value{numA}]\SetQuestionProperties{section-title=\nameref{sec:units}}
Classify the following species as Arrhenius acids, or bases:
\begin{inparaenum}[(a)]
\item \ce{H2SO4} % (acid)
\item \ce{NaOH} % (base)
\item \ce{HNO3} % (acid)
\item \ce{HCl} % (acid)
\item \ce{Ca(OH)2} % (base)
 \end{inparaenum}
\end{question}
\begin{solution}
\begin{inparaenum}[(a)]
\item \ce{H2SO4}   (acid)
\item \ce{NaOH}   (base)
\item \ce{HNO3}   (acid)
\item \ce{HCl}   (acid)
\item \ce{Ca(OH)2}   (base)
\end{inparaenum}
\hspace{0.1cm}\end{solution}\stepcounter{numA}%%%%%%%%%%%%


\begin{question}[ID=\the\value{numA}]\SetQuestionProperties{section-title=\nameref{sec:units}}
Classify the following species as Br\"{o}nsted-Lowry acids, bases or both:
\begin{inparaenum}[(a)]
\item \ce{H2O} % (both)
\item \ce{NaOH} % (base)
\item \ce{NH3} % (base)
\item \ce{NO2-} % (base)
 \end{inparaenum}
\end{question}
\begin{solution}
\begin{inparaenum}[(a)]
\item \ce{H2O}   (both)
\item \ce{NaOH}   (base)
\item \ce{NH3}   (base)
\item \ce{NO2-}   (base)
\end{inparaenum}
\hspace{0.1cm}\end{solution}\stepcounter{numA}%%%%%%%%%%%%

\begin{question}[ID=\the\value{numA}]\SetQuestionProperties{section-title=\nameref{sec:units}}
Classify the following species as Br\"{o}nsted-Lowry acids, bases or both:
\begin{inparaenum}[(a)]
\item \ce{HCO3-} % (both)
\item \ce{HI} % (acid)
\item \ce{HCN} % (acid)
\item \ce{HSO4-} % (both)
\item \ce{HCOONa} % (base)
 \end{inparaenum}
\end{question}
\begin{solution}
\begin{inparaenum}[(a)]
\item \ce{HCO3-}   (both)
\item \ce{HI}   (acid)
\item \ce{HCN}   (acid)
\item \ce{HSO4-}   (both)
\item \ce{HCOONa}   (base)
\end{inparaenum}
\hspace{0.1cm}\end{solution}\stepcounter{numA}%%%%%%%%%%%%

\begin{question}[ID=\the\value{numA}]\SetQuestionProperties{section-title=\nameref{sec:units}}
Classify the following species as Lewis acids or bases:
\begin{inparaenum}[(a)]
\item \ce{CO2} % Lewis acid
\item \ce{NH3} % Lewis base
\item \ce{F-} % Lewis base
 \end{inparaenum}
\end{question}
\begin{solution}
\begin{inparaenum}[(a)]
\item \ce{CO2}   (Lewis acid)
\item \ce{NH3}   (Lewis base)
\item \ce{F-}   (Lewis base)
\end{inparaenum}
\hspace{0.1cm}\end{solution}\stepcounter{numA}%%%%%%%%%%%%

\begin{question}[ID=\the\value{numA}]\SetQuestionProperties{section-title=\nameref{sec:units}}
Classify the following species as Lewis acids or bases:
\begin{inparaenum}[(a)]
\item \ce{SO2} % (Lewis acid)
\item \ce{BCl3} % (Lewis acid)
\item \ce{I-} % (Lewis base)
 \end{inparaenum}
\end{question}
\begin{solution}
\begin{inparaenum}[(a)]
\item \ce{SO2}   (Lewis acid)
\item \ce{BCl3}   (Lewis acid)
\item \ce{I-}   (Lewis base)
\end{inparaenum}
\hspace{0.1cm}\end{solution}\stepcounter{numA}%%%%%%%%%%%%


{\raggedright\textsc{\textbf{Dissociations of acids \& bases}}\par}

%%%%%PROBLEM
\begin{question}[ID=\the\value{numA}]\SetQuestionProperties{section-title=\nameref{sec:units}}
Write down the formula of the conjugate bases:
\begin{inparaenum}[(a)]
\item  \ce{H2O} %( \ce{OH^-})
\item  \ce{HCl} %( \ce{Cl^-})
\item  \ce{HNO3} %( \ce{NO3^-})
\item  \ce{H2SO4} %( \ce{HSO4^-})
 \end{inparaenum}
\end{question}
\begin{solution}
\begin{inparaenum}[(a)]
\item  \ce{H2O}  ( \ce{OH^-})
\item  \ce{HCl}  ( \ce{Cl^-})
\item  \ce{HNO3}  ( \ce{NO3^-})
\item  \ce{H2SO4}  ( \ce{HSO4^-}) 
 \end{inparaenum}
\hspace{0.1cm}\end{solution}\stepcounter{numA}%%%%%%%%%%%%

%%%%%PROBLEM
\begin{question}[ID=\the\value{numA}]\SetQuestionProperties{section-title=\nameref{sec:units}}
Write down the formula of the conjugate bases:
\begin{inparaenum}[(a)]
\item  \ce{HSO4^-} %( \ce{SO4^{2-}})
\item  \ce{H2S} %( \ce{HS^{-}})
\item  \ce{HCOOH} %( \ce{HCOO^{-}})
\item  \ce{H2PO4^-} %( \ce{HPO4^{2-}})
 \end{inparaenum}
\end{question}
\begin{solution}
\begin{inparaenum}[(a)]
\item  \ce{HSO4^-}  ( \ce{SO4^{2-}})
\item  \ce{H2S}  ( \ce{HS^{-}})
\item  \ce{HCOOH}  ( \ce{HCOO^{-}})
\item  \ce{H2PO4^-}  ( \ce{HPO4^{2-}}) 
 \end{inparaenum}
\hspace{0.1cm}\end{solution}\stepcounter{numA}%%%%%%%%%%%%


%%%%%PROBLEM
\begin{question}[ID=\the\value{numA}]\SetQuestionProperties{section-title=\nameref{sec:units}}
Write down the formula of the conjugate acids:
\begin{inparaenum}[(a)]
\item  \ce{NH4^{+}} %(\ce{NH3})
 \item  \ce{CH3COO^{-}} %(\ce{CH3COOH})
 \item  \ce{HS^{-}} %(\ce{H2S})
 \item  \ce{CN^{-}} %(\ce{HCN})
 \end{inparaenum}
\end{question}
\begin{solution}
\begin{inparaenum}[(a)]
 \item  \ce{NH4^{+}}  (\ce{NH3})
 \item  \ce{CH3COO^{-}}  (\ce{CH3COOH})
 \item  \ce{HS^{-}}  (\ce{H2S})
 \item  \ce{CN^{-}}  (\ce{HCN})
 \end{inparaenum}
\hspace{0.1cm}\end{solution}\stepcounter{numA}%%%%%%%%%%%%


%%%%%%PROBLEM
%\begin{question}[ID=\the\value{numA}]\SetQuestionProperties{section-title=\nameref{sec:units}}
%Write down the formula of the conjugate acids:
%\begin{inparaenum}[(a)]
%\item  \ce{NH4^{+}} %(\ce{NH3})
% \item  \ce{CH3COO^{-}} %(\ce{CH3COOH})
% \item  \ce{HS^{-}} %(\ce{H2S})
% \item  \ce{CN^{-}} %(\ce{HCN})
% \end{inparaenum}
%\end{question}
%\begin{solution}
%\begin{inparaenum}[(a)]
% \item  \ce{NH4^{+}}  (\ce{NH3})
% \item  \ce{CH3COO^{-}}  (\ce{CH3COOH})
% \item  \ce{HS^{-}}  (\ce{H2S})
% \item  \ce{CN^{-}}  (\ce{HCN})
% \end{inparaenum}
%\hspace{0.1cm}\end{solution}\stepcounter{numA}%%%%%%%%%%%%


%%%%%PROBLEM
\begin{question}[ID=\the\value{numA}]\SetQuestionProperties{section-title=\nameref{sec:units}}
Identify the conjugate acid-base pairs:
\begin{enumerate}[label=(\alph*)]
\item  \ce{HCl_{(g)} + H2O_{(l)}  <=> Cl^-_{(aq)} + H3O^+_{(aq)} } %(\ce{HCl}  /  \ce{Cl^-} ) and (\ce{H2O}  /  \ce{H3O^+} )
\item  \ce{CH3COO^-_{(l)} + HCl_{(g)}  <=> Cl^-_{(aq)} + CH3COOH_{(aq)} } %(\ce{CH3COO^-}  /  \ce{CH3COOH} ) and (\ce{HCl}  /  \ce{Cl^-} )
\item  \ce{CO3^{2-}_{(aq)} + HCN_{(g)}  <=> CN^-_{(aq)} + HCO3^{-}_{(aq)} } %(\ce{CO3^{2-}}  /  \ce{CN^-} ) and (\ce{HCN}  /  \ce{HCO3^{-}} )
\item  \ce{HNO3_{(aq)} + OH^{-}_{(aq)}  <=> NO3^-_{(aq)} + H2O_{(aq)} } %(\ce{HNO3}  /  \ce{NO3^-} ) and (\ce{OH^{-}}  /  \ce{H2O} )
 \end{enumerate}
\end{question}
\begin{solution}
\begin{inparaenum}[(a)]
 \item   (\ce{HCl}  /  \ce{Cl^-} ) and (\ce{H2O}  /  \ce{H3O^+} )
\item  (\ce{CH3COO^-}  /  \ce{CH3COOH} ) and (\ce{HCl}  /  \ce{Cl^-} )
\item  (\ce{CO3^{2-}}  /  \ce{CN^-} ) and (\ce{HCN}  /  \ce{HCO3^{-}} )
\item   (\ce{HNO3}  /  \ce{NO3^-} ) and (\ce{OH^{-}}  /  \ce{H2O} )
 \end{inparaenum}
\hspace{0.1cm}\end{solution}\stepcounter{numA}%%%%%%%%%%%%

%%%%%PROBLEM
\begin{question}[ID=\the\value{numA}]\SetQuestionProperties{section-title=\nameref{sec:units}}
Write down the following dissociation or acid-base reaction involving the exchange of one proton:
\begin{enumerate}[label=(\alph*)]
\item  \ce{HNO3_{(l)} + H2O_{(l)}  <=>   } %(\ce{HNO3_{(l)} + H2O_{(l)}  <=> NO3^-_{(aq)} + H3O^+_{(aq)} })
\item  \ce{H2SO4_{(l)} + H2O_{(l)}  <=>   } %(\ce{H2SO4_{(l)} + H2O_{(l)}  <=> HSO4^-_{(aq)} + H3O^+_{(aq)} })
\item  \ce{HCl_{(g)} + H2O_{(l)}  <=>   } %(\ce{HCl_{(g)} + H2O_{(l)}  <=> Cl^-_{(aq)} + H3O^+_{(aq)} })
\item  \ce{NH3_{(g)} + H2O_{(l)}  <=>   } %(\ce{NH3_{(g)} + H2O_{(l)}  <=> NH4^+_{(aq)} + OH^-_{(aq)} })
 \end{enumerate}
\end{question}
\begin{solution}
\begin{inparaenum}[(a)]
\item   \ce{HNO3_{(l)} + H2O_{(l)}  <=> NO3^-_{(aq)} + H3O^+_{(aq)} } 
\item  \ce{H2SO4_{(l)} + H2O_{(l)}  <=> HSO4^-_{(aq)} + H3O^+_{(aq)} } 
\item  \ce{HCl_{(g)} + H2O_{(l)}  <=> Cl^-_{(aq)} + H3O^+_{(aq)} } 
\item   \ce{NH3_{(g)} + H2O_{(l)}  <=> NH4^+_{(aq)} + OH^-_{(aq)} } 
 \end{inparaenum}
\hspace{0.1cm}\end{solution}\stepcounter{numA}%%%%%%%%%%%%

%%%%%PROBLEM
\begin{question}[ID=\the\value{numA}]\SetQuestionProperties{section-title=\nameref{sec:units}}
Write down the following dissociation or acid-base reaction involving the exchange of one proton:
\begin{enumerate}[label=(\alph*)]
\item  \ce{CO3^{2-}_{(aq)} + HCN_{(g)}  <=>   } %(\ce{CO3^{2-}_{(aq)} + HCN_{(g)}  <=> HCO3^-_{(aq)} + CN^-_{(aq)} })
\item  \ce{HCO3^{-}_{(aq)} + HCN_{(g)}  <=>   } %(\ce{HCO3^{-}_{(aq)} + HCN_{(g)}  <=> H2CO3_{(aq)} + CN^-_{(aq)} })
\item  \ce{HCO3^{-}_{(aq)} + OH^-_{(aq)}  <=>   } %(\ce{HCO3^{-}_{(aq)} + OH^-_{(aq)} <=> CO3^{2-}_{(aq)} + H2O_{(l)} })
 \end{enumerate}
\end{question}
\begin{solution}
\begin{inparaenum}[(a)]
\item   \ce{CO3^{2-}_{(aq)} + HCN_{(g)}  <=> HCO3^-_{(aq)} + CN^-_{(aq)} } 
\item   \ce{HCO3^{-}_{(aq)} + HCN_{(g)}  <=> H2CO3_{(aq)} + CN^-_{(aq)} } 
\item   \ce{HCO3^{-}_{(aq)} + OH^-_{(aq)} <=> CO3^{2-}_{(aq)} + H2O_{(l)} } 
 \end{inparaenum}
\hspace{0.1cm}\end{solution}\stepcounter{numA}%%%%%%%%%%%%


%%%%%PROBLEM
\begin{question}[ID=\the\value{numA}]\SetQuestionProperties{section-title=\nameref{sec:units}}
From the following pairs, select the strongest acid:
\begin{enumerate}[label=(\alph*)]
\item  \ce{HIO3} ($K_a=1.6\cdot 10^{-1}$) or \ce{H2SO3} ($K_a=1.5\cdot 10^{-2}$) %(\ce{HIO3})
\item  \ce{HN3} ($K_a=1.9\cdot 10^{-5}$) or \ce{H2CO3} ($K_a=4.3\cdot 10^{-7}$)($K_a=1.5\cdot 10^{-2}$) %(\ce{H2CO3} )
 \end{enumerate}
\end{question}
\begin{solution}
\begin{inparaenum}[(a)]
 \item    \ce{HIO3} 
\item    \ce{H2CO3}  
 \end{inparaenum}
\hspace{0.1cm}\end{solution}\stepcounter{numA}%%%%%%%%%%%%

%%%%%PROBLEM
\begin{question}[ID=\the\value{numA}]\SetQuestionProperties{section-title=\nameref{sec:units}}
From the following pairs, select the strongest base:
\begin{enumerate}[label=(\alph*)]
\item  \ce{CN^-} ($K_a=6.2\cdot 10^{-10}$) or \ce{H2O} ($K_a=1.0\cdot 10^{-14}$) %(\ce{CN^-})
\item  \ce{H2C6H5O^-} ($K_a=1.8\cdot 10^{-5}$) or \ce{HCOOH} ($K_a=1.7\cdot 10^{-4}$) %(\ce{HCOOH} )
 \end{enumerate}
\end{question}
\begin{solution}
\begin{inparaenum}[(a)]
\item   \ce{CN^-} 
\item  \ce{HCOOH}   
 \end{inparaenum}
\hspace{0.1cm}\end{solution}\stepcounter{numA}%%%%%%%%%%%%
 







{\raggedright\textsc{\textbf{The PH scale}}\par}


%%%%%PROBLEM
\begin{question}[ID=\the\value{numA}]\SetQuestionProperties{section-title=\nameref{sec:units}}
Answer the following questions:
\begin{inparaenum}[(a)]
\item The proton concentration of a solution is $3\times 10^{-3}$M. Calculate the hydroxyl concentration in the same solution %$3.3\times 10^{-12}$M
\item The hydroxyl concentration of a solution is $8\times 10^{-6}$M. Calculate the proton concentration in the same solution.%$1.25\times 10^{-9}$M
 \end{inparaenum}
\end{question}
\begin{solution}
\begin{inparaenum}[(a)]
\item  $3.3\times 10^{-12}$M
\item  $1.25\times 10^{-9}$M
 \end{inparaenum}
\hspace{0.1cm}\end{solution}\stepcounter{numA}%%%%%%%%%%%%


%%%%%PROBLEM
\begin{question}[ID=\the\value{numA}]\SetQuestionProperties{section-title=\nameref{sec:units}}
Answer the following questions:
\begin{inparaenum}[(a)]
\item  The PH of a solution is 1.34. Calculate the POH of the same solution %12.66
\item  The POH of a solution is 12. Calculate the PH of the same solution.%12
 \end{inparaenum}
\end{question}
\begin{solution}
\begin{inparaenum}[(a)]
\item   12.66
\item   12
 \end{inparaenum}
\hspace{0.1cm}\end{solution}\stepcounter{numA}%%%%%%%%%%%%

%%%%%PROBLEM
\begin{question}[ID=\the\value{numA}]\SetQuestionProperties{section-title=\nameref{sec:units}}
Answer the following questions:
\begin{inparaenum}[(a)]
\item  The PH of a solution is 1.56. Calculate the concentration of protons%0.027M
\item  The POH of a solution is 10.34. Calculate the hydroxyl concentration in the same solution%$4.57\times 10^{-11}$M
\item  The PH of a solution is 12.4. Calculate the hydroxyl concentration in the same solution.%$2.5\times 10^{-2}$M
 \end{inparaenum}
\end{question}
\begin{solution}
\begin{inparaenum}[(a)]
\item   0.027M
\item   $4.57\times 10^{-11}$M
\item   $2.5\times 10^{-2}$M
 \end{inparaenum}
\hspace{0.1cm}\end{solution}\stepcounter{numA}%%%%%%%%%%%%


%%%%%PROBLEM
\begin{question}[ID=\the\value{numA}]\SetQuestionProperties{section-title=\nameref{sec:units}}
Fill the table below:
\begin{center}\begin{tabularx}{0.9\columnwidth}{>{ \arraybackslash}p{5em}>{ \arraybackslash}p{5em}>{\centering\arraybackslash}p{2em}   }
  \toprule
\heading{$\big[ \ce{H^+} \big]$} & \heading{$\big[ \ce{OH^-} \big]$}  &  Acidic/Basic/Neutral?      \\
    \midrule
 { \small $1.5\times 10^{-4}$}	&   { \small $6.6\times 10^{-11}$}		&	 	       \\
 { \small $4.9\times 10^{-12}$}	&   { \small $2.0\times 10^{-3}$}		&	 	       \\
 { \small $1.9\times 10^{-6}$}	&   { \small $5.3\times 10^{-9}$}		&	 	       \\
 { \small $1.0\times 10^{-7}$}	&   { \small $1.0\times 10^{-7}$}		&	 	       \\
    \bottomrule
\end{tabularx}\end{center}
\end{question}
\begin{solution}
\begin{center}\begin{tabularx}{0.9\columnwidth}{>{ \arraybackslash}p{5em}>{ \arraybackslash}p{5em}>{\centering\arraybackslash}p{2em}   }
  \toprule
\heading{$\big[ \ce{H^+} \big]$} & \heading{$\big[ \ce{OH^-} \big]$}  &  Acidic/Basic/Neutral?      \\
    \midrule
 { \small $1.5\times 10^{-4}$}	&   { \small $6.6\times 10^{-11}$}		&Acidic	 	       \\
 { \small $4.9\times 10^{-12}$}	&   { \small $2.0\times 10^{-3}$}		&	 basic	       \\
 { \small $1.9\times 10^{-6}$}	&   { \small $5.3\times 10^{-9}$}		&	Acidic 	       \\
 { \small $1.0\times 10^{-7}$}	&   { \small $1.0\times 10^{-7}$}		&	 neutral	       \\
	    \bottomrule
\end{tabularx}\end{center}
\hspace{0.1cm}\end{solution}\stepcounter{numA}%%%%%%%%%%%%



%%%%%PROBLEM
\begin{question}[ID=\the\value{numA}]\SetQuestionProperties{section-title=\nameref{sec:units}}
Fill the table below:
\begin{center}\begin{tabularx}{0.9\columnwidth}{>{ \arraybackslash}p{5em}>{ \arraybackslash}p{5em}>{\centering\arraybackslash}p{2em}>{\centering\arraybackslash}p{2em}  }
  \toprule
\heading{$\big[ \ce{H^+} \big]$} & \heading{$\big[ \ce{OH^-} \big]$}  &  \heading{PH} & \heading{POH}     \\
    \midrule
	 { \small $4.5\times 10^{-3}$}	&  	&	-- 	&	--       \\
	 	& { \small $3.2\times 10^{-7}$}		&	-- 	&	--      \\
	  	--&  	--	&	{ \small 5.1}	&	        \\
	  	--&  	--	&		&	{ \small 6.9}       \\
    \bottomrule
\end{tabularx}\end{center}
\end{question}
\begin{solution}
\begin{center}\begin{tabularx}{0.9\columnwidth}{>{ \arraybackslash}p{5em}>{ \arraybackslash}p{5em}>{\centering\arraybackslash}p{2em}>{\centering\arraybackslash}p{2em}  }
  \toprule
\heading{$\big[ \ce{H^+} \big]$} & \heading{$\big[ \ce{OH^-} \big]$}  &  \heading{PH} & \heading{POH}     \\
    \midrule
% { \small $4.5\times 10^{-3}$}	&		&	--	&	-- &     \\
%   	&	{ \small $3.4\times 10^{-1}$}	&	--	&	-- &     \\
	 { \small $4.5\times 10^{-3}$}	& { \small $2.2\times 10^{-12}$}		&	{ \small 2.3}	&	{ \small 11.6}       \\
	 { \small $3.1\times 10^{-8}$}	& { \small $3.2\times 10^{-7}$}		&	{ \small 7.5}	&	{ \small 6.5}       \\
	 { \small $7.1\times 10^{-6}$}	& { \small $1.4\times 10^{-9}$}		&	{ \small 5.1}	&	{ \small 8.8}       \\
	 { \small $9.0\times 10^{-8}$}	& { \small $1.1\times 10^{-7}$}		&	{ \small 7.1}	&	{ \small 6.9}       \\
    \bottomrule
\end{tabularx}\end{center}
\hspace{0.1cm}\end{solution}\stepcounter{numA}%%%%%%%%%%%%




%%%%%PROBLEM
\begin{question}[ID=\the\value{numA}]\SetQuestionProperties{section-title=\nameref{sec:units}}
Fill the table below:
\begin{center}\begin{tabularx}{0.9\columnwidth}{>{ \arraybackslash}p{5em}>{ \arraybackslash}p{5em}>{\centering\arraybackslash}p{2em}>{\centering\arraybackslash}p{2em}  }
  \toprule
\heading{$\big[ \ce{H^+} \big]$} & \heading{$\big[ \ce{OH^-} \big]$}  &  \heading{PH} & \heading{POH}     \\
    \midrule
 { \small $3.5\times 10^{-1}$}	& --		&	 -- 	&	     \\
	 --	& --	&	--	&	{ \small 2}       \\

    \bottomrule
\end{tabularx}\end{center}
\end{question}
\begin{solution}
\begin{center}\begin{tabularx}{0.9\columnwidth}{>{ \arraybackslash}p{5em}>{ \arraybackslash}p{5em}>{\centering\arraybackslash}p{2em}>{\centering\arraybackslash}p{2em}  }
  \toprule
\heading{$\big[ \ce{H^+} \big]$} & \heading{$\big[ \ce{OH^-} \big]$}  &  \heading{PH} & \heading{POH}     \\
    \midrule
% { \small $4.5\times 10^{-3}$}	&		&	--	&	-- &     \\
%   	&	{ \small $3.4\times 10^{-1}$}	&	--	&	-- &     \\
	 { \small $3.5\times 10^{-1}$}	& { \small $2.8\times 10^{-14}$}		&	{ \small 0.4}	&	{ \small 13.5}       \\
	 { \small $1\times 10^{-12}$}	& { \small $1.0\times 10^{-2}$}		&	{ \small 12}	&	{ \small 2}       \\
	    \bottomrule
\end{tabularx}\end{center}
\hspace{0.1cm}\end{solution}\stepcounter{numA}%%%%%%%%%%%%


%%%%%PROBLEM
\begin{question}[ID=\the\value{numA}]\SetQuestionProperties{section-title=\nameref{sec:units}}
Answer the following questions:
\begin{inparaenum}[(a)]
\item Calculate the PH of a 0.34M HCl solution %0.46
\item Calculate the PH of a 0.04M \ce{HNO3} solution %1.39
\item Calculate the PH of a 0.08M \ce{NaOH} solution %12.9
 \end{inparaenum}
\end{question}
\begin{solution}
\begin{inparaenum}[(a)]
\item  0.46
\item  1.39
\item  12.9

 \end{inparaenum}
\hspace{0.1cm}\end{solution}\stepcounter{numA}%%%%%%%%%%%%


%%%%%PROBLEM
\begin{question}[ID=\the\value{numA}]\SetQuestionProperties{section-title=\nameref{sec:units}}
Answer the following questions:
\begin{inparaenum}[(a)]
\item Calculate the PH of a 0.27M \ce{H2SO4} solution %0.27
\item Calculate the PH of a 0.03M \ce{Ba(OH)2} solution %12.77
 \end{inparaenum}
\end{question}
\begin{solution}
\begin{inparaenum}[(a)]
\item  0.27
\item  12.77
 \end{inparaenum}
\hspace{0.1cm}\end{solution}\stepcounter{numA}%%%%%%%%%%%%

					%%%%%PROBLEM
\begin{question}[ID=\the\value{numA}]\SetQuestionProperties{section-title=\nameref{sec:units}}
Solve the following quadratic equations and select the positive root:
\begin{enumerate}[label=(\alph*)]
\item  $x^2 +5.5 \times 10^{-11}\cdot x -5.5 \times 10^{-12}=0$  % $2.3\times 10^{-6}$
\item  $x^2 +6.6 \times 10^{-5}\cdot x -1.32 \times 10^{-5}=0$  % $3.6\times 10^{-3}$
 \end{enumerate}
\end{question}
\begin{solution}
\begin{inparaenum}[(a)]
\item    $2.3\times 10^{-6}$
\item    $3.6\times 10^{-3}$
 \end{inparaenum}
\hspace{0.1cm}\end{solution}\stepcounter{numA}%%%%%%%%%%%%	


%%%%%PROBLEM
\begin{question}[ID=\the\value{numA}]\SetQuestionProperties{section-title=\nameref{sec:units}}
Answer the following questions:
\begin{inparaenum}[(a)]
\item Calculate the PH of a 0.23M \ce{C6H5COOH} (benzoic acid, $K_a=6.25 \times 10^{-5}$) solution %2.42
\item Calculate the PH of a 0.08M \ce{C6H5NH2}	 (Aniline, $K_b=7.40 \times 10^{-10}$) solution %8.8
 \end{inparaenum}
\end{question}
\begin{solution}
\begin{inparaenum}[(a)]
\item Calculate the PH of a 0.23M \ce{C6H5COOH} (benzoic acid, $K_a=6.25 \times 10^{-5}$) solution %2.42
\item Calculate the PH of a 0.08M \ce{C6H5NH2}	 (Aniline, $K_b=7.40 \times 10^{-10}$) solution %8.8
 \end{inparaenum}
\hspace{0.1cm}\end{solution}\stepcounter{numA}%%%%%%%%%%%%

		%%%%%PROBLEM
\begin{question}[ID=\the\value{numA}]\SetQuestionProperties{section-title=\nameref{sec:units}}
Answer the following questions:
\begin{inparaenum}[(a)]
\item Calculate the PH of a 0.05M \ce{C6H5OH} (phenol, $K_a=1.00 \times 10^{-10}$) solution %5.65
\item Calculate the PH of a 0.09M \ce{C5H5N}	 (pyridine, $K_b=1.70 \times 10^{-9}$) solution %9.1
 \end{inparaenum}
\end{question}
\begin{solution}
\begin{inparaenum}[(a)]
\item Calculate the PH of a 0.05M \ce{C6H5OH} (phenol, $K_a=1.00 \times 10^{-10}$) solution %5.65
\item Calculate the PH of a 0.09M \ce{C5H5N}	 (pyridine, $K_b=1.70 \times 10^{-9}$) solution %9.1
 \end{inparaenum}
\hspace{0.1cm}\end{solution}\stepcounter{numA}%%%%%%%%%%%%		


			
{\raggedright\textsc{\textbf{Buffer solutions}}\par}


%%%%%PROBLEM
\begin{question}[ID=\the\value{numA}]\SetQuestionProperties{section-title=\nameref{sec:units}}
Which of the following mixtures of solutions can act as a buffer:
\begin{enumerate}[label=(\alph*)]
\item 0.1M-\ce{H2SO4}/0.1M-\ce{Na(SO4)2} % (No buffer)
\item 0.01M-\ce{NH3}/0.01M-\ce{NH4Cl} % (buffer)
\item 0.2M-\ce{HNO2}/0.2M-\ce{NaNO2} % (buffer)
 \end{enumerate}
\end{question}
\begin{solution}
\begin{inparaenum}[(a)]
\item 0.1M-\ce{H2SO4}/0.1M-\ce{Na(SO4)2}   (No buffer)
\item 0.01M-\ce{NH3}/0.01M-\ce{NH4Cl}   (buffer)
\item 0.2M-\ce{HNO2}/0.2M-\ce{NaNO2}   (buffer)
 \end{inparaenum}
\hspace{0.1cm}\end{solution}\stepcounter{numA}%%%%%%%%%%%%


%%%%%PROBLEM
\begin{question}[ID=\the\value{numA}]\SetQuestionProperties{section-title=\nameref{sec:units}}
Which of the following mixtures of solutions can act as a buffer:
\begin{enumerate}[label=(\alph*)]
\item 0.4M-\ce{H2SO4}/0.1M-\ce{NaHSO4} % (No buffer)
\item 0.23M-\ce{HCl}/0.20M-\ce{KCl} % (No buffer)
\item 0.56M-\ce{HCN}/0.22M-\ce{NaCN} % ( buffer)
 \end{enumerate}
\end{question}
\begin{solution}
\begin{inparaenum}[(a)]
\item 0.4M-\ce{H2SO4}/0.1M-\ce{NaHSO4}   (No buffer)
\item 0.23M-\ce{HCl}/0.20M-\ce{KCl}   (No buffer)
\item 0.56M-\ce{HCN}/0.22M-\ce{NaCN}   ( buffer)
 \end{inparaenum}
\hspace{0.1cm}\end{solution}\stepcounter{numA}%%%%%%%%%%%%

%%%%%PROBLEM
\begin{question}[ID=\the\value{numA}]\SetQuestionProperties{section-title=\nameref{sec:units}}
Calculate the PH of the following buffers:
\begin{enumerate}[label=(\alph*)]
\item 0.15 M-\ce{HCN}/0.35M-\ce{NaCN}, $K_a=6.20 \times 10^{-10}$ % (PH=9.57)
\item 0.15 M-\ce{HCN}/0.15M-\ce{NaCN}, $K_a=6.20 \times 10^{-10}$ % (PH=9.21)
 \item 0.25 M-\ce{HCN}/0.15M-\ce{NaCN}, $K_a=6.20 \times 10^{-10}$ % (PH=9.98)
 \end{enumerate}
\end{question}
\begin{solution}
\begin{inparaenum}[(a)]
\item 0.15 M-\ce{HCN}/0.35M-\ce{NaCN}, $K_a=6.20 \times 10^{-10}$   (PH=9.57)
\item 0.15 M-\ce{HCN}/0.15M-\ce{NaCN}, $K_a=6.20 \times 10^{-10}$   (PH=9.21)
 \item 0.25 M-\ce{HCN}/0.15M-\ce{NaCN}, $K_a=6.20 \times 10^{-10}$   (PH=9.98)
 \end{inparaenum}
\hspace{0.1cm}\end{solution}\stepcounter{numA}%%%%%%%%%%%%

%%%%%PROBLEM
\begin{question}[ID=\the\value{numA}]\SetQuestionProperties{section-title=\nameref{sec:units}}
Calculate the PH of the following buffers:
\begin{enumerate}[label=(\alph*)]
\item 0.25 M-\ce{NH3}/0.45M-\ce{NH4Cl}, $K_b=1.80 \times 10^{-5}$ % (PH=9)
\item 0.15 M-\ce{HNO2}/0.05M-\ce{NaNO2}, $K_a=5.60 \times 10^{-4}$ % (PH=2.8)
 \end{enumerate}
\end{question}
\begin{solution}
\begin{inparaenum}[(a)]
\item 0.25 M-\ce{NH3}/0.45M-\ce{NH4Cl}, $K_b=1.80 \times 10^{-5}$  (PH=9)
\item 0.15 M-\ce{HNO2}/0.05M-\ce{NaNO2}, $K_a=5.60 \times 10^{-4}$   (PH=2.8)
 \end{inparaenum}
\hspace{0.1cm}\end{solution}\stepcounter{numA}%%%%%%%%%%%%

%%%%%PROBLEM
\begin{question}[ID=\the\value{numA}]\SetQuestionProperties{section-title=\nameref{sec:units}}
Calculate the PH of the following buffers:
\begin{enumerate}[label=(\alph*)]
\item 0.02 M-\ce{C6H5NH2}/0.05M-\ce{C6H5NHCl}, $K_b=7.40 \times 10^{-10}$ % (PH=4.47)
\item 0.4 M-\ce{HCNO}/0.5M-\ce{NaCNO}, $K_a=3.50 \times 10^{-4}$ % (PH=3.5)
 \end{enumerate}
\end{question}
\begin{solution}
\begin{inparaenum}[(a)]
\item 0.02 M-\ce{C6H5NH2}/0.05M-\ce{C6H5NHCl}, $K_b=7.40 \times 10^{-10}$   (PH=4.47)
\item 0.4 M-\ce{HCNO}/0.5M-\ce{NaCNO}, $K_a=3.50 \times 10^{-4}$   (PH=3.5)
 \end{inparaenum}
\hspace{0.1cm}\end{solution}\stepcounter{numA}%%%%%%%%%%%%


%%%%%PROBLEM
\begin{question}[ID=\the\value{numA}]\SetQuestionProperties{section-title=\nameref{sec:units}}
The PH of a  \ce{C6H5NH2}/\ce{C6H5NHCl} ($K_b=7.40 \times 10^{-10}$) buffer is 4.0. Calculate the ratio of [\ce{C6H5NH2}]/[\ce{C6H5NHCl}].
\end{question}
\begin{solution}
0.13
\hspace{0.1cm}\end{solution}\stepcounter{numA}%%%%%%%%%%%%


%%%%%PROBLEM
\begin{question}[ID=\the\value{numA}]\SetQuestionProperties{section-title=\nameref{sec:units}}
Calculate the PH for the following scenarios:
\begin{inparaenum}[(a)]
\item A 0.3 M-\ce{HCNO}/0.4M-\ce{NaCNO}, $K_a=3.50 \times 10^{-4}$ buffer  %(PH=3.58)
\item A 0.3 M-\ce{HCNO}/0.4M-\ce{NaCNO}, $K_a=3.50 \times 10^{-4}$ buffer after adding 3mL of 0.1HCl into 5mL of the buffer %(PH=3.43)
\item A 0.3 M-\ce{HCNO}/0.4M-\ce{NaCNO}, $K_a=3.50 \times 10^{-4}$ buffer after adding 3mL of 0.1NaOH into 5mL of the buffer %(PH=3.73)
\item A 0.3 M-\ce{HCNO}/0.4M-\ce{NaCNO}, $K_a=3.50 \times 10^{-4}$ buffer after adding 1mL of 0.1NaOH into 5mL of the buffer %(PH=3.63)
 \end{inparaenum}
\end{question}
\begin{solution}
\begin{inparaenum}[(a)]
\item  PH=3.58 
\item  PH=3.43 
\item  PH=3.73 
\item  PH=3.63 
 \end{inparaenum}
\hspace{0.1cm}\end{solution}\stepcounter{numA}%%%%%%%%%%%%




%%%%%PROBLEM
\begin{question}[ID=\the\value{numA}]\SetQuestionProperties{section-title=\nameref{sec:units}}
Calculate the PH for the following scenarios:
\begin{inparaenum}[(a)]
\item A 0.8 M-\ce{HF}/0.1M-\ce{NaF}, $K_a=6.30 \times 10^{-4}$ buffer  %(PH=2.30 )
\item A 0.8 M-\ce{HF}/0.1M-\ce{NaF}, $K_a=6.30 \times 10^{-4}$ buffer after adding 1mL of 0.1HCl into 2mL of the buffer %(PH=1.97)
\item A 0.8 M-\ce{HF}/0.1M-\ce{NaF}, $K_a=6.30 \times 10^{-4}$ buffer after adding 1mL of 0.1NaOH into 2mL of the buffer %(PH=2.50)
\item Is the buffer more resistant to acids or bases, and why?
 \end{inparaenum}
\end{question}
\begin{solution}
\begin{inparaenum}[(a)]
\item  PH=2.30  
\item PH=1.97 
\item  PH=2.50 
\item bases
 \end{inparaenum}
\hspace{0.1cm}\end{solution}\stepcounter{numA}%%%%%%%%%%%%

%%%%%PROBLEM
\begin{question}[ID=\the\value{numA}]\SetQuestionProperties{section-title=\nameref{sec:units}}
Calculate the PH for the following scenarios:
\begin{inparaenum}[(a)]
\item A 0.1 M-\ce{HF}/0.9M-\ce{NaF}, $K_a=6.30 \times 10^{-4}$ buffer  %(PH=4.15 )
\item A 0.1 M-\ce{HF}/0.9M-\ce{NaF}, $K_a=6.30 \times 10^{-4}$ buffer after adding 1mL of 0.1HCl into 2mL of the buffer %(PH=3.95)
\item A 0.1 M-\ce{HF}/0.9M-\ce{NaF}, $K_a=6.30 \times 10^{-4}$ buffer after adding 1mL of 0.1NaOH into 2mL of the buffer %(PH=4.48)
\item Is the buffer more resistant to acids or bases, and why?
 \end{inparaenum}
\end{question}
\begin{solution}
\begin{inparaenum}[(a)]
\item  PH=4.15 
\item  PH=3.95 
\item  PH=4.48 
\item acids
 \end{inparaenum}
\hspace{0.1cm}\end{solution}\stepcounter{numA}%%%%%%%%%%%%


{\raggedright\textsc{\textbf{Titrations}}\par}
%%%%%PROBLEM
\begin{question}[ID=\the\value{numA}]\SetQuestionProperties{section-title=\nameref{sec:units}}
Solve the following titration scenarios:
\begin{inparaenum}[(a)]
\item In a titration experiment, 13.5 mL of 0.34 M \ce{HCl} neutralize 34.3 mL of KOH. What is the concentration of the \ce{KOH} solution? %0.13M
\item In a titration experiment, 20.4 mL of 0.10 M \ce{HCl} neutralize 12.4 mL of \ce{Ca(OH)2}. What is the concentration of the base solution? %0.08M
\item In a titration experiment, 10.4 mL of 0.20 M \ce{H2SO4} neutralize 8.4 mL of \ce{Ca(OH)2}. What is the concentration of the base solution? %0.24M
 \end{inparaenum}
\end{question}
\begin{solution}
\begin{inparaenum}[(a)]
\item  0.13M
\item  0.08M
\item  0.24M
 \end{inparaenum}
\hspace{0.1cm}\end{solution}\stepcounter{numA}%%%%%%%%%%%%


%%%%%PROBLEM
\begin{question}[ID=\the\value{numA}]\SetQuestionProperties{section-title=\nameref{sec:units}}
A 15 mL solution of 0.2 M \ce{HCl} is titrated with a 0.3 M KOH solution. Calculate the pH after the following additions of the base solution: 
\begin{inparaenum}[(a)]
\item  0mL  	%PH=0.70
\item  5mL		%PH=1.12
\item  10mL	%PH=7
\item  15mL	%PH=12.70
 \end{inparaenum}
\end{question}
\begin{solution}
\begin{inparaenum}[(a)]
 \item  PH=0.70
\item   PH=1.12
\item   PH=7
\item   PH=12.70
 \end{inparaenum}
\hspace{0.1cm}\end{solution}\stepcounter{numA}%%%%%%%%%%%%



%%%%%PROBLEM
\begin{question}[ID=\the\value{numA}]\SetQuestionProperties{section-title=\nameref{sec:units}}
A 25 mL solution of 0.1 M \ce{HNO2} ($K_a=5.60 \times 10^{-4}$) is titrated with a 0.05 M KOH solution. Calculate the pH after the following additions of the base solution: 
\begin{inparaenum}[(a)]
\item  0mL  	%PH=1.00
\item  25mL	%PH=3.25 
\item  50mL	%PH=11.61 
\item  65mL	%PH= 11.92
 \end{inparaenum}
\end{question}
\begin{solution}
\begin{inparaenum}[(a)]
\item  0mL  	%PH=1.00
\item  25mL	%PH=3.25 
\item  50mL	%PH=11.61 
\item  65mL	%PH= 11.92
 \end{inparaenum}
\hspace{0.1cm}\end{solution}\stepcounter{numA}%%%%%%%%%%%%

%%%%%PROBLEM
\begin{question}[ID=\the\value{numA}]\SetQuestionProperties{section-title=\nameref{sec:units}}
A 20 mL solution of 0.1 M \ce{KOH} is titrated with a 0.5 M HCl solution. Calculate the pH after the following additions of the acid solution: 
\begin{inparaenum}[(a)]
\item  0mL  	%PH=13
\item  2mL		%PH=12.65
\item  4mL		%PH=7
\item  5mL		%PH=1.69
 \end{inparaenum}
\end{question}
\begin{solution}
\begin{inparaenum}[(a)]
\item   PH=13
\item   PH=12.65
\item   PH=7
\item   PH=1.69
 \end{inparaenum}
\hspace{0.1cm}\end{solution}\stepcounter{numA}%%%%%%%%%%%%

%%%%%PROBLEM
\begin{question}[ID=\the\value{numA}]\SetQuestionProperties{section-title=\nameref{sec:units}}
A 10 mL solution of 0.2 M \ce{NH3} ($K_b=2.50 \times 10^{-5}$) is titrated with a 0.1 M HCl solution. Calculate the pH after the following additions of the base solution: 
\begin{inparaenum}[(a)]
\item  0mL  	%PH=11.34 
\item  10mL	%PH=9.40  
\item  20mL	%PH=2.89  
\item  30mL	%PH=1.60  
 \end{inparaenum}
\end{question}
\begin{solution}
\begin{inparaenum}[(a)]
 \item  PH=11.34 
\item   PH=9.40  
\item   PH=2.89  
\item   PH=1.60 
 \end{inparaenum}
\hspace{0.1cm}\end{solution}\stepcounter{numA}%%%%%%%%%%%%


\end{multicols*}

\newpage
\begin{answersenvironment}
\begin{minipage}[c]{1\textwidth}
\begin{localsize}{10}
{\Large \bf Answers}
\SetupExSheets{
  headings = inline-nr , % numbered and inline
  counter-format = qu) , % numbers 1) 2) ... 
}
%\printsolutions 
  \printsolutions[byID={1,3,5,7,9,11,13,15,17,19,21,23,25,27,29,31,33,35,37}]
\end{localsize}
\end{minipage}\end{answersenvironment}
\end{document}

