\documentclass[main.tex]{subfiles}
\begin{document}
\pagestyle{empty}

\newgeometry{a4paper,left=1in,right=1in,top=1in,bottom=1in,nohead}

%%%%%%%%%%%%HEADING TITLE
\hfill
\vspace{0.2cm}
\begin{center}
{\large \bfseries 
EXPERIMENT 
\par
\Huge
Titration of a weak acid
\\[5pt] \par}
\vspace{0.2cm}
\end{center}
\par

%\uline{  \hfill \normalsize \hfill       }
%%%%%%%%%%%%HEADING
\vspace{0.2cm}{\large \bfseries Goal}
Weak acids are acids that do not dissociate completely, releasing only some of its hydrogen atoms into the solution. Acetic acid (\ce{CH3COOH})  is a very important weak acid, produced from the fermentation ethanol from the wine. Commercial acetic acid--what you know as vinegar--is just an aqueous solution of acetic acid. The goal of this experiment is to calculate the \textit{molar concentration} of a sample of acetic acid by means of a standard chemical procedure known as \begin{it}titration\end{it}. In order to do that you will react the weak acid with a basic solution of sodium hydroxide (\ce{NaOH}), which has a known concentration. You will also use phenolphthalein as \begin{it}indicator\end{it}. 



\vspace{0.2cm}{\large \bfseries Background}
A titration is a technique where a solution of known concentration--often times a base--is used to determine the unknown concentration of another solution--often times an acid. Both substances react with each other in an acid-base reaction. The solution of known concentration is delivered carefully from a buret until an indicator--a third substance added to indicate the end of the titration--changes color. 
We can express the composition of a solution as the \textit{mass percentage} (\%mass) of each component: solute and solvent.
\[\%mass=\frac{\text{mass of solute}}{\text{mass of solute + mass of solvent}}\times 100.\% \]
For example, if we dissolve 15 g of NaCl in 60. g of water, the total mass of the solution --solute plus solvent-- is 75 g and the mass percentage of NaCl in the solution is $(15\:g/75\:g) \cdot 100\%= 20.\%$ NaCl. 
In chemistry, the molar concentration, $c$, of a solute in a solution, the \begin{it}molarity\end{it} of the solute, is the moles of solute $n$ present in a given volume, $V$, of the solution in liters. The units of molarity are moles per liter ($mol\cdot L^{-1}$), and it is denoted as $M$.
\[M=\frac{\text{moles of solute}}{\text{L of solution}}=\frac{n}{V}\]
Density, although not a measurement of concentration, is a property of liquid that we can use to convert mass into volume or volume into mass. The formula for density $d$ is:
\[d=\frac{\text{grams of solution}}{\text{mL of solution}}\]
\begin{center}\begin{example}{Example}
\begin{it}
What is the molarity of a sodium hydroxide solution (\ce{NaOH}, $M_W$=39.997 $g\cdot mol^{-1}$) prepared by dissolving 15.00 g of the solute in enough water to make 350.0 mL of solution?
\end{it}
\Sepline
\begin{bf}Answer\end{bf}: the molecular mass of \ce{NaOH} is $39.997\:g\cdot mol^{-1}$ and the number of \ce{NaOH} moles are:
\[n_{solute}=15.00\:g\times \frac{1\:mol}{39.997\:g}=0.3750\:mol\]
Do not forget to convert the volume from mL to L.
\[350.0\:mL \times \frac{1\:L}{1000\:mL}=0.3500\:L\]
The molarity will be:
\[M=\frac{0.3750\:mol}{0.3500\:L}=1.071\:M\]
\end{example}\end{center}

\vspace{0.2cm}{\large \bfseries Volumetric analysis} The determination of concentration by measuring volumes is called \textit{volumetric analysis}. Titrations are volumetric analyses where a buret is used to add and measure the volume of one of the reactants. Acid-base titrations are extensively used chemical techniques employed to determine solute concentration in a solution. In a \textit{acid-base titration}, an acid reacts with a base by gradually adding one solution to the other. The volume of the second solution is known, and the volume of the first solution required for the complete reaction is measured. The formula to use in a titration is:
\[ c_a\cdot V_a= c_b\cdot V_b,\]
$c_a$ and $V_a$ are the concentration of the acid and the volume of acid employed, and  $c_b$ and $V_b$ are the concentration of the base and the volume of base employed. 
\begin{center}\begin{example}{Example}
\begin{it}A 25 mL solution of perchloric acid, \ce{HClO4} --which has two acidic protons-- is titrated with \ce{NaOH} 0.10 M. The end point for the reaction is reached after 40. mL of the \ce{NaOH} solution are added. Find the molarity of the acid solution.\end{it}
\Sepline
\begin{bf}Answer\end{bf}: the balanced equation for the acid-base reaction is:\\
\begin{center}\ce{HClO4(aq) + NaOH(aq) -> H2O +NaClO4  }\end{center}
The molarity of the acid, $c_a$, is unknown whereas the volume used, $V_a=25\:mL$, is given.\\
The base concentration, $c_b=0.10\:M$, and volume, $V_b=40.\:mL$, are given:
\[c_a=\frac{ c_b\cdot V_b}{ V_a}\]
\[c_a=\frac{ 0.10\:M\cdot 40.\:mL}{25\:mL}=0.080\:M\]
\end{example}\end{center}


An \textit{indicator} is used to indicate the exact end of the reaction.  The indicator chosen will have one color before the reaction is complete and a different color when the acid-base reaction finishes. 
For example in the reaction between acetic acid (\ce{CH3COOH}) and sodium hydroxide (\ce{NaOH}):
\begin{center}\ce{CH3COOH + NaOH -> CH3COONa + H2O}\end{center}
using phenolphtalein as the indicator, the solution will be colorless before the competition of this reaction but pink after completion. At a specific point during the titration, a single drop of the NaOH from the buret will cause the solution being titrated to turn from colorless to a barely discernible pink color. This point is called the \textit{endpoint}. 

\restoregeometry





%%%%%%%%%%%%HEADING
\begin{fullwidth}


\begin{multicols}{2}
\begin{tcolorbox}[enhanced jigsaw,breakable,size=title,
colback=mybrown!05,colframe=black,fonttitle=\bfseries,
title=STUDENT INFO,pad at break=1mm, break at=15cm/0pt ]
\vspace{0.2cm}
\noindent Name: \rule{5cm}{0.4pt}Date:\rule{1cm}{0.4pt}\\
Pre-lab Done: \tikzcheckmark[scale=2,black]{no mark}\quad
\end{tcolorbox}
\end{multicols}
\hfill
\vspace{0.2cm}
\begin{center}
{\large \bfseries 
Pre-lab Questions 
\par
\Huge
Acid-Base titration
\\[5pt] \par}
\vspace{0.2cm}
\end{center}
\par
\noindent
\uline{  \hfill \normalsize \hfill       }
%%%%%%%%%%%%HEADING

\begin{enumerate}
% PELAB 1
% PELAB 1
\item A 10.00 mL sample of aqueous HCl requires 31.00 mL of 0.0900 M NaOH to reach the endpoint. What is the molar concentration of HCl. The equation for the reaction is:\\
\begin{center}\ce{HCl + NaOH -> NaCl + H2O}\end{center}\vspace{3cm}
\item The molarity of a vinegar solution is 0.90 M. Calculate the number of acetic acid moles in 10. mL of this solution.\vspace{4cm}

\item  Nitric acid (\ce{HPO3}) is an acid with three protons. Suppose you titrate 5.00 mL of of this acid with \ce{NaOH} 0.10 M. Knowing that the end point is reached after 25.00 mL of the base is added, find the molarity of the acid solution.\vspace{4cm}




\end{enumerate}


\clearpage\thispagestyle{empty}\mbox{}\clearpage



%%%%%%%%%%%%HEADING
\begin{multicols}{2}
\begin{tcolorbox}[enhanced jigsaw,breakable,size=title,
colback=mybrown!05,colframe=black,fonttitle=\bfseries,
title=STUDENT INFO,pad at break=1mm, break at=15cm/0pt ]
\vspace{0.2cm}
\noindent Name: \rule{5cm}{0.4pt}Date:\rule{1cm}{0.4pt}\\
Pre-lab Done: \tikzcheckmark[scale=2,black]{no mark}\quad
\end{tcolorbox}
\end{multicols}
\hfill
\vspace{0.2cm}
\begin{center}
{\large \bfseries 
Experiment
\par
\Huge
Acid-Base titration
\\[5pt] \par}
\vspace{0.2cm}
\end{center}
\par
\noindent
\uline{  \hfill \normalsize \hfill       }
%%%%%%%%%%%%HEADING

\vspace{0.2cm}{\large \bfseries 1. Acetic acid titration}

\begin{steps}
      \newstep[] Obtain a 5 mL pipet and a 50 mL buret with a stand and buret clamp.
    \newstep[] Obtain about 30 mL of acetic acid solution in a 50 mL beaker and about 80 mL of the NaOH solution in a clean, dry Erlenmeyer flask. Keep the NaOH solution containing Erlenmeyer closed with a rubber stopper. 
    \newstep[] Clean your buret and fill it with the NaOH solution using a plastic funnel. 
    \newstep[] Record the initial volume in the buret. Read accordingly to the tool precision, including your significant or estimated value.
  \newstep[] Pipet 5.00 mL of acetic acid into a clean 125 mL Erlenmeyer flask that has 20 mL of distilled water and 2 drops of phenolphthalein. 
       \newstep[] Record the molarity of the NaOH solution indicated in the stock solution bottle.
       \newstep[] Place the flask under the buret. Use a piece of white paper under the flask to distinguish better the color change.
       \newstep[] Add the NaOH solution from the buret in 1 mL portions, while swirling the solution in the flask. 
       \newstep[] The titration is completed when an addition of 1 mL causes the color to change from colorless to any shade of pink.
       \newstep[] Record the final buret volume.
              \newstep[] Repeat the steps above four times and average the resulting acetic acid concentration.

\end{steps}

\begin{center}\begin{tabular}{ |m{1cm} m{5.5cm}|m{1.7cm}|p{1.7cm}|p{1.7cm}|p{1.7cm}|  }
\cline{3-6}
\multicolumn{2}{r|}{} & 1 & 2& 3& 4 \\
\hline
 \mycircled{1}&Initial Buret Volume (mL)\vspace{0.5cm}  &    &  &  &  \\ [10pt]
\hline
 \mycircled{2}&Final Buret Volume (mL)\vspace{0.5cm}  &    & & &  \\[13pt]
\hline
 \mycircled{3}&NaOH Volume used (mL)\vspace{0.5cm}  &    & & &  \\[10pt]
\hline
 \mycircled{4}&\ce{CH3COOH} Concentration (M)\vspace{0.5cm}  &    & & &  \\[10pt]
\hline\hline
 \mycircled{5}&Mean \ce{CH3COOH} Concentration (M)\vspace{0.5cm}  &    \multicolumn{4}{c|}{}  \\[10pt]
\hline
\end{tabular}\end{center}


\end{fullwidth}




\newpage
\begin{fullwidth}

\vspace{0.2cm}{\large \bfseries Calculations }\\
\mycircled{1} Record the initial volume of the buret. This value is not necessarily 0.00 mL.\vspace{0.5cm}\\
\mycircled{2} Record the final volume of the buret, after you reached the end point.\vspace{0.5cm}\\
\mycircled{3} The volumen of \ce{NaOH} used should be: $\mycircled{2} \: - \: \mycircled{1}$ \vspace{0.5cm}\\
\mycircled{4} You can calculate the molarity of the acetic acid solution by means of:
\[c_a=\frac{\mycircled{2} \: \cdot c_b}{\text{5 mL}}\]
where $c_b$ is the given molarity of the \ce{NaOH} solution.\\
\mycircled{5} Is the average of the 4 concentrations calculated. \[\frac{\sum{\mycircled{4}}}{4}\]


\vspace{0.2cm}{\large \bfseries PostLab questions }
\begin{enumerate}
% PROBLEM 1
\item You need to prepare a sample containing 0.20 g of \ce{CuSO4} from a solution that is 10.\% \ce{CuSO4} by mass. What mass of solution do you need?\vspace{3cm}

\item A 10.00 mL sample of aqueous \ce{HNO3} requires 20.00 mL of 0.201 M NaOH to reach the endpoint. Calculate the molarity of \ce{HNO3}. \vspace{3cm}

\item You titrate a vinegar sample--an acetic acid solution in water--with 0.30 M \ce{NaOH}. Using 10. mL of vinegar, you reach the endpoint after 10. mL of the bases are added. Indicate the molarity of the acetic acid solution.
\end{enumerate}



\end{fullwidth}









\end{document}