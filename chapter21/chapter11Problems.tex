\documentclass[main.tex]{subfiles}
\begin{document}\newpage
\setdoublesep{0.35700 em}  % 'Bond Spacing'
\setatomsep{1.78500 em}    % 'Fixed Length'
\setbondoffset{0.18265 em} % 'Margin Width'
\newcommand{\bondwidth}{0.06642 em} % 'Line Width'
\setbondstyle{line width = \bondwidth}
\newgeometry{left=0.8in,right=0.8in, top=2.5cm,bottom=2cm}
\fancyhfoffset[E,O]{0pt}
\setlength{\columnsep}{30pt}
\begin{conclusion}
\end{conclusion}
\setstretch{0.3}
\begin{multicols*}{2}

{\raggedright\textsc{\textbf{Acids \& Bases}}\par}
\begin{enumerate}

\item The formula for carbonic acid is:
\begin{enumerate}[label=(\alph*)]
\begin{multicols*}{2}
\item \ce{HMnO4}
\item \ce{H2CO3}
\item \ce{HNO3}
\item \ce{HNO2}
\item \ce{H3PO4}
\end{multicols*}\begin{flushright}\small Ans: (b)\end{flushright}
\end{enumerate}

\item The formula for nitric acid is:
\begin{enumerate}[label=(\alph*)]
\begin{multicols*}{2}
\item \ce{HMnO4}
\item \ce{H2CO3}
\item \ce{HNO3}
\item \ce{HNO2}
\item \ce{H3PO4}
\end{multicols*}\begin{flushright}\small Ans: (a)\end{flushright}
\end{enumerate}

\item The formula for nitrous acid is:
\begin{enumerate}[label=(\alph*)]
\begin{multicols*}{2}
\item \ce{HMnO4}
\item \ce{H2CO3}
\item \ce{HNO3}
\item \ce{HNO2}
\item \ce{H3PO4}
\end{multicols*}\begin{flushright}\small Ans: (d)\end{flushright}
\end{enumerate}


\item The formula for permanganic acid is:
\begin{enumerate}[label=(\alph*)]
\begin{multicols*}{2}
\item \ce{HMnO4}
\item \ce{H2CO3}
\item \ce{HNO3}
\item \ce{HNO2}
\item \ce{H3PO4}
\end{multicols*}\begin{flushright}\small Ans: (a)\end{flushright}
\end{enumerate}

\item The name of the following chemical is: \ce{Ca(OH)2}
\begin{enumerate}[label=(\alph*)]
\begin{multicols*}{2}
\item Calcium oxygen hydate
\item Calcium hydrate
\item Calcium oxyhidrate
\item Calcium(II) hydroxide
\item Calcium hydroxide
\end{multicols*}\begin{flushright}\small Ans: (e)\end{flushright}
\end{enumerate}


\item The following chemical is an acid or a base: \ce{Ca(OH)2}
\begin{enumerate}[label=(\alph*)]
\begin{multicols*}{2}
\item Acid
\item Base
\end{multicols*}\begin{flushright}\small Ans: (b)\end{flushright}
\end{enumerate}

\item The following chemical is an acid or a base: \ce{HF}
\begin{enumerate}[label=(\alph*)]
\begin{multicols*}{2}
\item Acid
\item Base
\end{multicols*}\begin{flushright}\small Ans: (a)\end{flushright}
\end{enumerate}


\item The following chemical is an acid or a base: \ce{F^-}
\begin{enumerate}[label=(\alph*)]
\begin{multicols*}{2}
\item Acid
\item Base
\end{multicols*}\begin{flushright}\small Ans: (b)\end{flushright}
\end{enumerate}

{\raggedright\textsc{\textbf{Dissociation of acids \& bases}}\par}

\item Write down the dissociation reaction of the following chemical: \ce{HI_{(Aq)}}.
\begin{flushright}\small Ans: $\ce{HI_{(Aq)} ->[H2O] H^+_{(Aq)} + I^-_{(Aq)}}$ \end{flushright}


\item Write down the dissociation reaction of the following chemical: \ce{HMnO4_{(Aq)}}.
\begin{flushright}\small Ans: $\ce{HMnO4_{(Aq)} ->[H2O] H^+_{(Aq)} + MnO4^-_{(Aq)}}$ \end{flushright}

\item Write down the complete dissociation reaction of the following chemical: \ce{H2SO3_{(Aq)}}.
\begin{flushright}\small Ans: $\ce{H2SO3_{(Aq)} ->[H2O] 2H^+_{(Aq)} + SO3^{2-}_{(Aq)}}$ \end{flushright}

\item Write down the complete dissociation reaction of the following chemical: \ce{H3PO4_{(Aq)}}.
\begin{flushright}\small Ans: $\ce{H3PO4_{(Aq)} ->[H2O] 3H^+_{(Aq)} + PO4^{2-}_{(Aq)}}$ \end{flushright}

\item Write down the complete dissociation reaction of the following chemical: \ce{Ca(OH)2_{(Aq)}}.
\begin{flushright}\small Ans: $\ce{Ca(OH)2_{(Aq)} ->[H2O] Ca^{2+}_{(Aq)} + 2OH^{-}_{(Aq)}}$ \end{flushright}

\item Write down the complete dissociation reaction of the following chemical: \ce{SO4^{2-}_{(Aq)}}.
\begin{flushright}\small Ans: $\ce{SO4^{2-}_{(Aq)} + H2O_{(l)} -> HSO4^{-}_{(Aq)} + OH^{-}_{(Aq)}}$ \end{flushright}

\item In the following dissociation reaction, identify the acid, the base, the conjugate base and conjugate acid:
\begin{center}\ce{SO4^{2-}_{(Aq)} + H2O_{(l)} -> HSO4^{-}_{(Aq)} + OH^{-}_{(Aq)}}\end{center}
\begin{flushright}\small Ans: A (\ce{H2O}); B (\ce{SO4^{2-}}); Cacid (\ce{HSO4^{-}}); Cbase (\ce{OH^-});  \end{flushright}

\item In the following dissociation reaction, identify the acid, the base, the conjugate base and conjugate acid:
\begin{center}\ce{Cl^{-}_{(Aq)} + H2O_{(l)} -> HCl_{(Aq)} + OH^{-}_{(Aq)}}\end{center}
\begin{flushright}\small Ans: A (\ce{H2O}); B (\ce{Cl^{-}}); Cacid (\ce{HCl}); Cbase (\ce{OH^-});  \end{flushright}

\item In the following dissociation reaction, identify the acid, the base, the conjugate base and conjugate acid:
\begin{center}\ce{HCN_{(Aq)} + NO2^-_{(l)} -> CN^-_{(Aq)} + HNO2_{(Aq)}}\end{center}
\begin{flushright}\small Ans: A (\ce{HCN}); B (\ce{NO2^-}); CA (\ce{HNO2}); CB (\ce{CN^-});  \end{flushright}






\item Pick the strongest acid: \ce{HIO3} ($K_a=1.6\cdot 10^{-1}$) or \ce{H2SO3} ($K_a=1.5\cdot 10^{-2}$).
\begin{flushright}\small Ans: \ce{HIO3}  \end{flushright}

\item Pick the weakest acid: \ce{HN3} ($K_a=1.9\cdot 10^{-5}$) or \ce{H2CO3} ($K_a=4.3\cdot 10^{-7}$).
\begin{flushright}\small Ans: \ce{H2CO3}  \end{flushright}

\item Pick the weakest base: \ce{CN^-} ($K_a=6.2\cdot 10^{-10}$) or \ce{H2O} ($K_a=1.0\cdot 10^{-14}$).
\begin{flushright}\small Ans: \ce{CN^-}  \end{flushright}


\item Pick the weakest base: \ce{H2C6H5O^-} ($K_a=1.8\cdot 10^{-5}$) or \ce{HCOOH} ($K_a=1.7\cdot 10^{-4}$).
\begin{flushright}\small Ans: \ce{HCOOH}  \end{flushright}

{\raggedright\textsc{\textbf{The PH scale}}\par}

\item The PH of the stomach is 1.2. Calculate the concentration of protons in the solution.
\begin{flushright}\small Ans: $0.06M$  \end{flushright}

\item The PH of a can of pepsi is 2.5. Calculate the concentration of protons in the solution.
\begin{flushright}\small Ans: $3.2\cdot 10^{-3}M$  \end{flushright}

\item The PH of a baking soda solution is 8.2. Calculate the concentration of protons in the solution.
\begin{flushright}\small Ans: $5.0\cdot 10^{-9}M$  \end{flushright}

\item The proton concentration of a solution is $4.3\cdot 10^{-6}M$. Calculate the concentration of hydroxyls in the solution.
\begin{flushright}\small Ans: $2.3\cdot 10^{-9}M$  \end{flushright}

\item The hydroxyl concentration of a solution is $1.6\cdot 10^{-2}M$. Calculate the concentration of protons in the solution.
\begin{flushright}\small Ans: $6.2\cdot 10^{-13}M$  \end{flushright}

\item The PH of a bleach is 12.9. Calculate the concentration of hydroxyls in the solution.
\begin{flushright}\small Ans: $1.2\cdot 10^{-13}M$  \end{flushright}


{\raggedright\textsc{\textbf{Titrations and buffers}}\par}
\item A 5mL sample of an unknown acid is neutralized with 40 mL of a \ce{KOH} 0.5M solution. Calculate the molarity of the unknown acid.
\begin{flushright}\small Ans: $4M$ \end{flushright}
\item A 0.05L sample of an unknown acid is neutralized with 5 mL of a \ce{KOH} 2M solution. Calculate the molarity of the unknown acid.
\begin{flushright}\small Ans: $2M$ \end{flushright}
\item In order to standardize an HCl solution of unknown concentration, you add 25.00 mL of this acid in a  flask, and then add a few drops of an indicator. On the buret you use 0.2 M \ce{NaOH}. Before the titration, the buret reads 1 mL and 31 mL at the end point. Find the molarity of the HCl solution.
\begin{flushright}\small Ans: 0.24M \end{flushright}
\item What volume of 0.5 M \ce{KOH} would neutralize 10 mL of the 3M-\ce{HCl}. 
\begin{flushright}\small Ans: $60mL$ \end{flushright}









\restoregeometry
\end{enumerate}
\end{multicols*}
\pagecolor{green!10}\afterpage{\nopagecolor}\newpage
\end{document}
