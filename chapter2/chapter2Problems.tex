\documentclass[main.tex]{subfiles}
\begin{document}\newpage
\setdoublesep{0.35700 em}  % 'Bond Spacing'
\setatomsep{1.78500 em}    % 'Fixed Length'
\setbondoffset{0.18265 em} % 'Margin Width'
\newcommand{\bondwidth}{0.06642 em} % 'Line Width'
\setbondstyle{line width = \bondwidth}
\newgeometry{left=0.8in,right=0.8in, top=2.5cm,bottom=2cm}
\fancyhfoffset[E,O]{0pt}
\setlength{\columnsep}{30pt}
\begin{conclusion}
\end{conclusion}
%\setstretch{0.3}
\begin{multicols*}{2}\setcounter{numA}{1}





{\raggedright\textsc{\textbf{Energy and temperature }}\par}

%%%%%PROBLEM
\begin{question}[ID=\the\value{numA}]\SetQuestionProperties{section-title=\nameref{sec:units}}
Answer the following questions:
\begin{inparaenum}[(a)]	
\item What is the name of the energy associated with the motion of particles in a substance?
\item	 What is the name of the energy stored in the chemical bonds of a carbohydrate molecule? 
\item  What is the name of the energy stored in heigh?
\end{inparaenum} 
\end{question}
\begin{solution}\begin{inparaenum}[(a)]
\item kinetic energy
\item chemical energy
\item potential energy
\end{inparaenum}\hspace{0.1cm}\end{solution}\stepcounter{numA}%%%%%%%%%%%%

%%%%%PROBLEM
\begin{question}[ID=\the\value{numA}]\SetQuestionProperties{section-title=\nameref{sec:units}}
Carry the following conversions:
\begin{inparaenum}[(a)]	
\item 650J into calories
\item	  3.25 kcal into joules
\item  41$^{\circ}$F into celsius
\end{inparaenum} 
\end{question}
\begin{solution}\begin{inparaenum}[(a)]
\item  155 cal
\item  13598 J
\item  5$^{\circ}$C
\end{inparaenum}\hspace{0.1cm}\end{solution}\stepcounter{numA}%%%%%%%%%%%%
%%%%%PROBLEM
\begin{question}[ID=\the\value{numA}]\SetQuestionProperties{section-title=\nameref{sec:units}}
Carry the following conversions:
\begin{inparaenum}[(a)]	
\item 20$^{\circ}$C to F
\item	  300K to $^{\circ}$C
\end{inparaenum} 
\end{question}
\begin{solution}\begin{inparaenum}[(a)]
\item  68$^{\circ}F$
\item  27$^{\circ}$C
\end{inparaenum}\hspace{0.1cm}\end{solution}\stepcounter{numA}%%%%%%%%%%%%



{\raggedright\textsc{\textbf{The first law of thermodynamics}}\par}

%%%%%%%PROBLEM
\begin{question}[ID=\the\value{numA}]\SetQuestionProperties{section-title=\nameref{sec:units}}
A sample of gas expands from 3 to 4 L at constant pressure. Using $1L\cdot atm=101.3J$, calculate the work done in J under the following conditions:
\begin{inparaenum}[(a)]	
\item The gas expands against the vacuum. 
\item The gas expands against a constant pressure of 5atm 
\end{inparaenum} 
\end{question}
\begin{solution}
\begin{inparaenum}[(a)]	
\item 0
\item	 -506J 
\end{inparaenum} 
 \hspace{0.1cm}\end{solution}\stepcounter{numA}
%%%%%%%%%%%%%%

%%%%%%%PROBLEM
\begin{question}[ID=\the\value{numA}]\SetQuestionProperties{section-title=\nameref{sec:units}}
A 50g piece of aluminum ($c_e=0.214\frac{cal}{g^{\circ}C}$) initially at 25$^{\circ}$C absorbs 100cal. Calculate the final temperature of the aluminum piece.
\end{question}
\begin{solution}
$34.34^{\circ}$C
 \hspace{0.1cm}\end{solution}\stepcounter{numA}
%%%%%%%%%%%%%%

%%%%%%%PROBLEM
\begin{question}[ID=\the\value{numA}]\SetQuestionProperties{section-title=\nameref{sec:units}}
A sample of gas expands carrying work on the surroundings of 120J and absorbing 150J of heat from the surroundings. Calculate the change of the internal energy of the system in J.
\end{question}
\begin{solution}
30J
 \hspace{0.1cm}\end{solution}\stepcounter{numA}
%%%%%%%%%%%%%%



%%%%%%%PROBLEM
\begin{question}[ID=\the\value{numA}]\SetQuestionProperties{section-title=\nameref{sec:units}}
A 200g piece of iron ($c_e=0.1\frac{cal}{g^{\circ}C}$) initially at 15$^{\circ}$C absorbs 1000cal. Calculate the final temperature of the metal piece.
\end{question}
\begin{solution}
$T_{Final}=65^{\circ}$C
 \hspace{0.1cm}\end{solution}\stepcounter{numA}
%%%%%%%%%%%%%%

%%%%%%%PROBLEM
\begin{question}[ID=\the\value{numA}]\SetQuestionProperties{section-title=\nameref{sec:units}}
How many calories are required to raise the temperature of a 35 g sample of iron from 25$^{\circ}$C to 35$^{\circ}$C?  Iron has a specific heat of $0.108\frac{cal}{g^{\circ}C}$.
\end{question}
\begin{solution}
38 cal
 \hspace{0.1cm}\end{solution}\stepcounter{numA}
%%%%%%%%%%%%%%

%%%%%%%PROBLEM
\begin{question}[ID=\the\value{numA}]\SetQuestionProperties{section-title=\nameref{sec:units}}
What is the final temperature of a 35 g sample of iron at 25$^{\circ}$C after receiving 50cal?  Iron has a specific heat of $0.108\frac{cal}{g^{\circ}C}$.
\end{question}
\begin{solution}
38$^{\circ}$C
 \hspace{0.1cm}\end{solution}\stepcounter{numA}
%%%%%%%%%%%%%%
%%%%%%%PROBLEM
\begin{question}[ID=\the\value{numA}]\SetQuestionProperties{section-title=\nameref{sec:units}}
What is the initial temperature of a 50 g sample of aluminum that after receiving 50cal reaches a temperature of 50$^{\circ}$C?  Al has a specific heat of $0.2\frac{cal}{g^{\circ}C}$.
\end{question}
\begin{solution}
45$^{\circ}$C
 \hspace{0.1cm}\end{solution}\stepcounter{numA}
%%%%%%%%%%%%%%
%%%%%%%PROBLEM
\begin{question}[ID=\the\value{numA}]\SetQuestionProperties{section-title=\nameref{sec:units}}
What is the specific heat of a metal if a 100 g sample at 25$^{\circ}$C warms up until 50$^{\circ}$C after receiving 100cal? 
\end{question}
\begin{solution}
$0.04cal/g^{\circ}$C
 \hspace{0.1cm}\end{solution}\stepcounter{numA}
%%%%%%%%%%%%%%

{\raggedright\textsc{\textbf{Calorimetry }}\par}

%%%%%%%PROBLEM
\begin{question}[ID=\the\value{numA}]\SetQuestionProperties{section-title=\nameref{sec:units}}
A 3 moles sample of \ce{C(s)} is burned in a constant-volume calorimeter containing 40g of water. The temperature inside the calorimeter increases from 25.0$^{\circ}$C to 25.89 $^{\circ}$C. The calorimeter constant is 9.90 $\frac{kJ}{^{\circ}C}$.  Calculate the molar heat of the reaction. 
\end{question}
\begin{solution}
-3KJ/mol
 \hspace{0.1cm}\end{solution}\stepcounter{numA}
%%%%%%%%%%%%%%
%%%%%%%PROBLEM
\begin{question}[ID=\the\value{numA}]\SetQuestionProperties{section-title=\nameref{sec:units}}
A 10 grams sample of fructose (MW=180g/mol) is burned in a constant-volume calorimeter containing 50g of water. The temperature inside the calorimeter increases 7$^{\circ}$C . The calorimeter constant is 10.8 $\frac{kJ}{^{\circ}C}$.  Calculate the molar heat of the reaction.
\end{question}
\begin{solution}
-1387KJ/mol
 \hspace{0.1cm}\end{solution}\stepcounter{numA}
%%%%%%%%%%%%%%
%%%%%%%PROBLEM
\begin{question}[ID=\the\value{numA}]\SetQuestionProperties{section-title=\nameref{sec:units}}
When a 0.09-g sample of trinitrotoluene (TNT, MW=213g/mol), is burned in a bomb calorimeter, the temperature increases from 23.5 $^{\circ}$C to 27.1$^{\circ}$C. The heat capacity of the calorimeter is 400 $\frac{J}{^{\circ}C}$, and it contains 100 mL of water. Calculate the molar heat of the reaction. Remember that the density of water is 1g/mL.
\end{question}
\begin{solution}
-6973KJ/mol
 \hspace{0.1cm}\end{solution}\stepcounter{numA}
%%%%%%%%%%%%%%
%%%%%%%%PROBLEM
\begin{question}[ID=\the\value{numA}]\SetQuestionProperties{section-title=\nameref{sec:units}}
We mix 50mL of 2M HCl with 100mL of 1.5M NaOH in a coffee-cup calorimeter. Both solutions are initially at 20$^{\circ}$C. Calculate the final temperature of the solution in the calorimeter considering that the specific heat of the mixture is 4.184$\frac{J}{g^{\circ}C}$ and the density of the solution is 1g/mL. The molar heat of the reaction is -56kJ/mol.
\end{question}
\begin{solution}
29$^{\circ}$C
 \hspace{0.1cm}\end{solution}\stepcounter{numA}
%%%%%%%%%%%%%%%



%%%%%%%%PROBLEM
%\begin{question}[ID=\the\value{numA}]\SetQuestionProperties{section-title=\nameref{sec:units}}
%You warm up 10 g of a solid to 200$^{\circ}$C and add it to 60 g of water in a coffee-cup calorimeter. The water temperature changes from 24$^{\circ}$C to 29$^{\circ}$C. Calculate the specific heat of the solid. You can consider negligible the heat capacity of the coffee-cup calorimeter.
%\end{question}
%\begin{solution}
%0.17J/$^{\circ}$C
% \hspace{0.1cm}\end{solution}\stepcounter{numA}
%%%%%%%%%%%%%%%


{\raggedright\textsc{\textbf{Enthalpy }}\par}

%%%%%%%%%%%%%%
\begin{question}[ID=\the\value{numA}]\SetQuestionProperties{section-title=\nameref{sec:units}}
Identify the following reaction as endothermic or exothermic.
 \begin{center}\ce{  	C6H12O6(s) + 6O2_{(g)} -> 6CO2_{(g)} + 6H2O_{(g)}		} \hspace*{0pt}\hfill $\Delta H^{\circ}_R=	 -2800KJ/mol$\end{center}
\end{question}
\begin{solution}
exothermic
 \hspace{0.1cm}\end{solution}\stepcounter{numA}
%%%%%%%%%%%%%%
%%%%%%%%%%%%%%
\begin{question}[ID=\the\value{numA}]\SetQuestionProperties{section-title=\nameref{sec:units}}
Identify the following reaction as endothermic or exothermic.
 \begin{center}\ce{  	B2O3(s) + 3H2O_{(g)} -> 3O2_{(g)} + B2H6_{(g)}	} \hspace*{0pt}\hfill $\Delta H^{\circ}_R=	   2035KJ/mol$\end{center}
\end{question}
\begin{solution}
endothermic
 \hspace{0.1cm}\end{solution}\stepcounter{numA}
%%%%%%%%%%%%%%
%%%%%%%%%%%%%%
\begin{question}[ID=\the\value{numA}]\SetQuestionProperties{section-title=\nameref{sec:units}}
For the following reaction:
 \begin{center}\ce{  	C6H12O6(s) + 6O2_{(g)} -> 6CO2_{(g)} + 6H2O_{(g)}	} \hspace*{0pt}\hfill $\Delta H^{\circ}_R=	   -2800KJ/mol$\end{center}
Fill the conversion factor:
\begin{equation*}
\frac{\hlmath{\hspace{35pt}}\text{ moles of }\ce{O2}}{\text{ -2800 KJ }} 
\end{equation*}
\end{question}
\begin{solution}
6
 \hspace{0.1cm}\end{solution}\stepcounter{numA}
%%%%%%%%%%%%%%
%%%%%%%%%%%%%%
\begin{question}[ID=\the\value{numA}]\SetQuestionProperties{section-title=\nameref{sec:units}}
In the following combustion reaction:
 \begin{center}\ce{  	C6H12O6(s) + 6O2_{(g)} -> 6CO2_{(g)} + 6H2O_{(g)}	} \hspace*{0pt}\hfill $\Delta H^{\circ}_R=	   -2800KJ/mol$\end{center}
glucose (\ce{C6H12O6}) burns to produce carbon dioxide and water. Calculate the heat involved in the combustion of 3 moles of glucose.\end{question}
\begin{solution}
8400KJ
 \hspace{0.1cm}\end{solution}\stepcounter{numA}
%%%%%%%%%%%%%%
%%%%%%%%%%%%%%
\begin{question}[ID=\the\value{numA}]\SetQuestionProperties{section-title=\nameref{sec:units}}
Calculate the enthalpy of reaction for:
\begin{center}\ce{ 2OF2_{(g)}    -> O2_{(g)}  +  2F2_{(g)}}\end{center}
given:
\begin{center}
$\Delta H^{\circ}_f(\ce{OF2_{(g)}})=24.5KJ$
\end{center}
\end{question}
\begin{solution}
-49KJ
 \hspace{0.1cm}\end{solution}\stepcounter{numA}
%%%%%%%%%%%%%%
%%%%%%%%%%%%%%
\begin{question}[ID=\the\value{numA}]\SetQuestionProperties{section-title=\nameref{sec:units}}
Calculate the enthalpy of reaction for:
\begin{center}\ce{ 2ClF_{(g)}  +  O2_{(g)}    ->  Cl2O_{(g)}  +  OF2_{(g)} }\end{center}
given:
\begin{center}
$\Delta H^{\circ}_f(\ce{ClF_{(g)}})=-56KJ$\\
$\Delta H^{\circ}_f(\ce{Cl2O_{(g)}})=88KJ$\\
$\Delta H^{\circ}_f(\ce{OF2_{(g)}})=25KJ$
\end{center}
\end{question}
\begin{solution}
225KJ
 \hspace{0.1cm}\end{solution}\stepcounter{numA}
%%%%%%%%%%%%%%

%%%%%%%%%%%%%%
\begin{question}[ID=\the\value{numA}]\SetQuestionProperties{section-title=\nameref{sec:units}}
Calculate the enthalpy of reaction for:
\begin{center}\ce{ ClF3_{(g)}  +  O2_{(g)}    ->  Cl2O_{(g)}  +  3/2OF2_{(g)} }\end{center}
given:
\begin{center}
$\Delta H^{\circ}_f(\ce{ClF3_{(g)}})=-156KJ$\\
$\Delta H^{\circ}_f(\ce{Cl2O_{(g)}})=88KJ$\\
$\Delta H^{\circ}_f(\ce{OF2_{(g)}})=25KJ$
\end{center}\end{question}
\begin{solution}
281KJ
 \hspace{0.1cm}\end{solution}\stepcounter{numA}
%%%%%%%%%%%%%%

{\raggedright\textsc{\textbf{Hess's Law }}\par}
%%%%%%%%%%%%%%
\begin{question}[ID=\the\value{numA}]\SetQuestionProperties{section-title=\nameref{sec:units}}
Using the following reactions:\\
\begin{tabularx}{\columnwidth}{>{}m{.65\linewidth} *{2}{Y} }
\multicolumn{2}{l}{\hspace{\linewidth} }   \\
\multicolumn{2}{l}{\ce{2OF2_{(g)}    -> O2_{(g)}  +  2F2_{(g)} } }   \\
\multicolumn{2}{r}{ $\Delta H_1=-49KJ$  }    \\
\multicolumn{2}{l}{\ce{2ClF_{(g)}  +  O2_{(g)}    ->  Cl2O_{(g)}  +  OF2_{(g)} } }   \\
\multicolumn{2}{r}{ $\Delta H_2=225KJ$   }    \\
\multicolumn{2}{l}{\ce{ClF3_{(g)}  +  O2_{(g)}    ->  1/2Cl2O_{(g)}  +  3/2OF2_{(g)} } }   \\
\multicolumn{2}{r}{ $\Delta H_3=324KJ$   }    \\
\end{tabularx}\\
Determine the enthalpy change for:
\begin{center}\ce{ ClF_{(g)} + F2_{(g)} -> ClF3_{(g)}}\end{center}\end{question}
\begin{solution}
-187KJ
 \hspace{0.1cm}\end{solution}\stepcounter{numA}
%%%%%%%%%%%%%%
%%%%%%%%%%%%%%
\begin{question}[ID=\the\value{numA}]\SetQuestionProperties{section-title=\nameref{sec:units}}
Using the following reactions:\\
\begin{tabularx}{\columnwidth}{>{}m{.65\linewidth} *{2}{Y} }
\multicolumn{2}{l}{\hspace{\linewidth} }   \\
\multicolumn{2}{l}{\ce{N2_{(g)} + 3H2_{(g)} -> 2NH3_{(g)} } }   \\
\multicolumn{2}{r}{ $\Delta H_1=-92KJ$  }    \\
\multicolumn{2}{l}{\ce{C(s) + 2H2_{(g)} -> CH4_{(g)} } }   \\
\multicolumn{2}{r}{ $\Delta H_2=-75KJ$   }    \\
\multicolumn{2}{l}{\ce{H2_{(g)} + 2C(s) + N2_{(g)} -> 2HCN_{(g)} } }   \\
\multicolumn{2}{r}{ $\Delta H_3=270KJ$   }    \\
\end{tabularx}\\
Determine the enthalpy change for:
\begin{center}\ce{ CH4_{(g)} + NH3_{(g)} -> HCN_{(g)} + 3H2_{(g)}}\end{center}\end{question}
\begin{solution}
256KJ
 \hspace{0.1cm}\end{solution}\stepcounter{numA}
%%%%%%%%%%%%%%
%%%%%%%%%%%%%%
\begin{question}[ID=\the\value{numA}]\SetQuestionProperties{section-title=\nameref{sec:units}}
Using the following reactions:\\
\begin{tabularx}{\columnwidth}{>{}m{.65\linewidth} *{2}{Y} }
\multicolumn{2}{l}{\hspace{\linewidth} }   \\
\multicolumn{2}{l}{\ce{3C(s) + 3H2_{(g)} + 1/2 O2_{(g)} -> C3H6O_{(l)} } }   \\
\multicolumn{2}{r}{ $\Delta H_1=-285KJ$  }    \\
\multicolumn{2}{l}{\ce{C(s) + O2_{(g)} -> CO2_{(g)} } }   \\
\multicolumn{2}{r}{ $\Delta H_2=-394KJ$   }    \\
\multicolumn{2}{l}{\ce{H2_{(g)} + 1/2 O2_{(g)} -> H2O_{(l)} } }   \\
\multicolumn{2}{r}{ $\Delta H_3=-286KJ$   }    \\
\end{tabularx}\\
Determine the enthalpy change for:
\begin{center}\ce{ C3H6O_{(l)} + 4O2_{(g)} -> 3CO2_{(g)} + 3H2O_{(l)}}\end{center}\end{question}

\begin{solution}
-1755KJ
 \hspace{0.1cm}\end{solution}\stepcounter{numA}
%%%%%%%%%%%%%%

%%%%%%%%%%%%%%
\begin{question}[ID=\the\value{numA}]\SetQuestionProperties{section-title=\nameref{sec:units}}
Using the following reactions:\\
\begin{tabularx}{\columnwidth}{>{}m{.65\linewidth} *{2}{Y} }
\multicolumn{2}{l}{\hspace{\linewidth} }   \\
\multicolumn{2}{l}{\ce{2C2H6 + 7O2 -> 4CO2 + 6H2O	 } }   \\
\multicolumn{2}{r}{ $\Delta H_1=-3120KJ$  }    \\
\multicolumn{2}{l}{\ce{2H2 + O2 -> 2H2O } }   \\
\multicolumn{2}{r}{ $\Delta H_2=-479KJ$   }    \\
\multicolumn{2}{l}{\ce{2CO + O2 -> 2CO2 } }   \\
\multicolumn{2}{r}{ $\Delta H_3=-566KJ$   }    \\
\end{tabularx}\\
Determine the enthalpy change for:
\begin{center}\ce{ C2H6 + O2 -> 3H2 + 2CO}\end{center}\end{question}

\begin{solution}
-276KJ
 \hspace{0.1cm}\end{solution}\stepcounter{numA}
%%%%%%%%%%%%%%




\end{multicols*}
\newpage
\begin{answersenvironment}
\begin{minipage}[c]{1\textwidth}
\begin{localsize}{10}
{\Large \bf Answers}
\SetupExSheets{
  headings = inline-nr , % numbered and inline
  counter-format = qu) , % numbers 1) 2) ... 
}
%\printsolutions 
\printsolutions[byID={1,3,5,7,9,11,13,15,17,19,21,23,25 }]
\end{localsize}
\end{minipage}\end{answersenvironment}
\end{document}

























