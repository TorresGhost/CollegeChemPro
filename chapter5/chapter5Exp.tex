\documentclass[main.tex]{subfiles}
\begin{document}\newpage
\setdoublesep{0.35700 em}  % 'Bond Spacing'
\setatomsep{1.78500 em}    % 'Fixed Length'
\setbondoffset{0.18265 em} % 'Margin Width'
\newcommand{\bondwidth}{0.06642 em} % 'Line Width'
\setbondstyle{line width = \bondwidth}

\begin{fullwidth}





%%%%%%%%%%%%HEADING
\begin{multicols}{2}
\begin{tcolorbox}[enhanced jigsaw,breakable,size=title,
colback=mybrown!05,colframe=black,fonttitle=\bfseries,
title=STUDENT INFO,pad at break=1mm, break at=15cm/0pt ]
\vspace{0.2cm}
\noindent Name: \rule{5cm}{0.4pt}Date:\rule{1cm}{0.4pt}\\
Pre-lab Done: \tikzcheckmark[scale=2,black]{no mark}\quad
\end{tcolorbox}
\end{multicols}
\hfill
\vspace{0.2cm}
\begin{center}
{\large \bfseries 
Pre-lab Questions 
\par
\Huge
Chemical formulas
\\[5pt] \par}
\vspace{0.2cm}
\end{center}
\par
\noindent
\uline{  \hfill \normalsize \hfill       }
%%%%%%%%%%%%HEADING

\begin{enumerate}
% PELAB 1
\item Fill the following conversion to convert  15g of N into moles (AW(N)=14$g\cdot mol^{-1}$)
 \begin{equation*}
15\cancel{\text{ g of }\ce{N}} \times \dfrac{\hlmath{\hspace{35pt}}\text{ moles of }\ce{N}}{\hlmath{\hspace{35pt}}\cancel{\text{g of }\ce{N}}} = \text{ moles of }\ce{N}.
\end{equation*}
\vspace{3cm}


\item Fill the following conversion to convert  2 moles of O into moles (AW(O)=16$g\cdot mol^{-1}$)
 \begin{equation*}
2\cancel{\text{ moles of }\ce{O}} \times \dfrac{\hlmath{\hspace{35pt}}\text{ g of }\ce{O}}{\hlmath{\hspace{35pt}}\cancel{\text{moles of }\ce{O}}} = \text{ g of }\ce{O}.
\end{equation*}
\vspace{1cm}

\item The thermal decomposition of 5g of an hydrate gives a final product mass of 2.5g. Calculate the percent of water in the hydrate.
\vspace{3cm}

\item Name or give the formula of the following compounds:
\begin{center} \begin{tabular}{ p{5cm} p{4cm}    }
 Magnesium sulfate&\rule{4cm}{0.4pt}        \\
   \ce{MgSO4 . H2O} &\rule{4cm}{0.4pt}        \\
     Barium chloride &\rule{4cm}{0.4pt}        \\
    \ce{BaCl2 . 2H2O} &\rule{4cm}{0.4pt}        \\
     Chromium(III) chloride &\rule{4cm}{0.4pt}    \\
      \ce{CoCl2 . 6H2O}  &\rule{4cm}{0.4pt}    \\
     Nickel(II) sulfate heptahydrate&\rule{4cm}{0.4pt}    \\
  \end{tabular}\end{center}


\vspace{3cm}

 

\end{enumerate}


\clearpage\mbox{}\clearpage



%%%%%%%%%%%%HEADING
\begin{multicols}{2}
\begin{tcolorbox}[enhanced jigsaw,breakable,size=title,
colback=mybrown!05,colframe=black,fonttitle=\bfseries,
title=STUDENT INFO,pad at break=1mm, break at=15cm/0pt ]
\vspace{0.2cm}
\noindent Name: \rule{5cm}{0.4pt}Date:\rule{1cm}{0.4pt}\\
Pre-lab Done: \tikzcheckmark[scale=2,black]{no mark}\quad
\end{tcolorbox}
\end{multicols}
\hfill
\vspace{0.2cm}
\begin{center}
{\large \bfseries 
Experiment
\par
\Huge
Chemical formulas
\\[5pt] \par}
\vspace{0.2cm}
\end{center}
\par
\noindent
\uline{  \hfill \normalsize \hfill       }
%%%%%%%%%%%%HEADING

\vspace{0.2cm}{\large \bfseries 1. Empirical formula of an oxyde}
The goal of this mini experiment is to calculate the formula of an oxide. You will do this by burning up a metal .
\begin{steps}
      \newstep[] Obtain a crucible with a lid, a clay triangle and an iron ring attached to a ring stand. Place the covered crucible in the clay triangle on an iron ring attached to a ring stand. Adjust the height of the ring so that the bottom of the crucible will be in the hottest part of the flame. The correct arrangement of the equipment, crucible, and burner is shown in the figure (Right panel).   
     


\begin{figure*}[h] % FUL FIGURE
        \includegraphics[width=\linewidth,scale=0.3]{./chapter5/formu.png}
\caption{(Left panel) A crucible. (Central panel) Use of the crucible tongs. (Right panel) Correct arrangement of the ring stand.}\label{fig:reactions}
\end{figure*}
\vspace{1cm}


        \newstep[] Learn how to use the Bunsen burner \ding{45}. Heat the covered crucible in the hottest part of the flame for about 5 min while keeping the lid ajar, making sure that the bottom of the crucible attains a red glow.
       \newstep[] Stop the burner and allow the crucible to cool down completely. Weight the covered crucible and record the mass of the covered crucible.     
\end{steps}
\begin{steps}[resume]
    \newstep[] Obtain 0.2 g of magnesium ribbon. Clean the surface of the metal with metallic wool until it shines. Cut the magnesium ribbon into tiny bits, and place them inside the crucible. Cover the crucible, obtain and record the mass again. Now you know the mass of the crucible+lid+Mg.
       \newstep[] Set the crucible on the clay triangle with the lid on and heat the crucible \ding{45} in the hottest part of the flame another 5 min. Keep the lid close. Using the crucible tongs, lift the lid carefully by a slight amount. The metal should glow brightly without flames. Continue until all Mg is burned and the product does not glow.
              \newstep[] Patiently cool down the crucible with lid. The content should be white or slightly gray. At this point, add a few drops of water using a plastic pipet on the crucible content. You might notice a smell of ammonia at this point.
       \newstep[] Place the lid back onto the crucible, slightly ajar, and heat the crucible  in the hottest part of the flame for 15 more minutes. After that time, allow the covered crucible and its content to cool down. Obtain the mass of the covered crucible.
       \end{steps}
     
      
\end{fullwidth}






 






\newpage
\begin{fullwidth}
\begin{center}\begin{tabular}{ p{1cm}p{6cm} p{5cm}  }
\hline
&&\\
\begin{center} \mycircled{1}\end{center}    & \begin{center} Mass of empty crucible and lid (g)\end{center}  &\begin{center}\rule{3.0cm}{0.4pt}\end{center}   \\

\begin{center}\mycircled{2}\end{center}   &\begin{center}Mass of crucible and lid with Mg (g)\end{center} &\begin{center}\rule{3.0cm}{0.4pt}\end{center}   \\

\begin{center}\mycircled{3}\end{center}   &\begin{center}Mass of  Mg (g)\end{center} &\begin{center}\rule{3.0cm}{0.4pt}\end{center}    \\

\begin{center}\mycircled{4} \end{center}  &\begin{center}Moles of  Mg (mole)\end{center} &\begin{center}\rule{3.0cm}{0.4pt}\end{center}   \\

\begin{center}\mycircled{5}\end{center}  &  \begin{center}Mass of  crucible and lid with MgO (g)\end{center} &\begin{center}\rule{3.0cm}{0.4pt}\end{center}  \\

\begin{center} \mycircled{6} \end{center}    &\begin{center}Mass of  MgO (g) \end{center} &\begin{center}\rule{3.0cm}{0.4pt}\end{center}  \\

\begin{center} \mycircled{7} \end{center}     &\begin{center}Mass of  O (g) \end{center} &\begin{center}\rule{3.0cm}{0.4pt}\end{center} \\

\begin{center}\mycircled{8} \end{center}  &\begin{center}Moles of  O (moles) \end{center} &\begin{center}\rule{3.0cm}{0.4pt}\end{center}    \\
\hline
\end{tabular}\end{center}



\begin{center}\begin{tabular}{ |p{4cm}|p{4cm}|p{4cm}|  }
\hline
\vspace{0.2cm} & Mg & O  \\
\hline

Moles of (moles) \vspace{0.6cm} &     & \\
\hline
Moles/smallest amount  \vspace{0.6cm} &     & \\
\hline
Empirical Formula \vspace{0.6cm} &   \multicolumn{2}{c|}{}  \\
\hline
\end{tabular}\end{center}


\end{fullwidth}


\newpage 
\begin{fullwidth}
\subsection*{Calculations}
\mycircled{1} This is the mass of the empty and clean crucible with lid.\vspace{0.5cm}\\
\mycircled{2} This is the mass of the clean crucible with lid and the Mg.\vspace{0.5cm}\\
\mycircled{3} This is the mass of Mg added to the crucible: $\mycircled{2}\: - \: \mycircled{1}$ \vspace{0.5cm}\\
\mycircled{4} This is the moles of Mg (Atomic weight 24.305 $g\cdot mol^{-1}$):
$n_{Mg}=\frac{\mycircled{3}\: g}{24.305 \: g\cdot mol^{-1}}$\vspace{0.5cm}\\
\mycircled{5} This is the mass of the clean crucible with lid and the product.\vspace{0.5cm}\\
\mycircled{6} This is the mass of product: $\mycircled{5} \: - \: \mycircled{1}$ \vspace{0.5cm}\\
\mycircled{7} This is the mass of O in the product: $\mycircled{6} \: - \: \mycircled{3} $ \vspace{0.5cm}\\
\mycircled{8} This is the moles of O (Atomic weight 15.999 $g\cdot mol^{-1}$) in the product: $n_{O}=\frac{\mycircled{7} \:g}{15.999 \:g\cdot mol^{-1}}$


\vspace{0.6cm}{\large \bfseries PostLab questions }
\begin{enumerate}
\item[1 a)] Calculate the formula of the oxide resulting of mixing Mg and O according to the respective valences of the elements?
\vspace{4.cm}
\item[1 b)] Calculate the formula of the oxide resulting of mixing Mg and N according to the respective valences of the elements?
\vspace{4.cm}
\item[1 c)] A nitride contains 1.5 moles of Ca and 1 moles of N. Calculate the formula of the nitride.
\vspace{4.cm}
\end{enumerate}

\end{fullwidth}


\newpage 
\begin{fullwidth}
\vspace{0.2cm}{\large \bfseries 2. Thermal decomposition of an hydrate}
The goal of this mini experiment is to calculate the percentage of water contained in an hydrate. You will achieve this goal by heating up the hydrate and measuring its mass before and after heating. The difference in mass will be the mass of water contained in the hydrate.
\begin{steps}
    \newstep[] Place a clean, covered crucible in a clay triangle on an iron ring attached to a ring stand. Adjust the height of the ring so that the bottom of the crucible will be in the hottest part of the flame. The correct arrangement of the equipment, crucible, and burner is shown in the figure (Right panel).   
        \newstep[] Learn how to use the Bunsen burner \ding{45}. Heat the covered crucible in the hottest part of the flame for about 5 min, making sure  that the bottom of the crucible attains a red glow.
       \newstep[] Stop the burner and allow the crucible to cool down completely.         
       \newstep[] Weight the covered crucible  and record the mass of the covered crucible.

       \newstep[] Weight about 1.5 g of the hydrate.
       \newstep[] Add the the hydrate sample onto the crucible and cover the crucible again. Weight the covered crucible with the chemical and record the exact mass in the results sheet.
       \newstep[] Heat up the crucible in the hottest part of the flame for about 15 min. The bottom of the crucible should be red hot during this time.
       \newstep[] When the crucible is cool, weight and record the mass of the product. 
       \end{steps}
\vspace{2cm}
\begin{center}\begin{tabular}{ p{3cm}p{8cm}p{5cm}  }

\hline
\multicolumn{3}{c}{\Large \begin{bf}Formula of an oxide Data\end{bf} }\\
\hline
\begin{center}\vspace{0.02cm} \mycircled{1}\end{center} & \begin{center}Mass of empty crucible and lid (g)\end{center}   &\begin{center}\rule{3.0cm}{0.4pt}\end{center}       \\[10pt]

\begin{center}\vspace{0.02cm} \mycircled{2}\end{center} &\begin{center}Mass of crucible and lid with  hydrate (g)\end{center}  &\begin{center}\rule{3.0cm}{0.4pt}\end{center}      \\[10pt]

\begin{center}\vspace{0.02cm} \mycircled{3}\end{center} &\begin{center}Mass of  hydrate (g)\end{center}  &\begin{center}\rule{3.0cm}{0.4pt}\end{center}       \\[10pt]

\begin{center}\vspace{0.02cm} \mycircled{4}\end{center} &\begin{center}Mass of crucible, lid and product (g)\end{center}   &\begin{center}\rule{3.0cm}{0.4pt}\end{center}      \\[10pt]

\begin{center}\vspace{0.02cm} \mycircled{5}\end{center} &\begin{center}Mass of  product (g)\end{center} &\begin{center}\rule{3.0cm}{0.4pt}\end{center}      \\[10pt]

\begin{center}\vspace{0.02cm} \mycircled{6}\end{center} &\begin{center}Mass \% of water\end{center}  &\begin{center}\rule{3.0cm}{0.4pt}\end{center}       \\[10pt]
\hline
\end{tabular}\end{center}
\end{fullwidth}


\newpage
\begin{fullwidth}
\subsection*{Calculations}
\mycircled{1} Record the mass of the empty crucible with the lid. Remember to weight the crucible in the balance only when completely cool.\vspace{0.5cm}\\
\mycircled{2} Record the mass of the empty crucible with the lid with hydrate.\vspace{0.5cm}\\
\mycircled{3} The mass of hydrate added to the crucible should be:
\[\text{Mass of }\ce{hydrate}= \: \mycircled{2}\: - \: \mycircled{1}\] \vspace{0.5cm}
\mycircled{4} After you heat the crucible with hydrate a product will form. Weight the crucible and lid with the final product inside.\\ \vspace{0.5cm}
\mycircled{5} You should calculate the mass of product by doing:
\[\text{Mass Product=} \: \mycircled{4}\:-\: \mycircled{1}\] \vspace{0.5cm}
\mycircled{6} Calculate the mass \% of water in the hydrate:
\[\frac{\text{(Mass  hydrate)} - \text{(Mass Product) }}{\text{(Mass of hydrate)}}\times 100 = \frac{\mycircled{3} \: - \: \mycircled{5} }{\mycircled{3} }\times 100 \] \vspace{0.5cm}


\vspace{0.6cm}{\large \bfseries PostLab questions }
\begin{enumerate}
\item[2 a)] The product of burning 5 grams of a hydrate weights 4.5g. Calculate the water \% mass of the hydrate.
\vspace{3.7cm}
\item[2 b)] The formula for an hydrate is  \ce{FeSO4 . 7H2O}. Calculate the water \% mass of the hydrate.
\vspace{3.7cm}
\item[2 c)] Name the following chemical:  \ce{FeSO4 . 7H2O}.
\vspace{3.7cm}
\end{enumerate}


\end{fullwidth}




\end{document}