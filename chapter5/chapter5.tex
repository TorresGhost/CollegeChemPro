\documentclass[main.tex]{subfiles}

\begin{document}

\linenumbers


\chapter[Chemical naming ]{Chemical naming}
\label{ch:naming}


   
            \begin{marginfigure}
\begin{tikzpicture} \node (a) at (0,0) {\includegraphics[width=4cm]{chapter5/figure1}} node[rotate=90, font=\tiny] at ([yshift=.5cm,xshift=.1cm]a.south east) {\textsuperscript{\textcopyright} Pixnio} ;
\end{tikzpicture}
\label{fig:naming1}
\end{marginfigure}
   
\lettrine[lines=4]{\color{black!45}A}{ll} elements in the periodic table with the exception of the noble gases--He, Ne, Ar, Kr, Xe and Rn--combine to produce chemical compounds. Most of these chemicals are useful in your every day life, and you drink water to quench your thirst, use Clorox to clean your house or baking soda to get rid of a stinky refrigerator. In this chapter you will learn not only how to name these chemicals but also to read chemical formulas--we call this to formulate chemicals. Still, chemical elements such as hydrogen and oxygen do not combine randomly and they only choose specific elemental partners to form a compound. As an example, hydrogen combines with oxygen using specific proportions to produce \ce{H2O} and not \ce{HO2}. In this chapter you will also learn the rules that chemical elements use to combine.
\begin{marginfigure}%LEARNING GOALS BOX
\begin{mytcbox}{GOALS}
\begin{enumerate}[label=\protect\circled{\color{white}\arabic*}]
\item Name and formulate ionic compounds
\item Name and formulate covalent compounds
\item Name and formulate acids and bases 
\item Name and formulate oxosalts 
\item Name and formulate common chemicals
\end{enumerate}
\end{mytcbox}
\vspace{1cm}
\begin{tcolorbox}[enhanced,colback=red!5!white,colframe=black!50!red,boxrule=1pt,
  arc=0pt,outer arc=0pt,drop heavy lifted shadow]
\faGears\ 
\docenvdef{Discussion:} think about your household and the chemicals you use at home. List three chemicals you found around you, with its correct chemical name \end{tcolorbox}
\end{marginfigure}%LEARNING GOALS BOX

\section{Ions \& ionic charges}
Atoms gain and loose electrons to produce ions. An ion is just an atom with a positive or negative charge. Ions result from an electron transfer. Positive ions have lost negatively charged electrons, whereas negative ions have gained electrons. The reason for this electron transfer is that atoms try to achieve a very stable electronic configuration with eight electrons in the valence, and this is called the octet electron configuration. Examples of ions are: \ce{H+}, \ce{Ca^{2+}} or \ce{O^{2-}}. This section covers the properties of ions and the ionic charges.\sloppy 
%\begin{marginfigure}%%%%%%%MARGIN FIGURE
%      \includegraphics{chapter5/figure2}
%      \label{image51}
%      \caption{Figure with the different ionic charges.}              
%   \end{marginfigure}%%%%%%%MARGIN FIGURE
   
   
\begin{description}
\item[\docfilehook{Cations}{Cations}] Atoms that loose electrons become positively charged. These ions are called cations. Example of cations are \ce{Li+} or \ce{Mg^{2+}} called lithium cation and magnesium cation, respectively.
\item[\docfilehook{Anions}{Anions}] Atoms that gain electrons become negatively charged, as electrons have negative charge. These ions are called anions. Example of anions are \ce{F-}  called fluoride or \ce{N^{3-}} called nitride. The way to name anions is by using the name of the element and the suffix -ide.
\item[\docfilehook{Ionic charges: the valences}{Ionic charges: the valences}] How do we know that hydrogen produces a \ce{H+} ion and nitrogen a \ce{N^{3-}} anion. The charge of an ion is called ionic charges, and the numbers are coming from the periodic table. H, Na or K are in the group IA (left of the table) and hence the ionic charge will be $1+$.  Similarly, Mg or Ca are in the group IIA (left of the table) and hence the ionic charge will be $2+$. Differently, F, Cl or Br are in the group 7A (right of the table) and its charge will be $1-$. Oxygen is in group 6A (right of the table) and the ionic charge will be $2-$. Figure \ref{fig:naming2} contains all ionic charges. What if the element is not in this list such as the case of Iron (Fe)? In that case, very probably it will have several ionic charges and this charge has to be indicated in the chemical name. An example would be \ce{Fe}, which ionic charge is not in Figure \ref{fig:naming2}  as iron can have several ionic charges.






\begin{example} %%%%%%%%%%%%%%%%%%%%%%%% EXAMPLE BOX
Identify the correct ionic state of: Cl, K, O and C. \\
\textlcsc{ \textcolor{dgreen}{\Large \textbf{Solution}} }\\
Cl is on the 7A group and hence its charge is $1-$, whereas potassium belongs to 1A and its charge will be $1+$. Oxygen and carbon will have $2-$ and $4-$ charges. The final ionic states are: \ce{Cl^{-}}, \ce{K^{+}}, \ce{O^{2-}} and \ce{C^{4-}}.
\\
\faDiamond\ \textlcsc{ \textcolor{dgreen}{\Large \textbf{Study Check}} }\\
Identify the correct ionic state of: N and Br.\\
\flushright Answer: \ce{N^{3-}} and \ce{Br^{-}}.
\end{example}%%%%%%%%%%%%%%%%%%%%%%%% EXAMPLE BOX


\begin{figure}[h] % FUL FIGURE
\begin{center}\EmptyTable\end{center}
\caption{Ionic charges (valences) for different elements}
\label{fig:naming2}
\end{figure}% FUL FIGURE

\end{description}











\section{Ionic compounds}
Ionic compounds are chemicals resulting from the combination of a nonmetallic element with a metallic element. And example is \ce{NaCl}, which results of combining sodium (a metal) with chloride (a non metal).
\sloppy 
%\begin{marginfigure}%%%%%%%MARGIN FIGURE
%      \includegraphics{chapter5/figure4}
%      \label{fig:marginfig}
%      \caption{Table salt is an ionic compound}
%	\end{marginfigure}%%%%%%%MARGIN FIGURE
\begin{description}
\item[\docfilehook{Combining ions}{Combining ions}] Ionic compounds are the result of combining two ions: a positive (cation) and a negative (anion) ions. Each ion has a charge, depending on its location on the table. When combining two atoms you first need to arrange the ions starting from positive and followed by negative. The charges of an ion would become the coefficient of the other ion. For example \ce{Mg^{2+}} and \ce{N^{3-}} are combined as \ce{Mg3N2}:
\vspace{.5cm}\begin{center}
  \schemestart
    \chemfig{3@{e1}Mg^{2+} }  \hspace{1cm}  \chemfig{+}\hspace{1cm}   \chemfig{2@{e2}N^{3-}}   \arrow \ce{Mg3N2}
  \schemestop
  \chemmove{
    \draw[elmove=left] (e2) ..controls +(45:2cm) and +(-45:1cm) .. (e1) ;
      \draw[elmove=left]   (e1) ..controls +(45:1cm) and +(-45:2cm) .. (e2) ;
  }\end{center}\vspace{.5cm}
Another example would be the combination of \ce{Na^{+}} and \ce{O^{2-}} that would be \ce{Na2O}. You need to simplify the indexes of the formula by diving by the smallest one, always using integer values. For example,  \ce{Mg^{2+}} and \ce{O^{2-}} give  \ce{Mg2O2} that should be written as \ce{MgO}
\vspace{.5cm}\begin{center}
  \schemestart
    \chemfig{@{f1}Mg^{2+} }  \hspace{1cm}  \chemfig{+}\hspace{1cm}   \chemfig{@{f2}O^{2-}}   \arrow \ce{Mg_{\cancel{2}}O_{\cancel{2}}} \arrow  \ce{MgO} 
  \schemestop
  \chemmove{
    \draw[elmove=left] (f2) ..controls +(45:2cm) and +(-45:1cm) .. (f1) ;
      \draw[elmove=left]   (f1) ..controls +(45:1cm) and +(-45:2cm) .. (f2) ;
  }\end{center}\vspace{.5cm}

Another example that involves simplifying the formula is the chemical resulting of combining \ce{Ca^{2+}} and \ce{C^{4-}}. After combining the charges we obtain \ce{Ca4C2} that needs to be simplified dividing by the smallest number leading to \ce{Ca2C}. 
\vspace{.5cm}\begin{example} %%%%%%%%%%%%%%%%%%%%%%%% EXAMPLE BOX
Combine the following ions or give the ions given the final compound: \\
\begin{center}\begin{tabularx}{0.75\textwidth}{
    >{\centering}m{.185\linewidth} 
    *{3}{Y} }
  \toprule
\heading{Ions} & \multicolumn{3}{c}{\textbf{Combination}}   \\
    \midrule
   \ce{Li^{+}} and  \ce{O^{2-}} & 	\multicolumn{3}{c}{     }    \\
      \ce{Ca^{2+}} and  \ce{O^{2-}} & 	\multicolumn{3}{c}{     }    \\
         & 	\multicolumn{3}{c}{  \ce{Li3N}   }    \\
         & 	\multicolumn{3}{c}{  \ce{Mg2C}  }    \\
      \bottomrule
\end{tabularx}\end{center}\vspace{.5cm}
\textlcsc{ \textcolor{dgreen}{\Large \textbf{Solution}} }\\
The result of combining \ce{Li^{+}} and  \ce{O^{2-}}  is \ce{Li2O}. For \ce{Ca^{2+}} and  \ce{O^{2-}}, the resulting chemical is  \ce{CaO}. \ce{Li3N} results from the combination of  \ce{Li^{+}} and  \ce{N^{3-}}, and  \ce{Mg2C} results from  \ce{Mg^{2+}} and  \ce{C^{4-}}.\\
\faDiamond\ \textlcsc{ \textcolor{dgreen}{\Large \textbf{Study Check}} }\\
Combine the following ions or give the ions given the final compound: \ce{Na^{+}} and  \ce{F^{-}} and \ce{Na3N}.\\
\flushright Answer: \ce{NaF};  \ce{Na^{+}} and  \ce{N^{3-}}.
\end{example}%%%%%%%%%%%%%%%%%%%%%%%% EXAMPLE BOX


\item[\docfilehook{Simple ionic naming (type I ionic)}{Simple ionic naming (type I ionic)}] Type I ionic compounds result from the combination of a metal with given valence (Li, Ca, Mg, etc.) and a non metal. In order to name an ionic compound (type I ionic) you need to (a) use the name of the first element in the compound, (b) use the first syllable of the second element, and (c) finish the name of the molecule in the suffix \emph{-ide}. As an example, the formula \ce{NaCl} is named as sodium chloride and  \ce{MgCl_2} is named magnesium chloride. Another example would be:
 \begin{namingbox} {}
  \ce{CaCl2}\hfill calcium chloride 	\hfill   \ce{LiO2} \hfill lithium oxide  \\
\end{namingbox}
In order to formulate an ionic compound based on a name, we need to combine both ions by exchanging the valences (the ionic charges). For example,\ce{MgCl_2} results of the combination of \ce{Mg^{2+}} and \ce{Cl^{-}} so that the number 2 in \ce{MgCl_2} near the Cl atom is coming from the \ce{Mg^{2+}}. In other words:
\vspace{.5cm}\begin{center}
  \schemestart
    \chemfig{@{d1}Mg^{2+} }  \hspace{1cm}  \chemfig{+}\hspace{1cm}   \chemfig{2@{d2}Cl^{-}}   \arrow \ce{MgCl2}
  \schemestop
  \chemmove{
    \draw[elmove=left] (d2) ..controls +(45:2cm) and +(-45:1cm) .. (d1) ;
      \draw[elmove=left]   (d1) ..controls +(45:1cm) and +(-45:2cm) .. (d2) ;
  }\end{center}\vspace{.5cm}



The sign of the charges only indicate which element goes first in the formula: the positive element (cation) first following by the negative element (anion). For example the result of combining \ce{Na^{+}} and \ce{Cl^{-}} is \ce{NaCl} and not \ce{ClNa} as Na has positive ionic charge and has to appear first in the formula.

\begin{example} %%%%%%%%%%%%%%%%%%%%%%%% EXAMPLE BOX
Name or give the formula for the following ionic compounds: \\
\begin{tabularx}{0.75\textwidth}{
    >{\centering}m{.185\linewidth} 
    *{3}{Y} }
  \toprule
\heading{Formula} & \multicolumn{3}{c}{\textbf{Name}}   \\
    \midrule
  \ce{MgO} & 	\multicolumn{3}{c}{     }    \\
   \ce{Mg3N2}    & 	\multicolumn{3}{c}{     }    \\
         & 	\multicolumn{3}{c}{  Lithium nitride   }    \\
        & 	\multicolumn{3}{c}{  Magnesium carbide   }    \\
      \bottomrule
\end{tabularx}\\
\textlcsc{ \textcolor{dgreen}{\Large \textbf{Solution}} }\\
The name for  \ce{MgO}  is magnesium oxide. \ce{Mg3N2} is called magnesium nitride. The formula for Lithium nitride is \ce{Li3N} and the formula for Magnesium carbide is \ce{Mg2C}, result of simplifying \ce{Mg4C2} dividing by two, the smallest number.\\
\faDiamond\ \textlcsc{ \textcolor{dgreen}{\Large \textbf{Study Check}} }\\
Name or give the formula for the following ionic compounds: Sodium fluoride and \ce{Na3N}.\\
\flushright Answer: \ce{NaF};  Sodium nitride.
\end{example}%%%%%%%%%%%%%%%%%%%%%%%% EXAMPLE BOX

\item[\docfilehook{Complex ionic naming}{Complex ionic naming}] The ionic chemical \ce{NaCl} results from the combination of \ce{Na^+} and \ce{Cl^-}. The ionic charges of Na and Cl are given in Figure \ref{fig:naming2} according to the group. If the ionic chemical contains a transition metal with variable ionic charge, that is, which is not in Figure \ref{fig:naming2} then the ionic naming becomes a bit more complex. The reason is that one needs to specify the charge of the metal, explicitly in the name of the chemical. An example would be \ce{NiCl2} named as Nickel(II) chloride or  \ce{Co2O3} named as Cobalt(III) oxide. 


\item[\docfilehook{Formulate complex ionic chemicals}{Formulate complex ionic chemicals}] In this section we will learn how to name ionic chemicals containing a metal with several possible charges, that is a metal which is not included in Figure \ref{fig:naming2}. The charge of the metal has to be given in the name. As an example, the formula for Nickel(III) oxide is \ce{Ni2O3}. The reason for the formula is the combination of Nickel(III) \ce{Ni^{3+}} and oxygen \ce{O^{2-}}, that gives \ce{Ni2O3}, after crossing the charges from top to bottom. Another example is Nickel(II) oxide formulated as \ce{NiO}. This results from combining Nickel(II) \ce{Ni^{2+}} and oxygen \ce{O^{2-}} that gives \ce{Ni2O2}. After simplifying one obtains \ce{NiO}.


\item[\docfilehook{Name complex ionic chemicals}{Name complex ionic chemicals}] This section covers how to name ionic chemicals containing a metal with variable charge. In this case you need to specify the charge of the metal in the name. In order to calculate this number you will solve a simple math equation. For example, the name of \ce{Mn2O3} is Manganese(III) oxide. How do we get this name? Manganese has several charges as it is not in Figure \ref{fig:naming2}, lets use x for its charge \ce{Mn^x} and oxygen has a charge of two \ce{O^{2-}}. After combining \ce{Mn^x} and \ce{O^{2-}} the resulting formula would be  \ce{Mn2Ox}. By comparison with the given formula, \ce{Mn2O3} , x has to be three and hence the charge of Mn has to be three. Therefore, the final name would be Manganese(III) oxide.

\item[\docfilehook{Properties of ionic compounds}{Properties of ion compounds}] Ionic compounds normally have high melting points and are solid at normal conditions. An typical ionic compound would be \ce{NaCl}, cooking salt.
\item[\docfilehook{The ionic bond}{The ionic bond}] Atoms of an ionic compound are connected by means of an ionic bond. In an ionic bond, one element gives away electrons (the cation) and the other one receives electrons (the anion). As an example, in the \ce{NaCl} molecule Na gives away an electron to Cl and the molecule results from the combinations of \ce{Na^+} and \ce{Cl^+}. In an ionic compound the element on the left is positive and the one on the right is negative.

\end{description}

\begin{example} %%%%%%%%%%%%%%%%%%%%%%%% EXAMPLE BOX
Name or give the formula for the following ionic compounds: \\
\begin{tabularx}{0.75\textwidth}{
    >{\centering}m{.185\linewidth} 
    *{3}{Y} }
  \toprule
\heading{Formula} & \multicolumn{3}{c}{\textbf{Name}}   \\
    \midrule
  \ce{MnO} & 	\multicolumn{3}{c}{     }    \\
   \ce{Fe3N2}    & 	\multicolumn{3}{c}{     }    \\
         & 	\multicolumn{3}{c}{  Cobalt(II) carbide   }    \\
        & 	\multicolumn{3}{c}{  Iron(II) oxide   }    \\
      \bottomrule
\end{tabularx}\\
\textlcsc{ \textcolor{dgreen}{\Large \textbf{Solution}} }\\
All the chemicals on this example contain a metal that can have several charges, and hence, we need to specify the ionic charge on the name. \ce{MnO} results from  \ce{Mn^x} and  \ce{O^{2-}}. After combining the ions, the formula would be  \ce{Mn2Ox}, a formula that needs to be compared to  \ce{MnO}. The formulas do not look similar, so lets make them more similar by dividing by two so that \ce{MnO$\frac{x}{2}$} resembles \ce{MnO}. By comparing x has to be 2 and hence the name is Manganese(II) oxide. The name for \ce{Fe3N2}  would be Iron(II) nitride. The valence of Iron comes from combining  \ce{Fe^x} and  \ce{N^{3-}} that gives \ce{Fe3Nx}. By comparison with \ce{Fe3N2}  x has to be two and the name is Iron(II) nitride. the formula for Cobalt(II) carbide would be  \ce{Co2C} as Cobalt(II) is \ce{Co^{2+}} and carbide is \ce{C^{4-}}. After combining the ions one obtains   \ce{Co4C2} that gives \ce{Co2C}. Finally, the formula for Iron(II) oxide is \ce{FeO} as Iron(II) is \ce{Fe^{2+}} and oxide is \ce{O^{2-}} that gives \ce{Fe2O2} and simplifying one obtains \ce{FeO}.\\
\faDiamond\ \textlcsc{ \textcolor{dgreen}{\Large \textbf{Study Check}} }\\
Name or give the formula for the following ionic compounds: Manganese(IV) oxide and \ce{AuCl}.\\
\flushright Answer: \ce{MnO2};  Gold(I) chloride.
\end{example}%%%%%%%%%%%%%%%%%%%%%%%% EXAMPLE BOX

\section{Covalent compounds}
Covalent compounds are chemicals resulting from the combination of nonmetallic elements. And example is \ce{CO2}, which results of combining carbon (a non metal) with oxygen (a non metal).
\sloppy 
%\begin{marginfigure}%%%%%%%MARGIN FIGURE
%      \includegraphics{chapter5/figure3}
%      \label{fig:marginfig}
%      \caption{Water is a covalent compound}
%	\end{marginfigure}%%%%%%%MARGIN FIGURE
\begin{description}
\item[\docfilehook{Covalent naming}{Covalent naming}] In order to name a covalent compound you need to (a) use the name of the first element in the compound, (b) use the first syllable of the second element, and (c) finish the name of the molecule in the suffix \emph{-ide}. More importantly, you need to use prefixes that indicate the number of atoms in the molecule. See Table \ref{fig:naming1} for a list of the different equivalencies between prefixes and number.  As an example, the formula \ce{CH4} is named as carbon tetrahydride. Similarly, a covalent chemical name can be translated into a formula (we call this to formulate a chemical with a given name), and the formula for carbon monoxide would be \ce{CO}. When the vowels \emph{a} and \emph{o} appear together, the first vowel is omitted as in carbon monoxide instead of carbon \sout{monooxide}. Another example would be \ce{N2O} named as dinitrogen oxide, and the name sulfur hexafluoride corresponds to the formula \ce{SF6}. The prefix mono is omitted in the first element of the name, and for example you will not name the chemical \ce{CO} as \sout{monocarbon} monoxide, you would just say carbon monoxide. A final example of a covalent compound:
 \begin{namingbox} {}
  \ce{N2O5}\hfill dinitrogen pentoxide 	\hfill   \ce{BCl3} \hfill Boron trichloride  \\
%    \ce{Mn3(BO3)2}\hfill Manganese(II) borate 	\hfill   \ce{Mn(H2BO3)2}\hfill Manganese(II) dihydrogenborate ({\small hydrosalt}) 
\end{namingbox}




%\begin{marginfigure}%%%%%%%MARGIN FIGURE
%\begin{tcolorbox}[tab2,tabularx={X|Y}]%%%% FANCY COLOR TABLE
%Prefix & number              \\\hline\hline
%\end{tcolorbox}%%%% FANCY COLOR TABLE
%\caption{Prefixes used to name covalent compounds} 
%\label{fig:covalent3}
%         
% \end{marginfigure}%%%%%%%MARGIN FIGURE



\begin{center}
\refstepcounter{table} \label{tab:naming1}
%\begin{table}[ht]
\fontfamily{ppl}\selectfont
\begin{tabular}{llll}
\rowcolor{black!45}
\toprule
\multicolumn{4}{l}{\hypersetup{colorlinks,linkcolor={white}} \cellcolor{black}\color{white}\bfseries\small Table \ref{tab:naming1} Prefixes used to name covalent compounds } \\
\midrule
 \rowcolor{gray!10} Prefix & number & Prefix & number \\
\midrule
Mono &    1    & Hexa&    6        \\
Di &     2      &      Hepta &  7  \\  
Tri &     3       &    Octa &  8  \\
Tetra &     4   &      Nona&  9   \\  
Penta &    5   &       Deca &  10 \\    
\bottomrule
\end{tabular}\end{center}

\begin{example} %%%%%%%%%%%%%%%%%%%%%%%% EXAMPLE BOX
Name of give the name of the following covalent chemicals: \\
\begin{tabularx}{0.75\textwidth}{
    >{\centering}m{.185\linewidth} 
    *{3}{Y} }
  \toprule
\heading{Formula} & \multicolumn{3}{c}{\textbf{Name}}   \\
    \midrule
   \ce{NO } & 	\multicolumn{3}{c}{     }    \\
    \ce{CS2 } & 	\multicolumn{3}{c}{     }    \\
         & 	\multicolumn{3}{c}{  Sulfur Dioxide   }    \\
         & 	\multicolumn{3}{c}{  Nitrogen Trichloride   }    \\
      \bottomrule
\end{tabularx}\\
\textlcsc{ \textcolor{dgreen}{\Large \textbf{Solution}} }\\
All chemicals in this example are covalent as they result of the combination of nonmetals. In order to name them, we need to use prefixes and finish the sufix with -ide. The first chemical is called nitrogen oxide. \ce{CS2 } is called carbon disulfide. The formula for sulfur dioxide and nitrogen trichoride are respectively  \ce{SO2 } and  \ce{NCl3}.\\
\faDiamond\ \textlcsc{ \textcolor{dgreen}{\Large \textbf{Study Check}} }\\
Name of give the name of the following covalent chemicals: \ce{SCl2} and diboron thrioxide.\\
\flushright Answer: sulfur dichloride and \ce{B2O3}.
\end{example}%%%%%%%%%%%%%%%%%%%%%%%% EXAMPLE BOX

\item[\docfilehook{Properties of covalent compounds}{Properties of covalent compounds}] At normal conditions, covalent compounds may exist as solids, liquids, or gases. Covalent compounds do not exhibit any electrical conductivity, either in pure form or when dissolved in water. A typical covalent compound would be \ce{H2O}, water.
\item[\docfilehook{The covalent bond}{The covalent bond}] Atoms in a covalent compound are connected by means of a covalent chemical bond. In a covalent bond, both atoms connected share the electrons. As an example, the \ce{HCl} molecule has an hydrogen and a chlorine atom connected by means of a covalent bond, in which each atoms share the electrons of the bond. 

\end{description}


%\begin{marginfigure}[-5cm]%%%%%%%MARGIN FIGURE
%\label{tab:acids1}
%\begin{tcolorbox}[tab2,tabularx={X|Y}]%%%% FANCY COLOR TABLE
%Acid & Name             \\\hline\hline
%\ce{HCl} &  Hydrochloric acid             \\\hline
%\ce{HI} &  Hydroiodic acid             \\\hline
%\ce{HF} &  Hydrofluoric acid             \\\hline
%\ce{HCN} &  Hydrocyanic acid           
%\end{tcolorbox}%%%% FANCY COLOR TABLE
%\caption{List of hydracids} \label{table5:2}             
% \end{marginfigure}%%%%%%%MARGIN FIGURE
 %\resizeableyellownote{2.5}{1}{Add this table into your flashcard.}

\section{Naming acids \& bases}
In this section we will learn how to name acids and bases. Acids normally have common names (e.g. sulfuric acid) and its naming does not follow modern rules. Names and formulas of acids are listed in tables. Differently, bases (e.g. sodium hydroxide) are named in a standard way.\sloppy 
\begin{description}
\item[\docfilehook{Bases or hydroxides}{Bases or hydroxides}] Bases (hydroxides) result from the combination of a metal and the hydroxide anion (\ce{OH^-}). Examples are \ce{NaOH} or \ce{Ca(OH)2}. The name of a base starts by the name of the cation finishing by the word \begin{it}hydroxide\end{it}. An example is \ce{NaOH} named as \begin{it}sodium hydroxide\end{it}, or \ce{Ca(OH)2}, named as \begin{it}calcium hydroxide\end{it}. The word  \begin{it}hydroxide\end{it} refers to the \ce{OH^-} ion, and hence Sodium hydroxide results from combining \ce{Na^+} and \ce{OH^-}, and Calcium hydroxide from combining \ce{Ca^{2+}} and \ce{OH^-}. More examples of hydroxides:
 \begin{namingbox} {}
  \ce{Mg(OH)2}\hfill Magnesium hydroxide\\
\end{namingbox}



\item[\docfilehook{Acids}{Acids}] Acids--in particular inorganic acids--are chemicals that normally contain hydrogen at the beginning of its formula. For example, \ce{HCl} or \ce{H2SO4}. \ce{HCl} is an hydracid and is named as \begin{it}hydrochloric acid\end{it}, whereas  \ce{H2SO4} is an oxoacid that contains oxygen named as \begin{it}sulfuric acid\end{it}.
The names of acids are not standard and they come from common names employed in the field for many years. Table \ref{tab:naming2} contains a list of the most important oxoacids and hydracids. 
More examples of acids:
 \begin{namingbox} {}
  \ce{HNO3}\hfill Nitric acid\hfill 
    \ce{HF}\hfill Hydrofluoric acid\\
\end{namingbox}





\begin{example} %%%%%%%%%%%%%%%%%%%%%%%% EXAMPLE BOX
Name or give the formula for the following acids and bases. Indicate whether the compound is an acid or a base. \\
\begin{center}\begin{tabularx}{0.75\textwidth}{
    >{\centering}m{.185\linewidth} 
    *{4}{Y} }
  \toprule
\heading{Formula} & \textbf{Acid/Base}& \multicolumn{2}{c}{\textbf{Name}}   \\
    \midrule
  \ce{HCN} & &	\multicolumn{2}{c}{     }    \\
   \ce{KOH}   & & 	\multicolumn{2}{c}{     } &   \\
         & &	\multicolumn{2}{l}{  Carbonic acid   } &   \\
        & &	\multicolumn{2}{l}{  Lithium hydroxide   }  &  \\
      \bottomrule
\end{tabularx}\end{center}\vspace{.5cm}
\textlcsc{ \textcolor{dgreen}{\Large \textbf{Solution}} }\\
 \ce{HCN} is an acid named hydrocyanic acid. \ce{KOH}  is a base called potassium hydroxide. The formula for Carbonic acid is \ce{H2CO3}, and Lithium hydroxide is a base with formula \ce{LiOH}. \\
\faDiamond\ \textlcsc{ \textcolor{dgreen}{\Large \textbf{Study Check}} }\\
Name or give the formula for the following ionic compounds: phosphoric acid and \ce{Mg(OH)2}.\\
\flushright Answer: \ce{H3PO4};  magnesium hydroxide.
\end{example}%%%%%%%%%%%%%%%%%%%%%%%% EXAMPLE BOX
\item[\docfilehook{Oxidation states of oxoacids}{Oxidation states of oxoacids}] Consider the following set of acids: \ce{HClO}, \ce{HClO2}, \ce{HClO3} and \ce{HClO4}. We say \ce{Cl} in these acids have different oxidation state or different oxidation number. This section will cover the calculation of the oxidation state of the central atom of an oxoacid. \\
Let us address the oxoacid: \ce{H\underline{Cl}O3}. The goal is to calculate the oxidation number of the underline element, \ce{Cl}. In order to do this we will follow a set of simple rules. First, we will use the valences as the oxidation number of the elements to the right and to the left of the central atom. Then, we will assign an unknown oxidation state of $x$ to the central atom. After that we will set up a equation so that the sum of all oxidation numbers equals to the charge of the acid, if any. In this formula, we will include the atomic coefficients. In the case of \ce{H\underline{Cl}O3}, the equation would be:
\[1+x+3\cdot (-2)=0\]
as the number of oxygens is three, we will have to time by three the valence of oxygen. The number zero results from the charge of the acid. If we solve for x, we obtain: $x=5$. That is, the oxidation state of \ce{Cl} on \ce{H\underline{Cl}O3} is $5$ and this is represented as\ce{H\underline{Cl\textsuperscript{V}}O3}.

\item[\docfilehook{Oxidizing and reducing character of oxoacids}{Oxidizing and reducing character of oxoacids}]    
The importance of the oxidation state of the central elements of an oxoacid results from the fact that acids with high or low oxidation states, tends to be very reactive, being sometimes capable of completely dissolving metals. We call these oxidizing (or reducing) acids. For example, \ce{HNO3} and \ce{H2SO4} and both oxidizing acids and these acids will solve for example a piece of copper. Similarly, acids with very small or negative oxidation numbers can be very reactive as well. These acids are called reducing acid sor agents.
Let us compare two oxoacids in order to elaborate more on the terminology used to describe redox numbers. For example, let us compare \ce{H\underline{Cl\textsuperscript{V}}O3} and \ce{H\underline{Cl\textsuperscript{III}}O2}. 
We say  Cl on \ce{H\underline{Cl\textsuperscript{V}}O3} has a larger redox number than \ce{H\underline{Cl\textsuperscript{III}}O2}. We can also say, Cl in \ce{H\underline{Cl\textsuperscript{V}}O3} is more oxidized than Cl on \ce{H\underline{Cl\textsuperscript{III}}O2}. 
Finally, we can also say, \ce{H\underline{Cl\textsuperscript{V}}O3} is more reducing than \ce{H\underline{Cl\textsuperscript{III}}O2}. Again, the terms associated with high redox numbers are oxidized and reducing, and the terms associated with low redox numbers are reduced and oxidizing.

It is important to note that ultimately the oxidation state of an element is related to the number of electrons of the element. The more electrons the smaller--the more negative--the oxidation state. In other words, large oxidation states result from losing electrons.

\begin{example} %%%%%%%%%%%%%%%%%%%%%%%% EXAMPLE BOX
Calculate the redox number of S in the following acids and indicate the more oxidizing acid: \ce{H2\underline{S}2O6} named dithionic acid and \ce{H2\underline{S}O4} named sulfuric acid.\\
\textlcsc{ \textcolor{dgreen}{\Large \textbf{Solution}} }\\
We will set up the redox formula for the first acid (\ce{H2\underline{S}2O6}), given that the redox number of H is $+1$ and the redox number of O is $-2$.
\[2\cdot 1+2\cdot x+6\cdot (-2)=0\]
Solving for x: 
\[2 +2\cdot x-12=0\;\;\; \text{we have that } x=\frac{12-2}{2}\]
The oxidation state of \ce{S} in \ce{H2\underline{S}2O3} is $+5$. For the second acid (\ce{H2\underline{S}O4}):
\[2\cdot 1+x+4\cdot (-2)=0\]
Solving for x: 
\[2 +x-8=0\;\;\; \text{we have that } x=\frac{8-2}{1}\]
that gives a redox of $6$. If we compare both acids the smaller the redox number the more reduced is the central element and the more oxidizing the acid is. Therefore, \ce{H2\underline{S}2O3} is more oxidizing than \ce{H2\underline{S}O4}. 
\\
\faDiamond\ \textlcsc{ \textcolor{dgreen}{\Large \textbf{Study Check}} }\\
Calculate the redox number of the following acids: \ce{H2MnO4} and \ce{H2Cr2O7}.\\
\flushright Answer: $+6$.
\end{example}%%%%%%%%%%%%%%%%%%%%%%%% EXAMPLE BOX


\end{description}

\begin{center}
\refstepcounter{table} \label{tab:naming2}
%\begin{table}[ht]
\fontfamily{ppl}\selectfont
\begin{tabular}{lllll}
\rowcolor{black!45}
\toprule
\multicolumn{5}{l}{\hypersetup{colorlinks,linkcolor={white}} \cellcolor{black}\color{white}\bfseries\small Table \ref{tab:naming2} Names of oxoacids and oxosalts (top table) and hyracids (bottom table). Yellow rows need to be remembered. } \\
\midrule
 \rowcolor{gray!10} Element&Oxoacid & Oxoacid Name & Oxoasalt & Oxoasalt  Name \\
\midrule
\rowcolor{yellow!10} Manganese &\ce{HMnO4}  &  Permanganic  Acid      &    \ce{MnO4^-}    &     Permanganate    \\
 &\ce{H2MnO4}  &  Manganic acid     &    \ce{MnO4^{-2}}    &     Manganate    \\

\rowcolor{yellow!10}Carbon&   \ce{H2CO3}  &  Carbonic  Acid      &    \ce{CO3^{-2}}    &     Carbonate    \\
\rowcolor{yellow!10}Nitrogen&   \ce{HNO3}  &  Nitric   Acid     &    \ce{NO3^-}    &     Nitrate    \\
&   \ce{HNO2}  &  Nitrous  Acid      &    \ce{NO2^-}    &     Nitrite    \\
\rowcolor{yellow!10}Phosphorus&   \ce{H3PO4}  &  Phosphoric   Acid     &    \ce{PO4^{-3}}    &     Phosphate    \\
\rowcolor{yellow!10}Sulfur&   \ce{H2SO4}  &  Sulfuric  Acid      &    \ce{SO4^{-2}}    &     Sulfate    \\
&   \ce{H2SO3}  &  Sulfurous  Acid      &    \ce{SO3^{-2}}    &     Sulfite   \\
 &  \ce{H2S2O2}  &  Thiosulfurous  Acid    &    \ce{S2O2^{-2}}    &     Thiosulfite   \\
&      \ce{H2S2O3}  &  Thiosulfuric  Acid    &    \ce{S2O3^{-2}}    &     Thiosulfate   \\
&      \ce{H2S2O7}  &  Disulfuric acid    &    \ce{S2O7^{-2}}    &    Disulfate   \\
&      \ce{H2S2O8}  &  Peroxodisulfuric acid    &    \ce{S2O8^{-2}}    &    Peroxodisulfate   \\
Chlorine&  \ce{HClO4}  &  Perchloric Acid       &    \ce{ClO4^-}    &     Perchlorate    \\
&  \ce{HClO3}  &  Chloric acid       &    \ce{ClO3^-}    &     Chlorate    \\
&  \ce{HClO2}  &  Chlorous acid	      &    \ce{ClO2^-}    &     Chlorite    \\
&  \ce{HClO}  &  Hypochlorous acid	      &    \ce{ClO^-}    &     Hypochlorite    \\

Iodine &  \ce{HIO4}  &  Periodic  Acid      &    \ce{IO4^-}    &     Periodate    \\
Chromium &      \ce{H2CrO4}  &  Chromic acid      &    \ce{CrO4^{2-}}    &     Chromate    \\
\rowcolor{yellow!10} &     \ce{H2Cr2O7}  &  Dichromic acid      &    \ce{Cr2O7^{2-}}    &     Dichromate    \\
Boron &     \ce{H3BO3}  &  Boric acid      &    \ce{BO3^{3-}}    &     Borate    \\
\midrule
 \rowcolor{gray!10}Hydracid &Hydracid Name &Hydracid  & Hydracid Name&  \\
\midrule
   \ce{HCl}  &  Hydrochloric acid      &    \ce{HBr}    &     Hydrobromic acid &   \\
   \ce{HI}  &  Hydroiodic acid      &    \ce{HF}    &     Hydrofluoric acid &   \\
   \ce{HCN}  &  Hydrocyanic acid      &    \ce{H2S}    &     Hydrosulfuric acid &   \\
\bottomrule
\end{tabular}\end{center}
%\caption{Names of oxoacids and oxosalts }
%\label{table5:3}
%\end{table} 







\section{Naming complex salts  \& common chemicals}
At this point we saw the naming and formulation of ionic (e.g. \ce{NaCl}) and covalent compounds (e.g. \ce{CO2}). This section covers the naming of complex salts: oxosalts and hydrosalts. In general, salts (oxosalts or hydrosalts) are the result of mixing an oxoacid and a base. They tend to look more complex than simple ionic or covalent compounds as they have at least three different elements. An example of oxosalt would be \ce{CaSO4} called calcium carbonate. An example of hydrosalt would be \ce{NaHSO4} which is called sodium monohydrosulfate. This section will also cover the naming of hydrates (e.g. \ce{CaSO4 . H2O}), that are compounds containing water molecules inside its structure. Before being able to name these complex chemicals it is convenient to practice combining ions, without paying attention to the naming.
\sloppy 
\begin{description}
\item[\docfilehook{Combining ions }{Combining ions}] 
In order to combine two ions, you first arrange the positive ion in the left followed by the negative ion in the right, to then cross the ionic charges from the top of the ion to the bottom of the opposite ion. The positive and negative charges are not carried. If the ions have more than one element we have to use parenthesis. An example would be combining \ce{Ca^{2+}} and \ce{PO4^{3-}} leading  to \ce{Ca3(PO4)2}: 

\vspace{.5cm}\begin{center}
  \schemestart
    \chemfig{3@{Cl1}Ca^{2+} }  \hspace{1cm}  \chemfig{+}\hspace{1cm}   \chemfig{2P@{Cl2}O_4^{3-}}   \arrow \ce{Ca3(PO4)2}
  \schemestop
  \chemmove{
    \draw[elmove=left] (Cl2) ..controls +(45:2cm) and +(-45:1cm) .. (Cl1) ;
      \draw[elmove=left]   (Cl1) ..controls +(45:1cm) and +(-45:2cm) .. (Cl2) ;
  }\end{center}\vspace{.5cm}
We would simplify in case the charges compensate with each other. An example would be combining \ce{Mg^{2+}} and \ce{HPO4^{2-}} leading  to \ce{MgHPO4}

\vspace{.5cm}\begin{center}
  \schemestart
    \chemfig{@{a1}Mg^{2+} }  \hspace{1cm}  \chemfig{+}\hspace{1cm}   \chemfig{HP@{a2}O_4^{2-}}   \arrow \ce{Mg_{\cancel{2}}(HPO4)_{\cancel{2}}} \arrow\ce{MgHPO4}
  \schemestop
  \chemmove{
    \draw[elmove=left] (a2) ..controls +(45:2cm) and +(-45:1cm) .. (a1) ;
      \draw[elmove=left]   (a1) ..controls +(45:1cm) and +(-45:2cm) .. (a2) ;
  }\end{center}\vspace{.5cm}

%\begin{example} %%%%%%%%%%%%%%%%%%%%%%%% EXAMPLE BOX
%Combine the following ions or break down the following chemicals into ions:\\
%\begin{tabularx}{0.75\textwidth}{>{\centering}m{.4\textwidth} *{2}{Y} }
%  \toprule
%\heading{Ions} & \multicolumn{2}{c}{\textbf{Combination}}   \\
%    \midrule
%   \ce{Li^{+}} and  \ce{PO4^{3-}} & 	\multicolumn{2}{c}{     }    \\
%      \ce{Ca^{2+}} and  \ce{NO3^{-}} & 	\multicolumn{2}{c}{     }    \\
%         & 	\multicolumn{2}{c}{  \ce{Li2CO3}   }    \\
%         & 	\multicolumn{2}{c}{  \ce{Mg2NO3}  }    \\
%      \bottomrule
%\end{tabularx}\\
%\textlcsc{ \textcolor{dgreen}{\Large \textbf{Solution}} }\\
%By combining   \ce{Li^{+}} and  \ce{PO4^{3-}} one obtains \ce{Li3PO4}, and  \ce{Ca^{2+}} and  \ce{NO3^{-}} gives  \ce{Ca(NO3)2}. \ce{Li2CO3} results from \ce{Li^{+}} and  \ce{CO3^{2-}}  and \ce{Mg2NO3} from combining \ce{Mg^{2+}} with \ce{NO3^{-}}.  \\
%\faDiamond\ \textlcsc{ \textcolor{dgreen}{\Large \textbf{Study Check}} }\\
%Combine the following ions or give the ions given the final compound: \ce{Ca^{2+}} and  \ce{NO3^{-}} and \ce{K2SO4}.\\
%\flushright Answer: \ce{Ca2NO3};  \ce{K^{+}} and  \ce{SO4^{2-}}.
%\end{example}%%%%%%%%%%%%%%%%%%%%%%%% EXAMPLE BOX




\item[\docfilehook{Naming Oxosalts}{Naming Oxosalts}] 
The names of the oxosalts are constructed by combining the name of the first element--you need to specify its charge in the case of a transition metal element with different possible charges--followed by the name of the oxosalt from Table \ref{tab:naming2}. For example, the name of \ce{MgSO4} is magnesium sulfate, as \ce{Mg^{2+}} is magnesium and \ce{SO4^{2-}} is sulfate. Another example is\ce{Fe2(CO3)3} called Iron(III) carbonate. A final example:
 \begin{namingbox} {}
  \ce{NO3^{-}}\hfill Nitrate 	\hfill   \ce{LiNO3} \hfill Litium nitrate ({\small oxosalt})\\
%    \ce{Mn3(BO3)2}\hfill Manganese(II) borate 	\hfill   \ce{Mn(H2BO3)2}\hfill Manganese(II) dihydrogenborate ({\small hydrosalt}) 
\end{namingbox}

\item[\docfilehook{Formulating Oxosalts}{Formulating Oxosalts}] 
In the case that you know the name of an oxosalt and you want to obtain its formula, you first need to arrange the positive ion in the left followed by the negative ion in the right, to then cross the ionic charges from the top of the ion to the bottom of the opposite ion.
For example, calcium nitrate results from the combination of \ce{Ca^{2+}} calcium and \ce{NO3^{-}}, nitrate. By combining the two ions we obtain the final formula as \ce{Ca(NO3)2}:
\vspace{.5cm}\begin{center}
  \schemestart
    \chemfig{@{b1}Ca^{2+} }  \hspace{1cm}  \chemfig{+}\hspace{1cm}   \chemfig{N@{b2}O_3^{-}}   \arrow \ce{Ca(NO3)2} 
  \schemestop
  \chemmove{
    \draw[elmove=left] (b2) ..controls +(45:2cm) and +(-45:1cm) .. (b1) ;
      \draw[elmove=left]   (b1) ..controls +(45:1cm) and +(-45:2cm) .. (b2) ;
  }\end{center}\vspace{.5cm}


\begin{example} %%%%%%%%%%%%%%%%%%%%%%%% EXAMPLE BOX
Name of give the name of the following oxosalts: \\
\begin{tabularx}{0.75\textwidth}{>{\centering}m{.185\linewidth}  *{3}{Y} }
  \toprule
\heading{Formula} & \multicolumn{3}{c}{\textbf{Name}}   \\
    \midrule
   \ce{K2SO4 } & 	\multicolumn{3}{c}{     }    \\
    \ce{Na2CO3 } & 	\multicolumn{3}{c}{     }    \\
         & 	\multicolumn{3}{c}{  Nickel(II) carbonate   }    \\
         & 	\multicolumn{3}{c}{  Sodium phosphate   }    \\
      \bottomrule
\end{tabularx}\\
\textlcsc{ \textcolor{dgreen}{\Large \textbf{Solution}} }\\
\ce{K2SO4} is named potassium sulfate, as \ce{K^+} is potassium and \ce{SO4^{2-}} stands for sulfate. \ce{Na2CO3 }  is sodium carbonate. Nickel(II) carbonate is \ce{NiCO3} and sodium phosphate  is \ce{Na3PO4 }.\\
\faDiamond\ \textlcsc{ \textcolor{dgreen}{\Large \textbf{Study Check}} }\\
Name of give the name of the following oxosalts: \ce{FeSO4} and Iron(III) sulfate.\\
\flushright Answer: Iron(II) sulfate and \ce{Fe2(SO4)3}.
\end{example}%%%%%%%%%%%%%%%%%%%%%%%% EXAMPLE BOX

\item[\docfilehook{Naming Hydrosalts}{Naming Hydrosalts}] 
Hydrosalts are related to oxosalts (e.g. \ce{Na2SO4}) and they resemble these chemicals while having hydrogen atoms in their chemical formula, in between the oxosalt cation and anion (e.g. \ce{NaHSO4}). That is the reason they are called hydrosalts as they are oxosalts with hydrogen. For example, \ce{NaHSO4} is named sodium monohydrogensulfate. In order to understand this name, we will first focus on the second part on the name, monohydrogensulfate that represents the hydrosalt anion.

The name monohydrogensulfate (\ce{HSO4^{-}}) comes from adding a proton (\ce{H^+}) to a sulfate cation (\ce{SO4^{2-}}). Mind that protons (\ce{H^+}) are positively charged and therefore if we add a single \ce{H^+} to a sulfate cation (\ce{SO4^{2-}}) the charge will have to decrease a single unit, giving us \ce{HSO4^{-}}. As we can see, the name of hydrosalt anions are directly related to the oxosalt anion and the number of hydrogens in the hydrosalt name. For example, phosphate (\ce{PO4^{3-}}) is an oxosalt anion whereas hydrogenphosphate (\ce{HPO4^{2-}}) and dihydrogenphosphate (\ce{H2PO4^{-}}) are both hydrosalt anions. An explanation about the charges: as phosphate has three negative charges, hydrogenphosphate has to have one less charge (that is $2-$) and dihydrogenphosphate hast to have two less negative charges (that is $-1$). Some final hydrosalt anions examples:

 \begin{namingbox} {}
  \ce{CO3^{-2}}\hfill carbonate ({\small oxosalt})	\hfill   \ce{HCO3^{-}}\hfill monohydrogen carbonate ({\small hydrosalt})\\
    \ce{BO3^{-3}}\hfill borate ({\small oxosalt})	\hfill   \ce{H2BO3^{-}}\hfill dihydrogen borate ({\small hydrosalt}) 
\end{namingbox}
Above we saw how to name just the ending of the hydrosalt. Now we can move forward to the naming of hyrdrosalts. We just need to add the name of the element in the first place, and for example \ce{NaH2BO3} would be named sodium dihydrogenborate. If the first ion--the cation--is a transition metal cation (a type two cation) we need to include in parenthesis the valence of the cation. For example, \ce{Fe(H2BO3)2} would be named iron(II) dihydrogenborate. More examples:
 \begin{namingbox} {}
  \ce{Na2CO3}\hfill sodium carbonate 	\hfill   \ce{NaHCO3} \hfill sodium monohydrogen carbonate ({\small hydrosalt})\\
%    \ce{Mn3(BO3)2}\hfill Manganese(II) borate 	\hfill   \ce{Mn(H2BO3)2}\hfill Manganese(II) dihydrogenborate ({\small hydrosalt}) 
\end{namingbox}

\begin{example} %%%%%%%%%%%%%%%%%%%%%%%% EXAMPLE BOX
Name or formulate the following hydrosalts:\\
\begin{center}\begin{tabularx}{0.95\textwidth}{>{\centering}m{.4\textwidth} *{2}{Y} }
  \toprule
\heading{Formula} & \multicolumn{2}{c}{\textbf{Name}}   \\
    \midrule
     & 	\multicolumn{2}{c}{ Magnesium hydrogensulfate    }    \\
       & 	\multicolumn{2}{c}{  Sodium hydrogen carbonate   }    \\
       \ce{LiHCO3}   & 	\multicolumn{2}{c}{    }    \\
       \ce{MgH2PO4}   & 	\multicolumn{2}{c}{   }    \\
      \bottomrule
\end{tabularx}\end{center}\vspace{0.5cm}
\textlcsc{ \textcolor{dgreen}{\Large \textbf{Solution}} }\\
The formula of Magnesium hydrogensulfate   is \ce{Mg(HSO4)2} as the formula for monohydrogen sulfate is  \ce{HSO4^-} and the valence of magnesium is \ce{Mg^{2+}}. The formula for Sodium monohydrogen carbonate is \ce{NaHCO3} as it results from combining \ce{Na^+} and \ce{HCO3^-}. Mind monohydrogen carbonate results from adding a hydrogen ion \ce{H^+} to a carbonate \ce{CO3^{2-}} ion. The name for   \ce{LiHCO3}  is lithium monohydrogen carbonate, whereas the name for \ce{MgH2PO4}  is magnesium dihydrogenphosphate.
 \\
\faDiamond\ \textlcsc{ \textcolor{dgreen}{\Large \textbf{Study Check}} }\\
Name or formulate the following hydrogensalts: \ce{LiHS2O3}, \ce{LiH2PO4} and sodium hydrogenphosphate.\\
\flushright Answer: lithium monohydrogenthiosulfate;  Lithium dihydrogenphosphate and  \ce{Na2HPO3}.
\end{example}%%%%%%%%%%%%%%%%%%%%%%%% EXAMPLE BOX

\item[\docfilehook{Hydrates }{Hydrates}] 
Some chemicals contain water molecules trapped in its structure and therefore water molecules (\ce{H2O}) are often indicated in chemical formulas.  These types of chemicals containing water are called \emph{hydrates}, precisely because hydrate means water. Examples of hydrates are: \ce{BeSO4 . 4H2O} or \ce{CuSO4 . 5H2O} called respectively beryllium sulfate tetrahydrate and cupper(II) sulfate pentahydrate. In order to formulate hydrates you just need to use prefixes such as mono, di, tetra--the same ones we use to name covalent chemicals--to indicate the number of water molecules in the chemical and end the name with \emph{hydrate}. As a note,  warming up hydrates (e.g. \ce{BeSO4 . 4H2O}) results on the release of water producing a dehydrated or \emph{anhydrous} compound (e.g. \ce{BeSO4}). A final example of hydrate naming:
 \begin{namingbox} {}
\ce{Na2SO4 . 5H2O} \hfill {\small (a hydrate)}\hfill Sodium sulfate pentahydrate 
\end{namingbox}
\begin{example} %%%%%%%%%%%%%%%%%%%%%%%% EXAMPLE BOX
Name or formulate the following hydrates:\\
\begin{center}\begin{tabularx}{0.95\textwidth}{>{\centering}m{.4\textwidth} *{2}{Y} }
  \toprule
\heading{Formula} & \multicolumn{2}{c}{\textbf{Name}}   \\
    \midrule
     & 	\multicolumn{2}{c}{ Nickel(II) permanganate dihydrate  }    \\
       & 	\multicolumn{2}{c}{  Sodium nitrate monohydrate   }    \\
       \ce{Na2CO3 . 10H2O}   & 	\multicolumn{2}{c}{    }    \\
       \ce{MgSO4 . 7H2O}   & 	\multicolumn{2}{c}{   }    \\
      \bottomrule
\end{tabularx}\end{center}\vspace{0.5cm}
\textlcsc{ \textcolor{dgreen}{\Large \textbf{Solution}} }\\
The formula for Nickel(II) permanganate  is \ce{Ni(MnO4)2}, therefore the formula for Nickel(II) permanganate dihydrate is \ce{Ni(MnO4)2 . 2H2O}. The formula for Sodium nitrate is  \ce{NaNO3}, therefore \ce{NaNO3 . H2O} is Sodium nitrate monohydrate. The name for \ce{Na2CO3 . 10H2O} is sodium carbonate decahydrate and \ce{MgSO4 . 7H2O} is magnesium sulfate heptahydrate.
 \\
\faDiamond\ \textlcsc{ \textcolor{dgreen}{\Large \textbf{Study Check}} }\\
Name or formulate the following hydrates: \ce{LiNO3 . H2O}, \ce{Na3PO4 . 3H2O} and sodium sulfate tetrahydrate.\\
\flushright Answer: lithium nitrate monohydrate;  sodium phosphate trihydrate and  \ce{Na2SO4 . 4H2O}.
\end{example}%%%%%%%%%%%%%%%%%%%%%%%% EXAMPLE BOX


\item[\docfilehook{Common naming}{Common naming}] 
Some of the chemicals are normally referred by a common name that does not involve the use of any chemical naming rules. An example would be \ce{H2O} normally referred as water instead of its standard name that is dihydrogen oxide. You can find more names in Table \ref{tab:naming3}. Another example:\\
 \begin{namingbox} {}
 \ce{NaCl} \hfill Sodium chloride ({\small standard name})\hfill  Table salt ({\small common name})  
\end{namingbox}
%\resizeableyellownote{2.5}{1}{Add this table into your flashcard.}
%\begin{marginfigure}%%%%%%%MARGIN FIGURE
%\label{tab:common}
%\begin{tcolorbox}[tab2,tabularx={X|Y}]%%%% FANCY COLOR TABLE
%Chemical & Name             \\\hline\hline
%\ce{H2O} &  Water             \\\hline
%\ce{NH3} &  Ammonia             \\\hline
%\ce{CH4} &  Methane             \\\hline
%\ce{CO2} &  Dry ice             \\\hline
%\ce{NaCl} &  Table salt             \\\hline
%\ce{NaHCO3} &  Sodium Bicarbonate             \\\hline
%\ce{Mg(OH)2} &  Milk of magnesia        \\\hline
%\ce{N2O} &  Laughing gas        \\\hline
%\ce{CaCO3} &  Marble        \\\hline
%\ce{CaO} &  Quicklime        \\\hline
%\ce{NaHCO3} &  Baking Soda       \\\hline
%\ce{MgSO4}$\cdot$ 7 \ce{H2O} &  Epsom Salt     
%
%\end{tcolorbox}%%%% FANCY COLOR TABLE
%\caption{List of common chemicals}\label{table5:4}              
% \end{marginfigure}%%%%%%%MARGIN FIGURE
 
 
 \begin{center}
\refstepcounter{table} \label{tab:naming3}
%\begin{table}[ht]
\fontfamily{ppl}\selectfont
\begin{tabular}{llll}
\rowcolor{black!45}
\toprule
\multicolumn{4}{l}{\hypersetup{colorlinks,linkcolor={white}} \cellcolor{black}\color{white}\bfseries\small Table \ref{tab:naming3} List of common chemicals } \\
\midrule
 \rowcolor{gray!10} Chemical & Name & Chemical & Name \\
\midrule
\ce{H2O} &  Water  &           \ce{Mg(OH)2} &  Milk of magnesia  \\  
\ce{NH3} &  Ammonia &      \ce{N2O} &  Laughing gas       \\
\ce{CH4} &  Methane  &     \ce{CaCO3} &  Marble        \\
\ce{CO2} &  Dry ice  &     \ce{CaO} &  Quicklime       \\
\ce{NaCl} &  Table salt   &       \ce{NaHCO3} &  Baking Soda     \\
\ce{NaHCO3} &  Sodium Bicarbonate       &       \ce{MgSO4}$\cdot$ 7 \ce{H2O} &  Epsom Salt \\  
\bottomrule
\end{tabular}\end{center}
 
 
 

\begin{example} %%%%%%%%%%%%%%%%%%%%%%%% EXAMPLE BOX
Name or formulate the following common chemicals: milk of magnesia and dry ice. \\
\textlcsc{ \textcolor{dgreen}{\Large \textbf{Solution}} }\\
The formula for milk of magnesia is \ce{Mg(OH)2} (magnesium hydroxide), whereas dry ice is the common name for \ce{CO2}, carbon dioxide.\\
\faDiamond\ \textlcsc{ \textcolor{dgreen}{\Large \textbf{Study Check}} }\\
Name or formulate the following common chemicals: ammonia and methane.\\
\flushright Answer: \ce{NH3} (nitrogen trihydride) and \ce{CH4} (carbon tetrahydride).
\end{example}%%%%%%%%%%%%%%%%%%%%%%%% EXAMPLE BOX
\end{description}








\clearpage\thispagestyle{empty}\mbox{}\clearpage



\end{document}


%\textquotesingle
%\section{The atom}\marginnote{ \faEnvelope\myemail{dtorresrangel@bmcc.cuny.edu}{Error in the Book}{ Send Me typos!}}
%XXXXX
%\sloppy
%\begin{description}
%\item[\docfilehook{Elements and Symbols}{Elements and Symbols}] 
%\end{description}
%

%\begin{marginfigure}%%%%%%%QUOTES
%    \begin{shadequote}[l]{Democritus}
%Nothing exists except atoms and empty space; everything else is opinion.
%\end{shadequote}   \end{marginfigure}%%%%%%QUOTES


%\begin{marginfigure}%%%%%%%MARGIN FIGURE
%      \includegraphics{chapter2/figure1}
%      \label{fig:marginfig}
%   \end{marginfigure}%%%%%%%MARGIN FIGURE

%\begin{definition}[Political Factors]%%%%%%%%%ADITIONAL INFO BOX
%\begin{minipage}{0.25\linewidth}
%\includegraphics[width = \linewidth]{example-image-a}
%\end{minipage}%
%\hfill
%\begin{minipage}{0.7\linewidth}
%Analyses to what degree the government intervenes in the
%economy. It includes regulations and legal issues and defines
%both formal and informal rules under which the firm must
%operate. Political factors include: tax policy, employment laws,
%environmental regulations, trade restriction tariffs and political
%stability.
%\end{minipage}%
%\end{definition}%%%%%%%%%ADITIONAL INFO BOX
%
%\begin{example} %%%%%%% EXAMPLE BOX with explanation
%Obtain the electronic configuration of C.\\
%\textlcsc{ \textcolor{dgreen}{\Large Solution} }\\
%The atomic number of C is Z=6 and that means C has 6 electrons. The orbital order from Figure \ref{fig:orbitaltable} is: $1s$,$2s$, $2p$, $3s$, etc. Each $s$ orbital can fit two electrons, whereas the occupancy of  the $p$ orbitals is six electrons. Hence the electronic configuration of C is: $1s^2 2s^2 2p^2$. The $s$ orbitals are all filled, whereas the $p$ orbital is only occupied with two electrons.
%\\
%\faDiamond\ \textlcsc{ \textcolor{dgreen}{\Large \textbf{Study Check}} }\\Obtain the electronic configuration of Ni.
%\flushright Answer: $1s^2 2s^2 2p^6 3s^2 3p^6 4s^2 3d^8$. 
%\end{example}
%\begin{marginfigure}
%\begin{work} % 
%\textlcsc{ \textcolor{olive}{\Large Get the Answer by: } }
%\begin{enumerate}
%\item Get the electrons
%\item Check the orbital order table
%\item Fill each orbital following the order
%\end{enumerate}
%\end{work}% 
%\end{marginfigure}%%%%%%% EXAMPLE BOX with explanation

% \begin{marginfigure}%%%%% DISCUSSION
%\begin{tcolorbox}[enhanced,colback=red!5!white,colframe=black!50!red,boxrule=1pt,
%  arc=0pt,outer arc=0pt,drop heavy lifted shadow]
%\faGears\ 
%\docenvdef{Discussion:} Look around your apartment and list a pure substance, a compound, a heterogeneous mixture and a homogeneous mixture?\end{tcolorbox}
% \end{marginfigure} %%%%% DISCUSSION
%Table \ref{tab:units} % TABLE OR FIGURE REFERENCE




%\vspace{6mm} \begin{equation*}%%%% COMMENTED EQUATION
%    \mathcal{A} = (\,\tikzmark{identity}{\texttt{I}} -\tikzmark[red]{G}{\texttt{G}}\,\,\, 
%    \tikzmark[blue]{L}{\texttt{L}} - \tikzmark[purple]{C}{\texttt{C }}\,)
%\end{equation*}
%\begin{tikzpicture}[overlay, remember picture,node distance =1.5cm]
%    \node (identitydescr) [below left=of identity ]{words};
%    \draw[,->,thick] (identitydescr) to [in=-90,out=90] (identity);
%    \node[red] (Gdescr) [below =of G]{other words};
%    \draw[red,->,thick] (Gdescr) to [in=-90,out=90] (G);
%    \node[blue,xshift=1cm] (Ldescr) [above right =of L]{some words};
%    \draw[blue,->,thick] (Ldescr) to [in=45,out=-90] (L.north);
%    \node[purple] (Cdescr) [below right =of C]{more words};
%    \draw[purple,->,thick] (Cdescr) to [in=-90,out=90] (C.south);
%\end{tikzpicture}\vspace{10mm} %%%% COMMENTED EQUATION


%\begin{tcolorbox}[tab2,tabularx={X||Y|Y|Y|Y||Y}]%%%% FANCY COLOR TABLE
%Group & One     & Two     & Three    & Four     & Sum      \\\hline\hline
%Red   & 1000.00 & 2000.00 &  3000.00 &  4000.00 & 10000.00 \\\hline
%Green & 2000.00 & 3000.00 &  4000.00 &  5000.00 & 14000.00 \\\hline
%Blue  & 3000.00 & 4000.00 &  5000.00 &  6000.00 & 18000.00 \\\hline\hline
%Sum   & 6000.00 & 9000.00 & 12000.00 & 15000.00 & 42000.00
%\end{tcolorbox}%%%% FANCY COLOR TABLE


%\begin{center}\schemestart%%% ANNONATED CHEM EQUATION
%\chemname{\ce{H2O}}{Alcohol\\test1}
%\arrow(.mid east--.mid west)
%\chemname{\ce{H2O}}{Ester}
%\+
%\chemname{\ce{H2O}}{Ester}
%\schemestop\end{center}%%% ANNONATED CHEM EQUATION
%


%\begin{figure}[h!]%%%%% EMBEDED MOVIES
%\includemovie[
%  poster=FlashPoster.jpg,width=0.25\textwidth
%  text={\Large\bf Click to start\hspace*{400pt}}
%]{550pt}{400pt}{blendone.swf}
%\end{figure}%%%%% EMBEDED MOVIES



%\begin{marginfigure}%%%%%%%MARGIN PLOT
%\begin{tikzpicture}[yscale=0.8]
%\begin{axis}[axis background/.style = {%
%      shade,
%      top color = blue!10,
%      bottom color = white},
%    x tick label style={
%        /pgf/number format/1000 sep=},
%    ymax=19,%
%    ymin=1,%
%    xmax=40,%
%    xmin=0,%
%    width=\linewidth,
%    enlargelimits=0.1,
%    legend style={at={(0.5,0.5)},
%      anchor=north,legend columns=-1},
%    ylabel={\Large $N(t)$ of \ce{^131 Ir} (g)},xlabel={\Large Time (days)},
%    bar width=5mm, y=4mm,
%    symbolic x coords={0, 8, 24, 32, 40},
%    xtick=data,
%    nodes near coords align={vertical},
%    ]
%\addplot[ybar,  fill=black!10] 
%    coordinates {(0,20) (8,5) (24,2.5) (32,1.25) (40,0.6)} ;
%    \node [above] at (axis cs:  8,5) {$t_{1/2}$};
%        \node [above] at (axis cs:  24,2.5) {$2t_{1/2}$};
%        \node [above] at (axis cs:  32,1.25) {$3t_{1/2}$};
%\end{axis}
%    \end{tikzpicture}
%\end{marginfigure}%%%%%%%MARGIN PLOT
