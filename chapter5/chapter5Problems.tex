\documentclass[main.tex]{subfiles}
\begin{document}\newpage
\setdoublesep{0.35700 em}  % 'Bond Spacing'
\setatomsep{1.78500 em}    % 'Fixed Length'
\setbondoffset{0.18265 em} % 'Margin Width'
\newcommand{\bondwidth}{0.06642 em} % 'Line Width'
\setbondstyle{line width = \bondwidth}
\newgeometry{left=0.8in,right=0.8in, top=2.5cm,bottom=2.5cm}
\fancyhfoffset[E,O]{0pt}
\setlength{\columnsep}{30pt}
\begin{conclusion}
\end{conclusion}
%\setstretch{0.3}
\begin{multicols*}{2}\setcounter{numA}{1}  %\newcounter{numA}


{\raggedright\textsc{\textbf{Ions \& Ionic charges}}\par}

%%%%%%%PROBLEM
\begin{question}[ID=\the\value{numA}]\SetQuestionProperties{section-title=\nameref{sec:units}}
Indicate if the following chemical species represent an atom, and anion or a cation:
\begin{inparaenum}[(a)]
\item \ce{Fe^{2+}}
\item \ce{Cl^-} 
 \item \ce{Ag} 
\end{inparaenum}
\end{question}
\begin{solution}
\begin{inparaenum}[(a)]
\item \ce{Fe^{2+}} (cation)
\item \ce{Cl^-} (anion)
 \item \ce{Ag} (atom)
\end{inparaenum}\hspace{0.1cm}\end{solution}\stepcounter{numA}
%%%%%%%%%%%%%%


%%%%%%%PROBLEM
\begin{question}[ID=\the\value{numA}]\SetQuestionProperties{section-title=\nameref{sec:units}}
Identify the ionic state of the following elements. If needed, indicate the existence of multiple ionic states:
\begin{inparaenum}[(a)]
\item \ce{H} %\ce{H^+}
\item \ce{O} % \ce{O^{2-}}
\item \ce{N} %\ce{N^{3-}}
\item \ce{F} %\ce{F^{-}}
\item \ce{Mn} % (multiple)
\end{inparaenum}
\end{question}
\begin{solution}
\begin{inparaenum}[(a)]
\item \ce{H^+}
\item  \ce{O^{2-}}
 \item \ce{N^{3-}}
  \item \ce{F^{-}}
  \item \ce{Mn}  (multiple)
\end{inparaenum}\hspace{0.1cm}\end{solution}\stepcounter{numA}
%%%%%%%%%%%%%%

{\raggedright\textsc{\textbf{Covalent compounds}}\par}

%%%%%%%PROBLEM
\begin{question}[ID=\the\value{numA}]\SetQuestionProperties{section-title=\nameref{sec:units}}
Name or formulate the following covalent compounds:
\begin{inparaenum}[(a)]
\item \ce{NO}				% nitrogen monoxide
\item Dichlorine monofluoride	%\ce{Cl2F}
\item \ce{NO2}				%nitrogen dioxide
\end{inparaenum}
\end{question}
\begin{solution}
\begin{inparaenum}[(a)]
\item \ce{NO}	 (nitrogen monoxide)
\item Dichlorine monofluoride	(\ce{Cl2F})
\item \ce{NO2}		(nitrogen dioxide)
\end{inparaenum}\hspace{0.1cm}\end{solution}\stepcounter{numA}
%%%%%%%%%%%%%%


%%%%%%%PROBLEM
\begin{question}[ID=\the\value{numA}]\SetQuestionProperties{section-title=\nameref{sec:units}}
Name or formulate the following covalent compounds:
\begin{inparaenum}[(a)]
\item Chlorine Monofluoride	%\ce{ClF}
\item \ce{N2O}				%dinitrogen monoxide
\item Nitrogen trifluoride		%\ce{NF3}
\end{inparaenum}
\end{question}
\begin{solution}
\begin{inparaenum}[(a)]
\item Chlorine Monofluoride	(\ce{ClF})
\item \ce{N2O}		(dinitrogen monoxide)
\item Nitrogen trifluoride(	\ce{NF3})
\end{inparaenum}\hspace{0.1cm}\end{solution}\stepcounter{numA}
%%%%%%%%%%%%%%




%%%%%%%PROBLEM
\begin{question}[ID=\the\value{numA}]\SetQuestionProperties{section-title=\nameref{sec:units}}
Name or formulate the following covalent compounds:
\begin{inparaenum}[(a)]
\item \ce{SO3}				%(sulfur trioxide)
\item Disulfur dichloride		%(S2Cl)	
\item \ce{SO2}				%(sulfur dioxide)
\item Disulfur tetrachloride		%(S2Cl4)	
\end{inparaenum}
\end{question}
\begin{solution}
\begin{inparaenum}[(a)]
\item \ce{SO3}		(sulfur trioxide)
\item Disulfur dichloride (S2Cl)	
\item \ce{SO2}		(sulfur dioxide)
\item Disulfur tetrachloride (S2Cl4)	
\end{inparaenum}\hspace{0.1cm}\end{solution}\stepcounter{numA}
%%%%%%%%%%%%%%
%%%%%%%PROBLEM
\begin{question}[ID=\the\value{numA}]\SetQuestionProperties{section-title=\nameref{sec:units}}
Name or formulate the following covalent compounds:
\begin{inparaenum}[(a)]
\item \ce{P4S3}				%(tetraphosporus trisulfide)
\item Sulfur Tetrafluoride		%(SF4)	
\item \ce{As2O5}			%(diarsenic pentoxide)	
\item Sulfur trioxide			%(SO3)
\end{inparaenum}
\end{question}
\begin{solution}
\begin{inparaenum}[(a)]
\item \ce{P4S3}		(tetraphosporus trisulfide)
\item Sulfur Tetrafluoride (SF4)	
\item \ce{As2O5}	 (diarsenic pentoxide)	
\item Sulfur trioxide	(SO3)
\end{inparaenum}\hspace{0.1cm}\end{solution}\stepcounter{numA}
%%%%%%%%%%%%%%






{\raggedright\textsc{\textbf{Ionic compounds}}\par}


%%%%%%%PROBLEM
\begin{question}[ID=\the\value{numA}]\SetQuestionProperties{section-title=\nameref{sec:units}}
Name or formulate the following ionic (Type I) compounds:
\begin{inparaenum}[(a)]
\item Magnesium iodide%(\ce{MgI2})
\item \ce{Ca3P2}	%(Calcium phosphide)	
\item Lithium nitride%(\ce{Li3N})
\item \ce{MgF}	%(Magnesium fluoride)
\end{inparaenum}
\end{question}
\begin{solution}
\begin{inparaenum}[(a)]
\item Magnesium iodide (\ce{MgI2})
\item \ce{Ca3P2}	 (Calcium phosphide)	
\item Lithium nitride (\ce{Li3N})
\item \ce{MgF}	 (Magnesium fluoride)
 \end{inparaenum}\hspace{0.1cm}\end{solution}\stepcounter{numA}
%%%%%%%%%%%%%%
%%%%%%%PROBLEM
\begin{question}[ID=\the\value{numA}]\SetQuestionProperties{section-title=\nameref{sec:units}}
Name or formulate the following ionic (Type I) compounds:
\begin{inparaenum}[(a)]
\item Magnesium fluoride%(\ce{MgF2})
\item \ce{CaS}	%(Calcium sulfide)
\item Barium phosphide%(\ce{Ba3P2})
\item \ce{Mg3N2}	%(magnesium nitride)
\end{inparaenum}
\end{question}
\begin{solution}
\begin{inparaenum}[(a)]
\item Magnesium fluoride (\ce{MgF2})
\item \ce{CaS}	 (Calcium sulfide)
\item Barium phosphide (\ce{Ba3P2})
\item \ce{Mg3N2}	(magnesium nitride)
 \end{inparaenum}\hspace{0.1cm}\end{solution}\stepcounter{numA}
%%%%%%%%%%%%%%




%%%%%%%PROBLEM
\begin{question}[ID=\the\value{numA}]\SetQuestionProperties{section-title=\nameref{sec:units}}
Name or formulate the following ionic (Type II) compounds:
\begin{inparaenum}[(a)]
\item \ce{Fe3P2}		%Iron(II) phosphide	
\item Copper(II) iodide 	%\ce{CuI2}
\item \ce{Fe3N2}		%Iron(II) nitride
\item Iron(II) sulfide		%\ce{FeS}
\end{inparaenum}
\end{question}
\begin{solution}
\begin{inparaenum}[(a)]
\item \ce{Fe3P2}		(Iron(II) phosphide)	
\item Copper(II) iodide 	(\ce{CuI2})
\item \ce{Fe3N2}		(Iron(II) nitride)
\item Iron(II) sulfide		(\ce{FeS})
 \end{inparaenum}\hspace{0.1cm}\end{solution}\stepcounter{numA}
%%%%%%%%%%%%%%
%%%%%%%PROBLEM
\begin{question}[ID=\the\value{numA}]\SetQuestionProperties{section-title=\nameref{sec:units}}
Name or formulate the following ionic (Type II) compounds:
\begin{inparaenum}[(a)]
\item \ce{Fe2S3}		%(Iron(III) sulfide)
\item Gold(III) chloride	%\ce{AuCl3}
\item \ce{FeO}			% Iron(II) oxide
\item Vanadium(V) nitride	%\ce{V3N5}
\end{inparaenum}
\end{question}
\begin{solution}
\begin{inparaenum}[(a)]
\item \ce{Fe2S3}		(Iron(III) sulfide)
\item Gold(III) chloride	(\ce{AuCl3})
\item \ce{FeO}			(Iron(II) oxide)
\item Vanadium(V) nitride	(\ce{V3N5} )
 \end{inparaenum}\hspace{0.1cm}\end{solution}\stepcounter{numA}
%%%%%%%%%%%%%%





%%%%%%%PROBLEM
\begin{question}[ID=\the\value{numA}]\SetQuestionProperties{section-title=\nameref{sec:units}}
Name or formulate the following ionic (Type II) compounds:
\begin{inparaenum}[(a)]
\item \ce{FeI2}			%(Iron(II) iodide)
\item Lead(IV) sulfide	%(\ce{PbS2})
\item \ce{FeBr2}		%(Iron(II) bromide)	
\end{inparaenum}
\end{question}
\begin{solution}
\begin{inparaenum}[(a)]
\item \ce{FeI2}			(Iron(II) iodide)
\item Lead(IV) sulfide	(\ce{PbS2})
\item \ce{FeBr2}		(Iron(II) bromide)	
 \end{inparaenum}\hspace{0.1cm}\end{solution}\stepcounter{numA}
%%%%%%%%%%%%%%
%%%%%%%PROBLEM
\begin{question}[ID=\the\value{numA}]\SetQuestionProperties{section-title=\nameref{sec:units}}
Name or formulate the following ionic (Type II) compounds:
\begin{inparaenum}[(a)]
\item Manganese(IV) oxide	%(\ce{MnO2})
\item \ce{FeCl2}			%(Iron(II) chloride)	
\item Copper(I) oxide%(\ce{Cu2O})
\end{inparaenum}
\end{question}
\begin{solution}
\begin{inparaenum}[(a)]
\item Manganese(IV) oxide	(\ce{MnO2})
\item \ce{FeCl2}			(Iron(II) chloride)	
\item Copper(I) oxide (\ce{Cu2O})
 \end{inparaenum}\hspace{0.1cm}\end{solution}\stepcounter{numA}
%%%%%%%%%%%%%%

{\raggedright\textsc{\textbf{Acids and hydroxides}}\par}


%%%%%%%PROBLEM
\begin{question}[ID=\the\value{numA}]\SetQuestionProperties{section-title=\nameref{sec:units}}
Name or formulate the following acids or bases:
\begin{inparaenum}[(a)]
\item \ce{HCl}			%(hydrochloric acid)
\item Hydrofluoric Acid	%(\ce{HF})
\item \ce{Mg(OH)2}		%(magnesium hydroxide)
\end{inparaenum}
\end{question}
\begin{solution}
\begin{inparaenum}[(a)]
\item \ce{HCl}			(hydrochloric acid)
\item Hydrofluoric Acid	(\ce{HF})
\item \ce{Mg(OH)2}		(magnesium hydroxide)
 \end{inparaenum}\hspace{0.1cm}\end{solution}\stepcounter{numA}
%%%%%%%%%%%%%%
%%%%%%%PROBLEM
\begin{question}[ID=\the\value{numA}]\SetQuestionProperties{section-title=\nameref{sec:units}}
Name or formulate the following acids or bases:
\begin{inparaenum}[(a)]
\item Sulfuric Acid		%(\ce{H2SO4})
\item \ce{H2CO3}		%(carbonic acid)
\item Lithium hydroxide		%(\ce{LiOH})
\end{inparaenum}
\end{question}
\begin{solution}
\begin{inparaenum}[(a)]
\item Sulfuric Acid		(\ce{H2SO4})
\item \ce{H2CO3}		(carbonic acid)
\item Lithium hydroxide	(\ce{LiOH})
 \end{inparaenum}\hspace{0.1cm}\end{solution}\stepcounter{numA}
%%%%%%%%%%%%%%


%%%%%%%PROBLEM
\begin{question}[ID=\the\value{numA}]\SetQuestionProperties{section-title=\nameref{sec:units}}
From the following chemicals identify acids and bases:
\begin{inparaenum}[(a)]
\item \ce{KOH}			%(base)
\item \ce{LiOH}			%(base)
\item \ce{CH3OH}			%(acid, organic)
\end{inparaenum}
\end{question}
\begin{solution}
\begin{inparaenum}[(a)]
\item \ce{KOH}			(base)
\item \ce{LiOH}			(base)
\item \ce{CH3OH}		(acid, organic)
 \end{inparaenum}\hspace{0.1cm}\end{solution}\stepcounter{numA}
%%%%%%%%%%%%%%
%%%%%%%PROBLEM
\begin{question}[ID=\the\value{numA}]\SetQuestionProperties{section-title=\nameref{sec:units}}
From the following chemicals identify acids and bases:
\begin{inparaenum}[(a)]
\item \ce{H2SO3}			%(acid, inorganic)
\item \ce{NH3}			%(base)
\item \ce{Ca(OH)2}			%(base)
\end{inparaenum}
\end{question}
\begin{solution}
\begin{inparaenum}[(a)]
\item \ce{H2SO3}		(acid, inorganic)
\item \ce{NH3}			(base)
\item \ce{Ca(OH)2}		(base)
 \end{inparaenum}\hspace{0.1cm}\end{solution}\stepcounter{numA}
%%%%%%%%%%%%%%


%%%%%%%PROBLEM
\begin{question}[ID=\the\value{numA}]\SetQuestionProperties{section-title=\nameref{sec:units}}
From the following chemicals identify hydracids and oxoacids:
\begin{inparaenum}[(a)]
\item \ce{HF}		%(hydracid)
\item \ce{H2SO3}	%(oxacid)
\item \ce{H2S}		%(hydracid)
%\item \ce{HClO4}	%(oxacid)
\end{inparaenum}
\end{question}
\begin{solution}
\begin{inparaenum}[(a)]
\item \ce{HF}		(hydracid)
\item \ce{H2SO3}	(oxacid)
\item \ce{H2S}		(hydracid)
%\item \ce{HClO4}	(oxacid)
 \end{inparaenum}\hspace{0.1cm}\end{solution}\stepcounter{numA}
%%%%%%%%%%%%%%
%%%%%%%PROBLEM
\begin{question}[ID=\the\value{numA}]\SetQuestionProperties{section-title=\nameref{sec:units}}
From the following chemicals identify hydracids and oxoacids:
\begin{inparaenum}[(a)]
\item \ce{H3BO3}	%(oxacid)
\item \ce{HCl}		%(hydracid)
\item \ce{HI}		%(hydracid)
%\item \ce{HClO4}	%(oxacid)
\end{inparaenum}
\end{question}
\begin{solution}
\begin{inparaenum}[(a)]
\item \ce{H3BO3}	(oxacid)
\item \ce{HCl}		(hydracid)
\item \ce{HI}		(hydracid)
%\item \ce{HClO4}	(oxacid)
 \end{inparaenum}\hspace{0.1cm}\end{solution}\stepcounter{numA}
%%%%%%%%%%%%%%


%%%%%%%PROBLEM
\begin{question}[ID=\the\value{numA}]\SetQuestionProperties{section-title=\nameref{sec:units}}
Identify the redox number of the central atom of the following oxoacids:
\begin{inparaenum}[(a)]
\item	\ce{H2CrO4}   	%redox=6
\item	\ce{H2Cr2O7}	%redox=6
\item	\ce{HMnO4}	%redox=7
\end{inparaenum}
\end{question}
\begin{solution}
\begin{inparaenum}[(a)]
\item	\ce{H2CrO4}   	redox=6
\item	\ce{H2Cr2O7}	redox=6
\item	\ce{HMnO4}	redox=7
\end{inparaenum}\hspace{0.1cm}\end{solution}\stepcounter{numA}
%%%%%%%%%%%%%%
%%%%%%%PROBLEM
\begin{question}[ID=\the\value{numA}]\SetQuestionProperties{section-title=\nameref{sec:units}}
Identify the redox number of the central atom of the following oxoacids:
\begin{inparaenum}[(a)]
\item	\ce{H2MnO4} 	%redox=6
\item	\ce{HReO3} 	%redox=5
\item	\ce{H2SiO3}	%redox=4
\end{inparaenum}
\end{question}
\begin{solution}
\begin{inparaenum}[(a)]
\item	\ce{H2MnO4} 	redox=6
\item	\ce{HReO3} 	redox=5
\item	\ce{H2SiO3}	redox=4 
\end{inparaenum}\hspace{0.1cm}\end{solution}\stepcounter{numA}
%%%%%%%%%%%%%%





%%%%%PROBLEM
\begin{question}[ID=\the\value{numA}]\SetQuestionProperties{section-title=\nameref{sec:units}}
Identify the most oxidated acid: 
 \noindent
 \begin{multicols}{2}
  \begin{enumerate} [topsep=0pt, partopsep=1pt, label=(\alph*), leftmargin=0.5cm]	
\item \ce{H3AsO4} or \ce{H3AsO3} \iffalse \ce{H3As\textsuperscript{V}O4   } \fi
\item \ce{H2XeO4} or \ce{H4XeO6} \iffalse \ce{H4Xe\textsuperscript{VIII}O6} \fi
\end{enumerate}
\end{multicols}
\end{question}
\begin{solution}
\begin{inparaenum}[(a)]
\item   \ce{H3As\textsuperscript{V}O4   } 
\item  \ce{H4Xe\textsuperscript{VIII}O6}  
 \end{inparaenum}
\hspace{0.1cm}\end{solution}\stepcounter{numA}%%%%%%%%%%%%

%%%%%PROBLEM
\begin{question}[ID=\the\value{numA}]\SetQuestionProperties{section-title=\nameref{sec:units}}
Identify the most reduced acid: 
 \noindent
  \begin{multicols}{2}
  \begin{enumerate} [topsep=0pt, partopsep=1pt, label=(\alph*), leftmargin=0.5cm]	
\item \ce{H2RuO4} or \ce{HRuO4} \iffalse \ce{H2Ru\textsuperscript{VI}O4} \fi
\item \ce{HTcO4} or \ce{H2TcO4} \iffalse \ce{HTc\textsuperscript{VII}O4} \fi
\end{enumerate}
\end{multicols}
\end{question}
\begin{solution}
\begin{inparaenum}[(a)]
\item   \ce{H2Ru\textsuperscript{VI}O4}  
\item   \ce{HTc\textsuperscript{VII}O4}   
 \end{inparaenum}
\hspace{0.1cm}\end{solution}\stepcounter{numA}%%%%%%%%%%%%

%%%%%PROBLEM
\begin{question}[ID=\the\value{numA}]\SetQuestionProperties{section-title=\nameref{sec:units}}
Identify the most oxidant acid: 
 \noindent
  \begin{multicols}{2}
  \begin{enumerate} [topsep=0pt, partopsep=1pt, label=(\alph*), leftmargin=0.5cm]	
\item \ce{H2S2O6} or \ce{H2SO4} \iffalse \ce{H2S2\textsuperscript{V}O6} \fi
\item \ce{H2SeO4} or \ce{H2SeO3} \iffalse \ce{H2Se\textsuperscript{IV}O3} \fi
\end{enumerate}
 \end{multicols} 
\end{question}
\begin{solution}
\begin{inparaenum}[(a)]
\item   \ce{H2S2\textsuperscript{V}O6}  
\item   \ce{H2Se\textsuperscript{IV}O3}    
 \end{inparaenum}
\hspace{0.1cm}\end{solution}\stepcounter{numA}%%%%%%%%%%%%



{\raggedright\textsc{\textbf{Naming of oxosalts}}\par}


%%%%%%%PROBLEM
\begin{question}[ID=\the\value{numA}]\SetQuestionProperties{section-title=\nameref{sec:units}}
Name or formulate the following (Type I) oxosalts:
\begin{inparaenum}[(a)]
\item \ce{Mg(NO3)2}			%(magnesium nitrate) 	
\item Sodium permanganate 	%(\ce{NaMnO4})
\item \ce{KMnO4}			%(potassium permanganate)
\item Calcium carbonate		%(\ce{CaCO3})
\item \ce{Li3PO4}			%(lithium phosphate)
\end{inparaenum}
\end{question}
\begin{solution}
\begin{inparaenum}[(a)]
\item \ce{Mg(NO3)2}			 (magnesium nitrate) 	
\item Sodium permanganate 	 (\ce{NaMnO4})
\item \ce{KMnO4}			 (potassium permanganate)
\item Calcium carbonate		 (\ce{CaCO3})
\item \ce{Li3PO4}			 (lithium phosphate)
 \end{inparaenum}\hspace{0.1cm}\end{solution}\stepcounter{numA}
%%%%%%%%%%%%%%
%%%%%%%PROBLEM
\begin{question}[ID=\the\value{numA}]\SetQuestionProperties{section-title=\nameref{sec:units}}
Name or formulate the following (Type I) oxosalts:
\begin{inparaenum}[(a)]
\item Lithium sulfate			%(\ce{Li2SO4})	
\item \ce{Na2CrO4} 			%(sodium dichromate)
\item Lithium sulfite			%(Li2SO3)
\item \ce{Cs2Cr2O7}			%(caesium dichromate )
\item Calcium sulfate			%(CaSO4)
\end{inparaenum}
\end{question}
\begin{solution}
\begin{inparaenum}[(a)]
\item Lithium sulfate			 (\ce{Li2SO4})	
\item \ce{Na2CrO4} 			 (sodium dichromate)
\item Lithium sulfite			 (Li2SO3)
\item \ce{Cs2Cr2O7}			 (caesium dichromate )
\item Calcium sulfate			(CaSO4)
 \end{inparaenum}\hspace{0.1cm}\end{solution}\stepcounter{numA}
%%%%%%%%%%%%%%
	 	
%%%%%%%PROBLEM
\begin{question}[ID=\the\value{numA}]\SetQuestionProperties{section-title=\nameref{sec:units}}
Name or formulate the following (Type II) oxosalts:
\begin{inparaenum}[(a)]
\item \ce{Cr2(SO4)3}	 		%(chromium(III) sulfate)
\item zinc(II) carbonate		%(ZnCO3)
\item \ce{Fe(MnO4)3}	 	%(iron(III) permanganate)	
%\item cobalt(II) carbonate		%(CoCO3)
\end{inparaenum}
\end{question}
\begin{solution}
\begin{inparaenum}[(a)]
\item \ce{Cr2(SO4)3}	 		 (chromium(III) sulfate)
\item zinc(II) carbonate		 (ZnCO3)
\item \ce{Fe(MnO4)3}	 	 (iron(III) permanganate)	
%\item cobalt(II) carbonate		 (CoCO3)
 \end{inparaenum}\hspace{0.1cm}\end{solution}\stepcounter{numA}
%%%%%%%%%%%%%%
%%%%%%%PROBLEM
\begin{question}[ID=\the\value{numA}]\SetQuestionProperties{section-title=\nameref{sec:units}}
Name or formulate the following (Type II) oxosalts:
\begin{inparaenum}[(a)]	
\item cobalt(III) carbonate		%(Co2(CO3)2)
\item \ce{Fe(ClO4)3}	 		%(iron(III) perchlorate)
\item zinc(II) carbonate		%(ZnCO3)
%\item cobalt(II) carbonate		%(CoCO3)
\end{inparaenum}
\end{question}
\begin{solution}
\begin{inparaenum}[(a)]
\item cobalt(III) carbonate		 (\ce{Co2(CO3)2})
\item \ce{Fe(ClO4)3}	 		(iron(III) perchlorate)
\item zinc(II) carbonate		 (\ce{ZnCO3})
%\item cobalt(II) carbonate		 (CoCO3)
 \end{inparaenum}\hspace{0.1cm}\end{solution}\stepcounter{numA}
%%%%%%%%%%%%%%

{\raggedright\textsc{\textbf{ Hydrosalts, hydrates  \& common chemicals}}\par}

%%%%%%%PROBLEM
\begin{question}[ID=\the\value{numA}]\SetQuestionProperties{section-title=\nameref{sec:units}}
Name or formulate the following hydrosalts:
\begin{inparaenum}[(a)]	
\item \ce{NaHCO3}				%(sodium hydrogen carbonate)	
\item Calcium Hydrogencarbonate 	%(\ce{Ca(HCO3)2})
\item \ce{Al(HSO4)3}				%(aluminum hydrogencarbonate)
\end{inparaenum}
\end{question}
\begin{solution}
\begin{inparaenum}[(a)]
\item \ce{NaHCO3}				 (sodium hydrogen carbonate)	
\item Calcium Hydrogencarbonate 	 (\ce{Ca(HCO3)2})
\item \ce{Al(HSO4)3}				 (aluminum hydrogencarbonate)
 \end{inparaenum}\hspace{0.1cm}\end{solution}\stepcounter{numA}
%%%%%%%%%%%%%%
%%%%%%%PROBLEM
\begin{question}[ID=\the\value{numA}]\SetQuestionProperties{section-title=\nameref{sec:units}}
Name or formulate the following hydrosalts:
\begin{inparaenum}[(a)]	
\item Sodium dihydrogenphosphate	%(\ce{NaH2PO4})
\item \ce{LiH2PO4}				%(lithium dihydrogenphosphate)
\item Silver monohydrogenphosphate	%(\ce{Ag2HPO4})
\end{inparaenum}
\end{question}
\begin{solution}
\begin{inparaenum}[(a)]
\item Sodium dihydrogenphosphate	 (\ce{NaH2PO4})
\item \ce{LiH2PO4}				 (lithium dihydrogenphosphate)
\item Silver monohydrogenphosphate	 (\ce{Ag2HPO4})
 \end{inparaenum}\hspace{0.1cm}\end{solution}\stepcounter{numA}
%%%%%%%%%%%%%%

%%%%%%%PROBLEM
\begin{question}[ID=\the\value{numA}]\SetQuestionProperties{section-title=\nameref{sec:units}}
Name or formulate the following hydrates:
\begin{inparaenum}[(a)]	
\item \ce{Al2(SO4)3 . 3H2O}		%(aluminum phosphate trihydrate)	
\item Silver phosphate dihydrate	%(\ce{Ag3PO4 . 2H2O})
\item \ce{KMnO4 . 4H2O}			%(potassium permanganate tetrahydrate)
\item Lithium sulfate tetrahydrate	%(\ce{Li2SO4 . 4H2O})
\end{inparaenum}
\end{question}
\begin{solution}
\begin{inparaenum}[(a)]
\item \ce{Al2(SO4)3 . 3H2O}		 (aluminum phosphate trihydrate)	
\item Silver phosphate dihydrate	 (\ce{Ag3PO4 . 2H2O})
\item \ce{KMnO4 . 4H2O}			 (potassium permanganate tetrahydrate)
\item Lithium sulfate tetrahydrate	 (\ce{Li2SO4 . 4H2O})
 \end{inparaenum}\hspace{0.1cm}\end{solution}\stepcounter{numA}
%%%%%%%%%%%%%%




%%%%%%%PROBLEM
\begin{question}[ID=\the\value{numA}]\SetQuestionProperties{section-title=\nameref{sec:units}}
Name or formulate the following compounds:
\begin{inparaenum}[(a)]	
\item  \ce{MgSO4}				%(magnesium phosphate)
\item  \ce{Ni(SO4)3}				%(nickel(III) sulfate)
\item  Cobalt(II) nitrate			%(\ce{Co(NO3)2})
\item  Cobalt(II) sulfate dihydrate	%(\ce{CoSO4 . 2H2O})
\item \ce{KHCO3}				%(potassium monohydrogencarbonate)
\end{inparaenum}
\end{question}
\begin{solution}
\begin{inparaenum}[(a)]
\item  \ce{MgSO4}					(magnesium phosphate)
\item  \ce{Ni(SO4)3}					(nickel(III) sulfate)
\item  Cobalt(II) nitrate				(\ce{Co(NO3)2})
\item  Cobalt(II) sulfate dihydrate		(\ce{CoSO4 . 2H2O})
\item \ce{KHCO3}					(potassium monohydrogencarbonate)
 \end{inparaenum}\hspace{0.1cm}\end{solution}\stepcounter{numA}
%%%%%%%%%%%%%%

%%%%%%%PROBLEM
\begin{question}[ID=\the\value{numA}]\SetQuestionProperties{section-title=\nameref{sec:units}}
Name or formulate the following compounds:
\begin{inparaenum}[(a)]	
\item  \ce{Ca(NO3)2}				%(calcium nitrate)
\item  \ce{Na(HCO3)2}			%(sodium monohydrogencarbonate)
\item  Nickel(II) sulfate			%(\ce{NiSO4})
\item  Nicke(II) sulfate tetrahydrate	%(\ce{NiSO4 . 4H2O})
\item \ce{NaH2PO4}				%(sodium dihydrogenphosphate)
\end{inparaenum}
\end{question}
\begin{solution}
\begin{inparaenum}[(a)]
\item  \ce{Ca(NO3)2}				 (calcium nitrate)
\item  \ce{Na(HCO3)2}			 (sodium monohydrogencarbonate)
\item  Nickel(II) sulfate			 (\ce{NiSO4})
\item  Nicke(II) sulfate tetrahydrate	 (\ce{NiSO4 . 4H2O})
\item \ce{NaH2PO4}				 (sodium dihydrogenphosphate)
 \end{inparaenum}\hspace{0.1cm}\end{solution}\stepcounter{numA}
%%%%%%%%%%%%%%


{\raggedright\textsc{\textbf{General problems}}\par}


%%%%%%%PROBLEM
\begin{question}[ID=\the\value{numA}]\SetQuestionProperties{section-title=\nameref{sec:units}}
Classify the following chemicals in two groups, justifying your classification:
\begin{inparaenum}[(a)]
\item \ce{NaCl}  %(group ionic)
\item \ce{CO2} %(group covalent)
\item \ce{FeCl3} %(group ionic)
\item \ce{N2O3}%(group covalent)
\item \ce{SO3}%(group covalent)
\item \ce{Ca3N2} %(group ionic)
\end{inparaenum}
\end{question}
\begin{solution}
\begin{inparaenum}[(a)]
\item \ce{NaCl}   (group ionic)
\item \ce{CO2}  (group covalent)
\item \ce{FeCl3}  (group ionic)
\item \ce{N2O3} (group covalent)
\item \ce{SO3} (group covalent)
\item \ce{Ca3N2}  (group ionic)
\end{inparaenum}\hspace{0.1cm}\end{solution}\stepcounter{numA}
%%%%%%%%%%%%%%


%%%%%%%PROBLEM
\begin{question}[ID=\the\value{numA}]\SetQuestionProperties{section-title=\nameref{sec:units}}
Name the following compounds:\begin{multicols}{2}
  \noindent
  \begin{enumerate} [topsep=0pt, partopsep=1pt, label=(\alph*), leftmargin=.5cm]
\item \ce{NaCl}  %(sodium chloride)
\item \ce{Ca3N2} %(calcium nitride)
\item \ce{MgI2} %(magnesium iodide)
\item \ce{SrS} %(strontium sulfide)
\item \ce{RbCl} %(rubidium chloride)
\item \ce{KF} %(potassium fluoride)
\end{enumerate}
\end{multicols}    
\end{question}
\begin{solution}
\begin{inparaenum}[(a)]
\item \ce{NaCl}   (sodium chloride)
\item \ce{Ca3N2}  (calcium nitride)
\item \ce{MgI2}  (magnesium iodide)
\item \ce{SrS}  (strontium sulfide)
\item \ce{RbCl}  (rubidium chloride)
\item \ce{KF}  (potassium fluoride)
\end{inparaenum}\hspace{0.1cm}\end{solution}\stepcounter{numA}
%%%%%%%%%%%%%%

%%%%%%%PROBLEM
\begin{question}[ID=\the\value{numA}]\SetQuestionProperties{section-title=\nameref{sec:units}}
Combine the following ions:
\begin{multicols}{2}
  \noindent
  \begin{enumerate} [topsep=0pt, partopsep=1pt, label=(\alph*), leftmargin=.5cm]
\item \ce{Na^{+}} + \ce{Cl^{-}}   %(\ce{NaCl})
\item \ce{Na^{+}} + \ce{Se^{2-}}  %(\ce{Na2S})
\item \ce{Na^{+}} + \ce{P^{3-}}  %(\ce{Na3P})
\item   \ce{Mg^{2+}} + \ce{Cl^{-}}  %(\ce{MgCl2})
\item  \ce{Mg^{2+}} + \ce{O^{2-}}  %(\ce{MgO})
\item   \ce{Mg^{2+}} + \ce{N^{3-}}  %(\ce{Mg3N2})
\end{enumerate}
\end{multicols}    
\end{question}
\begin{solution}
\begin{inparaenum}[(a)]
\item \ce{Na^{+}} + \ce{Cl^{-}}    (\ce{NaCl})
\item \ce{Na^{+}} + \ce{Se^{2-}}   (\ce{Na2S})
\item \ce{Na^{+}} + \ce{P^{3-}}   (\ce{Na3P})
\item   \ce{Mg^{2+}} + \ce{Cl^{-}}   (\ce{MgCl2})
\item   \ce{Mg^{2+}} + \ce{O^{2-}}   (\ce{MgO})
\item  \ce{Mg^{2+}} + \ce{N^{3-}}   (\ce{Mg3N2})
\end{inparaenum}\hspace{0.1cm}\end{solution}\stepcounter{numA}
%%%%%%%%%%%%%%

%%%%%%%PROBLEM
\begin{question}[ID=\the\value{numA}]\SetQuestionProperties{section-title=\nameref{sec:units}}
Memorize the following oxoacids:
  \noindent
  \begin{enumerate} [topsep=0pt, partopsep=1pt, label=(\alph*), leftmargin=.5cm]
\item \ce{H2SO4} Sulfuric acid
\item \ce{H2CO3} Carbonic acid
\item \ce{HMnO4} Permanganic acid
\item \ce{HNO3} Nitric acid
\item \ce{H3PO4} Carbonic acid
\item \ce{H2Cr2O7} Dicromic acid
\end{enumerate}
\end{question}
\begin{solution}
No answer.\hspace{0.1cm}\end{solution}\stepcounter{numA}
%%%%%%%%%%%%%%

%%%%%%%PROBLEM
\begin{question}[ID=\the\value{numA}]\SetQuestionProperties{section-title=\nameref{sec:units}}
Classify the following chemicals in two groups. Justify your classification.
\begin{multicols}{3}
  \noindent
  \begin{enumerate} [topsep=0pt, partopsep=0pt, label=(\alph*), leftmargin=.5cm]
\item \ce{NaCl}  %(group ionic type I)
\item \ce{MnO2} %(group ionic type II)
\item \ce{FeCl3} %(group ionic type II)
\item \ce{SrO} %(group ionic type I)
\item \ce{Li3N} %(group ionic type I)
\item \ce{NiO} %(group ionic type II)
\end{enumerate}
\end{multicols}    
\end{question}
\begin{solution}
\begin{inparaenum}[(a)]
\item \ce{NaCl}   (group ionic type I)
\item \ce{MnO2}  (group ionic type II)
\item \ce{FeCl3}  (group ionic type II)
\item \ce{SrO}  (group ionic type I)
\item \ce{Li3N}  (group ionic type I)
\item \ce{NiO}  (group ionic type II)
\end{inparaenum}\hspace{0.1cm}\end{solution}\stepcounter{numA}
%%%%%%%%%%%%%%


%%%%%%%PROBLEM
\begin{question}[ID=\the\value{numA}]\SetQuestionProperties{section-title=\nameref{sec:units}}
Formulate the following compounds:
 \begin{multicols}{2}
 \noindent
  \begin{enumerate} [topsep=0pt, partopsep=0pt, label=(\alph*), leftmargin=.4cm]
\item  Copper(I) oxide %(\ce{Cu2O})
\item  Copper(II) nitride %(\ce{Cu3N2})
\item  Nickel(III) oxide %(\ce{Ni3O2})
\item  Manganese(IV) oxide %(\ce{MnO2})
\end{enumerate}
\end{multicols}
\end{question}
\begin{solution}
\begin{inparaenum}[(a)]
\item  Copper(I) oxide  (\ce{Cu2O})
\item  Copper(II) nitride  (\ce{Cu3N2})
\item  Nickel(III) oxide  (\ce{Ni3O2})
\item  Manganese(IV) oxide  (\ce{MnO2})
\end{inparaenum}\hspace{0.1cm}\end{solution}\stepcounter{numA}
%%%%%%%%%%%%%%

%%%%%%%PROBLEM
\begin{question}[ID=\the\value{numA}]\SetQuestionProperties{section-title=\nameref{sec:units}}
Name the following compounds:
 \begin{multicols}{2}
  \noindent
  \begin{enumerate} [topsep=0pt, partopsep=0pt, label=(\alph*), leftmargin=.5cm]
\item  \ce{NiO} %(Nicke(II) oxide)
\item  \ce{Cr2O3} %(Chromium(III) oxide)
\item  \ce{VO} %(Vanadium oxide)
\item  \ce{MnO4} %(Manganese(VII) oxide)
\end{enumerate} \end{multicols}

\end{question}
\begin{solution}
\begin{inparaenum}[(a)]
\item  \ce{NiO}  (Nicke(II) oxide)
\item  \ce{Cr2O3}  (Chromium(III) oxide)
\item  \ce{VO}  (Vanadium oxide)
\item  \ce{MnO4}  (Manganese(VII) oxide)
\end{inparaenum}\hspace{0.1cm}\end{solution}\stepcounter{numA}
%%%%%%%%%%%%%%


%%%%%%%PROBLEM
\begin{question}[ID=\the\value{numA}]\SetQuestionProperties{section-title=\nameref{sec:units}}
Combine the following polyatomic ions:
\begin{multicols}{2}
  \noindent
  \begin{enumerate} [topsep=0pt, partopsep=1pt, label=(\alph*), leftmargin=.5cm]
\item \ce{Na^{+}} + \ce{NO3^{-}}   %(\ce{NaNO3})
 \item \ce{Na^{+}} + \ce{CO3^{2-}}   %(\ce{Na2CO3})
 \item \ce{Na^{+}} + \ce{PO4^{3-}}   %(\ce{Na3PO4})
 \item \ce{Ca^{2+}} + \ce{CO3^{2-}}   %(\ce{CaCO3})
 \item \ce{Ca^{2+}} + \ce{PO4^{3-}}   %(\ce{Ca3(PO4)2})
\end{enumerate}
\end{multicols}    
\end{question}
\begin{solution}
\begin{inparaenum}[(a)]
\item \ce{Na^{+}} + \ce{NO3^{-}}    (\ce{NaNO3})
 \item \ce{Na^{+}} + \ce{CO3^{2-}}    (\ce{Na2CO3})
 \item \ce{Na^{+}} + \ce{PO4^{3-}}    (\ce{Na3PO4})
 \item \ce{Ca^{2+}} + \ce{CO3^{2-}}    (\ce{CaCO3})
 \item \ce{Ca^{2+}} + \ce{PO4^{3-}}    (\ce{Ca3(PO4)2})
\end{inparaenum}\hspace{0.1cm}\end{solution}\stepcounter{numA}
%%%%%%%%%%%%%%

%%%%%%%PROBLEM
\begin{question}[ID=\the\value{numA}]\SetQuestionProperties{section-title=\nameref{sec:units}}
Identify the ionic state of the following elements. If needed, indicate the existence of multiple ionic states:
\begin{inparaenum}[(a)]
\item \ce{Li} %\ce{Li^+}
\item \ce{V} % (multiple)
\item \ce{Cl} % \ce{Cl^{-}}
\item \ce{S} %\ce{S^{2-}}
\item \ce{Cr} % (multiple)
\item \ce{Sr} %\ce{Sr^{2+}}
\item \ce{Ni} % (multiple)
\end{inparaenum}
\end{question}
\begin{solution}
\begin{inparaenum}[(a)]
\item \ce{Li}  \ce{Li^+}
\item \ce{V}   (multiple)
\item \ce{Cl}   \ce{Cl^{-}}
\item \ce{S}  \ce{S^{2-}}
\item \ce{Cr}   (multiple)
\item \ce{Sr}  \ce{Sr^{2+}}
\item \ce{Ni}   (multiple)
\end{inparaenum}\hspace{0.1cm}\end{solution}\stepcounter{numA}




%%%%%%%PROBLEM
\begin{question}[ID=\the\value{numA}]\SetQuestionProperties{section-title=\nameref{sec:units}}
Combine the following ions:
\begin{multicols}{2}
  \noindent
  \begin{enumerate} [topsep=0pt, partopsep=1pt, label=(\alph*), leftmargin=.5cm]
\item \ce{Cs^{+}} + \ce{Ni^{-}}   %(\ce{CsNi})
\item \ce{Sr^{2+}} + \ce{Mn^{2-}}   %(\ce{CrMn})
\item \ce{Be^{2+}} + \ce{Co^{4-}}   %(\ce{Be2Co})
\item \ce{Li^{+}} + \ce{Cu^{-}}   %(\ce{LiCu})
\item \ce{Mg^{2+}} + \ce{Cr^{-6}}   %(\ce{Mg3Cr})
\end{enumerate}
\end{multicols}    
\end{question}
\begin{solution}
\begin{inparaenum}[(a)]
 \item \ce{Cs^{+}} + \ce{Ni^{-}}   (\ce{CsNi})
\item \ce{Sr^{2+}} + \ce{Mn^{2-}}    (\ce{CrMn})
\item \ce{Be^{2+}} + \ce{Co^{4-}}    (\ce{Be2Co})
\item \ce{Li^{+}} + \ce{Cu^{-}}    (\ce{LiCu})
\item \ce{Mg^{2+}} + \ce{Cr^{-6}}    (\ce{Mg3Cr})
\end{inparaenum}\hspace{0.1cm}\end{solution}\stepcounter{numA}




%%%%%%%PROBLEM
\begin{question}[ID=\the\value{numA}]\SetQuestionProperties{section-title=\nameref{sec:units}}
Formulate the following compounds:
 \noindent
  \begin{enumerate} [topsep=0pt, partopsep=0pt, label=(\alph*), leftmargin=.4cm]
\item  Iron(II) nitride %(\ce{Fe3N2})
 \item  Copper(I) sulfide %(\ce{Cu2S})
 \item  Chromium(III) iodide %(\ce{CrI3})
 \item  Palladium(IV) phosphide %(\ce{Pd3P4})
 \item  Manganese(VI) oxide %(\ce{MnO3})
\end{enumerate}
\end{question}
\begin{solution}
\begin{inparaenum}[(a)]
 \item  Iron(II) nitride  (\ce{Fe3N2})
 \item  Copper(I) sulfide  (\ce{Cu2S})
 \item  Chromium(III) iodide  (\ce{CrI3})
 \item  Palladium(IV) phosphide  (\ce{Pd3P4})
 \item  Manganese(VI) oxide  (\ce{MnO3})
\end{inparaenum}\hspace{0.1cm}\end{solution}\stepcounter{numA}
%%%%%%%%%%%%%%

%%%%%%%PROBLEM
\begin{question}[ID=\the\value{numA}]\SetQuestionProperties{section-title=\nameref{sec:units}}
Name the following compounds:
   \begin{multicols}{2}\noindent
  \begin{enumerate} [topsep=0pt, partopsep=0pt, label=(\alph*), leftmargin=.5cm]
\item  \ce{Ni2O3} %(nickel(III) oxide)
\item  \ce{Fe3N2} %(iron(II) nitride)
\item  \ce{Cr2O3} %(chromium(III) oxide)
\item  \ce{Ni3P2} %(nickel(II) phosphide)
\item  \ce{Ru2Se3} %(rubidium(III) selenide)
\end{enumerate}  \end{multicols} 
\end{question}
\begin{solution}
\begin{inparaenum}[(a)]
 \item  \ce{Ni2O3}  (nickel(III) oxide)
\item  \ce{Fe3N2}  (iron(II) nitride)
\item  \ce{Cr2O3}  (chromium(III) oxide)
\item  \ce{Ni3P2}  (nickel(II) phosphide)
\item  \ce{Ru2Se3}  (rubidium(III) selenide)
\end{inparaenum}\hspace{0.1cm}\end{solution}\stepcounter{numA}
%%%%%%%%%%%%%%


%%%%%%%PROBLEM
\begin{question}[ID=\the\value{numA}]\SetQuestionProperties{section-title=\nameref{sec:units}}
Name the following compounds:
   \begin{multicols}{2}\noindent
  \begin{enumerate} [topsep=0pt, partopsep=0pt, label=(\alph*), leftmargin=.5cm]
\item  \ce{FeO} %(Iron(II) oxide )
\item  \ce{CrN} %(Chromium(III) nitride )
\item  \ce{ZnI2} %(Zinc  iodide)
\item  \ce{CoS} %(Cobalt(II) sulfide )
\item  \ce{MnF3} %(Manganese(III)  fluoride)
\item  \ce{Cu2C} %(copper(IV)  carbide)
\item  \ce{Ag2O} %(silver oxide)
\end{enumerate}  \end{multicols} 
\end{question}
\begin{solution}
\begin{inparaenum}[(a)]
\item  \ce{FeO}  (Iron(II) oxide )
\item  \ce{CrN}  (Chromium(III) nitride )
\item  \ce{ZnI2}  (Zinc  iodide)
\item  \ce{CoS}  (Cobalt(II) sulfide )
\item  \ce{MnF3}  (Manganese(III)  fluoride)
\item  \ce{Cu2C}  (copper(IV)  carbide)
\item  \ce{Ag2O}  (silver oxide)
\end{inparaenum}\hspace{0.1cm}\end{solution}\stepcounter{numA}
%%%%%%%%%%%%%%

%%%%%%%PROBLEM
\begin{question}[ID=\the\value{numA}]\SetQuestionProperties{section-title=\nameref{sec:units}}
Name or formulate the following oxoanions:
 \begin{multicols}{2}
 \noindent
  \begin{enumerate} [topsep=0pt, partopsep=0pt, label=(\alph*), leftmargin=.4cm]
\item  \ce{ClO4^{-}} %(perchlorate)
\item  \ce{PO4^{3-}} %(phosphate)
\item  \ce{SO4^{2-}} %(sulfate)
\item  \ce{CO3^{2-}} %(carbonate)
\item  \ce{NO3^{-}} %(nitrate)
\item  \ce{CrO4^{2-}} %(chromate)
\item  \ce{BO3^{3-}} %(borate)
\end{enumerate}
\end{multicols}
\end{question}
\begin{solution}
\begin{inparaenum}[(a)]
 \item  \ce{ClO4^{-}}  (perchlorate)
\item  \ce{PO4^{3-}}  (phosphate)
\item  \ce{SO4^{2-}}  (sulfate)
\item  \ce{CO3^{2-}}  (carbonate)
\item  \ce{NO3^{-}}  (nitrate)
\item  \ce{CrO4^{2-}} (chromate)
\item  \ce{BO3^{3-}}  (borate)
\end{inparaenum}\hspace{0.1cm}\end{solution}\stepcounter{numA}
%%%%%%%%%%%%%%


%%%%%%%PROBLEM
\begin{question}[ID=\the\value{numA}]\SetQuestionProperties{section-title=\nameref{sec:units}}
Combine the following ions:
\begin{multicols}{2}
  \noindent
  \begin{enumerate} [topsep=0pt, partopsep=1pt, label=(\alph*), leftmargin=.5cm]
 \item \ce{Na^{+}} + \ce{PO4^{3-}}   %(\ce{Na3PO4})
\item \ce{Li^{+}} + \ce{MnO4^{-}}   %(\ce{LiMnO4})
\item \ce{Mg^{2+}} + \ce{NO3^{-}}   %(\ce{Mg(NO3)2})
\item \ce{Ca^{2+}} + \ce{CO3^{2-}}   %(\ce{CaCO3})
\item \ce{Cs^{+}} + \ce{Cr2O7^{2-}}   %(\ce{Cs2Cr2O7})
\item \ce{K^{+}} + \ce{BO3^{3-}}   %(\ce{K3BO3})
\end{enumerate}
\end{multicols}    
\end{question}
\begin{solution}
\begin{inparaenum}[(a)]
  \item \ce{Na^{+}} + \ce{PO4^{3-}}    (\ce{Na3PO4})
\item \ce{Li^{+}} + \ce{MnO4^{-}}    (\ce{LiMnO4})
\item \ce{Mg^{2+}} + \ce{NO3^{-}}    (\ce{Mg(NO3)2})
\item \ce{Ca^{2+}} + \ce{CO3^{2-}}    (\ce{CaCO3})
\item \ce{Cs^{+}} + \ce{Cr2O7^{2-}}    (\ce{Cs2Cr2O7})
\item \ce{K^{+}} + \ce{BO3^{3-}}    (\ce{K3BO3})
\end{inparaenum}\hspace{0.1cm}\end{solution}\stepcounter{numA}




%%%%%%%PROBLEM
\begin{question}[ID=\the\value{numA}]\SetQuestionProperties{section-title=\nameref{sec:units}}
Name or formulate the following compounds:
   \begin{multicols}{2}\noindent
  \begin{enumerate} [topsep=0pt, partopsep=0pt, label=(\alph*), leftmargin=.5cm]
\item  \ce{Na2SO4} %(sodium sulfate )
\item  \ce{KNO3} %( potassium nitrate )
\item  \ce{CaCO3} %(calcium carbonate  )
\item  \ce{Ca(NO2)2} %(calcium nitrate)
\item  \ce{SrSO3} %( strontium sulfite )
\end{enumerate}  \end{multicols} 
\end{question}
\begin{solution}
\begin{inparaenum}[(a)]
\item  \ce{Na2SO4}  (sodium sulfate)
\item  \ce{KNO3}  (potassium nitrate)
\item  \ce{CaCO3}  (calcium carbonate)
\item  \ce{Ca(NO2)2}  (calcium nitrate)
\item  \ce{SrSO3}  (strontium sulfite)
\end{inparaenum}\hspace{0.1cm}\end{solution}\stepcounter{numA}
%%%%%%%%%%%%%%


%%%%%%%PROBLEM
\begin{question}[ID=\the\value{numA}]\SetQuestionProperties{section-title=\nameref{sec:units}}
Name or formulate the following compounds:
   \begin{multicols}{2}\noindent
  \begin{enumerate} [topsep=0pt, partopsep=0pt, label=(\alph*), leftmargin=.5cm]
\item  \ce{MnSO4} %(manganese(II) sulfate )
\item  \ce{CuNO3} %(copper(I) nitrate )
\item  \ce{Cr2(CO3)3} %(chromium(III) carbonate  )
\item  \ce{V(NO2)2} %(vanadium(II) nitrate)
\item  \ce{FeSO3} %(Iron(II) sulfite )
\end{enumerate}  \end{multicols} 
\end{question}
\begin{solution}
\begin{inparaenum}[(a)]
\item  \ce{MnSO4}  (manganese(II) sulfate)
\item  \ce{CuNO3}  (copper(I) nitrate)
\item  \ce{Cr2(CO3)3}  (chromium(III) carbonate)
\item  \ce{V(NO2)2}  (vanadium(II) nitrate)
\item  \ce{FeSO3}  (Iron(II) sulfite)
\end{inparaenum}\hspace{0.1cm}\end{solution}\stepcounter{numA}
%%%%%%%%%%%%%%



%%%%%%%PROBLEM
\begin{question}[ID=\the\value{numA}]\SetQuestionProperties{section-title=\nameref{sec:units}}
Name or formulate the following pairs or ions:
   \begin{multicols}{2}\noindent
  \begin{enumerate} [topsep=0pt, partopsep=0pt, label=(\alph*), leftmargin=.5cm]
\item carbonate and monohydrogencarbonate %(\ce{CO3^{2-}} and \ce{HCO3^{-}})
 \item sulfate and monohydrogensulfate %(\ce{SO4^{2-}} and \ce{HSO4^{-}})
 \item cromate and monohydrogenchromate %(\ce{CrO4^{2-}} and \ce{HCrO4^{-}})
 \item phosphate and dihydrogenphosphate %(\ce{PO4^{3-}} and \ce{H2PO4^{-}})
 \item phosphate and monohydrogenphosphate %(\ce{PO4^{3-}} and \ce{HPO4^{2-}})
 \item borate and dihydrogenphosborate %(\ce{BO3^{3-}} and \ce{H2BO3^{-}})
\end{enumerate}  \end{multicols} 
\end{question}
\begin{solution}
\begin{inparaenum}[(a)]
 \item carbonate and monohydrogencarbonate  (\ce{CO3^{2-}} and \ce{HCO3^{-}})
 \item sulfate and monohydrogensulfate  (\ce{SO4^{2-}} and \ce{HSO4^{-}})
 \item cromate and monohydrogenchromate  (\ce{CrO4^{2-}} and \ce{HCrO4^{-}})
 \item phosphate and dihydrogenphosphate  (\ce{PO4^{3-}} and \ce{H2PO4^{-}})
 \item phosphate and monohydrogenphosphate  (\ce{PO4^{3-}} and \ce{HPO4^{2-}})
 \item borate and dihydrogenphosborate  (\ce{BO3^{3-}} and \ce{H2BO3^{-}})
\end{inparaenum}\hspace{0.1cm}\end{solution}\stepcounter{numA}
%%%%%%%%%%%%%%


%\documentclass[main.tex]{subfiles}
%\begin{document}\newpage
%\setdoublesep{0.35700 em}  % 'Bond Spacing'
%\setatomsep{1.78500 em}    % 'Fixed Length'
%\setbondoffset{0.18265 em} % 'Margin Width'
%\newcommand{\bondwidth}{0.06642 em} % 'Line Width'
%\setbondstyle{line width = \bondwidth}
%\newgeometry{left=0.8in,right=0.8in, top=2.5cm,bottom=2cm}
%\fancyhfoffset[E,O]{0pt}
%\setlength{\columnsep}{30pt}
%\begin{conclusion}
%\end{conclusion}
%\setstretch{0.3}
%\begin{multicols*}{2}

%{\raggedright\textsc{\textbf{Ions \& Ionic charges}}\par}
%
%\begin{enumerate}
%
%\item \ce{Fe^{2+}} is:
%\begin{enumerate}[label=(\alph*)]
%\begin{multicols*}{2}
%\item an atom
%\item an element
%\item an anion
%\item a cation
%\item none of the above
%\end{multicols*}\flushright  {\small Ans: (d)}
%\end{enumerate}
%
%\item \ce{F^{-}} is:
%\begin{enumerate}[label=(\alph*)]
%\begin{multicols*}{2}
%\item an atom
%\item an element
%\item an anion
%\item a cation
%\item none of the above
%\end{multicols*}\flushright  {\small Ans: (c)}
%\end{enumerate}
%
%\item \ce{Cl} is:
%\begin{enumerate}[label=(\alph*)]
%\begin{multicols*}{2}
%\item an atom
%\item an ion
%\item an anion
%\item a cation
%\item none of the above
%\end{multicols*}\flushright  {\small Ans: (a)}
%\end{enumerate}
%
%
%
%\item Identify the correct ionic state of Mg :
%\begin{enumerate}[label=(\alph*)]
%\begin{multicols*}{2}
%\item \ce{Mg^{2+}}
%\item \ce{Mg^{2-}}
%\item \ce{Mg^{+}}
%\item \ce{Mg^{2-}}
%\item \ce{Mg}
%\end{multicols*}\flushright  {\small Ans: (a)}
%\end{enumerate}
%
%\item Identify the correct ionic state of O:
%\begin{enumerate}[label=(\alph*)]
%\begin{multicols*}{2}
%\item \ce{O^{2+}}
%\item \ce{O^{2-}}
%\item \ce{O}
%\item \ce{O^{+}}
%\item \ce{O^{-}}
%\end{multicols*}\flushright  {\small Ans: (b)}
%\end{enumerate}
%
%
%\item Identify the correct ionic state of N:
%\begin{enumerate}[label=(\alph*)]
%\begin{multicols*}{2}
%\item \ce{N^{2+}}
%\item \ce{N^{2-}}
%\item \ce{N^{3-}}
%\item \ce{N}
%\item \ce{N^{-}}
%\end{multicols*}\flushright  {\small Ans: (c)}
%\end{enumerate}
%
%
%
%{\raggedright\textsc{\textbf{Covalent Compounds}}\par}
%
%\item Name or formulate the following compound: \ce{NO}
%\begin{enumerate}[label=(\alph*)]
%\begin{multicols*}{2}
%\item nitrogen monoxide
%\item nitrogen(I) oxide
%\item nitrogen(II) oxide
%\item mononitrogen monoxide
%\item oxygen nitride
%\end{multicols*}\flushright  {\small Ans: (a)}
%\end{enumerate}
%
%\item Name or formulate the following compound: phosphorus trichloride
%\begin{enumerate}[label=(\alph*)]
%\begin{multicols*}{2}
%\item \ce{PCl5}
%\item \ce{PCl3}
%\item \ce{PCl2}
%\item \ce{P3Cl}
%\item \ce{PCl}
%\end{multicols*}\flushright  {\small Ans: (b)}
%\end{enumerate}
%
%
%\item Name or formulate the following compound: \ce{SO3}
%\begin{enumerate}[label=(\alph*)]
%\begin{multicols*}{2}
%\item sulfur(III) oxide
%\item sulfur(VI) oxide
%\item sulfur trioxide
%\item trisulfur monoxide
%\item trisulfur oxide
%\end{multicols*}\flushright  {\small Ans: (c)}
%\end{enumerate}
%
%\item Name or formulate the following compound: sulfur hexafluoride
%\begin{enumerate}[label=(\alph*)]
%\begin{multicols*}{2}
%\item \ce{SF6}
%\item \ce{S6F}
%\item \ce{SF}
%\item \ce{SF2}
%\item \ce{S6F3}
%\end{multicols*}\flushright  {\small Ans: (a)}
%\end{enumerate}
%
%
%{\raggedright\textsc{\textbf{Ionic Compounds}}\par}
%
%\item Name or formulate the following compound: \ce{Li3N}
%\begin{enumerate}[label=(\alph*)]
%\begin{multicols*}{2}
%\item Trilithium nitride
%\item Lithium(I) nitride
%\item Lithium nitride
%\item Lithium Dinitride
%\item Lithium trinitride
%\end{multicols*}\flushright  {\small Ans: (c)}
%\end{enumerate}
%
%\item Name or formulate the following compound: \ce{MgCl2}
%\begin{enumerate}[label=(\alph*)]
%\begin{multicols*}{2}
%\item Magnesium(II) chloride
%\item Magnesium chloride(II)
%\item Magnesium Dichloride
%\item Dimagnesium chloride
%\item Magnesium chloride
%\end{multicols*}\flushright  {\small Ans: (e)}
%\end{enumerate}
%
%\item Name or formulate the following compound: calcium sulfide
%\begin{enumerate}[label=(\alph*)]
%\begin{multicols*}{2}
%\item \ce{Ca2S2}
%\item \ce{CaS2}
%\item \ce{Ca2S}
%\item \ce{CaS}
%\item \ce{Ca2S1}
%\end{multicols*}\flushright  {\small Ans: (d)}
%\end{enumerate}
%
%\item Combine the following ions: \ce{Ba^{2+}} and \ce{P^{3-}} 
%\begin{enumerate}[label=(\alph*)]
%\begin{multicols*}{2}
%\item \ce{BaP}
%\item \ce{Ba2P3}
%\item \ce{Ba3P2}
%\item \ce{BaP2}
%\item \ce{Ba3P3}
%\end{multicols*}\flushright  {\small Ans: (c)}
%\end{enumerate}
%
%
%\item Name or formulate the following compound: \ce{VO}
%\begin{enumerate}[label=(\alph*)]
%\begin{multicols*}{2}
%\item Vanadium monoxide
%\item Vanadium(I) oxide
%\item Vanadium(II) oxide
%\item Vanadium oxide
%\end{multicols*}\flushright  {\small Ans: (c)}
%\end{enumerate}
%
%
%
%
%\item Name or formulate the following compound: \ce{Ni2O3}
%\begin{enumerate}[label=(\alph*)]
%\begin{multicols*}{2}
%\item Nickel(II) oxide
%\item Nickel(I) oxide
%\item Nickel oxide
%\item Nickel monoxide
%\end{multicols*}\flushright  {\small Ans: (a)}
%\end{enumerate}
%
%\item Name or formulate the following compound: Iron(III) chloride
%\begin{enumerate}[label=(\alph*)]
%\begin{multicols*}{2}
%\item \ce{FeCl3}
%\item \ce{FeCl2}
%\item \ce{Fe3Cl}
%\item \ce{Fe2Cl3}
%\item \ce{FeCl}
%\end{multicols*}\flushright  {\small Ans: (a)}
%\end{enumerate}
%
%
%
%{\raggedright\textsc{\textbf{Acids and Bases Naming}}\par}
%
%
%\item  Name or formulate the following compound: hydrochloric Acid
%\begin{enumerate}[label=(\alph*)]
%\begin{multicols*}{2}
%\item \ce{HI}
%\item \ce{HClO4}
%\item \ce{HCl}
%\item \ce{HClO3}
%\item \ce{ClH}
%\end{multicols*}\flushright  {\small Ans: (c)}
%\end{enumerate}
%
%
%
%\item  Name or formulate the following compound: hydroiodic Acid
%\begin{enumerate}[label=(\alph*)]
%\begin{multicols*}{2}
%\item \ce{HI}
%\item \ce{HClO4}
%\item \ce{HCl}
%\item \ce{HClO3}
%\item \ce{ClH}
%\end{multicols*}\flushright  {\small Ans: (a)}
%\end{enumerate}
%
%
%\item  Name or formulate the following compound: perchloric acid
%\begin{enumerate}[label=(\alph*)]
%\begin{multicols*}{2}
%\item \ce{HMnO4}
%\item \ce{H2CO3}
%\item \ce{HNO3}
%\item \ce{H2SO4}
%\item \ce{HClO4}
%\end{multicols*}\flushright  {\small Ans: (e)}
%\end{enumerate}
%
%\item  Name or formulate the following compound: nitric acid
%\begin{enumerate}[label=(\alph*)]
%\begin{multicols*}{2}
%\item \ce{HMnO4}
%\item \ce{H2CO3}
%\item \ce{HNO3}
%\item \ce{H2SO4}
%\item \ce{HClO4}
%\end{multicols*}\flushright  {\small Ans: (c)}
%\end{enumerate}
%
%\item  Name or formulate the following compound: permanganic acid
%\begin{enumerate}[label=(\alph*)]
%\begin{multicols*}{2}
%\item \ce{HMnO4}
%\item \ce{H2CO3}
%\item \ce{HNO3}
%\item \ce{H2SO4}
%\item \ce{HClO4}
%\end{multicols*}\flushright  {\small Ans: (a)}
%\end{enumerate}
%
%\item  Name or formulate the following compound: sulfuric acid
%\begin{enumerate}[label=(\alph*)]
%\begin{multicols*}{2}
%\item \ce{HMnO4}
%\item \ce{H2CO3}
%\item \ce{HNO3}
%\item \ce{H2SO4}
%\item \ce{HClO4}
%\end{multicols*}\flushright  {\small Ans: (d)}
%\end{enumerate}
%
%
%\item  Name or formulate the following compound: \ce{Ca(OH)2}
%\begin{enumerate}[label=(\alph*)]
%\begin{multicols*}{2}
%\item calcium hydroxyde
%\item calcium(II) hydroxyde
%\item calcium(I) hydroxyde
%\item Calcium dihydroxide
%\item Calcium hydrate
%\end{multicols*}\flushright  {\small Ans: (a)}
%\end{enumerate}
%
%
%
%{\raggedright\textsc{\textbf{Naming of oxosalts, hydrosalts, hydrates  \& common chemicals}}\par}
%\item  Combine the chemical: \ce{SO4^{2-}} and \ce{Ca^{2+}}
%\begin{enumerate}[label=(\alph*)]
%\begin{multicols*}{2}
%\item \ce{CaSO4}
%\item \ce{Ca2(SO4)2}
%\item \ce{Ca2SO4}
%\item \ce{CaSO3}
%\item \ce{CaS4}
%\end{multicols*}\flushright  {\small Ans: (a)}
%\end{enumerate}
%
%\item  Give the ions forming the chemical: \ce{KNO3}
%\begin{enumerate}[label=(\alph*)]
%\begin{multicols*}{2}
%\item \ce{NO4^{-}} and \ce{K^{+}}
%\item  \ce{NO3^{-2}} and \ce{K^{+}}
%\item \ce{NO3^{-2}} and \ce{K^{+2}}
%\item  \ce{NO3^{-}} and \ce{K^{+}}
%\item \ce{NO3^{-}} and \ce{K^{2+}}
%\end{multicols*}\flushright  {\small Ans: (d)}
%\end{enumerate}
%
%
%%\item  Name or formulate the following compound: \ce{LiNO3}
%%\begin{enumerate}[label=(\alph*)]
%%\begin{multicols*}{2}
%%\item Lithium nitrate
%%\item Lithium nitrogen trioxide 
%%\item Lithium(I) nitrate
%%\item  Lithium nitrogen oxide
%%\item Lithium nitrite
%%\end{multicols*}\flushright  {\small Ans: (a)}
%%\end{enumerate}
%
%\item  Name or formulate the following compound: \ce{Na2CO3}
%\begin{enumerate}[label=(\alph*)]
%\begin{multicols*}{2}
%\item Sodium bicarbonate
%\item  Sodium(I) carbonate
%\item Sodium(II) carbonate
%\item  Sodium carbonate
%\item Sodium carbon oxide
%\end{multicols*}\flushright  {\small Ans: (d)}
%\end{enumerate}
%
%\item  Name or formulate the following compound: \ce{FeCO3}
%\begin{enumerate}[label=(\alph*)]
%\begin{multicols*}{2}
%\item Iron(II) carbonate
%\item  Iron(I) carbonate
%\item Iron carbonate
%\item Iron carbon oxide
%\item Iron(III) carbonate
%\end{multicols*}\flushright  {\small Ans: (a)}
%\end{enumerate}
%
%
%\item  Name or formulate the following compound: potassium permanganate
%\begin{enumerate}[label=(\alph*)]
%\begin{multicols*}{2}
%\item \ce{KMnO}
%\item  \ce{K2Mn}
%\item \ce{K2MnO4}
%\item \ce{KMnO3}
%\item \ce{KMnO4}
%\end{multicols*}\flushright  {\small Ans: (e)}
%\end{enumerate}
%
%\item  Name or formulate the following compound: sodium bicarbonate(common name)
%\begin{enumerate}[label=(\alph*)]
%\begin{multicols*}{2}
%\item  \ce{NaCO4}
%\item  \ce{NaHCO3}
%\item  \ce{NaCO3}
%\item  \ce{Na2CO3}
%\item \ce{NaCO2}
%\end{multicols*}\flushright  {\small Ans: (b)}
%\end{enumerate}
%
%%\item  Name or formulate the following compound: table salt(common name)
%%\begin{enumerate}[label=(\alph*)]
%%\begin{multicols*}{2}
%%\item  \ce{CH4}
%%\item  \ce{NH3}
%%\item  \ce{NaCO3}
%%\item  \ce{Mg(OH)2}
%%\item \ce{NaCl}
%%\end{multicols*}\flushright  {\small Ans: (e)}
%%\end{enumerate}
%
%\item  Name or formulate the following compound: ammonia(common name)
%\begin{enumerate}[label=(\alph*)]
%\begin{multicols*}{2}
%\item  \ce{CH4}
%\item  \ce{NH3}
%\item  \ce{NaCO3}
%\item  \ce{Mg(OH)2}
%\item \ce{NaCl}
%\end{multicols*}\flushright  {\small Ans: (b)}
%\end{enumerate}
%
%
%
%
%
%

%
%
%
%
%\restoregeometry
%\end{enumerate}
%\end{multicols*}
%\pagecolor{green!10}\afterpage{\nopagecolor}\newpage
%\end{document}



%\restoregeometry
\end{multicols*}


\newpage
\begin{answersenvironment}
\begin{minipage}[c]{1\textwidth}
\begin{localsize}{10}
{\Large \bf Answers}
\SetupExSheets{
  headings = inline-nr , % numbered and inline
  counter-format = qu) , % numbers 1) 2) ... 
}
%\printsolutions 
\printsolutions[byID={1,3,5,7,9,11,13,15,17,19,21,23,25,27,29,31}]
\end{localsize}
\end{minipage}\end{answersenvironment}
%\clearpage\thispagestyle{empty}\mbox{}
\end{document}
